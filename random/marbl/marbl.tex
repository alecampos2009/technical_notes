\documentclass[11pt]{article}

\usepackage{"../../info/packages"}
\usepackage{"../../info/nomenclature"}
\usepackage{fullpage}


\title{Marbl}


\begin{document}

\maketitle

\section{Governing equations}
Define $\rho = \rho(\xvec,t)$, $\uvec = \uvec(\xvec,t)$, $e = e(\xvec,t)$, and $p = p(\xvec,t)$.
\begin{equation}
    \frac{\partial \rho}{\partial t} + \uvec \cdot \nabla \rho = -\rho \nabla \cdot \uvec
\end{equation}
\begin{equation}
    \rho \left ( \frac{\partial \uvec}{\partial t} + \uvec \cdot \nabla \uvec \right ) = -\nabla p
\end{equation}
\begin{equation}
    \rho \left ( \frac{\partial e}{\partial t} + \uvec \cdot \nabla e \right ) = -p \nabla \cdot \uvec
\end{equation}

\section{Finite element expansion}
Define the coefficients $R_i = R_i(t)$, $\Uvec_i = \Uvec_i(t)$, $e_i = e_i(t)$, $p_i = p_i(t)$, as well as the basis functions $\phi_i = \phi_i(\xvec,t)$, and $w_i = w_i(\xvec,t)$.
\begin{equation}
    \rho = \sum_i^{N_\rho} R_i \phi_i,
\end{equation}
\begin{equation}
    \uvec = \sum_i^{N_u} \Uvec_i w_i,
\end{equation}
\begin{equation}
    e = \sum_i^{N_e} E_i \phi_i,
\end{equation}
\begin{equation}
    p = \sum_i^{N_p} P_i \phi_i.
\end{equation}
The basis functions satisfy
\begin{equation}
    \frac{\partial \phi_i}{\partial t} + \uvec \cdot \nabla \phi_i = 0,
\end{equation}
\begin{equation}
    \frac{\partial w_i}{\partial t} + \uvec \cdot \nabla w_i = 0.
\end{equation}

\section{Semi-discrete momentum conservation}
We begin by showing that
\begin{align}
    \frac{\partial \uvec}{\partial t} + \uvec \cdot \nabla \uvec &= \left ( \sum_i^{N_u} \frac{d \Uvec_i}{d t} w_i + \Uvec_i \frac{\partial w_i}{\partial t} \right ) + \uvec \cdot \left ( \sum_i^{N_u} \Uvec_i \nabla w_i \right ) \nonumber \\
    &= \sum_i^{N_u} \frac{d \Uvec_i}{d t} w_i + \Uvec_i \left ( \frac{\partial w_i}{\partial t} + \uvec \cdot \nabla w_i \right ) \nonumber \\
    &= \sum_i^{N_u} \frac{d \Uvec_i}{d t} w_i.
\end{align}
Define $\Omega = \Omega(t)$. The finite element formulation of the momentum equation is thus 
\begin{equation}
    \int_\Omega \rho \sum_i^{N_u} \frac{d \Uvec_i}{d t} w_i w_j \, dV = \int_\Omega \sum_i^{N_p} P_i \phi_i \nabla w_j \, dV \qquad \text{for } j = 1,...,N_u.
\end{equation}
The above is re-written as
\begin{equation}
    \label{eq:semi_discrete_momentum_bilinear}
    \sum_i^{N_u} \frac{d \Uvec_i}{d t} m_{ij} = \sum_i^{N_p} P_i \mathbf{d}_{ij} \qquad \text{for } j = 1,...,N_u.
\end{equation}
where the mass bilinear form $m_{ij}$ is given by 
\begin{equation}
    m_{ij} = \int_\Omega \rho w_i w_j \, dV,
\end{equation}
and the derivative bilinear form $\mathbf{d}_{ij}$ by 
\begin{equation}
    \mathbf{d}_{ij} = \int_\Omega \phi_i \nabla w_j \, dV.
\end{equation}
In terms of its components, \cref{eq:semi_discrete_momentum_bilinear} becomes
\begin{align}
    \sum_i^{N_u} \frac{d U_{x,i}}{d t} m_{ij} &= \sum_i^{N_p} P_i d_{x,ij} \qquad \text{for } j = 1,...,N_u, \nonumber \\
    \sum_i^{N_u} \frac{d U_{y,i}}{d t} m_{ij} &= \sum_i^{N_p} P_i d_{y,ij} \qquad \text{for } j = 1,...,N_u, \nonumber \\
    \sum_i^{N_u} \frac{d U_{z,i}}{d t} m_{ij} &= \sum_i^{N_p} P_i d_{z,ij} \qquad \text{for } j = 1,...,N_u.
\end{align}
In matrix notation, the above can be written as 


\end{document}