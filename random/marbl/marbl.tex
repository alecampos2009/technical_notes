\documentclass[11pt]{article}

\usepackage{"../../info/packages"}
\usepackage{"../../info/nomenclature"}
\usepackage{fullpage}


\title{Marbl}


\begin{document}

\maketitle

\section{Governing equations}
We introduce the flow variables $\rho = \rho(\xvec,t)$, $\uvec = \uvec(\xvec,t)$, $e = e(\xvec,t)$, and $p = p(\xvec,t)$. The governing equations that dictate their evolution are
\begin{equation}
    \frac{\partial \rho}{\partial t} + \uvec \cdot \nabla \rho = -\rho \nabla \cdot \uvec,
\end{equation}
\begin{equation}
    \rho \left ( \frac{\partial \uvec}{\partial t} + \uvec \cdot \nabla \uvec \right ) = -\nabla p,
\end{equation}
\begin{equation}
    \rho \left ( \frac{\partial e}{\partial t} + \uvec \cdot \nabla e \right ) = -p \nabla \cdot \uvec.
\end{equation}

\section{Finite element expansion}
We introduce the coefficients $R_i = R_i(t)$, $\Uvec_i = \Uvec_i(t)$, $E_i = E_i(t)$, $P_i = P_i(t)$, as well as the basis functions $\phi_i = \phi_i(\xvec,t)$, and $w_i = w_i(\xvec,t)$. We note that $\Uvec_i$ is a vector whose components are $U_{i,\alpha} = U_{i,\alpha}(t)$ for $\alpha = x,y,z$. These coefficients are used in the following expansions
\begin{equation}
    \rho = \sum_j^{N_\rho} R_j \phi_j,
\end{equation}
\begin{equation}
    \uvec = \sum_j^{N_u} \Uvec_j w_j,
\end{equation}
\begin{equation}
    e = \sum_j^{N_e} E_j \phi_j,
\end{equation}
\begin{equation}
    p = \sum_j^{N_p} P_j \phi_j.
\end{equation}
The basis functions are defined so that they satisfy
\begin{equation}
    \frac{\partial \phi_j}{\partial t} + \uvec \cdot \nabla \phi_j = 0,
\end{equation}
\begin{equation}
    \frac{\partial w_j}{\partial t} + \uvec \cdot \nabla w_j = 0.
\end{equation}

\section{Semi-discrete momentum conservation}
We begin by showing that
\begin{align}
    \frac{\partial \uvec}{\partial t} + \uvec \cdot \nabla \uvec &= \sum_j^{N_u} \left ( \frac{d \Uvec_j}{d t} w_j + \Uvec_j \frac{\partial w_j}{\partial t} \right ) + \uvec \cdot \left ( \sum_j^{N_u} \Uvec_j \nabla w_j \right ) \nonumber \\
    &= \sum_j^{N_u} \left [ \frac{d \Uvec_j}{d t} w_j + \Uvec_j \left ( \frac{\partial w_j}{\partial t} + \uvec \cdot \nabla w_j \right ) \right ] \nonumber \\
    &= \sum_j^{N_u} \frac{d \Uvec_j}{d t} w_j.
\end{align}
Define the domain of the problem under consideration as $\Omega = \Omega(t)$. The finite element formulation of the momentum equation is thus 
\begin{equation}
    \int_\Omega \rho \sum_j^{N_u} \frac{d \Uvec_j}{d t} w_j w_i \, dV = \int_\Omega \sum_j^{N_p} P_j \phi_j \nabla w_i \, dV \qquad \text{for } i = 1,...,N_u.
\end{equation}
The above is re-written as
\begin{equation}
    \label{eq:semi_discrete_momentum_bilinear}
    \sum_j^{N_u} \frac{d \Uvec_j}{d t} m_{ij} = \sum_j^{N_p} P_j \mathbf{d}_{ji} \qquad \text{for } i = 1,...,N_u.
\end{equation}
where the mass bilinear form $m_{ij}$ is given by 
\begin{equation}
    m_{ij} = \int_\Omega \rho w_i w_j \, dV,
\end{equation}
and the derivative bilinear form $\mathbf{d}_{ij}$ by 
\begin{equation}
    \mathbf{d}_{ij} = \int_\Omega \phi_i \nabla w_j \, dV.
\end{equation}
Note that $\mathbf{d}_{ij}$ is a vector whose components are $d_{ij,\alpha}$, for $\alpha = x,y,z$, where $\alpha$ determines which component of the $\nabla$ operator is being used. \Cref{eq:semi_discrete_momentum_bilinear} can thus be expanded as
\begin{align}
    \label{eq:semi_discrete_momentum_bilinear_expanded}
    \sum_j^{N_u} \frac{d U_{j,x}}{d t} m_{ij} &= \sum_i^{N_p} P_j d_{ji,x} \qquad \text{for } i = 1,...,N_u, \nonumber \\
    \sum_j^{N_u} \frac{d U_{j,y}}{d t} m_{ij} &= \sum_i^{N_p} P_j d_{ji,y} \qquad \text{for } i = 1,...,N_u, \nonumber \\
    \sum_j^{N_u} \frac{d U_{j,z}}{d t} m_{ij} &= \sum_i^{N_p} P_j d_{ji,z} \qquad \text{for } i = 1,...,N_u.
\end{align}
We'll now write the above in matrix notation. Introduce the vector $\Uvec_x = \Uvec_x(t)$ whose components are $U_{i,x}$ for $i=1,...,N_v$. The analogous holds for $\Uvec_y$ and $\Uvec_z$. Similarly, we introduce the matrix $\Dvec_x = \Dvec_x(t)$ whose components are $d_{ij,x}$. The analogous holds for $D_y$ and $D_z$. Finally, the matrix $\Mvec=\Mvec(t)$ is that with components $m_{ij}$ and the vector $\Pvec = \Pvec(t)$ is that with components $P_j$. \Cref{eq:semi_discrete_momentum_bilinear_expanded} can now be written as 
\begin{align}
    M \frac{d \Uvec_x}{dt} &= \Dvec_x^T \Pvec, \nonumber \\
    M \frac{d \Uvec_y}{dt} &= \Dvec_y^T \Pvec, \nonumber \\
    M \frac{d \Uvec_z}{dt} &= \Dvec_z^T\Pvec.
\end{align}

\end{document}