\documentclass[11pt]{article}

\usepackage{"../../info/packages"}
\usepackage{"../../info/nomenclature"}
\usepackage{fullpage}


\title{Marbl}


\begin{document}

\maketitle

%------------------------------------------------------------------------
\section{Governing equations}
%------------------------------------------------------------------------
We introduce the flow variables for density $\rho = \rho(\xvec,t)$, velocity $\uvec = \uvec(\xvec,t)$, internal energy $e = e(\xvec,t)$, and stress tensor $\sigmavec = \sigmavec(\xvec,t)$. These are defined within a domain $\Omega = \Omega(t)$, which can be moving. The governing equations that dictate their evolution in the laboratory reference frame are
\begin{equation}
    \frac{\partial \rho}{\partial t} + \uvec \cdot \nabla \rho = -\rho \nabla \cdot \uvec,
\end{equation}
\begin{equation}
    \rho \left ( \frac{\partial \uvec}{\partial t} + \uvec \cdot \nabla \uvec \right ) = \nabla \cdot \sigmavec,
\end{equation}
\begin{equation}
    \rho \left ( \frac{\partial e}{\partial t} + \uvec \cdot \nabla e \right ) = \sigmavec : \nabla \uvec.
\end{equation}
A note on notation. The products that involve a tensor $\tauvec$ can be expressed in Einstein notation as
\begin{equation}
    \nabla \cdot \tauvec = \frac{\partial \tau_{ij}}{\partial x_j},
\end{equation}
\begin{equation}
    \tauvec \cdot \nabla \alpha = \tau_{ij} \frac{\partial \alpha}{\partial x_j},
\end{equation}
\begin{equation}
    \fvec \cdot \tauvec \cdot \nabla \alpha = f_i \tau_{ij} \frac{\partial \alpha}{\partial x_j},
\end{equation}
\begin{equation}
    \tauvec : \nabla \fvec = \tau_{ij} \frac{\partial f_i}{\partial x_j}.
\end{equation}
where $\alpha$ is a scalar and $\fvec$ a vector. In these notes we'll mostly be using indices $i$ and $j$ for FE expansions, rather than for Einstein notation.

%------------------------------------------------------------------------
\section{Finite element expansion}
%------------------------------------------------------------------------
We introduce the coefficients $\hat{d}_i = \hat{d}_i(t)$, $\hat{\uvec}_i = \hat{\uvec}_i(t)$ and $\hat{e}_i = \hat{e}_i(t)$, as well as the basis functions $\phi_i = \phi_i(\xvec,t) \in L^2$, and $w_i = w_i(\xvec,t) \in H^1$. We note that $\hat{\uvec}_i$ is a vector whose components are $\hat{u}_{i,\alpha} = \hat{u}_{i,\alpha}(t)$ for $\alpha = x,y,z$. These coefficients are used in the following expansions
\begin{equation}
    \rho = \sum_j^{N_\phi} \hat{d}_j \phi_j,
\end{equation}
\begin{equation}
    \uvec = \sum_j^{N_w} \hat{\uvec}_j w_j,
\end{equation}
\begin{equation}
    e = \sum_j^{N_\phi} \hat{e}_j \phi_j,
\end{equation}
The basis functions are defined so that they are Lagrangian, that is,
\begin{equation}
    \frac{\partial \phi_j}{\partial t} + \uvec \cdot \nabla \phi_j = 0,
\end{equation}
\begin{equation}
    \frac{\partial w_j}{\partial t} + \uvec \cdot \nabla w_j = 0.
\end{equation}

%------------------------------------------------------------------------
\section{Semi-discrete momentum conservation}
%------------------------------------------------------------------------
We begin by showing that
\begin{align}
    \frac{\partial \uvec}{\partial t} + \uvec \cdot \nabla \uvec &= \sum_j^{N_w} \left ( \frac{d \hat{\uvec}_j}{d t} w_j + \hat{\uvec}_j \frac{\partial w_j}{\partial t} \right ) + \uvec \cdot \left ( \sum_j^{N_w} \hat{\uvec}_j \nabla w_j \right ) \nonumber \\
    &= \sum_j^{N_w} \left [ \frac{d \hat{\uvec}_j}{d t} w_j + \hat{\uvec}_j \left ( \frac{\partial w_j}{\partial t} + \uvec \cdot \nabla w_j \right ) \right ] \nonumber \\
    &= \sum_j^{N_w} \frac{d \hat{\uvec}_j}{d t} w_j.
\end{align}
The finite element formulation of the momentum equation is thus 
\begin{equation}
    \int_\Omega \rho \sum_j^{N_w} \frac{d \hat{\uvec}_j}{d t} w_j w_i \, dV = -\int_\Omega \sigmavec \cdot \nabla w_i \, dV \qquad \text{for } i = 1,...,N_w.
\end{equation}
The above is re-written as
\begin{equation}
    \label{eq:semi_discrete_momentum_bilinear}
    \sum_j^{N_w} \frac{d \hat{\uvec}_j}{d t} m^{(w)}_{ij} = -\int_\Omega \sigmavec \cdot \nabla w_i \, dV \qquad \text{for } i = 1,...,N_w.
\end{equation}
where the mass bilinear form $m^{(w)}_{ij}$ is given by 
\begin{equation}
    m^{(w)}_{ij} = \int_\Omega \rho w_i w_j \, dV.
\end{equation}

We now introduce the vector $\Uvec$ whose components are $\hat{\uvec}_i$. We also introduce the matrix $\Mvec^{(w)}$ whose components are $m_{ij}^{(w)}$. Thus, the left-hand side of \cref{eq:semi_discrete_momentum_bilinear} can be written as $\Mvec^{(w)} \, d\Uvec / dt$.
We also introduce the vector bilinear form
\begin{equation}
    \fvec_{ij} = \int_\Omega \sigmavec \cdot \nabla w_i \phi_j dV.
\end{equation}
This is a \textit{vector} bilinear form since $\fvec_{ij}$ has components $f_{ij,\alpha} = f_{ij,\alpha}(t)$, for $\alpha = x,y,z$, where $\alpha$ denotes the first index of $\sigmavec$. We introduce the matrix $\Fvec$, whose components are $\fvec_{ij}$. We also expand the field with constant value of one as follows
\begin{equation}
    1 = \sum_i^{N_\phi} \hat{c}_i \phi_i.
\end{equation}
If we define the vector $\Cvec$ as that with components $\hat{c}_i$, we can show that 
\begin{align}
    \Fvec \Cvec &= \sum_j^{N_\phi} \fvec_{ij} \hat{c}_j &\text{for } i = 1,...,N_w \nonumber \\
    &= \sum_j^{N_\phi} \int_\Omega \sigmavec \cdot \nabla w_i \phi_j \, dV \hat{c}_j &\text{for } i = 1,...,N_w \nonumber \\
    &= \int_\Omega \sigmavec \cdot \nabla w_i \left ( \sum_j^{N_\phi} \hat{c}_j \phi_j \right ) \, dV &\text{for } i = 1,...,N_w \nonumber \\
    &= \int_\Omega \sigmavec \cdot \nabla w_i \, dV &\text{for } i = 1,...,N_w
\end{align}
The above is the negative of the right-hand side of \cref{eq:semi_discrete_momentum_bilinear}. Thus, combining all together we get
\begin{equation}
    \Mvec^{(w)} \frac{d\Uvec}{dt} = -\Fvec \Cvec.
\end{equation}
We note that since both the Lagrangian and Eulerian velocities share the same coefficients $\Uvec$, we now have a solution for both.

%------------------------------------------------------------------------
\section{Semi-discrete energy conservation}
%------------------------------------------------------------------------
As with momentum conservation, we have
\begin{align}
    \frac{\partial \evec}{\partial t} + \uvec \cdot \nabla \evec &= \sum_j^{N_\phi} \left ( \frac{d \hat{\evec}_j}{d t} \phi_j + \hat{\evec}_j \frac{\partial \phi_j}{\partial t} \right ) + \uvec \cdot \left ( \sum_j^{N_\phi} \hat{\evec}_j \nabla \phi_j \right ) \nonumber \\
    &= \sum_j^{N_\phi} \left [ \frac{d \hat{\evec}_j}{d t} \phi_j + \hat{\evec}_j \left ( \frac{\partial \phi_j}{\partial t} + \uvec \cdot \nabla \phi_j \right ) \right ] \nonumber \\
    &= \sum_j^{N_\phi} \frac{d \hat{\evec}_j}{d t} \phi_j.
\end{align}
For the right-hand side of the energy conservation equation, we have
\begin{equation}
    \sigmavec : \nabla \uvec = \sigmavec : \nabla \left ( \sum_k^{N_w} \hat{\uvec}_k w_k \right ) = \sum_k^{N_w} \hat{\uvec}_k \cdot \sigmavec \cdot \nabla w_k.
\end{equation}
The finite element formulation of the energy equation is thus 
\begin{align}
    \int_\Omega \rho \sum_j^{N_\phi} \frac{d \hat{\evec}_j}{d t} \phi_j \phi_i \, dV &= \int_\Omega \left ( \sum_k^{N_w} \hat{\uvec}_k \cdot \sigmavec \cdot \nabla w_k \right ) \phi_i \, dV &\text{for } i = 1,...,N_w \nonumber \\
    &=\sum_k^{N_w} \hat{\uvec}_k \cdot \int_\Omega \sigmavec \cdot \nabla w_k \phi_i \, dV &\text{for } i = 1,...,N_w
\end{align}
The above is re-written as
\begin{equation}
    \label{eq:semi_discrete_energy_bilinear}
    \sum_j^{N_\phi} \frac{d \hat{\evec}_j}{d t} m^{(\phi)}_{ij} = \sum_k^{N_w} \hat{\uvec}_k \cdot \fvec_{ki} \qquad \text{for } i = 1,...,N_w.
\end{equation}
where the mass bilinear form $m^{\phi}_{ij}$ is given by 
\begin{equation}
    m^{(\phi)}_{ij} = \int_\Omega \rho \phi_i \phi_j \, dV.
\end{equation}
Note that in \cref{eq:semi_discrete_energy_bilinear} there is a dot product in the right-hand side, that is, the right-hand side expanded out is  
\begin{equation}
    \sum_k^{N_w} \hat{\uvec}_k \cdot \fvec_{ki} = \sum_k^{N_w} \sum_{\alpha=x,y,z} \hat{u}_{k,\alpha} f_{ki,\alpha}.
\end{equation}

We now introduce the vector $\Evec$ whose components are $\hat{e}_i$. We also introduce the matrix $\Mvec^{(\phi)}$ whose components are $m_{ij}^{(\phi)}$. Thus, \cref{eq:semi_discrete_energy_bilinear} can be succinctly written as
\begin{equation}
    \Mvec^{(\phi)} \frac{d\Evec}{dt} = \Fvec^T \cdot \Uvec.
\end{equation}
Note again that on the right-hand side above there is a matrix-vector product \textit{and} a dot product. We also note that since both the Lagrangian and Eulerian internal energies share the same coefficients $\Evec$, we now have a solution for both.

%------------------------------------------------------------------------
\section{Semi-discrete equations for $\xvec^+$, $\Jvec^+$ and $\rho^+$}
%------------------------------------------------------------------------


\end{document}