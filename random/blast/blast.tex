\documentclass[11pt]{article}

\usepackage{"../../info/packages"}
\usepackage{"../../info/nomenclature"}
\usepackage{fullpage}


\title{Blast}


\begin{document}

\maketitle

%------------------------------------------------------------------------
\section{Governing equations}
%------------------------------------------------------------------------
We consider Lagrangian fluid particles, for which we define the position $\xvec^+=\xvec^+(t,\yvec)$, the determinant of the Jacobian $J^+ = J^+(t,\yvec)$, the density $\rho^+ = \rho^+(t,\yvec)$, the velocity $\uvec^+ = \uvec^+(t,\yvec)$, and the internal energy $e^+ = e^+(t,\yvec)$. The Eulerian counterparts for the density, velocity, and internal energy are, respectively, $\rho = \rho(t,\xvec)$, $\uvec = \uvec(t,\xvec)$, and $e = e(t,\xvec)$. Also consider the volume $\Omega_0$ as the set of all $\yvec$ vectors that make up the initial domain. The control volume $\Omega^+ = \Omega^+(t, \Omega_0)$ is then defined by
\begin{equation}
    \Omega^+ = \{ \xvec^+:\yvec \in \Omega_0 \}.
\end{equation}
Note that $\Omega^+(0,\Omega_0) = \Omega_0$.

The governing equations for the Lagrangian fluid particles are derived in my hydrodynamics notes (see section on kinematics, Lagrangian governing equations, etc.). These are shown below
\begin{align}
    \frac{\partial \xvec^+}{\partial t} &= \uvec^+, \label{eq:evol_x} \\
    \frac{\partial J^+}{\partial t} &= J^+ \left ( \nabla \cdot \uvec \right )_{\xvec = \xvec^+}, \label{eq:evol_J} \\
    \frac{\partial J^+ \rho^+}{\partial t} &= 0, \label{eq:evol_rho} \\
    \rho^+ \frac{\partial \uvec^+}{\partial t} &= \left ( \nabla \cdot \sigmavec \right )_{\xvec = \xvec^+}, \label{eq:evol_u} \\
    \rho^+ \frac{\partial e^+}{\partial t} &= \left ( \sigmavec : \nabla \uvec \right )_{\xvec = \xvec^+}. \label{eq:evol_e}
\end{align}

A note on notation. The products that involve a tensor $\tauvec$ can be expressed in Einstein notation as
\begin{equation}
    \nabla \cdot \tauvec = \frac{\partial \tau_{ij}}{\partial x_j},
\end{equation}
\begin{equation}
    \tauvec \cdot \nabla \alpha = \tau_{ij} \frac{\partial \alpha}{\partial x_j},
\end{equation}
\begin{equation}
    \fvec \cdot \tauvec \cdot \nabla \alpha = f_i \tau_{ij} \frac{\partial \alpha}{\partial x_j},
\end{equation}
\begin{equation}
    \tauvec : \nabla \fvec = \tau_{ij} \frac{\partial f_i}{\partial x_j}.
\end{equation}
where $\alpha$ is a scalar and $\fvec$ a vector. In these notes we'll mostly be using indices $i$ and $j$ for FE expansions, rather than for Einstein notation.

%------------------------------------------------------------------------
\section{Finite element expansion}
%------------------------------------------------------------------------
We introduce the coefficients $\hat{\xvec}_i = \hat{\xvec}_i(t)$, $\hat{\uvec}_i = \hat{\uvec}_i(t)$ and $\hat{e}_i = \hat{e}_i(t)$, as well as the Lagrangian basis functions $\phi^+_i = \phi^+_i(\yvec) \in L^2$, and $w^+_i = w^+_i(\yvec) \in H^1$. We note that $\hat{\xvec}_i$ and $\hat{\uvec}_i$ are each vectors, e.g., the components of $\hat{\uvec}_i$ are $\hat{u}_{i,\alpha} = \hat{u}_{i,\alpha}(t)$ for $\alpha = x,y,z$. We also note that $\phi^+_i$ and $w^+_i$ have Eulerian counterparts $\phi_i = \phi_i(t,\xvec)$ and $w_i = w_i(t,\xvec)$, respectively (see more details in section on finite elements in my notes for numerical methods). The coefficients are used in the following expansions
\begin{equation}
    \xvec^+ = \sum_j^{N_w} \hat{\xvec}_j w^+_j,
\end{equation}
\begin{equation}
    \uvec^+ = \sum_j^{N_w} \hat{\uvec}_j w^+_j,
\end{equation}
\begin{equation}
    e^+ = \sum_j^{N_\phi} \hat{e}_j \phi^+_j.
\end{equation}
We note that the expansion coefficients are the same for the Lagrangian and Eulerian variables. For example, for the Eulerian velocity, we have
\begin{equation}
    \uvec = \sum_j^{N_w} \hat{\uvec}_j w_j.
\end{equation}

%------------------------------------------------------------------------
\section{Semi-discrete equations for $\xvec^+$ and $\Jvec^+$}
%------------------------------------------------------------------------

%------------------------------------------------------------------------
\section{Semi-discrete equation for $\rho^+$}
%------------------------------------------------------------------------
\Cref{eq:evol_rho} allows us to write
\begin{equation}
    \label{eq:evol_rho_semi_discrete}
    \rho^+ = \frac{\rho^+_0}{J^+},
\end{equation}
where $\rho^+_0 = \rho^+(0,\yvec)$.

%------------------------------------------------------------------------
\section{Semi-discrete equation for $\uvec^+$}
%------------------------------------------------------------------------
Plugging in \cref{eq:evol_rho_semi_discrete} in \cref{eq:evol_u} we get
\begin{equation}
    \rho^+_0 \frac{\partial \uvec^+}{\partial t} = \left ( \nabla \cdot \sigmavec \right )_{\xvec = \xvec^+} J^+.
\end{equation}
We then multiply both sides of the above by the basis functions for velocity and integrate over all space to obtain
\begin{equation}
    \int_{\Omega_0} \rho^+_0 \frac{\partial \uvec^+}{\partial t} w^+_i \, dV_y = \int_{\Omega_0} \left ( \nabla \cdot \sigmavec \right )_{\xvec = \xvec^+} w^+_i J^+ \, dV_y.
\end{equation}
For the left-hand side we have
\begin{align}
    \int_{\Omega_0} \rho^+_0 \frac{\partial \uvec^+}{\partial t} w^+_i \, dV_y &= \int_{\Omega_0} \rho^+_0 \sum_j^{N_w} \frac{d \hat{\uvec}_j}{dt} w^+_j w^+_i \, dV_y, \nonumber \\
    &= \sum_j^{N_w} \frac{d \hat{\uvec}_j}{dt} \int_{\Omega_0} \rho^+_0 w^+_i w^+_j \, dV_y, \nonumber \\
    &= \sum_j^{N_w} \frac{d \hat{\uvec}_j}{dt} m_{ij}^{(w)},
\end{align}
where
\begin{equation}
    m_{ij}^{(w)} = \int_{\Omega_0} \rho^+_0 w^+_i w^+_j \, dV_y
\end{equation}
is a mass bilinear form (which is independent of time). For the right-hand side we have
\begin{align}
    \int_{\Omega_0} \left ( \nabla \cdot \sigmavec \right )_{\xvec = \xvec^+} w^+_i J^+ \, dV_y & = \int_{\Omega_0} \left ( \nabla \cdot \sigmavec w_i \right )_{\xvec = \xvec^+} J^+ \, dV_y \nonumber \\
    & = \int_{\Omega^+} \nabla \cdot \sigmavec w_i \, dV_x \nonumber \\
    & = -\int_{\Omega^+} \sigmavec \cdot \nabla w_i \, dV_x.
\end{align}
The second equality above follows from integration by substitution. Combining results we have
\begin{equation}
    \label{eq:evol_u_semi_discrete}
    \sum_j^{N_w} \frac{d \hat{\uvec}_j}{d t} m^{(w)}_{ij} = -\int_{\Omega^+} \sigmavec \cdot \nabla w_i \, dV_x.
\end{equation}

We now introduce the vector $\Uvec$ whose components are $\hat{\uvec}_i$. We also introduce the matrix $\Mvec^{(w)}$ whose components are $m_{ij}^{(w)}$. Thus, the left-hand side of \cref{eq:evol_u_semi_discrete} can be written as $\Mvec^{(w)} \, d\Uvec / dt$.
We also introduce the vector bilinear form
\begin{equation}
    \fvec_{ij} = \int_{\Omega^+} \sigmavec \cdot \nabla w_i \phi_j dV_x.
\end{equation}
This is a \textit{vector} bilinear form since $\fvec_{ij}$ has components $f_{ij,\alpha} = f_{ij,\alpha}(t)$, for $\alpha = x,y,z$, where $\alpha$ denotes the first index of $\sigmavec$. We introduce the matrix $\Fvec$, whose components are $\fvec_{ij}$. We also expand the field with constant value of one as follows
\begin{equation}
    1 = \sum_i^{N_\phi} \hat{c}_i \phi_i.
\end{equation}
If we define the vector $\Cvec$ as that with components $\hat{c}_i$, we can show that 
\begin{align}
    \Fvec \Cvec &= \sum_j^{N_\phi} \fvec_{ij} \hat{c}_j \nonumber \\
    &= \sum_j^{N_\phi} \int_{\Omega^+} \sigmavec \cdot \nabla w_i \phi_j \, dV_x \hat{c}_j \nonumber \\
    &= \int_{\Omega^+} \sigmavec \cdot \nabla w_i \left ( \sum_j^{N_\phi} \hat{c}_j \phi_j \right ) \, dV_x \nonumber \\
    &= \int_{\Omega^+} \sigmavec \cdot \nabla w_i \, dV_x.
\end{align}
The above is the negative of the right-hand side of \cref{eq:evol_u_semi_discrete}. Thus, combining all together we get
\begin{equation}
    \Mvec^{(w)} \frac{d\Uvec}{dt} = -\Fvec \Cvec.
\end{equation}
We note that since both the Lagrangian and Eulerian velocities share the same coefficients $\Uvec$, we now have a solution for both.

%------------------------------------------------------------------------
\section{Semi-discrete equation for $e^+$}
%------------------------------------------------------------------------
Plugging in \cref{eq:evol_rho_semi_discrete} in \cref{eq:evol_e} we get
\begin{equation}
    \rho^+_0 \frac{\partial e^+}{\partial t} = \left ( \sigmavec : \nabla \uvec \right )_{\xvec = \xvec^+} J^+.
\end{equation}
We then multiply both sides of the above by the basis functions for energy and integrate over all space to obtain
\begin{equation}
    \int_{\Omega_0} \rho^+_0 \frac{\partial e^+}{\partial t} \phi_i^+ \, dV_y = \int_{\Omega_0} \left ( \sigmavec : \nabla \uvec \right )_{\xvec = \xvec^+} \phi_i^+ J^+ \, dV_y.
\end{equation}
For the left-hand side we have
\begin{align}
    \int_{\Omega_0} \rho^+_0 \frac{\partial e^+}{\partial t} \phi_i^+ \, dV_y &= \int_{\Omega_0} \rho^+_0 \sum_j^{N_\phi} \frac{d \hat{e}_j}{dt} \phi_j^+ \phi_i^+ \, dV_y , \nonumber \\
    &= \sum_j^{N_\phi} \frac{d \hat{e}_j}{dt} \int_{\Omega_0} \rho^+_0 \phi_j^+ \phi_i^+ \, dV_y , \nonumber \\
    &= \sum_j^{N_\phi} \frac{d \hat{e}_j}{dt} m_{ij}^{(\phi)}
\end{align}
where
\begin{equation}
    m_{ij}^{(\phi)} = \int_{\Omega_0} \rho^+_0 \phi_j^+ \phi_i^+ \, dV_y
\end{equation}
is a mass bilinear form (which is independent of time). For the right-hand side we have
\begin{align}
    \int_{\Omega_0} \left ( \sigmavec : \nabla \uvec \right )_{\xvec = \xvec^+} \phi_i^+ J^+ \, dV_y &= \int_{\Omega_0} \left ( \sigmavec : \nabla \uvec \phi_i \right )_{\xvec = \xvec^+} J^+ \, dV_y \nonumber \\
    &= \int_{\Omega^+} \sigmavec : \nabla \uvec \phi_i \, dV_x. 
\end{align}
Combining results we have
\begin{equation}
    \sum_j^{N_\phi} \frac{d \hat{e}_j}{dt} m_{ij}^{(\phi)} = \int_{\Omega^+} \sigmavec : \nabla \uvec \phi_i \, dV_x.
\end{equation}
We no show that
\begin{equation}
    \sigmavec : \nabla \uvec = \sigmavec : \nabla \left ( \sum_k^{N_w} \hat{\uvec}_k w_k \right ) = \sum_k^{N_w} \hat{\uvec}_k \cdot \sigmavec \cdot \nabla w_k,
\end{equation}
and hence the previous result is written as
\begin{equation}
    \sum_j^{N_\phi} \frac{d \hat{e}_j}{dt} m_{ij}^{(\phi)} = \sum_k^{N_w} \hat{\uvec}_k \cdot \int_{\Omega^+} \sigmavec \cdot \nabla w_k \phi_i \, dV_x.
\end{equation}
The above is finally re-written as
\begin{equation}
    \label{eq:evol_e_semi_discrete}
    \sum_j^{N_\phi} \frac{d \hat{e}_j}{d t} m^{(\phi)}_{ij} = \sum_k^{N_w} \hat{\uvec}_k \cdot \fvec_{ki}.
\end{equation}
Note that in the above there is a dot product in the right-hand side, that is, the right-hand side expanded out is  
\begin{equation}
    \sum_k^{N_w} \hat{\uvec}_k \cdot \fvec_{ki} = \sum_k^{N_w} \sum_{\alpha=x,y,z} \hat{u}_{k,\alpha} f_{ki,\alpha}.
\end{equation}

We now introduce the vector $\Evec$ whose components are $\hat{e}_i$. We also introduce the matrix $\Mvec^{(\phi)}$ whose components are $m_{ij}^{(\phi)}$. Thus, \cref{eq:evol_e_semi_discrete} can be succinctly written as
\begin{equation}
    \Mvec^{(\phi)} \frac{d\Evec}{dt} = \Fvec^T \cdot \Uvec.
\end{equation}
Note again that on the right-hand side above there is a matrix-vector product \textit{and} a dot product. We also note that since both the Lagrangian and Eulerian internal energies share the same coefficients $\Evec$, we now have a solution for both.


\end{document}