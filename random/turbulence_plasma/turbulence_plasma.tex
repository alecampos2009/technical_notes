\documentclass[11pt]{article}

\usepackage{"../../info/packages"}
\usepackage{"../../info/nomenclature"}
\usepackage{fullpage}


\title{Plasma turbulence}


\begin{document}

\maketitle

%--------------------------------------------
\section{Hasegawa-Mima}
%--------------------------------------------
References for this model can be found in \cite{hasegawa1977,horton1994}.

%--------------------------------------------
\subsection{Assumptions}
%--------------------------------------------
\begin{enumerate}
    \item Singly-charged ions. \label{it:hm_single_charge_ions}
    \item No shear stresses, collisions, or sources. \label{it:hm_no_shear_source_coll}
    \item Cold ion approximation, i.e. $T_e \gg T_i$ and thus $\nabla p_i \approx 0$, \cite{hasegawa1977}. \label{it:hm_cold}
    \item Isothermal electron fluid, i.e.\@ $T_e$ is constant. \label{it:hm_isothermal_electron}
    \item Electrostatic field, i.e. $\Evec = - \nabla \phi$. \label{it:hm_electrostatic}
    \item Magnetic field is constant.
    \item Neglect parallel ion velocity, i.e. $u_{i,||} \approx 0$, \cite{hasegawa1977}. \label{it:hm_par_ion}
    \item Quasi-neutrality, i.e. $n_i \approx n_e$. \label{it:hm_quasineutrality}
    \item Adiabatic electrons, i.e. $n_e = n_0 \exp (e\phi/T_e)$, where $n_0 = n_0(x_1)$. \label{it:hm_adiabatic}
\end{enumerate}

%--------------------------------------------
\subsection{Derivation}
%--------------------------------------------
Using the assumptions in \cref{it:hm_single_charge_ions,it:hm_no_shear_source_coll}, the momentum equation for singly-charged ions
\begin{equation}
    \frac{\partial m_i n_i \uvec_i}{\partial t} + \nabla \cdot \left ( m_i n_i \uvec_i \uvec_i \right ) - Z e n_i \left ( \Evec + \uvec_i \times \Bvec \right ) = -\nabla p_i + \nabla \cdot \tvec_i + \ \Rvec_i + \hat{\Mvec}_i
\end{equation}
simplifies to
\begin{equation}
    m_i n_i \left (\frac{\partial \uvec_i}{\partial t} + \uvec_i \cdot \nabla \uvec_i \right ) = e n_i (\Evec + \uvec_i \times \Bvec) - \nabla p_i.
\end{equation}
Using the assumptions in \cref{it:hm_cold,it:hm_electrostatic}, the above becomes
\begin{equation}
    \frac{\partial \uvec_i}{\partial t} + \uvec_i \cdot \nabla \uvec_i = -\frac{e}{m_i} \nabla \phi + \frac{e}{m_i}\uvec_i \times \Bvec.
\end{equation}
Introduce the coordinate system $\evec_1$, $\evec_2$, $\evec_3$, and assume $\Bvec$ points in the $\evec_3$ direction. Defining the perpendicular velocity as $\uvec_{i,\perp} = [u_{i,1}, u_{i,2}, 0]^T$ and the perpendicular gradient as $\nabla_\perp = [\partial_1, \partial_2, 0]^T$, we have
\begin{equation}
    \frac{\partial \uvec_{i,\perp}}{\partial t} + \uvec_i \cdot \nabla \uvec_{i,\perp} = -\frac{e}{m_i} \nabla_\perp \phi + \frac{e}{m_i}\uvec_i \times \Bvec.
\end{equation}
Using the assumption in \cref{it:hm_par_ion} and noting that $\uvec_{i} \times \Bvec = \uvec_{i,\perp} \times \Bvec$, we obtain
\begin{equation}
\label{eq:hm_momentum_perpendicular}
    \frac{\partial \uvec_{i,\perp}}{\partial t} + \uvec_{i,\perp} \cdot \nabla_\perp \uvec_{i,\perp} = -\frac{e}{m_i} \nabla_\perp \phi + \frac{e}{m_i} \uvec_{i,\perp} \times \Bvec.
\end{equation}

We now introduce the scalings for a characteristic frequency $w$ and length scale $r$
\begin{equation}
    \frac{w}{w_{c,i}} \sim \epsilon \qquad \frac{r_s}{r} \sim \epsilon,
\end{equation}
where $w_{c,i} = eB/m_i$ is the cyclotron frequency, $r_s = v_s/w_{c,i}$ is a reference length scale, and $v_s = \sqrt{T_e/m_i}$ a reference velocity scale. Given these variables, we assume
\begin{equation}
    \frac{\partial \uvec_{i,\perp}}{\partial t} \sim \uvec_{i,\perp} w \qquad \nabla_\perp \uvec_{i,\perp} \sim \frac{\uvec_{i,\perp}}{r} \qquad \Evec \sim \uvec_{i,\perp} B.
\end{equation}
Finally, we introduce the decomposition $\uvec_{i,\perp} = \uvec_{i,\perp}^{(0)} + \uvec_{i,\perp}^{(1)}$, where $\uvec_{i,\perp}^{(0)} \sim v_s$ and $\uvec_{i,\perp}^{(1)} \sim \epsilon v_s$. We use this decomposition in \cref{eq:hm_momentum_perpendicular} and then divide the PDE by $w_{c,i} v_s$. The order of each element in the resulting equation is as follows
\begin{enumerate}
    \item $\frac{\partial \uvec_{i,\perp}^{(0)}}{\partial t} \sim \epsilon$.
    \item $\frac{\partial \uvec_{i,\perp}^{(1)}}{\partial t} \sim \epsilon^2$.
    \item $\uvec_{i,\perp}^{(0)} \cdot \nabla_\perp \uvec_{i,\perp}^{(0)} \sim \epsilon$.
    \item $\uvec_{i,\perp}^{(0)} \cdot \nabla_\perp \uvec_{i,\perp}^{(1)} \sim \epsilon^2$.
    \item $\uvec_{i,\perp}^{(1)} \cdot \nabla_\perp \uvec_{i,\perp}^{(0)} \sim \epsilon^2$.
    \item $\uvec_{i,\perp}^{(1)} \cdot \nabla_\perp \uvec_{i,\perp}^{(1)} \sim \epsilon^3$.
    \item $-\frac{e}{m_i} \nabla_\perp \phi \sim 1$.
    \item $\frac{e}{m_i} \uvec_{i,\perp}^{(0)} \times \Bvec \sim 1$.
    \item $\frac{e}{m_i} \uvec_{i,\perp}^{(1)} \times \Bvec \sim \epsilon$.
\end{enumerate}
Combining the first order terms we obtain
\begin{equation}
    0 = - \nabla_\perp \phi + \uvec_{i,\perp}^{(0)} \times \Bvec,
\end{equation}
which, upon crossing by $\Bvec$, gives
\begin{equation}
\label{eq:hm_e_cross_b_drift}
    \uvec_{i,\perp}^{(0)} = -\nabla_\perp \phi \times \frac{\bvec}{B}.
\end{equation}
Combining the terms of order $\epsilon$ we obtain
\begin{equation}
    \frac{\partial \uvec_{i,\perp}^{(0)}}{\partial t} + \uvec_{i,\perp}^{(0)} \cdot \nabla_\perp \uvec_{i,\perp}^{(0)} = \frac{e}{m_i} \uvec_{i,\perp}^{(1)} \times \Bvec,
\end{equation}
which, upon crossing by $\Bvec$, gives
\begin{equation}
\label{eq:hm_polarization_drift}
    \uvec_{i,\perp}^{(1)} = -\frac{1}{w_{c,i} B} \left [ \frac{\partial \nabla_\perp \phi}{\partial t} + \uvec_{i,\perp}^{(0)} \cdot \nabla_\perp \left ( \nabla_\perp \phi \right ) \right ].
\end{equation}
The above is the polarization drift. The velocity given by \cref{eq:hm_e_cross_b_drift} is referred to as the $E \times B$ drift, and the velocity given by \cref{eq:hm_polarization_drift} as the polarization drift.

Using the assumption in \cref{it:hm_no_shear_source_coll}, the continuity equation for ions is
\begin{equation}
    \frac{\partial n_i}{\partial t} + \nabla \cdot (n_i \uvec_i) = 0.
\end{equation}
Using the assumption in \cref{it:hm_par_ion} the above becomes
\begin{equation}
    \frac{\partial n_i}{\partial t} + \nabla_\perp \cdot (n_i \uvec_{i,\perp}) = 0,
\end{equation}
or
\begin{equation}
    \frac{\partial n_i}{\partial t} +  \uvec_{i,\perp} \cdot \nabla_\perp n_i + n_i \nabla_\perp \cdot \uvec_{i,\perp} = 0,
\end{equation}
One of the main components of the derivation of the Hasegawa-Mima equation is the assumption that advection is governed by the lowest-order velocity only; that is, by $\uvec^{(0)}_{i,\perp}$ and not $\uvec^{(1)}_{i,\perp}$. Thus, the above is written as
\begin{equation}
    \frac{\partial n_i}{\partial t} +  \uvec^{(0)}_{i,\perp} \cdot \nabla_\perp n_i + n_i \nabla_\perp \cdot \uvec_{i,\perp} = 0.
\end{equation}
We note that $\uvec^{(0)}_{i,\perp}$ is divergence free, and thus we have
\begin{equation}
    \frac{\partial n_i}{\partial t} +  \uvec^{(0)}_{i,\perp} \cdot \nabla_\perp n_i + n_i \nabla_\perp \cdot \uvec^{(1)}_{i,\perp} = 0.
\end{equation}
We divide by $n_i$ to express the density in terms of its logarithm 
\begin{equation}
    \label{eq:intermediate_hm_1}
    \frac{\partial \ln n_i}{\partial t} + \uvec^{(0)}_{i,\perp} \cdot \nabla_\perp \ln n_i + \nabla_\perp \cdot \uvec^{(1)}_{i,\perp} = 0.
\end{equation}

We now use the assumptions in \cref{it:hm_adiabatic,it:hm_quasineutrality} to obtain
\begin{equation}
    \ln n_i = \ln \left [ n_0 \exp \left ( \frac{e \phi}{T_e} \right ) \right ] = \ln n_0 + \frac{e \phi}{T_e}.
\end{equation}
Taking into account the fact that $n_0$ is time independent, the continuity equation becomes
\begin{equation}
    \frac{\partial}{\partial t} \left ( \frac{e\phi}{T_e} \right ) + \uvec^{(0)}_{i,\perp} \cdot \nabla_\perp \left [ \ln n_0 + \frac{e \phi}{T_e} \right ] +  \nabla_\perp \cdot \uvec^{(1)}_{i,\perp} = 0.
\end{equation}
Since $\uvec^{(0)}_{i,\perp}$ and $\nabla_\perp \phi$ are orthogonal, the above simplifies to
\begin{equation}
    \frac{\partial}{\partial t} \left ( \frac{e\phi}{T_e} \right ) + \uvec^{(0)}_{i,\perp} \cdot \nabla_\perp \ln n_0 +  \nabla_\perp \cdot \uvec^{(1)}_{i,\perp} = 0,
\end{equation}
which we re-write as
\begin{equation}
    \label{eq:intermediate_hm_2}
    \frac{\partial}{\partial t} \left ( \frac{e\phi}{T_e} \right ) + \uvec^{(0)}_{i,\perp} \cdot \nabla_\perp \ln \left ( \frac{n_0}{w_{c,i}} \right ) +  \nabla_\perp \cdot \uvec^{(1)}_{i,\perp} = 0.
\end{equation}

Given the definition of the polarization drift, we have
\begin{equation}
    \nabla_\perp \cdot \uvec_{i,\perp}^{(1)} = -\frac{1}{w_{c,i} B} \left \{ \frac{\partial \nabla_\perp^2 \phi}{\partial t} + \nabla_\perp \cdot \left [ \uvec_{i,\perp}^{(0)} \cdot \nabla_\perp \left ( \nabla_\perp \phi \right ) \right ] \right \}.
\end{equation}
The second term above is best computed using tensor notation, and we'll use $u_j$ to denote the components of $\uvec_{i,\perp}^{(0)}$. Thus,
\begin{equation}
    \frac{\partial}{\partial x_i} \left [ \left ( u_j \frac{\partial}{\partial x_j} \right) \frac{\partial \phi}{\partial x_i} \right ] = \frac{\partial u_j}{\partial x_i} \frac{\partial^2 \phi}{\partial x_j \partial x_i} + u_j \frac{\partial}{\partial x_j} \left (\frac{\partial^2 \phi}{\partial x_i \partial x_i} \right ).
\end{equation}
Using the definition of $\uvec_{i,\perp}^{(0)}$, the first term on the right-hand side above can be expressed as
\begin{align}
    \frac{\partial u_j}{\partial x_i} \frac{\partial^2 \phi}{\partial x_j \partial x_i} &= -\frac{1}{B^2} \epsilon_{jpq} \frac{\partial^2 \phi}{\partial x_p \partial x_i} B_q \frac{\partial^2 \phi}{\partial x_j \partial x_i} \nonumber \\
    &= -\frac{1}{B^2} \epsilon_{qjp} \frac{\partial^2 \phi}{\partial x_j \partial x_i} \frac{\partial^2 \phi}{\partial x_p \partial x_i} B_q  \nonumber \\
    &= -\frac{1}{B^2} \epsilon_{qjp} \left ( \frac{\partial^2 \phi}{\partial x_j \partial x_1} \frac{\partial^2 \phi}{\partial x_p \partial x_1} + \frac{\partial^2 \phi}{\partial x_j \partial x_2} \frac{\partial^2 \phi}{\partial x_p \partial x_2}  \right ) B_q.
\end{align}
Since $\epsilon_{qjp} \partial_j a \partial_p a \to \nabla a \times \nabla a = 0$ for any scalar $a$, the term above is identically zero. Thus, we have
\begin{equation}
    \label{eq:hm_divergence_polarization}
    \nabla_\perp \cdot \uvec_{i,\perp}^{(1)} = -\frac{1}{w_{c,i} B} \left [ \frac{\partial \nabla_\perp^2 \phi}{\partial t} + \uvec_{i,\perp}^{(0)} \cdot \nabla_\perp \left ( \nabla_\perp^2 \phi \right ) \right ],
\end{equation}
and \cref{eq:intermediate_hm_2} becomes
\begin{equation}
   \frac{\partial}{\partial t}\left ( \frac{1}{w_{c,i} B}\nabla_\perp^2 \phi - \frac{e \phi}{T_e} \right ) + \uvec_{i,\perp}^{(0)} \cdot \nabla_\perp \left [ \frac{1}{w_{c,i} B} \nabla_\perp^2 \phi - \ln \left( \frac{n_0}{w_{c,i}} \right) \right] = 0.
\end{equation}
Plugging in for $\uvec_{i,\perp}^{(0)}$,
\begin{equation}
\label{eq:intermediate_hm_3}
   \frac{\partial}{\partial t}\left ( \frac{1}{w_{c,i} B}\nabla_\perp^2 \phi - \frac{e \phi}{T_e} \right ) - \left( \nabla_\perp \phi \times \frac{\bvec}{B} \right) \cdot \nabla_\perp \left [ \frac{1}{w_{c,i} B} \nabla_\perp^2 \phi - \ln \left ( \frac{n_0}{w_{c,i}} \right) \right ] = 0.
\end{equation}
We now introduce the normalizations
\begin{equation}
    \phi(t,\xvec) = \frac{T_e}{e} \hat{\phi}(\hat{t},\hat{\xvec}) \qquad n_0(x_1) = \hat{n}_0(\hat{x}_1),
\end{equation}
where $\hat{t} = t w_{c,i}$ and $\hat{\xvec} = \xvec / r_s$. Neglecting the hat notation for the sake of simplicity, \cref{eq:intermediate_hm_3} finally becomes
\begin{equation}
    \label{eq:hasegawa_mima}
   \frac{\partial}{\partial t}\left ( \nabla_\perp^2 \phi - \phi \right ) - \left( \nabla_\perp \phi \times \bvec \right) \cdot \nabla_\perp \left [ \nabla_\perp^2 \phi - \ln \left( \frac{n_0}{w_{c,i}} \right) \right ] = 0.
\end{equation}

Using the following expansion
\begin{equation}
    \left ( \nabla_\perp \phi \times \bvec \right ) \cdot \nabla_\perp = \frac{\partial \phi}{\partial x_2} \frac{\partial}{\partial x_1} - \frac{\partial \phi}{\partial x_1} \frac{\partial}{\partial x_2},
\end{equation}
The Hasegawa-Mima equation can be written as
\begin{equation}
   \frac{\partial}{\partial t}\left ( \nabla_\perp^2 \phi - \phi \right ) - \frac{\partial \phi}{\partial x_2} \frac{\partial \nabla_\perp^2 \phi}{\partial x_1} + \frac{\partial \phi}{\partial x_1} \frac{\partial \nabla_\perp^2 \phi}{\partial x_2} + \beta \frac{\partial \phi}{\partial x_2} = 0,
\end{equation}
where 
\begin{equation}
    \beta = \frac{\partial}{\partial x_1} \ln \left ( \frac{n_0}{w_{c,i}} \right ).
\end{equation}

%--------------------------------------------
\subsection{Spectral space}
%--------------------------------------------
In this section we derive the equation for the Fourier coefficient $\hat{\phi}_\nvec = \hat{\phi}(t)_\nvec$, which relates to the potential through the following
\begin{equation}
    \phi(t,x) = \sum_{\nvec=-\infty}^\infty \hat{\phi}_\nvec(t) e^{i \kvec_\nvec \cdot \xvec},
\end{equation}
\begin{equation}
    \hat{\phi}_\nvec(t)= \frac{1}{L^2} \int_{L^2} \phi(t, \xvec) e^{-i \kvec_\nvec \cdot \xvec} \, d\xvec .
\end{equation}
We introduce the operator $\mathcal{F} \{ \}_\nvec$, which is defined by
\begin{equation}
    \mathcal{F} \{ \phi(t,\xvec) \}_\nvec = \frac{1}{L^2} \int_{L^2} \phi(t, \xvec) e^{-i \kvec_\nvec \cdot \xvec} \, d\xvec .
\end{equation}
The equation for $\hat{\phi}_\nvec$ is obtained by applying this operator to \cref{eq:hasegawa_mima}. Thus, the time derivative term in that equation becomes 
\begin{equation}
    \mathcal{F} \left \{ \frac{\partial}{\partial t} \left ( \nabla_\perp^2 - \phi \right ) \right \}_\nvec = \frac{\partial}{\partial t} \mathcal{F} \left \{ \nabla_\perp^2 \phi - \phi \right \}_\nvec = \frac{\partial}{\partial t} \left ( -k_\nvec^2 \hat{\phi}_\nvec - \hat{\phi}_\nvec \right ) = - \left ( 1 + k^2_\nvec \right ) \frac{\partial \hat{\phi}_\nvec}{\partial t}.
\end{equation}
We assume $\nabla \ln (n_o / w_{ci})$ is constant in space. Thus, the term containing the inhomogeneity becomes 
\begin{multline}
    \mathcal{F} \left \{ \left ( \nabla_\perp \phi \times \bvec \right ) \cdot \nabla_\perp \ln \left ( \frac{n_o}{w_{ci}} \right ) \right \}_\nvec \\
    = \left ( \mathcal{F} \left \{ \nabla_\perp \phi \right \}_\nvec \times \bvec \right ) \cdot \nabla_\perp \ln \left ( \frac{n_o}{w_{ci}} \right ) = i \left ( \kvec_\nvec \times \bvec \right ) \cdot \nabla_\perp \ln \left ( \frac{n_o}{w_{ci}} \right ) \hat{\phi}_\nvec.
\end{multline}
The remaining term is computed as follows
\begin{align}
    \mathcal{F} \left \{ - \left (\nabla_\perp \phi \times \bvec \right ) \right . & \left . \cdot \nabla^3_\perp \phi \right \}_\nvec \nonumber \\
    &= \mathcal{F} \left \{ -\sum_{\nvec'=-\infty}^\infty \sum_{\nvec''=-\infty}^\infty \left [ \hat{\phi}_{\nvec'} i \left ( \kvec_{\nvec'} \times \bvec \right ) e^{i \kvec_{\nvec'} \cdot \xvec} \right ] \cdot \left [ \hat{\phi}_{\nvec''} \left ( -i k^2_{\nvec''} \kvec_{\nvec''} \right ) e^{i \kvec_{\nvec''} \cdot \xvec} \right ] \right \}_\nvec \nonumber \\
    &= -\sum_{\nvec'=-\infty}^\infty \sum_{\nvec''=-\infty}^\infty \left ( \kvec_{\nvec'} \times \bvec \right ) \cdot \kvec_{\nvec''} k^2_{\nvec''} \hat{\phi}_{\nvec'} \hat{\phi}_{\nvec''} \mathcal{F} \left \{ e^{i \kvec_{\nvec'} \cdot \xvec} e^{i \kvec_{\nvec''} \cdot \xvec} \right \} \nonumber \\
    &= \sum_{\nvec'=-\infty}^\infty \sum_{\nvec''=-\infty}^\infty \left ( \kvec_{\nvec'} \times \kvec_{\nvec''} \right ) \cdot \bvec k^2_{\nvec''} \hat{\phi}_{\nvec'} \hat{\phi}_{\nvec''} \delta_{\nvec, \nvec' + \nvec''}
\end{align}
Since $\nvec'$ and $\nvec''$ are just symbolic variables for the summation, we can write the above as follows
\begin{align}
    \mathcal{F} \left \{ - \left (\nabla_\perp \phi \times \bvec \right ) \right . & \left . \cdot \nabla^3_\perp \phi \right \}_\nvec \nonumber \\
    =& \frac{1}{2} \sum_{\nvec'=-\infty}^\infty \sum_{\nvec''=-\infty}^\infty \left ( \kvec_{\nvec'} \times \kvec_{\nvec''} \right ) \cdot \bvec k^2_{\nvec''} \hat{\phi}_{\nvec'} \hat{\phi}_{\nvec''} \delta_{\nvec, \nvec' + \nvec''} \nonumber \\
    &+ \frac{1}{2} \sum_{\nvec''=-\infty}^\infty \sum_{\nvec'=-\infty}^\infty \left ( \kvec_{\nvec''} \times \kvec_{\nvec'} \right ) \cdot \bvec k^2_{\nvec'} \hat{\phi}_{\nvec''} \hat{\phi}_{\nvec'} \delta_{\nvec, \nvec'' + \nvec'} \nonumber \\
    =& \sum_{\nvec'=-\infty}^\infty \sum_{\nvec''=-\infty}^\infty \frac{1}{2} \left ( \kvec_{\nvec'} \times \kvec_{\nvec''} \right ) \cdot \bvec \left ( k^2_{\nvec''} - k^2_{\nvec'} \right ) \hat{\phi}_{\nvec'} \hat{\phi}_{\nvec''} \delta_{\nvec, \nvec' + \nvec''}
\end{align}
Thus, we finally have
\begin{equation}
    \mathcal{F} \left \{ - \left (\nabla_\perp \phi \times \bvec \right ) \cdot \nabla^3_\perp \phi \right \}_\nvec = \sum_{\nvec = \nvec' + \nvec''} \left ( \kvec_{\nvec'} \times \kvec_{\nvec''} \right ) \cdot \bvec \left ( k^2_{\nvec''} - k^2_{\nvec'} \right ) \hat{\phi}_{\nvec'} \hat{\phi}_{\nvec''}
\end{equation}
Combining the results above, we obtain
\begin{equation}
    \frac{\partial \hat{\phi}_\nvec}{\partial t} + i w_\nvec \hat{\phi}_\nvec = \sum_{\nvec = \nvec' + \nvec''} \Lambda^\nvec_{\nvec', \nvec''} \hat{\phi}_{\nvec'} \hat{\phi}_{\nvec''},
\end{equation}
where
\begin{equation}
    w_\nvec = - \frac{ \left (\kvec_\nvec \times \bvec \right ) }{1 + k^2_\nvec} \cdot \nabla_\perp \ln \left ( \frac{n_o}{w_{ci}} \right ),
\end{equation}
and
\begin{equation}
    \Lambda^\nvec_{\nvec', \nvec''} = \frac{1}{2} \frac{  \left ( \kvec_{\nvec'} \times \kvec_{\nvec''} \right ) \cdot \bvec \left ( k^2_{\nvec''} - k^2_{\nvec'} \right ) }{1 + k^2_\nvec}.
\end{equation}
Note that $w_\nvec$ can also be written as
\begin{equation}
    w_\nvec = -\frac{k_{2,\nvec} \evec_1 - k_{1,\nvec} \evec_2}{1 + k^2_\nvec} \cdot \beta \evec_1 = -\frac{k_{2,\nvec} \beta}{1 + k^2_\nvec}.
\end{equation}

%--------------------------------------------
\section{Hasegawa-Wakatani}
%--------------------------------------------
References for this model can be found in \cite{wakatani1984,hasegawa1987}.

%--------------------------------------------
\subsection{Assumptions}
%--------------------------------------------
\begin{enumerate}
    \item Singly-charged ions. \label{it:hw_single_charge_ions}
    \item No shear stresses in the electron momentum equation, no collisions in the ion momentum equation, no sources. \label{it:hw_no_shear_source_coll}
    \item Cold ion approximation, i.e. $T_e \gg T_i$ and thus $\nabla p_i \approx 0$. \label{it:hw_cold}
    \item Isothermal electron fluid, i.e.\@ $T_e$ is constant. \label{it:hw_isothermal_electron}
    \item Electrostatic field, i.e. $\Evec = -\nabla \phi$. \label{it:hw_electrostatic}
    \item Magnetic field is constant.
    \item Neglect parallel ion velocity, i.e. $u_{i,||} \approx 0$. \label{it:hw_par_ion}
    \item Quasi-neutrality, i.e. $n_i \approx n_e$. \label{it:hw_quasineutrality}
    \item $n_i = n_0 + n'$, where $n_0 = n_0(x_1)$ and $n'$ is smaller than $n_0$. \label{it:hw_ion_density}
    \item Perpendicular components of the ion shear-stress term are modeled as $(\nabla \cdot \tvec_i)_\perp/(m_i n_i) = \nu \nabla_\perp^2 \uvec_{i,\perp}$. \label{it:hw_shear_stress}
    \item Assume infinitesimally small electron mass, i.e.\@ $m_e \to 0$. \label{it:hw_small_electron_mass}
\end{enumerate}

%--------------------------------------------
\subsection{Derivation}
%--------------------------------------------
Using the assumptions in \cref{it:hw_single_charge_ions,it:hw_no_shear_source_coll}, the momentum equation for singly-charged ions
\begin{equation}
    \frac{\partial m_i n_i \uvec_i}{\partial t} + \nabla \cdot \left ( m_i n_i \uvec_i \uvec_i \right ) - Z e n_i \left ( \Evec + \uvec_i \times \Bvec \right ) = -\nabla p_i + \nabla \cdot \tvec_i + \ \Rvec_i + \hat{\Mvec}_i
\end{equation}
simplifies to
\begin{equation}
    m_i n_i \left (\frac{\partial \uvec_i}{\partial t} + \uvec_i \cdot \nabla \uvec_i \right ) = e n_i (\Evec + \uvec_i \times \Bvec) - \nabla p_i + \nabla \cdot \tvec.
\end{equation}
Using the assumptions in \cref{it:hw_cold,it:hw_electrostatic}, the above becomes
\begin{equation}
    \frac{\partial \uvec_i}{\partial t} + \uvec_i \cdot \nabla \uvec_i = -\frac{e}{m_i} \nabla \phi + \frac{e}{m_i}\uvec_i \times \Bvec + \frac{\nabla \cdot \tvec_i}{m_i n_i}.
\end{equation}
As before, introduce the coordinate system $\evec_1$, $\evec_2$, $\evec_3$, and assume $\Bvec$ points in the $\evec_3$ direction. Defining the perpendicular velocity as $\uvec_{i,\perp} = [u_{i,1}, u_{i,2}, 0]^T$, the perpendicular gradient as $\nabla_\perp = [\partial_1, \partial_2, 0]^T$, and the perpendicular shear stress as $(\nabla \cdot \tvec_i)_\perp = [(\nabla \cdot \tvec)_1, (\nabla \cdot \tvec)_2, 0]^T$, we have
\begin{equation}
    \frac{\partial \uvec_{i,\perp}}{\partial t} + \uvec_i \cdot \nabla \uvec_{i,\perp} = -\frac{e}{m_i} \nabla_\perp \phi + \frac{e}{m_i}\uvec_i \times \Bvec + \frac{(\nabla \cdot \tvec_i)_\perp}{m_i n_i}.
\end{equation}
Using the assumption in \cref{it:hw_par_ion} and noting that $\uvec_{i} \times \Bvec = \uvec_{i,\perp} \times \Bvec$, we obtain
\begin{equation}
    \frac{\partial \uvec_{i,\perp}}{\partial t} + \uvec_{i,\perp} \cdot \nabla_\perp \uvec_{i,\perp} = -\frac{e}{m_i} \nabla_\perp \phi + \frac{e}{m_i} \uvec_{i,\perp} \times \Bvec + \frac{(\nabla \cdot \tvec_i)_\perp}{m_i n_i}.
\end{equation}
Finally, using the assumption in \cref{it:hw_shear_stress}, we obtain
\begin{equation}
    \label{eq:hw_momentum_perpendicular}
        \frac{\partial \uvec_{i,\perp}}{\partial t} + \uvec_{i,\perp} \cdot \nabla_\perp \uvec_{i,\perp} = -\frac{e}{m_i} \nabla_\perp \phi + \frac{e}{m_i} \uvec_{i,\perp} \times \Bvec + \nu \nabla_\perp^2 \uvec_{i,\perp}.
\end{equation}
The same scaling analysis performed for the derivation of the Hasegawa Mima equation is now applied. The only new term in \cref{eq:hw_momentum_perpendicular} is the viscous term. We note that the kinematic viscosity $\nu$ scales as
\begin{equation}
    \nu \sim r^2 w.
\end{equation}
Thus, the viscous stress term leads to the following scalings
\begin{enumerate}
    \item $\nu \nabla^2_\perp \uvec^{(0)}_{i,\perp} \sim \epsilon$
    \item $\nu \nabla^2_\perp \uvec^{(1)}_{i,\perp} \sim \epsilon^2$
\end{enumerate}
As before, the first order terms lead to the $E \times B$ drift
\begin{equation}
    \label{eq:hw_e_cross_b_drift}
    \uvec^{(0)}_{i,\perp} = -\nabla_\perp \phi \times \frac{\bvec}{B}.
\end{equation}
However, the equation for terms of order $\epsilon$ now contains the viscous term as shown below
\begin{equation}
    \frac{\partial \uvec_{i,\perp}^{(0)}}{\partial t} + \uvec_{i,\perp}^{(0)} \cdot \nabla_\perp \uvec_{i,\perp}^{(0)} = \frac{e}{m_i} \uvec_{i,\perp}^{(1)} \times \Bvec + \nu \nabla_\perp^2 \uvec^{(0)}_{i,\perp}.
\end{equation}
Upon crossing by $\Bvec$, the above gives
\begin{equation}
\label{eq:hw_polarization_drift}
    \uvec_{i,\perp}^{(1)} = -\frac{1}{w_{c,i} B} \left [ \frac{\partial \nabla_\perp \phi}{\partial t} + \uvec_{i,\perp}^{(0)} \cdot \nabla_\perp \left ( \nabla_\perp \phi \right ) \right ] + \frac{\nu}{w_{c,i}B} \nabla^2_\perp (\nabla_\perp \phi).
\end{equation}
That is, an additional viscous term appears in the polarization drift.

As shown in the derivation of the Hasegawa-Mima equation, the continuity equation for ions can be expressed in the form of \cref{eq:intermediate_hm_1}, which is repeated below for convenience
\begin{equation}
    \frac{\partial \ln n_i}{\partial t} + \uvec^{(0)}_{i,\perp} \cdot \nabla_\perp \ln n_i + \nabla_\perp \cdot \uvec^{(1)}_{i,\perp} = 0.
\end{equation}
Given the assumption in \cref{it:hw_ion_density}, the natural logarithm of density is re-written as follows
\begin{equation}
    \ln n_i = \ln \left ( n_0 + n' \right ) = \ln \left [ n_0 \left ( 1 + \frac{n'}{n_0} \right ) \right ] = \ln n_0 + \ln \left (1 + \frac{n'}{n_0} \right ).
\end{equation}
We now introduce $n = n' / n_0$, which is small due to the assumption in \cref{it:hw_ion_density}. Thus, a Taylor series expansion would allow us to write
\begin{equation}
    \label{eq:hw_ion_density}
    \ln n_i = \ln n_0 + n.
\end{equation}
Since $n_0$ is time independent (assumption in \cref{it:hw_ion_density}), the continuity equation becomes
\begin{equation}
    \label{eq:hw_intermediate_ion_1}
    \frac{\partial n}{\partial t} + \uvec^{(0)}_{i,\perp} \cdot \nabla_\perp \left ( \ln n_0 + n \right ) + \nabla_\perp \cdot \uvec^{(1)}_{i,\perp} = 0.
\end{equation}
Using the same derivation for \cref{eq:hm_divergence_polarization}, we now have
\begin{equation}
    \nabla_\perp \cdot \uvec_{i,\perp}^{(1)} = -\frac{1}{w_{c,i} B} \left [ \frac{\partial \nabla_\perp^2 \phi}{\partial t} + \uvec_{i,\perp}^{(0)} \cdot \nabla_\perp \left ( \nabla_\perp^2 \phi \right ) \right ] + \frac{\nu}{w_{c,i} B} \nabla^4_\perp \phi.
\end{equation}
Plugging this in \cref{eq:hw_intermediate_ion_1}, we obtain
\begin{equation}
    \frac{\partial}{\partial t} \left ( \frac{1}{w_{c,i} B} \nabla^2_\perp \phi - n \right ) + \uvec^{(0)}_{i,\perp} \cdot \nabla_\perp \left ( \frac{1}{w_{c,i} B} \nabla^2_\perp \phi - n - \ln n_0 \right ) - \frac{\nu}{w_{c,i} B} \nabla^4_\perp \phi = 0.
\end{equation}
Plugging in for $\uvec^{(0)}_{i,\perp}$,
\begin{equation}
    \label{eq:hw_ion_continuity}
    \frac{\partial}{\partial t} \left ( \frac{1}{w_{c,i} B} \nabla^2_\perp \phi - n \right ) - \left (\nabla_\perp \phi \times \frac{\bvec}{B} \right ) \cdot \nabla_\perp \left ( \frac{1}{w_{c,i} B} \nabla^2_\perp \phi - n - \ln n_0 \right ) - \frac{\nu}{w_{c,i} B} \nabla^4_\perp \phi = 0.
\end{equation}

Using the assumptions in \cref{it:hw_single_charge_ions,it:hw_no_shear_source_coll}, the momentum equation for electrons 
\begin{equation}
    \frac{\partial m_e n_e \uvec_e}{\partial t} + \nabla \cdot \left ( m_e n_e \uvec_e \uvec_e \right ) + e n_e \left ( \Evec + \uvec_e \times \Bvec \right ) = -\nabla p_e + \nabla \cdot \tvec_e + \ \Rvec_e + \hat{\Mvec}_e
\end{equation}
simplifies to
\begin{equation}
    m_e n_e \left (\frac{\partial \uvec_e}{\partial t} + \uvec_e \cdot \nabla \uvec_e \right ) = -e n_e (\Evec + \uvec_e \times \Bvec) - \nabla p_e + \Rvec_e.
\end{equation}
Using the assumptions in \cref{it:hw_small_electron_mass,it:hw_isothermal_electron,it:hw_electrostatic}, the above simplifies to
\begin{equation}
    0 = -e n_e (-\nabla \phi + \uvec_e \times \Bvec) - T_e \nabla n_e + \Rvec_e.
\end{equation}
Dividing by $-e n_e$, we get
\begin{equation}
    \label{eq:hw_electron_momentum_no_inertia}
    0 = -\nabla \phi + \uvec_e \times \Bvec + \frac{T_e}{e} \nabla \ln n_e - \frac{1}{e n_e}\Rvec_e.
\end{equation}
We now focus on the component of the equation above that is parallel to $\Bvec$, that is 
\begin{equation}
    0 = -\nabla_{||} \phi + \frac{T_e}{e} \nabla_{||} \ln n_e - \frac{1}{e n_e} R_{e,||}.
\end{equation}
Using quasineutrality to replace $n_e$ by $n_i$, and plugging in \cref{eq:hw_ion_density} for $n_i$ gives 
\begin{equation}
    0 = -\nabla_{||} \phi + \frac{T_e}{e} \nabla_{||} (\ln n_0 + n) - \frac{1}{e n_e} R_{e,||}.
\end{equation}
We note that the gradient of $\ln n_0$ above is zero since $n_0$ does not vary along the direction of the magnetic field. The definition of the electron collision term is $\Rvec_e = (m_e \nu_{ei}/e) \Jvec$. Using the definition of the resistivity $\eta = m_e \nu_{ei}/e^2 n_e$, we get $\Rvec_e = e n_e \eta \Jvec$. Thus, we now have
\begin{equation}
    0 = -\nabla_{||} \phi + \frac{T_e}{e} \nabla_{||} n - \eta J_{||},
\end{equation}
which, upon re-arranging, gives
\begin{equation}
    \label{eq:hw_electron_momentum}
    J_{||} = \frac{T_e}{e \eta} \nabla_{||} \left ( n - \frac{e\phi}{T_e} \right ).
\end{equation}
The perpendicular component of \cref{eq:hw_electron_momentum_no_inertia} is as follows
\begin{equation}
    0 = -\nabla_\perp \phi + \uvec_e \times \Bvec + \frac{T_e}{e} \nabla_\perp \ln n_e - \frac{1}{e n_e}\Rvec_{e,\perp}.
\end{equation}
Since $\uvec_e \times \Bvec = \uvec_{e,\perp} \times \Bvec$ we have
\begin{equation}
    0 = -\nabla_\perp \phi + \uvec_{e,\perp} \times \Bvec + \frac{T_e}{e} \nabla_\perp \ln n_e - \frac{1}{e n_e}\Rvec_{e,\perp}.
\end{equation}
Again, using the definition of the electron collision term, we get
\begin{equation}
    0 = -\nabla_\perp \phi + \uvec_{e,\perp} \times \Bvec + \frac{T_e}{e} \nabla_\perp \ln n_e - \eta \Jvec_\perp.
\end{equation}
Crossing the above by $\Bvec$ gives 
\begin{equation}
    \uvec_{e,\perp} = -\nabla_\perp \phi \times \frac{\bvec}{B} - \frac{T_e}{e n_e B} {\bvec \times \nabla_\perp n_e} + \frac{\eta}{B} \bvec \times \Jvec_\perp.
\end{equation}
Typically the last term on the right-hand side above is significantly smaller, and thus it can be neglected. The electron velocity is thus
\begin{equation}
    \label{eq:hw_elec_vel}
    \uvec_{e,\perp} = -\nabla_\perp \phi \times \frac{\bvec}{B} - \frac{T_e}{e n_e B} {\bvec \times \nabla_\perp n_e}.
\end{equation}

The continuity equation for electrons is as follows
\begin{equation}
    \frac{\partial n_e}{\partial t} + \nabla \cdot (n_e \uvec_e) = 0.
\end{equation}
We split the convection term in the above into the perpendicular and parallel components
\begin{equation}
    \frac{\partial n_e}{\partial t} + \nabla_\perp \cdot (n_e \uvec_{e,\perp}) + \nabla_{||} \cdot (n_e u_{e,||}) = 0.
\end{equation}
Given the assumption in \cref{it:hw_par_ion}, we have $J_{||} = e n_e (u_{i,||} - u_{e,||}) = -en_e u_{e,||}$. Thus, the above becomes
\begin{equation}
    \frac{\partial n_e}{\partial t} + \nabla_\perp \cdot (n_e \uvec_{e,\perp}) = \frac{1}{e} \nabla_{||} J_{||}.
\end{equation}
Using the identity $\nabla \cdot (\Avec \times \Bvec) = \Bvec \cdot (\nabla \times \Avec) - \Avec \cdot ( \nabla \times \Bvec)$ we show that 
\begin{equation}
    \nabla_\perp \cdot (\bvec \times \nabla_\perp n_e) = \nabla_\perp n_e \cdot (\nabla_\perp \times \bvec) - \bvec \cdot (\nabla_\perp \times \nabla_\perp n_e) = 0.
\end{equation}
As a result, the second term on the right-hand side of \cref{eq:hw_elec_vel} does not contribute to $\nabla_\perp \cdot (n_e \uvec_{e,\perp})$. The electron continuity equation then becomes
\begin{equation}
    \frac{\partial n_e}{\partial t} - \left ( \nabla_\perp \phi \times \frac{\bvec}{B} \right ) \cdot \nabla_\perp n_e = \frac{1}{e} \nabla_{||} J_{||}.
\end{equation}
Dividing by $n_e$ and using quasi-neutrality
\begin{equation}
    \frac{\partial \ln n_i}{\partial t} - \left ( \nabla_\perp \phi \times \frac{\bvec}{B} \right ) \cdot \nabla_\perp \ln n_i = \frac{1}{e n_i} \nabla_{||} J_{||}.
\end{equation}
Using the expression for $\ln n_i$ in \cref{eq:hw_ion_density}, we get
\begin{equation}
    \frac{\partial n}{\partial t} - \left ( \nabla_\perp \phi \times \frac{\bvec}{B} \right ) \cdot \nabla_\perp (\ln n_0 + n ) = \frac{1}{e n_i} \nabla_{||} J_{||}.
\end{equation}
Finally, using the assumption in \cref{it:hw_ion_density}, we neglect $n'$ in the denominator of the right-hand side, and obtain
\begin{equation}
    \label{eq:hw_electron_continuity}
    \frac{\partial n}{\partial t} - \left ( \nabla_\perp \phi \times \frac{\bvec}{B} \right ) \cdot \nabla_\perp \left ( n + \ln n_0 \right ) = \frac{1}{en_0} \nabla_{||} J_{||}.
\end{equation}

Combining \cref{eq:hw_electron_momentum,eq:hw_ion_continuity,eq:hw_electron_continuity} leads to the dimensional form of the Hasegawa-Wakatani model
\begin{multline}
    \frac{\partial}{\partial t} \left ( \frac{1}{w_{c,i} B} \nabla^2_\perp \phi \right ) - \left (\nabla_\perp \phi \times \frac{\bvec}{B} \right ) \cdot \nabla_\perp \left ( \frac{1}{w_{c,i} B} \nabla^2_\perp \phi \right ) \\
    = \frac{T_e}{e^2 n_0 \eta} \nabla_{||}^2 \left (n - \frac{e \phi}{T_e} \right ) + \frac{\nu}{w_{c,i} B} \nabla^4_\perp \phi.
\end{multline}
\begin{equation}
    \frac{\partial n}{\partial t} - \left ( \nabla_\perp \phi \times \frac{\bvec}{B} \right ) \cdot \nabla_\perp \left ( n + \ln n_0 \right ) = \frac{T_e}{e^2 n_0 \eta} \nabla_{||}^2 \left (n - \frac{e \phi}{T_e} \right ).
\end{equation}
It is quite common to replace the parallel-gradient operator $\nabla_{||}$ by a coefficient, say $1/l^2$.

We now introduce the following non-dimensionalization
\begin{align}
    \phi(t,x_1,x_2,x_3) &= \frac{T_e}{e} \hat{\phi}(\hat{t},\hat{x}_1,\hat{x}_2, x_3) \\
    n_0(x_1) &= \hat{n}_0(\hat{x}_1) \\
    n(t,x_1,x_2,x_3) &= \hat{n}(\hat{t}, \hat{x}_1, \hat{x}_2, x_3),
\end{align}
where $\hat{t} = t w_{c,i}$, $\hat{x}_1 = x_1 / r_s$, and $\hat{x}_2 = x_2 / r_s$. Neglecting the hat notation for the sake of simplicity, the Hasegawa-Wakatani model in non-dimensional form is written as
\begin{equation}
    \label{eq:hw_potential}
    \frac{\partial \nabla^2_\perp \phi}{\partial t} - \left (\nabla_\perp \phi \times \bvec \right ) \cdot \nabla_\perp \left ( \nabla^2_\perp \phi \right ) = c_1 ( \phi - n) + c_2 \nabla^4_\perp \phi,
\end{equation}
\begin{equation}
    \label{eq:hw_density}
    \frac{\partial n}{\partial t} - \left ( \nabla_\perp \phi \times \bvec \right ) \cdot \nabla_\perp \left ( n + \ln n_0 \right ) = c_1 (\phi - n),
\end{equation}
where
\begin{equation}
    c_1 = -\frac{T_e}{e^2 n_0 \eta w_{c,i}} \nabla^2_{||} \qquad c_2 = \frac{\nu}{w_{c,i} r_s^2}.
\end{equation}
If $\nabla_{||}$ is replaced by $1/l^2$, then the $c_1$ operator is simply a coefficient.

%--------------------------------------------
\subsection{Relationship to other models}
%--------------------------------------------
The Hasegawa-Wakatani \cref{eq:hw_potential,eq:hw_density} contain two limits. Assuming $c_1$ is a coefficient rather than an operator, one of the limits is obtained by letting $c_1 = 0$. Then, the $\phi$ and $n$ equations are decoupled, and the $\phi$ equation corresponds to the third (and only non-zero) component of the 2D Navier-Stokes equations for vorticity
\begin{equation}
    \frac{\partial \wvec}{\partial t} + \uvec \cdot \nabla_\perp \wvec = \nu \nabla^2_\perp \wvec,
\end{equation}
where
\begin{equation}
    \uvec = \nabla_\perp \times (-\phi \bvec) = - \nabla_\perp \phi \times \bvec,
\end{equation}
and
\begin{equation}
    \wvec = \nabla_\perp \times \uvec = \nabla^2_\perp \phi \bvec.
\end{equation}
If, on the other hand, $c_1 \to \infty$, then dividing \cref{eq:hw_density} by $c_1$ shows that $n = \phi$. Subtracting \cref{eq:hw_density} from \cref{eq:hw_potential} and assuming $c_2=0$ one obtains 
\begin{equation}
    \frac{\partial}{\partial t} \left ( \nabla^2_\perp \phi - \phi \right ) - \left (\nabla_\perp \phi \times \bvec \right ) \cdot \nabla_\perp \left ( \nabla^2_\perp \phi - \ln n_0 \right ) = 0.
\end{equation}
The above is the Hasegawa-Mima equation.

\bibliographystyle{plainnat}
\bibliography{library}
\end{document}