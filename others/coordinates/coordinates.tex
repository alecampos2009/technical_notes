\documentclass[11pt]{article}

\usepackage{"../../info/packages"}
\usepackage{"../../info/nomenclature"}
\usepackage{fullpage}
\usepackage{booktabs}
\usepackage{array}
\usepackage{cellspace}
\newcommand{\uhat}{\hat{u}}
\newcommand{\xhat}{\hat{x}}

\title{Coordinate system for Tokamaks}


\begin{document}

\maketitle

\section{Basic definitions}
\subsection{Eulerian coordinates}
Consider our traditional Eucledian coordinate system given by coordinates $(x^1,x^2,x^3)$ and unit vectors $\xvec_1, \xvec_2, \xvec_2$. The position vector is $\xvec = x^1 \xvec_1 + x^2 \xvec_2 + x^3 \xvec_3$.

\subsection{Curvilinear Coordinates}
We will now define a new coordinate system, a curvilinear coordinate system, in relation to the standard Eucledian coordinates. To do so we first define the transformation 
\begin{equation}
\uhat^i = \uhat^i(x^1, x^2, x^3) .
\end{equation}
and we label its inverse as
\begin{equation}
\xhat^i = \xhat^i(u^1,u^2,u^3)
\end{equation}
Thus, we can write 
\begin{align}
    \xhat^i(\uhat^1, \uhat^2, \uhat^3) &= x^i \label{eq:trans1}\\
    \uhat^i(\xhat^1, \xhat^2, \xhat^3) &= u^i \label{eq:trans2}
\end{align}
We can now take the derivative of either \cref{eq:trans1} or \cref{eq:trans2}. For example, the derivative $d/du^j$ of \cref{eq:trans2} gives
\begin{equation}
    \left ( \frac{\partial \uhat^i}{\partial x^k} \right)_{x^i = \xhat^i} \frac{\partial \xhat^k}{\partial u^j} = \delta^i_j.
\end{equation}
We can evaluate the above at $u^i = \hat{u}^i$, so that
\begin{equation}
     \frac{\partial \uhat^i}{\partial x^k} \left( \frac{\partial \xhat^k}{\partial u^j} \right)_{u^i = \hat{u}^i} = \delta^i_j.
\end{equation}
We now define two basis vectors as follows
\begin{equation}
\label{eq:basis_sub}
    \evec_i = \left ( \frac{ \partial \xhat^1}{\partial u^i} \right)_{u^i = \hat{u}^i} \xvec_1 + \left ( \frac{ \partial \xhat^2}{\partial u^i} \right)_{u^i = \hat{u}^i} \xvec_2 + \left ( \frac{ \partial \xhat^3}{\partial u^i} \right)_{u^i = \hat{u}^i} \xvec_3
\end{equation}
\begin{equation}
\label{eq:basis_sup}
    \evec^i = \frac{ \partial \uhat^i}{\partial x^1} \xvec^1 + \frac{ \partial \uhat^i}{\partial x^2} \xvec^2 + \frac{ \partial \uhat^i}{\partial x^3}. \xvec^3
\end{equation}
The dot product of these two vectors is given by
\begin{equation}
\label{eq:basis_orthogonality}
    \evec^i \cdot \evec_j = \delta^i_j.
\end{equation}
The two coordinate bases are not necessarily constant, orthogonal, of unit length, or dimensionless.

\textbf{At the end, one way to think about it is that for every point $[x^1,x^2,x^3]$ in the Eucledian coordiante system, there is a coresponding coordinate given by $[\uhat^1,\uhat^2,\uhat^3]$, and that at every point of these new coordinates, there are two coordinate bases, given by \cref{eq:basis_sub} and \cref{eq:basis_sup}.} The latter basis is typically expresses as $\nabla \uhat^1$,$\nabla \uhat^2$, and $\nabla \uhat^3$.

\subsection{Vectors}
Since there are two coordinate bases, one can define two types of vectors at every point in the domain. One is a vector in terms of contravariant components $v^i$ 
\begin{equation}
    \vvec = v^1 \evec_1 + v^2 \evec_2 + v^3 \evec_3,
\end{equation}
and the other a vector in terms of covariant components $v_i$
\begin{equation}
    \vvec = v_1 \evec^1 + v_2 \evec^2 + v_3 \evec^3.
\end{equation}
Note that, due to \cref{eq:basis_orthogonality}, we have $v^i = \vvec \cdot \evec^i$ and $v_i = \vvec \cdot \evec_i$.

We now define the metric coefficients $g_{ij}$ and $g^{ij}$ as
\begin{align}
    g_{ij} &= \evec_i \cdot \evec_j \\
    g^{ij} &= \evec^i \cdot \evec^j \\
\end{align}
Thus, the dot product of two vectors in the curvilinear reference frame simplifies to the following
\begin{equation}
\vvec \cdot \wvec = v^i w_i = v_i w^i = g_{ij} v^i w^j = g^{ij} v_i w_j.
\end{equation}
The cross product can be computed as
\begin{align}
    \vvec \times \wvec &= \epsilon_{ijk} \sqrt{g} v^i w^j \evec^k \\
    \vvec \times \wvec &= \epsilon^{ijk} \frac{1}{\sqrt{g}} v_i w_j \evec_k,
\end{align}
where $g = \text{det}(g_{ij})$. One can also define $g^{-1} = \text{det}(g^{ij})$.

\section{Calculus}
\subsection{Integration}
\begin{itemize}

\item Volume integrals: Define $d V_u$ as an infinitesimal volume in curvilinear coordinates, $\Omega_u$ as a finite volume of integration in curvilinear coordinates, and $f_u$ as a function whose input is in curvilinear coordinates, that is, $f_u = f_u(u^1,u^2,u^3)$. Then, the volume integral can be computed using
\begin{equation}
\label{eq:vol_int}
    \int_{\Omega_u} f_u \, d V_u = \int_{\Omega_u} f_u \, |\text{det}(J_{ij})| du^1 du^2 du^3,
\end{equation}
where $J_{ij} = \partial \hat{x}^i / \partial u^j$ is the Jacobian. Given that Eulerian coordinates can be thought of as an instance of curvilinear coordinates, we have
\begin{equation}
\label{eq:vol_int_eulerian}
    \int_{\Omega_u} f_u \, dV_u =  \int_{\Omega_x} f_x \, dV_x = \int_{\Omega_x} f_x \, dx^1 dx^2 dx^3 
\end{equation}
One thing to note is that $f_x(x^1,x^2,x^3)$ and $f_u(u^1,u^2,u^3)$ are equal when evaluated at the same point in space.
In other words, these functions satisfy
\begin{equation}
    f_u(u^1,u^2,u^3) = f_x(\hat{x}^1, \hat{x}^2, \hat{x}^3).
\end{equation}
Equating \cref{eq:vol_int,eq:vol_int_eulerian} allows us to write the standard rule for integration by substitution
\begin{equation}
    \int_{\Omega_x} f_x(x^1,x^2,x^3) \, dx^1 dx^2 dx^3 = \int_{\Omega_u} f_x(\hat{x}^1, \hat{x}^2, \hat{x}^3) \, |\text{det}(J_{ij})| du^1 du^2 du^3 .
\end{equation}

\item Surface integrals: Define $d S_u$ as an infinitesimal surface in curvilinear coordinates, $\Gamma_u$ as a finite surface of integration in curvilinear coordinates that belongs to the $u^1 = \text{constant}$ surfaces, and $f_u$ as a function whose input is defined using curvilinear coordinates. Then, a surface integral can be computed using 
\begin{equation}
    \int_{\Gamma_u} f_u\, d S_u = \int_{\Gamma_u} f_u \, J | \nabla \hat{u}^1| du^2 du^3.
\end{equation}
Note that now we can write
\begin{multline}
\label{eq:int_from_vol_surf}
    \int_{\Omega_u} f_u \,dV_u = \int_{u^1_l}^{u^1_u} \int_{\Gamma_u} f_u \, J du^1 du^2 du^3 = \\ 
    \int_{u^1_l}^{u^1_u} \int_{\Gamma_u} f_u \, \frac{J | \nabla \hat{u}^1| du^2 du^3}{| \nabla \hat{u}^1 |} du^1 = \int_{u^1_l}^{u^1_u} \int_{\Gamma_u} f_u \, \frac{dS_u}{|\nabla \hat{u}^1|} du^1.
\end{multline}

\end{itemize}

\subsection{Differentiation}
\subsubsection{The grad operator}
Consider the function $f_x = f_x(x^1, x^2, x^3)$ and the grad operator, which is  
\begin{equation}
    \nabla f_x = \frac{\partial f_x}{\partial x^1} \xvec_1 + \frac{\partial f_x}{\partial x^2} \xvec_2 + \frac{\partial f_x}{\partial x^3} \xvec_3.
\end{equation}
We now introduce the function $f_u = f_u(u^1,u^2,u^3)$ and note that $f_x = f_u(\uhat^1,\uhat^2,\uhat^3)$. Thus
\begin{equation}
    \frac{\partial f_x}{\partial x^1} = \left ( \frac{ \partial f_u}{\partial u^1} \right)_{\uvec = \hat{\uvec}}        \frac{\partial \uhat^1}{\partial x^1} + 
                     \left ( \frac{ \partial f_u}{\partial u^2} \right)_{\uvec = \hat{\uvec}} \frac{\partial \uhat^2}{\partial x^1} +
                     \left ( \frac{ \partial f_u}{\partial u^3} \right)_{\uvec = \hat{\uvec}} \frac{\partial \uhat^3}{\partial x^1}, 
\end{equation}
\begin{equation}
    \frac{\partial f_x}{\partial x^2} = \left ( \frac{ \partial f_u}{\partial u^1} \right)_{\uvec = \hat{\uvec}} \frac{\partial \uhat^1}{\partial x^2} + 
                     \left ( \frac{ \partial f_u}{\partial u^2} \right)_{\uvec = \hat{\uvec}} \frac{\partial \uhat^2}{\partial x^2} +
                     \left ( \frac{ \partial f_u}{\partial u^3} \right)_{\uvec = \hat{\uvec}} \frac{\partial \uhat^3}{\partial x^2},
\end{equation}
\begin{equation}
    \frac{\partial f_x}{\partial x^3} = \left ( \frac{ \partial f_u}{\partial u^1} \right)_{\uvec = \hat{\uvec}} \frac{\partial \uhat^1}{\partial x^3} + 
                     \left ( \frac{ \partial f_u}{\partial u^2} \right)_{\uvec = \hat{\uvec}} \frac{\partial \uhat^2}{\partial x^3} +
                     \left ( \frac{ \partial f_u}{\partial u^3} \right)_{\uvec = \hat{\uvec}} \frac{\partial \uhat^3}{\partial x^3} .
\end{equation}
Using the definition of $\evec^i$, the grad operator can be written as
\begin{equation}
    \nabla f_x = \left ( \frac{\partial f_u }{\partial u^1} \right )_{\uvec = \hat{\uvec}} \evec^1 + \left ( \frac{\partial f_u }{\partial u^2} \right )_{\uvec = \hat{\uvec}} \evec^2 + \left ( \frac{\partial f_u }{\partial u^3} \right )_{\uvec = \hat{\uvec}} \evec^3 .
\end{equation}
This shows the equivalence between the grad operator in Eulerian coordinates and curvilinear coordinates.

\subsubsection{The divergence operator}

\subsubsection{The curl operator}
The curl is given by
\begin{equation}
    \nabla \times A = \epsilon^{ijk} \frac{1}{\sqrt{g}} \frac{\partial A_j}{\partial u^i} \evec_k
\end{equation}

\section{Flux coordinates}
Imagine that Eucledian space is permeated by a set of surfaces, which we call flux surfaces. Each of those surfaces is labeled with a different value of the variable $\psi$. We also note that the flux surfaces are not stationary, they can move around as time progresses.

We now introduce the function $\hat{\psi} = \hat{\psi}(t,x^1,x^2,x^3)$. This function is defined in such a way that for all values $x^1,x^2,x^3$ that are part of a given flux surface at a specific time $t$, then $\hat{\psi}$ will evaluate to the value of $\psi$ corresponding to that flux surface. The velocity of the flux surfaces is given by $\Vvec_\psi = \Vvec_\psi(t,x^1,x^2,x^3)$. Thus, by definition
\begin{equation}
\label{eq:cons_flux_surf}
    \frac{\partial \hat{\psi}}{\partial t} + \Vvec_\psi \cdot \nabla \hat{\psi} = 0.
\end{equation}

A flux coordinate is defined as one in which $\hat{u}^1 = \hat{\psi}$.


\subsection{Flux-surface averaging}
To begin, we define the following. $D(\psi,t)$ is the volume enclosed at time t by the flux surface labelled by $\psi$. The surface of $D(\psi,t)$ is labelled as $\partial D(\psi,t)$. Additionally, $\Delta(\psi,t) = D(\psi + \Delta \psi,t) - D(\psi,t)$.

The flux surface average of a function is given by
\begin{equation}
\label{eq:flux_surf_avg_1}
    \langle f \rangle_\psi = \lim_{\Delta \psi \to 0} \frac{ \int_{\Delta(\psi,t)} f \, d V}{\int_{\Delta(\psi,t)} d V}.
\end{equation}
This can be re-written as shown below
\begin{equation}
    \langle f \rangle_\psi = \lim_{\Delta \psi \to 0} \frac{ \frac{1}{\Delta \psi} \int_{\Delta(\psi,t)} f \, d V}{ \frac{1}{\Delta(\psi)} \int_{\Delta(\psi,t)} d V} = \lim_{\Delta \psi \to 0} \frac{ \frac{1}{\Delta \psi} \left ( \int_{D(\psi + \Delta \psi,t)} f\, d V - \int_{D(\psi,t)} f\, d V \right)}{ \frac{1}{\Delta \psi} \left (\int_{D(\psi + \Delta \psi,t)} d V - \int_{D(\psi,t)} d V \right )} = \frac{ \frac{\partial}{\partial \psi} \int_{D(\psi,t)} f \, d V}{ \frac{\partial}{\partial \psi} \int_{D(\psi,t)} d V} .
\end{equation}
Defining $V' = V'(\psi,t)$ as 
\begin{equation}
    V' = \frac{\partial}{\partial \psi} \int_{D(\psi,t)} \, dV.
\end{equation}
the second expression for the flux surface average is written as
\begin{equation}
\label{eq:flux_surf_avg_2}
    \langle f \rangle_\psi = \frac{1}{V'} \frac{\partial}{\partial \psi} \int_{D(\psi,t)} f \, d V .
\end{equation}
A third expression for $\langle g \rangle_\psi$ follows from using \cref{eq:int_from_vol_surf} for the above. Thus,
\begin{equation}
\label{eq:flux_surf_avg_3}
    \langle f \rangle_\psi = \frac{1}{V'} \frac{\partial}{\partial \psi} \int_0^\psi \int_{\partial D(\psi',t)} f \, \frac{d S}{| \nabla \hat{\psi} |} d\psi' = \frac{1}{V'}  \int_{\partial D(\psi,t)} f \, \frac{d S}{| \nabla \hat{\psi} |} .
\end{equation}

\subsubsection{Average of spatial derivatives}
We use the second definition of the flux-surface average, given by \cref{eq:flux_surf_avg_2}, and then the divergence theorem to obtain
\begin{equation}
    \langle \nabla \cdot \Avec \rangle_\psi = \frac{1}{V'} \frac{\partial}{\partial \psi} \int_{D(\psi,t)} \nabla \cdot \Avec \,dV = \frac{1}{V'} \frac{\partial}{\partial \psi} \int_{\partial D(\psi,t)} \Avec \cdot  \frac{\nabla \hat{\psi}}{|\nabla \hat{\psi}|} \, dS.
\end{equation}
We now use the third definition \cref{eq:flux_surf_avg_3} to obtain
\begin{equation}
    \langle \nabla \cdot \Avec \rangle_\psi = \frac{1}{V'} \frac{\partial}{\partial \psi} V' \langle \Avec \cdot \nabla \hat{\psi} \rangle_\psi.
\end{equation}

\subsubsection{Average of time derivatives}
Using the Reynolds transport theorem we show
\begin{align}
    \frac{\partial}{\partial t} \int_{D(\psi,t)} f \,dV &= \int_{D(\psi,t)} \frac{\partial f}{\partial t} \, dV + \int_{\partial D(\psi,t)} f \Vvec_\psi \cdot \frac{\nabla \hat{\psi}}{|\nabla \hat{\psi}|} \,dS \nonumber \\
    &= \int_{D(\psi,t)} \frac{\partial f}{\partial t} \, dV + V' \langle f \Vvec_\psi \cdot \nabla \hat{\psi} \rangle_\psi.
\end{align}
We now take the derivative of both sides by $\psi$ and then divide by $V'$.
\begin{equation}
    \frac{1}{V'} \frac{\partial}{\partial t} V' \langle f \rangle_\psi = \left < \frac{\partial f}{\partial t} \right>_\psi + \frac{1}{V'} \frac{\partial}{\partial \psi} V' \langle f \Vvec_\psi \cdot \nabla \hat{\psi} \rangle_\psi.
\end{equation}
Re-arranging and using \cref{eq:cons_flux_surf}
\begin{equation}
    \left < \frac{\partial f}{\partial t} \right >_\psi = \frac{1}{V'} \frac{\partial}{\partial t} V' \langle f \rangle_\psi + \frac{1}{V'} \frac{\partial}{\partial \psi} V' \left < f \frac{\partial \hat{\psi}}{\partial t} \right >_\psi.
\end{equation}

\end{document}