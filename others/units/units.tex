\documentclass[11pt]{article}

\usepackage{"../../info/packages"}
\usepackage{"../../info/nomenclature"}
\usepackage{fullpage}


\title{Units}


\begin{document}

\maketitle

\setlength{\cellspacetoplimit}{3pt}
\setlength{\cellspacebottomlimit}{3pt}

\section{Unit systems}
Note: the energy unit for the first section below is called the Jerk, whereas in cgs units it is called the erg.
\section{1st set of units}

\begin{center}
    \begin{tabular}{|Sc|Sc|Sc|Sc|Sc|}
        \hline
        variable & definition & in SI \\
        \hline 
        Length & cm & \\
        \hline
        Mass & g & \\
        \hline
        Time & sh & \\
        \hline
        Temperature & KeV & \\
        \hline
        Velocity & cm/sh & \\
        \hline
        Energy & g \( \cdot \) cm\textsuperscript{2} / sh\textsuperscript{2} & $10^9$ J  $(10^3$ MJ)\\
        \hline 
        Pressure & g / cm \( \cdot \) sh\textsuperscript{2} & $10^{15}$ Pa ($10^4$ Mbar) \\
        \hline
        Density & g / cm\textsuperscript{3} & \\
        \hline 
        Intensity & g / sh\textsuperscript{3} & $10^{21}$ W/m\textsuperscript{2} ($10^{17}$ W/cm\textsuperscript{2}) \\
        \hline
    \end{tabular}
\end{center}

The Stefan-Boltzman law is
\begin{equation}
    I = \frac{1}{\pi} \sigma T^4,
\end{equation} 
where 
\begin{equation}
    \sigma = \frac{2 \pi^5 k_B^4 }{15 c^2 h^3}.
\end{equation}
If we introduce 
\begin{equation}
    \eta = \frac{2 \pi^5 }{15 c^2 h^3},
\end{equation}
then we can re-write the Stefan-Boltzman law as
\begin{equation}
    I = \frac{1}{\pi} \eta (k_B T)^4.
\end{equation}
The value of $\eta$ in adive units can be be derived as follows
\begin{align}
    \eta &= 1.5605532983363822 \times 10^{84} \frac{1}{\text{s} \cdot \text{m}^2 \cdot \text{J}^3} \nonumber \\
    &= 1.5605532983363822 \times 10^{84} \frac{\text{J}}{\text{s} \cdot \text{m}^2 \cdot \text{J}^4} \left | \frac{1.602176634 \times 10^{-16} \text{J}}{1\text{keV}} \right |^4 \left | \frac{1\text{m}}{10^2 \text{cm}} \right |^2 \nonumber \\
    &= 1.028300817017691 \times 10^{17} \frac{\text{J}}{\text{s} \cdot \text{cm}^2 \cdot \text{keV}^4} \nonumber \\
    &= 1.028300817017691 \times 10^{17} \frac{\text{W}}{\text{cm}^2 \cdot \text{keV}^4} \nonumber \\
    &= 1.028300817017691 \frac{\text{g}}{\text{sh}^3 \cdot \text{keV}^4}.
\end{align}

\section{2nd set of units}

\begin{center}
    \begin{tabular}{|Sc|Sc|Sc|Sc|Sc|}
        \hline
        variable & definition & in SI \\
        \hline 
        Length & cm & \\
        \hline
        Mass & g & \\
        \hline
        Time & us & \\
        \hline
        Temperature & KeV & \\
        \hline
        Velocity & cm/us & \\
        \hline
        Energy & g \( \cdot \) cm\textsuperscript{2} / us\textsuperscript{2} & $10^5$ J  $(10^{-1}$ MJ)\\
        \hline 
        Pressure & g / cm \( \cdot \) us\textsuperscript{2} & \\
        \hline
        Density & g / cm\textsuperscript{3} & \\
        \hline 
        Intensity & g / us\textsuperscript{3} & $10^{15}$ W/m\textsuperscript{2} ($10^{11}$ W/cm\textsuperscript{2})\\
        \hline
    \end{tabular}
\end{center}

\end{document}