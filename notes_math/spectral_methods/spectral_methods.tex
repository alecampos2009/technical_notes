\documentclass[11pt]{article}

\usepackage{../../info/packages}
\usepackage{../../info/nomenclature}
\usepackage{fullpage}
\usepackage{tabu}

\newcommand{\mi}{\mathrm{i}}
\newcommand{\mj}{\mathrm{j}}
\newcommand{\mk}{\mathrm{k}}

\renewcommand{\arraystretch}{1.5}

\title{Spectral Methods}
\author{Alejandro Campos}

\begin{document}

\maketitle
\tableofcontents

A continuous function is denoted as $f(x)$ whereas a discrete function is denoted as $f_m$. Vectors are denoted in bold.

%----------------------------------------------------------------------------------------------------------
\section{Fourier Analysis}
%----------------------------------------------------------------------------------------------------------
%---------------------------------------
\subsection{Fourier Series}
%---------------------------------------
\begin{itemize}
\item Definition:
\begin{align*}
\fvec(\xvec) &= \sum_{\nvec=-\infty}^\infty \hat{\fvec}_\nvec e^{ i \kvec_\nvec \cdot \xvec} \\
\hat{\fvec}_\nvec &= \frac{1}{L^3}\int_{\mathbb{L}^3} \fvec(\xvec)e^{-i\kvec_\nvec \cdot \xvec}d\xvec
\end{align*}
where
\[ \kvec_\nvec = \frac{2 \pi}{L} \nvec \qquad \nvec = \begin{pmatrix} n_1 \\ n_2 \\ n_3 \end{pmatrix}  \]
Note: $ \sum_{\nvec = -\infty}^{\infty} = \sum_{n_1 = -\infty}^{\infty} \sum_{n_2 = -\infty}^{\infty} \sum_{n_3 = -\infty}^{\infty} $

\item Parseval's identity:
\[ \frac{1}{L^3} \int_{\mathbb{L}^3} \fvec(\xvec) \cdot \gvec^*(\xvec) \, d\xvec = \sum_{\nvec = -\infty}^{\infty} \hat{ \fvec }_\nvec \cdot \hat{ \gvec }^*_\nvec \]

\end{itemize}

%---------------------------------------
\subsection{Discrete Fourier Series}
%---------------------------------------
\begin{itemize}
\item Definition:
\[\fvec_\mvec = \sum_{\nvec=-N/2}^{N/2-1} \hat{\fvec}_\nvec e^{i \kvec_\nvec \cdot \xvec_\mvec}\]
\[\hat{\fvec}_\nvec =\frac{1}{N^3}\sum_{\mvec=0}^{N-1} \fvec_\mvec e^{-i \kvec_\nvec \cdot \xvec_\mvec}\]
 where 
 \[ \kvec_\nvec = \frac{2 \pi}{L} \nvec \qquad \xvec_\mvec = \frac{L}{N}\mvec \qquad \nvec = \begin{pmatrix} n_1 \\ n_2 \\ n_3 \end{pmatrix} \qquad \mvec = \begin{pmatrix} m_1 \\ m_2 \\m_3 \end{pmatrix} \]
 
 \item Parseval's identity:
 \[ \frac{1}{N^3} \sum_{\mvec=0}^{N-1} \fvec_\mvec \cdot \gvec^*_\mvec = \sum_{\nvec=-N/2}^{N/2-1} \hat{ \fvec }_\nvec \cdot \hat{ \gvec }^*_\nvec \]
 \end{itemize}
 
%---------------------------------------
\subsection{Fourier Transform}
%---------------------------------------
\begin{itemize}
\item Definition:
\[\fvec(\xvec) = \int_{\Rn} \hat{\fvec}(\kvec)e^{i\kvec \cdot \xvec} \,d\kvec \]
\[\hat{\fvec}(\kvec) = \frac{1}{(2\pi)^n}\int_{\Rn} \fvec(\xvec)e^{-i\kvec \cdot \xvec} \,d\xvec\]

\item Common functions
\begin{center}
\begin{tabular}{|c|c|}
\hline
$f(\xvec)$ & $\hat{f}(\kvec)$ \\
\hline
$e^ {i \boldsymbol{\lambda} \cdot \xvec}$ & $\delta(\kvec - \boldsymbol{\lambda})$ \\
\hline
$\delta(\xvec - \yvec)$ & $\frac{1}{(2\pi)^n}e^{- i \kvec \cdot \yvec} $ \\
\hline
\end{tabular}	
\end{center}

\item Parseval's Identity:
\[ \frac{1}{ ( 2 \pi )^3 } \int_\Rthree \fvec(\xvec) \cdot \gvec^*(\xvec) \,d\xvec =  \int_\Rthree \hat \fvec(\kvec) \hat \gvec^*(\kvec) \,d\kvec \]

\item Convolution: \\
Given
\[ h( \xvec ) = \int_{\Rthree} f( \xvec - \svec )g( \svec ) \, dx \]
then
\[\hat{h}(\kvec) = (2\pi)^3 \hat{f}(\kvec) \hat{g}(\kvec) \]

\end{itemize}

%---------------------------------------
\subsection{Spectral forms of common terms}
%---------------------------------------
We will use in this section both the hat notation and the $\mathcal{F}$ notation. That is, for the Fourier coefficient of a Fourier series, we use
\begin{equation*}
\hat{\fvec}_\nvec = \mathcal{F}^{(s)}\{ \fvec(\xvec) \}_\nvec,
\end{equation*}
and for the Fourier coefficient of a Fourier transform, we use
\begin{equation*}
 \hat{\fvec}(\kvec) = \mathcal{F}^{(t)}\{ \fvec(\xvec) \}(\kvec).
\end{equation*}

\begin{itemize}

\item General derivative:
\begin{equation*} 
  \mathcal{F}^{(s)} \left \{ \dfrac{\partial \fvec (\xvec)}{\partial x_j} \right \}_{\nvec}  = i \kappa_{\nvec,j} \hat{\fvec}_\nvec
\end{equation*} 
\begin{equation*} 
  \mathcal{F}^{(t)}\left \{ \dfrac{\partial \fvec (\xvec)}{\partial x_j} \right \}(\kvec) = i \kappa_j \hat{\fvec}(\kvec)
\end{equation*}

\item Derivative in spectral space:
\begin{equation*}
  \frac{\partial \hat{\fvec}(\kvec)}{\partial \kappa_j} = -i \mathcal{F}^{(t)} \left \{ x_j \fvec(\xvec) \right \} (\kvec)
\end{equation*}

\item Divergence:
\begin{equation*}
  \mathcal{F}^{(s)} \left \{ \nabla \cdot \fvec(\xvec) \right \}_\nvec = i \kvec_\nvec \cdot \hat{\fvec}_\nvec
\end{equation*}
\begin{equation*}
  \mathcal{F}^{(t)} \left \{ \nabla \cdot \fvec(\xvec) \right \} (\kvec) = i \kvec \cdot \hat{\fvec}(\kvec)
\end{equation*}

\item Curl:
\begin{equation*}
  \mathcal{F}^{(s)} \left \{ \nabla \times \fvec(\xvec) \right \}_\nvec = i \kvec_\nvec \times \hat{\fvec}_\nvec
\end{equation*}
\begin{equation*}
  \mathcal{F}^{(t)} \left \{ \nabla \times \fvec(\xvec) \right \} (\kvec) = i \kvec \times \hat{\fvec}(\kvec)
\end{equation*}

\item Laplacian:
\begin{equation*}
  \mathcal{F}^{(s)} \left \{ \nabla^2 \fvec(\xvec) \right \}_\nvec = - \kappa_\nvec^2 \hat{\fvec}(\kvec)
\end{equation*}
\begin{equation*}
  \mathcal{F}^{(t)} \left \{ \nabla^2 \fvec(\xvec) \right \} (\kvec) = - \kappa^2 \hat{\fvec}(\kvec)
\end{equation*}

\end{itemize}


%----------------------------------------------------------------------------------------------------------
\section{Chebyshev Analysis}
%----------------------------------------------------------------------------------------------------------
%---------------------------------------
\subsection{Chebyshev Series}
%---------------------------------------
\begin{itemize}
\item Definition:
\[f(x)=\sum_{n=0}^{\infty}a_nT_n(x)\]
\[a_n=\frac{2}{\pi C_n}\int_{-1}^{1}f(x)T_n(x)\frac{dx}{\sqrt{1-x^2}}\]
where
\[ C_n = \left\{ \begin{array}{cc}  2 & n=0\\ 1 & O.W. \end{array}\right. \]
\end{itemize}

\subsection{Discrete Chebyshev Series}
\begin{itemize}
\item Definition:
\[f_j=\sum_{n=0}^{\infty}a_nT_n(x_j)\]
\[a_n=\frac{2}{NC_n}\sum_{j=0}^{N}\frac{1}{C_j}f_jT_n(x_j)\]
where
\[ C_n = \left\{ \begin{array}{cc}  2 & n=0,N\\ 1 & O.W. \end{array}\right.\]
\end{itemize}

\section{Classical Orthogonal Polynimials}
Orthogonal polynomials are the members of the set $\{P_n(x)\}_{n=1}^{\infty}$, where $P_n(x)$ is a polynomial of degree n.

They satisfy the orthogonality relation:
\[\langle P_n,P_m \rangle = \int_a^b P_n(x)P_m(x)w(x)dx  = \langle P_n,P_n \rangle \delta_{nm}\]
These orthogonal polynomials satisfy the following ODE,
\[g_2(x) P''_n + g_1(x)P'_n + a_nP_n = 0\]
and are generated from the Rodrigues' formula:
\[P_n(x) = \frac{1}{e_nw(x)}\frac{d^n}{dx^n}\{w(x)[g(x)]^n\}\]
The polynomials considered in this file are also solutions of the Sturm-Liouville BVP, that is, they satisfy the following ODE and appropriate boundary conditions.
\[ \frac{1}{r(x)}\left[\frac{d}{dx}\left(p(x)\frac{d}{dx}\right)+q(x)\right]P_n = - \lambda_n^2P_n\]
The polynomials are also orthogonal with respect to $r(x)$.

%---------------------------------------
\subsection{Jacobi Polynomials}
%---------------------------------------
This is the family of polynomials for which:
\begin{eqnarray*}
p(x) &=& (1 - x)^{\alpha + 1}(1+x)^{\beta + 1}\\
q(x) &=& 0\\
r(x) &=& (1 - x)^\alpha(1+x)^\beta \\
\lambda_n^2 &=& n(n + \alpha + \beta + 1)
\end{eqnarray*}

\subsubsection{Chebyshev}
Orthogonal basis of $L^2_w[-1,1]$, with
\[w(x) = \frac{1}{\sqrt{1 - x^2}}\] 

\[\langle P_n,P_n\rangle = \left\{ 
  \begin{array}{l l}
    \pi/2 & \quad \text{if $n = m \neq 0$ }\\
    \pi & \quad \text{if $n=m=0$ }
  \end{array} \right.\]
  
Coefficients of common form ODE
\begin{eqnarray*}
g_1(x) &=& -x\\
g_2(x) &=& 1 - x^2\\
a_n &=& n^2
\end{eqnarray*}
Coefficients of Rodrigues' formula
 \begin{eqnarray*}
g(x) &=& 1 - x^2\\
e_n &=& (-1)^n(2n-1)(2n-3)...1
\end{eqnarray*}
Coefficients of Sturm-Liouville ODE
\begin{eqnarray*}
p(x) &=& \sqrt{1 - x^2}\\
q(x) &=& 0\\
r(x) &=& \frac{1}{\sqrt{1 - x^2}}\\
\lambda^2_n &=& n^2
\end{eqnarray*}
That is, $\alpha = \beta = -1/2$.

\subsubsection{Legendre}
Orthogonal basis of $L^2_w[-1,1]$, with
\[w(x) = \frac{1}{2}\] 
\[\langle P_n,P_n\rangle = \frac{1}{2n+1}\]
Coefficients of common form ODE
\begin{eqnarray*}
g_1(x) &=& -2x\\
g_2(x) &=& 1 - x^2\\
a_n &=& n(n+1)
\end{eqnarray*}
Coefficients of Rodrigues' formula
 \begin{eqnarray*}
g(x) &=& 1 - x^2\\
e_n &=& (-1)^n2^nn!
\end{eqnarray*}
Coefficients of Sturm-Liouville ODE
\begin{eqnarray*}
p(x) &=& 1 - x^2\\
q(x) &=& 0\\
r(x) &=& 1\\
\lambda^2_n &=& n(n+1)
\end{eqnarray*}
That is, $\alpha = \beta = 0$.
%---------------------------------------
\subsection{Hermite}
%---------------------------------------
Orthogonal basis of $L^2_w[-\infty,\infty]$, with
\[w(x) = \frac{1}{\sqrt{2\pi}}e^{-x^2/2}\] 
\[\langle P_n,P_n\rangle = n!\]
Coefficients of common form ODE
\begin{eqnarray*}
g_1(x) &=& -x\\
g_2(x) &=& 1\\
a_n &=& n
\end{eqnarray*}
Coefficients of Rodrigues' formula
 \begin{eqnarray*}
g(x) &=& 1 \\
e_n &=& (-1)^n
\end{eqnarray*}
Coefficients of Sturm-Liouville ODE
\begin{eqnarray*}
p(x) &=& e^{-x^2/2}\\
q(x) &=& 0\\
r(x) &=& e^{-x^2/2}\\
\lambda^2_n &=& n
\end{eqnarray*}

%---------------------------------------
\subsection{Laguerre}
%---------------------------------------
Orthogonal basis of $L^2_w[0,\infty]$, with
\[w(x) = \frac{1}{\sqrt{2\pi}}e^{-x}\] 
\[\langle P_n,P_n\rangle = 1\]
Coefficients of common form ODE
\begin{eqnarray*}
g_1(x) &=& 1 - x\\
g_2(x) &=& x\\
a_n &=& n
\end{eqnarray*}
Coefficients of Rodrigues' formula
 \begin{eqnarray*}
g(x) &=& x \\
e_n &=& n!
\end{eqnarray*}
Coefficients of Sturm-Liouville ODE
\begin{eqnarray*}
p(x) &=& xe^{-x}\\
q(x) &=& 0\\
r(x) &=& e^{-x}\\
\lambda^2_n &=& n
\end{eqnarray*}

\end{document}