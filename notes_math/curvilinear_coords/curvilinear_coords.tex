\documentclass[11pt]{article}

\usepackage{"../../info/packages"}
\usepackage{"../../info/nomenclature"}
\usepackage{fullpage}
\usepackage{booktabs}
\usepackage{array}
\usepackage{cellspace}
\newcommand{\yhat}{\hat{y}}
\newcommand{\xhat}{\hat{x}}

\title{Curvilinear coordinates}
\author{Alejandro Campos}

\begin{document}

\maketitle
\tableofcontents

\section{Cartesian coordinates}
Consider Euclidean space, which is represented by cartesian coordinates. We denote the coordinates as $(x^1,x^2,x^3)$ and the basis unit vectors as $\xvec_1, \xvec_2, \xvec_2$. A vector field $\vvec = \vvec(x^1, x^2, x^3)$ can be expanded as 
\begin{equation}
    \label{eq:vector_cartesian}
    \vvec = v^1 \xvec_1 + v^2 \xvec_2 + v^3 \xvec_3,
\end{equation}
where $v^1 = v^1(x^1, x^2, x^3)$, $v^2 = v^2(x^1, x^2, x^3)$, $v^3 = v^3(x^1, x^2, x^3)$. 

Notice that the components of the vector field were denoted with upper scripts and the unit vectors with lower scripts. For cartesian coordinates it doesn't matter if we use upper or lower scripts, that is, $v^i = v_i$ and $\xvec_i=\xvec^i$. For other coordinate systems, as we'll see in the sections below, upper scripts and lower scripts have different meanings.

\section{Curvilinear Coordinates}
We will now define a new coordinate system, a curvilinear coordinate system, in relation to the standard cartesian coordinates. The coordinates of this curvilinear system are denoted as $(y^1, y^2, y^3)$. We introduce the following two sets of functions
\begin{align}
\yhat^i &= \yhat^i(x^1, x^2, x^3), \\
\xhat^i &= \xhat^i(y^1,y^2,y^3),
\end{align}
such that they satisfy
\begin{align}
    \xhat^i(\yhat^1, \yhat^2, \yhat^3) &= x^i \label{eq:trans1}\\
    \yhat^i(\xhat^1, \xhat^2, \xhat^3) &= y^i \label{eq:trans2},
\end{align}
for $i=1,2,3$. In other words, $\xhat^i$ and $\yhat^i$ are functions that map between coordinates in the cartesian and curvilinear systems. 

We can now take the derivative of either \cref{eq:trans1} or \cref{eq:trans2}. For example, the derivative $d/dy^j$ of \cref{eq:trans2} gives
\begin{equation}
    \left( \frac{\partial \yhat^i}{\partial x^k} \right)_{x^i = \xhat^i} \frac{\partial \xhat^k}{\partial y^j} = \delta^i_j.
\end{equation}
We can evaluate the above at $y^i = \yhat^i$, so that
\begin{equation}
     \frac{\partial \yhat^i}{\partial x^k} \left( \frac{\partial \xhat^k}{\partial y^j} \right)_{y^i = \yhat^i} = \delta^i_j.
\end{equation}
We now define two basis vectors as follows
\begin{equation}
\label{eq:basis_sub}
    \evec_i = \left( \frac{ \partial \xhat^1}{\partial y^i} \right)_{y^i = \yhat^i} \xvec_1 + \left( \frac{ \partial \xhat^2}{\partial y^i} \right)_{y^i = \yhat^i} \xvec_2 + \left( \frac{ \partial \xhat^3}{\partial y^i} \right)_{y^i = \yhat^i} \xvec_3
\end{equation}
\begin{equation}
\label{eq:basis_sup}
    \evec^i = \frac{ \partial \yhat^i}{\partial x^1} \xvec^1 + \frac{ \partial \yhat^i}{\partial x^2} \xvec^2 + \frac{ \partial \yhat^i}{\partial x^3}. \xvec^3
\end{equation}
The dot product of these two vectors is given by
\begin{equation}
\label{eq:basis_orthogonality}
    \evec^i \cdot \evec_j = \delta^i_j.
\end{equation}
The two coordinate bases are not necessarily constant, orthogonal, of unit length, or dimensionless.

\textbf{At the end, one way to think about it is that for every point $[x^1,x^2,x^3]$ in the cartesian coordinate system, there is a corresponding coordinate given by $[\yhat^1,\yhat^2,\yhat^3]$, and that at every point of these new coordinates, there are two coordinate bases, given by \cref{eq:basis_sub} and \cref{eq:basis_sup}.} The latter basis is typically expresses as $\nabla \yhat^1$,$\nabla \yhat^2$, and $\nabla \yhat^3$.

\section{Vector fields}
As mentioned earlier, a vector field in cartesian space is given by \cref{eq:vector_cartesian}. We now introduce the vector field $\uvec = \uvec(y^1, y^1, y^2)$, which is written in terms of the basis vectors of the curvilinear coordinates. Since there are two sets of basis vectors for the curvilinear coordinates, there are two ways to express $\uvec$. One is in terms of the contravariant components $u^i = u^i(y^1, y^2, y^3)$
\begin{equation}
    \uvec = u^1 \evec_1 + u^2 \evec_2 + u^3 \evec_3,
\end{equation}
and the other is in terms of the covariant components $u_i = u_i(y^1, y^2, y^3)$
\begin{equation}
    \uvec = u_1 \evec^1 + u_2 \evec^2 + u_3 \evec^3.
\end{equation}
If we want to write the same vector in terms of cartesian or curvilinear coordinates, then we require
\begin{equation}
    \vvec(x^1, x^2, x^3) = \uvec(\yhat^1, \yhat^2, \yhat^3).
\end{equation}

Note that, due to \cref{eq:basis_orthogonality}, we have $u^i = \uvec \cdot \evec^i$ and $u_i = \uvec \cdot \evec_i$. We define the metric coefficients $g_{ij}$ and $g^{ij}$ as
\begin{align}
    g_{ij} &= \evec_i \cdot \evec_j \\
    g^{ij} &= \evec^i \cdot \evec^j \\
\end{align}
Thus, the dot product of two vectors in the curvilinear reference frame can be written as either of the four expressions below depending on whether the first or the second vector is written in covariant or contravariant coordinates
\begin{equation}
\avec \cdot \bvec = a^i b_i = a_i b^i = g_{ij} a^i b^j = g^{ij} a_i b_j.
\end{equation}
The cross product can be computed as
\begin{align}
    \avec \times \bvec &= \epsilon_{ijk} \sqrt{g} a^i b^j \evec^k \\
    \avec \times \bvec &= \epsilon^{ijk} \frac{1}{\sqrt{g}} a_i b_j \evec_k,
\end{align}
where $g = \text{det}(g_{ij})$. One can also define $g^{-1} = \text{det}(g^{ij})$.

\section{Differentiation}
\subsection{The grad operator}
Consider the function $v = v(x^1, x^2, x^3)$ and the grad operator, which is  
\begin{equation}
    \nabla v = \frac{\partial v}{\partial x^1} \xvec_1 + \frac{\partial v}{\partial x^2} \xvec_2 + \frac{\partial v}{\partial x^3} \xvec_3.
\end{equation}
We now introduce the function $u = u(y^1,y^2,y^3)$ such that 
\begin{equation}
    v(x^1, x^2, x^3) = u(\yhat^1, \yhat^2, \yhat^3).
\end{equation}
Thus
\begin{equation}
    \frac{\partial v}{\partial x^1} 
    = \left( \frac{ \partial u}{\partial y^1} \right)_{\yvec = \hat{\yvec}} \frac{\partial \yhat^1}{\partial x^1} 
    + \left( \frac{ \partial u}{\partial y^2} \right)_{\yvec = \hat{\yvec}} \frac{\partial \yhat^2}{\partial x^1} 
    + \left( \frac{ \partial u}{\partial y^3} \right)_{\yvec = \hat{\yvec}} \frac{\partial \yhat^3}{\partial x^1}, 
\end{equation}
\begin{equation}
    \frac{\partial v}{\partial x^2} 
    = \left( \frac{ \partial u}{\partial y^1} \right)_{\yvec = \hat{\yvec}} \frac{\partial \yhat^1}{\partial x^2} 
    + \left( \frac{ \partial u}{\partial y^2} \right)_{\yvec = \hat{\yvec}} \frac{\partial \yhat^2}{\partial x^2} 
    + \left( \frac{ \partial u}{\partial y^3} \right)_{\yvec = \hat{\yvec}} \frac{\partial \yhat^3}{\partial x^2},
\end{equation}
\begin{equation}
    \frac{\partial v}{\partial x^3} 
    = \left( \frac{ \partial u}{\partial y^1} \right)_{\yvec = \hat{\yvec}} \frac{\partial \yhat^1}{\partial x^3} 
    + \left( \frac{ \partial u}{\partial y^2} \right)_{\yvec = \hat{\yvec}} \frac{\partial \yhat^2}{\partial x^3} 
    + \left( \frac{ \partial u}{\partial y^3} \right)_{\yvec = \hat{\yvec}} \frac{\partial \yhat^3}{\partial x^3} .
\end{equation}
Using the definition of $\evec^i$, the grad operator can be written as
\begin{equation}
    \nabla v = \left ( \frac{\partial u }{\partial y^1} \right )_{\yvec = \hat{\yvec}} \evec^1 + \left ( \frac{\partial u }{\partial y^2} \right )_{\yvec = \hat{\yvec}} \evec^2 + \left ( \frac{\partial u }{\partial y^3} \right )_{\yvec = \hat{\yvec}} \evec^3 .
\end{equation}
This shows the equivalence between the grad operator in cartesian coordinates and curvilinear coordinates.

\subsection{The divergence operator}

\subsection{The curl operator}
The curl is given by
\begin{equation}
    \nabla \times A = \epsilon^{ijk} \frac{1}{\sqrt{g}} \frac{\partial A_j}{\partial y^i} \evec_k
\end{equation}

\section{Integration}

\subsection{Volume integrals}
Consider the function $v = v(x^1, x^2, x^3)$ and it's volume integral, which is
\begin{equation}
    \int_{\Omega_x} v \, dx^1 dx^2 dx^3.
\end{equation}
We again introduce the function $u = u(y^1, y^2, y^3)$ such that
\begin{equation}
    v(x^1, x^2, x^3) = u(\yhat^1, \yhat^2, \yhat^3).
\end{equation}
If we evaluate the above at $\xvec = \hat{\xvec}$ and use \cref{eq:trans2}, we get
\begin{equation}
    \label{eq:integration_vol_integrand}
    v(\xhat^1, \xhat^2, \xhat^3) = u(y^1, y^2, y^3).
\end{equation}
Integration by substitution for $v$ can be written as
\begin{equation}
    \int_{\Omega_x} v \, dx^1 dx^2 dx^3 = \int_{\Omega_y} v(\xhat^1, \xhat^2, \xhat^3) J \, dy^1 dy^2 dy^2,
\end{equation}
where $J$ is the determinant of the Jacobian $J_{ij}$, which is given by $J_{ij} = \partial \hat{x}^i / \partial y^j$.
Using \cref{eq:integration_vol_integrand} in the above, gives
\begin{equation}
    \int_{\Omega_x} v \, dx^1 dx^2 dx^3 = \int_{\Omega_y} u J \, dy^1 dy^2 dy^2.
\end{equation}

\subsection{Surface integrals}
Define $d S_y$ as an infinitesimal surface in curvilinear coordinates, $\Gamma_y$ as a finite surface of integration in curvilinear coordinates that belongs to the $y^1 = \text{constant}$ surfaces, and $u$ as a function whose input is defined using curvilinear coordinates. Then, a surface integral can be computed using 
\begin{equation}
    \int_{\Gamma_y} u \, d S_y = \int_{\Gamma_y} u \, J | \nabla \hat{y}^1| dy^2 dy^3.
\end{equation}
Note that now we can write
\begin{align}
\label{eq:int_from_vol_surf}
    \int_{\Omega_y} u \,dV_y &= \int_{y^1_l}^{y^1_u} \int_{\Gamma_y} u \, J dy^1 dy^2 dy^3 \nonumber \\
    &= \int_{y^1_l}^{y^1_u} \int_{\Gamma_y} u \, \frac{J | \nabla \hat{y}^1| dy^2 dy^3}{| \nabla \hat{y}^1 |} dy^1 \nonumber \\
    &= \int_{y^1_l}^{y^1_u} \int_{\Gamma_y} u \, \frac{dS_y}{|\nabla \hat{y}^1|} dy^1.
\end{align}

\section{Cylindrical coordinates}

\end{document}