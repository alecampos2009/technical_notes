\documentclass[a4paper,11pt]{report}
\usepackage{fullpage}

\usepackage{"../../info/packages"}
\usepackage{"../../info/nomenclature"}

\title{Radiation Transport}
\author{Alejandro Campos}

\begin{document}
\maketitle
\tableofcontents

%########################################################################
\chapter{Introduction}
%########################################################################

%------------------------------------------------------------------------
\section{Definitions}
%------------------------------------------------------------------------
Consider an infinitesimal amount of energy $dE$ which is the energy at location $\xvec$ and time $t$ with frequencies in the infinitesimal range $d\nu$ about the frequency $\nu$ and flowing in the direction of the solid angle $d\Omegavec$ about the vector $\Omegavec$ and passing through an infinitesimal area $d\Avec$ with unit normal $\Avec$. We express this energy in terms of a distribution $I_\nu = I_\nu(\xvec, t, \nu, \Omegavec)$ as follows
\begin{equation}
    dE = I_\nu (\Omegavec \cdot \Avec) dt d\nu d\Omega dA.
\end{equation}
$I_\nu$ is referred to as the spectral radiance, or spectral specific intensity. Various additional radiation quantities can be defined in terms of $I_\nu$, as shown in \cref{tab:definitions}.

\setlength{\cellspacetoplimit}{3pt}
\setlength{\cellspacebottomlimit}{3pt}

\begin{table}[ht]
    \centering
    \begin{tabular} { | Sc | Sc | Sc |}
        \hline

        & \multicolumn{2}{c|}{spectral} \\

        \hline

        & definition & units \\ 

        \hline 
         
        \begin{tabular}{c} Spectral radiance / \\ Spectral specific intensity \end{tabular} 
        & $I_\nu $ 
        & $ \left [ \frac{\text{J}}{\text{s$\cdot$m\textsuperscript{2}$\cdot$sr$\cdot$Hz}} \right ] $ \\ 

        \hline

        0\textsuperscript{th} moment 
        & $ \displaystyle J_\nu = \frac{1}{4\pi} \int_{4\pi} I_\nu \, d\Omega $ 
        & $ \left [ \frac{\text{J}}{\text{s$\cdot$m\textsuperscript{2}$\cdot$Hz}} \right ]$ \\

        \hline

        1\textsuperscript{st} moment 
        & $ \begin{aligned} \Hvec_\nu &= \frac{1}{4\pi} \int_{4\pi} I_\nu \Omegavec \, d\Omega \\ &= \frac{\Fvec_\nu}{4\pi} \end{aligned}$ 
        & $ \left [ \frac{\text{J}}{\text{s$\cdot$m\textsuperscript{2}$\cdot$Hz}} \right ]$ \\

        \hline

        2\textsuperscript{nd} moment 
        & $ \begin{aligned} \Kvec_\nu &= \frac{1}{4\pi} \int_{4\pi} I_\nu \Omegavec \Omegavec \, d\Omega \\ &= \frac{c}{4\pi} \Pvec_\nu \end{aligned} $ 
        & $ \left [ \frac{\text{J}}{\text{s$\cdot$m\textsuperscript{2}$\cdot$Hz}} \right ]$ \\

        \hline
        
        \begin{tabular}{c} Spectral radiant \\ energy density \end{tabular} 
        & $ \begin{aligned} E_\nu &= \frac{1}{c} \int_{4 \pi} I_\nu \, d\Omega \\ &= \frac{4 \pi}{c} J_\nu \end{aligned} $ 
        & $ \left [ \frac{\text{J}}{\text{m\textsuperscript{3}$\cdot$Hz}} \right ]$ \\

        \hline
        
        \begin{tabular}{c} One-sided \\ spectral radiant \\ energy flux \end{tabular} 
        & $ \displaystyle S_\nu^{\Avec} = \int_{\Omegavec \cdot \Avec > 0} I_\nu (\Omegavec \cdot \Avec) \, d\Omega $ 
        & $ \left [ \frac{\text{J}}{\text{s$\cdot$m\textsuperscript{2}$\cdot$Hz}} \right ]$ \\

        \hline
    \end{tabular}
    \caption{Radiation quantities. In the above $\Fvec_\nu$ is the radiation flux and $\Pvec_\nu$ the radiation pressure tensor.}
    \label{tab:definitions}
\end{table}

Any quantity dependent on $\nu$ can be integrated over all frequencies to obtain a total value. For example, for the spectral radiance/spectral specific intensity, we have
\begin{equation}
    I = \int_0^\infty I_\nu \, d\nu.
\end{equation}
In the above, $I = I(\xvec, t, \Omegavec)$ is the radiance, or specific intensity. 

%########################################################################
\chapter{Thermal radiation}
%########################################################################
For blackbody radiation we have
\begin{equation}
    I_\nu = \frac{2h\nu^3}{c^2} \frac{1}{\exp(h\nu/kT) - 1}.
\end{equation}
Consider the identity 
\begin{equation}
    \int_0^\infty \frac{x^3}{\exp(yx) - 1} \, dx = \frac{1}{15} \left ( \frac{\pi}{y} \right )^4.
\end{equation}
Using the above to integrate over all frequencies, we get
\begin{equation}
    I = \frac{2h}{c^2} \frac{1}{15} \left ( \frac{ \pi kT}{h} \right )^4.
\end{equation}
Defining the Stefan-Boltzmann constant as
\begin{equation}
    \sigma = \frac{2 \pi^5 k^4}{15 c^2 h^3},
\end{equation}
we have
\begin{equation}
    I = \frac{1}{\pi} \sigma T^4.
\end{equation}
In this case $I_\nu$ is isotropic, that is, it is independent of the direction $\Omegavec$. Thus $J_\nu = I_\nu$ and therefore $E_\nu = (4\pi/c) I_\nu$. Integrating over all frequencies leads to $E = (4\pi/c) I$.

\begin{figure}[ht]
    \centering
    \includegraphics[width=5cm]{../../images/spherical_coords_wiki.png}
    \caption{Spherical coordinates from Wikipedia.}
    \label{fig:spherical_coordinates}
\end{figure}

For the one-sided spectral radiant energy flux, we make reference to the diagram for spherical coordinates in \cref{fig:spherical_coordinates}. Let's assume $\Avec = \zvec$ without loss of generality. Then, we have
\begin{equation}
    S_\nu^{\hat{\zvec}} = \int_{\phi = 0}^{2\pi} \int_{\theta=0}^{\pi/2} I_\nu \cos \theta \, d\Omega = I_\nu \int_{\phi = 0}^{2\pi} \int_{\theta=0}^{\pi/2} \cos \theta \sin \theta \, d\theta d\phi = \pi I_\nu.
\end{equation}
Similarly as before, integrating over all frequencies leads to $S^{\hat{\zvec}} = \pi I$. These and the other relations derived above are shown in \cref{tab:blackbody_quantities}. 

\begin{table}
    \centering
    \begin{tabular} { | Sc | Sc | Sc |}
        \hline
         & total & spectral \\
        \hline
         \begin{tabular}{c} Radiance / \\ Specific intensity \end{tabular} & $ \displaystyle I = \frac{1}{\pi} \sigma T^4 $ & $\displaystyle I_\nu = \frac{2h\nu^3}{c^2} \frac{1}{\exp(h\nu/kT) - 1} $  \\
        \hline
        \begin{tabular}{c} Radiant \\ energy density \end{tabular}  & $\displaystyle E = \frac{4}{c} \sigma T^4 $ & $ \displaystyle E_\nu = \frac{8 \pi h\nu^3}{c^3} \frac{1}{\exp(h\nu/kT) - 1} $ \\
        \hline
        \begin{tabular}{c} One-sided \\ radiant \\ energy flux \end{tabular}  & $\displaystyle S^{\hat{\zvec}} = \sigma T^4 $ & $ \displaystyle S_\nu^{\hat{\zvec}} = \frac{2 \pi h\nu^3}{c^2} \frac{1}{\exp(h\nu/kT) - 1} $ \\
        \hline
    \end{tabular}
    \caption{Radiation quantities for a blackbody spectrum}
    \label{tab:blackbody_quantities}
\end{table}

%########################################################################
\chapter{Opacities}
%########################################################################
There are multiple processes that control the behavior of photons:
\begin{itemize}
    \item Absorption 
    \begin{itemize}
        \item Bound-bound excitation (stimulated absorption)
        \item Bound-free ionization (photoionization)
        \item Free-free photo-absorption (inverse breemstrahlung)
    \end{itemize}
    \item Emission 
    \begin{itemize}
        \item Bound-bound de-excitation (stimulated emission, spontaneous emission)
        \item Free-bound recombination
        \item Free-free photo-emission (breemstrahlung)
    \end{itemize}
    \item Scattering
    \begin{itemize}
        \item \textbf{Rayleigh} scattering: elastic scattering of a photon from an atom or molecule whose size is less than that of the wavelength of the photon. 
        \item \textbf{Mie} scattering: same as Rayleigh scattering but for cases where the sizes of the atoms or molecules are comparable to the wavelength of the incoming photon.
        \item \textbf{Raman} scattering: inelastic scattering of a photon from a molecule. The interaction changes the molecule's vibrational, rotational, or electron energy.
        \item \textbf{Brillouin} scattering: inelastic scattering of a photon caused by its interaction with material waves in a medium (i.e. mass oscillation modes, charge displacement modes, magnetic spin oscillation modes). 
        \item \textbf{Compton} scattering: inelastic scattering of a photon from a free charged particle. 
        \item \textbf{Thomson} scattering: low-energy limit of Compton scattering. The photon energy and the particle's kinetic energy do not change as a result of the scattering. Can be explained with classical electrodynamics.
    \end{itemize}
    \item Reflection
    \item Transmission (nothing happens)
    \item Pair production/annihilation
\end{itemize}

Each of the above would alter the beam of photons passing through the material (except for transmission). The cross-sections of each process, which would depend on the incoming photon frequency $\nu$, can be added up to obtain a total cross section $\sigma = \sigma(\nu)$. From the definition of a cross section, $\sigma$ can be used to determine how many photons keep their course as they traverse through the material and how many do not. Consider the material under consideration to have the shape of a thick slab, which starts at $x=0$ and continues on for a definite length along $x>0$. We want to know how $I(x)$, the number of photons crossing the slab at any location $x$, decreases as we travel along the $x$ direction. Let's focus on an infinitesimal thin lamina within the slab, of width $dx$ and located at some arbitrary location $x$. The number of target particles in that lamina will be $n dx$, where $n$ is the number volume density of particles in the target material. Then, the number of incident photons after crossing the lamina would be
\begin{equation}
    I(x+dx) = I(x) -\sigma I(x) n dx.
\end{equation}
This leads to the ODE $dI(x)/dx = -\sigma I(x) n$, which has as solution
\begin{equation}
    I(x) = I(0) \exp(-\sigma n x).
\end{equation}
The attenuation coefficient is defined as $\sigma n$ [1/cm]. The mass attenuation coefficient, also referred to as opacity, is then given by $\kappa = \sigma n / \rho$ [cm\textsuperscript{2}/g]. $\Lambda = 1 / \sigma n$ is referred to as the attenuation length. 

\end{document}
