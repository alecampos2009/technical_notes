%###############################################################################
%
%
\part{Kinetic description}
%
%
%###############################################################################
%-------------------------------------------------------------------------------
\chapter{Governing equations}
%-------------------------------------------------------------------------------

We denote the distribution function for a species $\alpha$ as $f_\alpha = f_\alpha(\rvec, \vvec, t)$, where $\rvec$ and $\vvec$ are the sample space variables for position and velocity. Note that the distribution function is appropriately normalized such that
\begin{equation}
\int f_\alpha \, d\rvec d\vvec = N_\alpha,
\end{equation}
where $N_\alpha$ is the total number of particles corresponding to species $\alpha$. 

The dynamics of a plasma can be characterized by the Boltzmann evolution equation for the distribution along with Maxwell's equations
\begin{equation}
\label{eq:kinetic_equation}
\frac{\partial f_\alpha}{\partial t} + \vvec \cdot \nabla f_\alpha + \frac{Z_\alpha e}{m_\alpha} ( \Evec + \vvec \times \Bvec ) \cdot \nabla_v f_\alpha = C_\alpha + S_\alpha
\end{equation}
\begin{equation}
\label{eq:maxwell_3}
\nabla \cdot \Evec = \frac{\rho_e}{\epsilon_0} 
\end{equation}
\begin{equation}
\label{eq:maxwell_4}
\nabla \cdot \Bvec = 0
\end{equation}
\begin{equation}
\label{eq:maxwell_1}
\nabla \times \Evec = -\frac{ \partial \Bvec}{\partial t}
\end{equation}
\begin{equation}
\label{eq:maxwell_2}
\nabla \times \Bvec = \mu_0 \Jvec + \mu_0 \epsilon_0 \frac{\partial \Evec}{\partial t}
\end{equation}
\begin{equation}
\Jvec = \sum_\alpha Z_\alpha e \int \vvec f_\alpha \, d\vvec
\end{equation}
\begin{equation}
\rho_e = \sum_\alpha Z_\alpha e \int f_\alpha \, d\vvec.
\end{equation}
In the above, 
\begin{itemize}
\item $m_\alpha$ is the species mass
\item $e$ is the charge
\item $Z_\alpha$ the charge number
\item $\Jvec = \Jvec(\rvec,t)$ the charge current
\item $\rho_e = \rho_e(\rvec,t)$ the charge density
\item $\Evec = \Evec(\rvec,t)$ the electric field
\item $\Bvec = \Bvec(\rvec,t)$ the magnetic field.
\end{itemize}
The terms $C_\alpha$ and $S_\alpha$ represent collision and source terms.

If we express the collision term in the usual way, that is $C_\alpha = \sum_\beta C_{\alpha \beta}$, then we can make the following statements:
\begin{enumerate}
\item Conservation of particles:
\begin{equation}
\int C_{\alpha \alpha} \, d\vvec = 0 \quad \forall \alpha \qquad \qquad
\int C_{\alpha \beta} \,d\vvec = 0 \quad \forall \alpha, \beta | \beta \ne \alpha.
\end{equation}

\item Conservation of momentum:
\begin{equation}
\int m_\alpha \vvec C_{\alpha \alpha} \, d\vvec = 0 \quad \forall \alpha \qquad \qquad \sum_\alpha \sum_{\beta, \beta \ne \alpha} \int m_\alpha \vvec C_{\alpha \beta} d\vvec = 0.
\end{equation}

\item Conservation of energy:
\begin{equation}
\int \frac{1}{2} m_\alpha v^2 C_{\alpha \alpha} \, d\vvec = 0 \quad \forall \alpha \qquad \qquad \sum_\alpha \sum_{\beta, \beta \ne \alpha} \int \frac{1}{2} m_\alpha v^2 C_{\alpha \beta} \, d\vvec = 0.
\end{equation}

\end{enumerate}

%-------------------------------------------------------------------------------
\section{Fluid equations}
%-------------------------------------------------------------------------------
We now define the particle density $n_\alpha = n_\alpha(\rvec,t)$, the fluid velocity $\uvec_\alpha = \uvec_\alpha(\rvec,t)$ and the fluid energy per unit mass $E_\alpha = E_\alpha(\rvec,t)$ as follows
\begin{align}
n_\alpha &= \int f_\alpha \, d\vvec \\
\uvec_\alpha &= \frac{1}{n_\alpha} \int \vvec f_\alpha \, d\vvec \\
E_\alpha &= \frac{1}{n_\alpha} \int \frac{1}{2} v^2 f_\alpha \, d\vvec.
\end{align}
Their evolution equations are obtained by taking the appropriate moments of the Boltzmann plasma equation. Before doing so, we re-write the Boltzmann equation as
\begin{equation}
\label{eq:boltz_mod}
\frac{\partial f_\alpha}{\partial t} + \nabla \cdot (\vvec f_\alpha) + \nabla_v \cdot \left [ \frac{Z_\alpha e}{m_\alpha} ( \Evec + \vvec \times \Bvec ) f_\alpha \right ] = C_\alpha + S_\alpha
\end{equation}

%--------------------------------------------
\subsection{Mass}
%--------------------------------------------
Integrating \cref{eq:boltz_mod} over all $\vvec$ we obtain
%\begin{empheq}[box=\widefbox]{equation}
\begin{equation}
\frac{\partial n_\alpha}{\partial t} + \nabla \cdot \left ( n_\alpha \uvec_\alpha \right ) = \hat{S}_\alpha
\end{equation}
%\end{empheq}
where 
\begin{equation}
\hat{S}_\alpha = \int S_\alpha \, d\vvec
\end{equation}
is an external source of mass.

%--------------------------------------------
\subsection{Momentum}
%--------------------------------------------
Multiplying \cref{eq:boltz_mod} by $\vvec$ and then integrating over all $\vvec$ leads to
\begin{multline}
\label{eq:mom_derv_1}
\frac{\partial n_\alpha \uvec_\alpha}{\partial t} + \nabla \cdot \left ( \int \vvec \vvec f_\alpha \, d\vvec \right) + \int \nabla_v \cdot \left [ \vvec \frac{Z_\alpha e}{m_\alpha} ( \Evec + \vvec \times \Bvec ) f_\alpha \right ] - \nabla_v \vvec \cdot \left [ \frac{Z_\alpha e}{m_\alpha} ( \Evec + \vvec \times \Bvec ) f_\alpha \right ] \, d\vvec =\\
\sum_{\beta, \beta \ne \alpha} \int \vvec C_{\alpha \beta} \, d\vvec + \int \vvec S_\alpha \, d\vvec.
\end{multline}
We note that the third term in \cref{eq:mom_derv_1} is zero since we are integrating over all space, and that $\nabla_v \vvec$ is the identity matrix. We thus have
\begin{multline}
\label{eq:mom_derv_2}
\frac{\partial n_\alpha \uvec_\alpha}{\partial t} + \nabla \cdot \left ( \int \vvec \vvec f_\alpha \, d\vvec \right) - \frac{Z_\alpha e n_\alpha}{m_\alpha} ( \Evec + \uvec_\alpha \times \Bvec ) =\\
\sum_{\beta, \beta \ne \alpha} \int \vvec C_{\alpha \beta} \, d\vvec + \int \vvec S_\alpha \, d\vvec.
\end{multline}

To proceed, we decompose $\vvec$ into a mean and a fluctuation, that is, $\vvec = \uvec_\alpha + \wvec_\alpha$. Using this decomposition 
\begin{equation}
\label{eq:identity_vv}
\int \vvec \vvec f_\alpha \, d\vvec = \int ( \uvec_\alpha \uvec_\alpha + 2 \uvec_\alpha \wvec_\alpha + \wvec_\alpha \wvec_\alpha) f_\alpha \, d\vvec = n_\alpha \uvec_\alpha \uvec_\alpha + \int \wvec_\alpha \wvec_\alpha f_\alpha \, d\vvec.
\end{equation}
Thus, \cref{eq:mom_derv_2} becomes
\begin{multline}
\label{eq:mom_derv_3}
\frac{\partial n_\alpha \uvec_\alpha}{\partial t} + \nabla \cdot \left ( n_\alpha \uvec_\alpha \uvec_\alpha \right) - \frac{Z_\alpha e n_\alpha}{m_\alpha} ( \Evec + \uvec_\alpha \times \Bvec ) = -\nabla \cdot \int \wvec_\alpha \wvec_\alpha f_\alpha d\vvec + \\
\sum_{\beta, \beta \ne \alpha} \int \vvec C_{\alpha \beta} \, d\vvec + \int \vvec S_\alpha \, d\vvec.
\end{multline}
Conservation of particles is used to modify the collisional term to thus obtain
\begin{multline}
\label{eq:mom_derv_4}
\frac{\partial n_\alpha \uvec_\alpha}{\partial t} + \nabla \cdot \left ( n_\alpha \uvec_\alpha \uvec_\alpha \right) - \frac{Z_\alpha e n_\alpha}{m_\alpha} ( \Evec + \uvec_\alpha \times \Bvec ) = -\nabla \cdot \int \wvec_\alpha \wvec_\alpha f_\alpha d\vvec + \\
\sum_{\beta, \beta \ne \alpha} \int \wvec_\alpha C_{\alpha \beta} \, d\vvec + \int \vvec S_\alpha \, d\vvec.
\end{multline}

Multiplying by mass leads to the following equation
%\begin{empheq}[box=\widefbox]{equation}
\begin{equation}
\label{eq:mom_derv_5}
\frac{\partial m_\alpha n_\alpha \uvec_\alpha}{\partial t} + \nabla \cdot \left ( m_\alpha n_\alpha \uvec_\alpha \uvec_\alpha \right) - Z_\alpha e n_\alpha ( \Evec + \uvec_\alpha \times \Bvec ) = \nabla \cdot \boldsymbol{\sigma}_\alpha + \Rvec_\alpha + \hat{\Mvec}_\alpha,
\end{equation}
%\end{empheq}
where the stress tensor is
\begin{equation}
\boldsymbol{\sigma}_\alpha = -\int m_\alpha \wvec_\alpha \wvec_\alpha f_\alpha \, d\vvec,
\end{equation}
the momentum transferred between unlike particles due to friction of collisions is
\begin{equation}
\Rvec_\alpha = \sum_{\beta, \beta \ne \alpha} \int m_\alpha \wvec_\alpha C_{\alpha \beta} \, d\vvec,
\end{equation}
and the external source of momentum is
\begin{equation}
\hat{\Mvec}_\alpha = \int m_\alpha \vvec S_\alpha \, d\vvec.
\end{equation}
 

The stress tensor is typically decomposed into isotropic $p_\alpha$ and anisotropic (shear) $\tvec_\alpha$ tensors as follows
\begin{equation}
\boldsymbol{\sigma}_\alpha = - p_\alpha \Ivec + \tvec_\alpha,
\end{equation}
where $P_\alpha$ is given by
\begin{equation}
p_\alpha = \frac{1}{3} \int m_\alpha (\wvec_\alpha \cdot \wvec_\alpha) f_\alpha d\vvec.
\end{equation}
Thus, conservation of momentum becomes
\begin{equation}
\label{eq:mom_derv_6}
\frac{\partial m_\alpha n_\alpha \uvec_\alpha}{\partial t} + \nabla \cdot \left ( m_\alpha n_\alpha \uvec_\alpha \uvec_\alpha \right) - Z_\alpha e n_\alpha ( \Evec + \uvec_\alpha \times \Bvec ) = - \nabla p_\alpha + \nabla \cdot \tvec_\alpha + \Rvec_\alpha + \hat{\Mvec}_\alpha.
\end{equation}

%--------------------------------------------
\subsection{Energy}
%--------------------------------------------
Multiplying \cref{eq:boltz_mod} by $\frac{1}{2} v^2$ and then integrating over all $\vvec$ leads to
\begin{multline}
\label{eq:energy_derv_1}
\frac{\partial n_\alpha E_\alpha}{\partial t} + \nabla \cdot \left [ \int \frac{1}{2} (\vvec \cdot \vvec) \vvec f_\alpha \, d\vvec \right ] + \int \nabla_v \cdot \left [ \frac{1}{2} (\vvec \cdot \vvec) \frac{Z_\alpha e}{m_\alpha} ( \Evec + \vvec \times \Bvec ) f_\alpha \right ] \\
- \nabla_v \left [ \frac{1}{2} (\vvec \cdot \vvec) \right ] \cdot \left [ \frac{Z_\alpha e}{m_\alpha} ( \Evec + \vvec \times \Bvec ) f_\alpha \right ] \, d\vvec = \sum_{\beta, \beta \ne \alpha} \int \frac{1}{2} (\vvec \cdot \vvec) C_{\alpha \beta} \, d\vvec + \int \frac{1}{2} (\vvec \cdot \vvec) S_\alpha \, d\vvec.
\end{multline}
We note that the third term above is zero since we are integrating over all space, and that $\nabla_v [ 1/2 (\vvec \cdot \vvec ) ] = \vvec$. Thus, we have
\begin{multline}
\label{eq:energy_derv_2}
\frac{\partial n_\alpha E_\alpha}{\partial t} + \nabla \cdot \left [ \int \frac{1}{2} (\vvec \cdot \vvec) \vvec f_\alpha \, d\vvec \right ] - \frac{Z_\alpha e n_\alpha}{m_\alpha} \Evec \cdot \uvec_\alpha =\\
\sum_{\beta, \beta \ne \alpha} \int \frac{1}{2} (\vvec \cdot \vvec) C_{\alpha \beta} \, d\vvec + \int \frac{1}{2} (\vvec \cdot \vvec) S_\alpha \, d\vvec.
\end{multline}

To proceed with the derivation we first note that
\begin{multline}
\int \frac{1}{2} (\vvec \cdot \vvec) \vvec f_\alpha \, d\vvec = \int \frac{1}{2} (\vvec \cdot \vvec) (\uvec_\alpha + \wvec_\alpha) f_\alpha \, d\vvec = n_\alpha E_\alpha \uvec_\alpha + \int \frac{1}{2} (\vvec \cdot \vvec ) \wvec_\alpha f_\alpha \, d\vvec
\end{multline}
The last term on the right-hand side above can be re-written as
\begin{align}
\int \frac{1}{2} ( \vvec \cdot \vvec ) \wvec_\alpha f_\alpha \, d\vvec &= \int \frac{1}{2} ( \uvec_\alpha \cdot \uvec_\alpha + 2\uvec_\alpha \cdot \wvec_\alpha + \wvec_\alpha \cdot \wvec_\alpha ) \wvec_\alpha f_\alpha \, d\vvec \\
& =  \uvec_\alpha \cdot \int \wvec_\alpha \wvec_\alpha f_\alpha \, d\vvec + \int \frac{1}{2} ( \wvec_\alpha \cdot \wvec_\alpha ) \wvec_\alpha f_\alpha \, d\vvec.
\end{align}
Using the expressions above, \cref{eq:energy_derv_2} becomes
\begin{multline}
\label{eq:energy_derv_3}
\frac{\partial n_\alpha E_\alpha}{\partial t} + \nabla \cdot (n_\alpha E_\alpha \uvec_\alpha ) - \frac{Z_\alpha e n_\alpha}{m_\alpha} \Evec \cdot \uvec_\alpha =  - \nabla \cdot \left ( \uvec_\alpha \cdot \int \wvec_\alpha \wvec_\alpha f_\alpha \, d\vvec \right ) - \nabla \cdot \int \frac{1}{2} ( \wvec_\alpha \cdot \wvec_\alpha ) \wvec_\alpha f_\alpha \, d\vvec\\
+ \sum_{\beta, \beta \ne \alpha} \int \frac{1}{2} (\vvec \cdot \vvec) C_{\alpha \beta} \, d\vvec + \int \frac{1}{2} (\vvec \cdot \vvec) S_\alpha \, d\vvec.
\end{multline}
Conservation of particles is used to modify the collisional term to thus obtain
\begin{multline}
\label{eq:energy_derv_4}
\frac{\partial n_\alpha E_\alpha}{\partial t} + \nabla \cdot (n_\alpha E_\alpha \uvec_\alpha ) - \frac{Z_\alpha e n_\alpha}{m_\alpha} \Evec \cdot \uvec_\alpha = - \nabla \cdot \left ( \uvec_\alpha \cdot \int \wvec_\alpha \wvec_\alpha f_\alpha \, d\vvec  \right ) - \nabla \cdot \int \frac{1}{2} ( \wvec_\alpha \cdot \wvec_\alpha ) \wvec_\alpha f_\alpha \, d\vvec \\
+ \uvec_\alpha \cdot \sum_{\beta,\beta \ne \alpha} \int \wvec_\alpha C_{\alpha \beta} \, d\vvec + \sum_{\beta, \beta \ne \alpha} \int \frac{1}{2} (\wvec_\alpha \cdot \wvec_\alpha) C_{\alpha \beta} \, d\vvec + \int \frac{1}{2} (\vvec \cdot \vvec) S_\alpha \, d\vvec.
\end{multline}

Multiplying by mass leads to the following equation
%\begin{empheq}[box=\widefbox]{multline}
\begin{multline}
\label{eq:energy_derv_5}
\frac{\partial m_\alpha n_\alpha E_\alpha}{\partial t} + \nabla \cdot (m_\alpha n_\alpha E_\alpha \uvec_\alpha ) - Z_\alpha e n_\alpha \Evec \cdot \uvec_\alpha = \nabla \cdot ( \uvec_\alpha \cdot \boldsymbol{\sigma}_\alpha ) - \nabla \cdot \qvec_\alpha \\
+ \uvec_\alpha \cdot \Rvec_\alpha + Q_{\alpha} + \hat{Q}_\alpha, 
\end{multline}
%\end{empheq}
where heat flux due to random motion is
\begin{equation}
\qvec_\alpha = \int \frac{1}{2} m_\alpha (\wvec_\alpha \cdot \wvec_\alpha ) \wvec_\alpha f_\alpha \, d\vvec,
\end{equation}
the heat generated and transferred between unlike particles due to collisional dissipation is 
\begin{equation}
Q_{\alpha} = \sum_{\beta,\beta \ne \alpha} \int \frac{1}{2} m_\alpha (\wvec_\alpha \cdot \wvec_\alpha) C_{\alpha \beta} \, d\vvec,
\end{equation}
and the external source of energy is
\begin{equation}
\hat{Q}_\alpha = \int \frac{1}{2} m_\alpha (\vvec \cdot \vvec) S_\alpha \, d\vvec.
\end{equation}

Using the decomposition for the stress tensor, the conservation of energy equation becomes
\begin{multline}
\label{eq:energy_derv_6}
\frac{\partial m_\alpha n_\alpha E_\alpha}{\partial t} + \nabla \cdot (m_\alpha n_\alpha E_\alpha \uvec_\alpha + p_\alpha \uvec_\alpha ) - Z_\alpha e n_\alpha \Evec \cdot \uvec_\alpha = \nabla \cdot ( \uvec_\alpha \cdot \tvec_\alpha ) - \nabla \cdot \qvec_\alpha \\
+ \uvec_\alpha \cdot \Rvec_\alpha + Q_{\alpha} + \hat{Q}_\alpha, 
\end{multline}
We also note that the energy $m_\alpha n_\alpha E_\alpha$ can be decomposed into internal and kinetic energies. Using the trace of the decomposition shown in \cref{eq:identity_vv} one obtains
\begin{align}
m_\alpha n_\alpha E_\alpha & = \int \frac{1}{2} m_\alpha (\vvec \cdot \vvec) f_\alpha \, d\vvec \nonumber \\
& = \int \frac{1}{2} m_\alpha (\wvec_\alpha \cdot \wvec_\alpha) f_\alpha \, d\vvec + \frac{1}{2} m_\alpha n_\alpha (\uvec_\alpha \cdot \uvec_\alpha) \nonumber \\
& = \frac{3}{2} P_\alpha + \frac{1}{2} m_\alpha n_\alpha (\uvec_\alpha \cdot \uvec_\alpha) \nonumber \\
& = \frac{3}{2} P_\alpha + m_\alpha n_\alpha K_\alpha.
\end{align}
where $K_\alpha = \frac{1}{2} \uvec_\alpha \cdot \uvec_\alpha$ is the kinetic energy of species $\alpha$.

%--------------------------------------------
\subsection{Kinetic and Internal Energies}
%--------------------------------------------
The equation for the kinetic energy is obtained by dotting \cref{eq:mom_derv_6} with $\uvec_\alpha$. For this, we first show that
\begin{align}
    \uvec_\alpha \cdot &\left [ \frac{\partial m_\alpha n_\alpha \uvec_\alpha}{\partial t} + \nabla \cdot \left ( m_\alpha n_\alpha \uvec_\alpha \uvec_\alpha \right) \right ] \\
    & =\uvec_\alpha \cdot \left \{ \left [ \frac{\partial m_\alpha n_\alpha }{\partial t} + \nabla \cdot \left ( m_\alpha n_\alpha \uvec_\alpha \right) \right] \uvec_\alpha  + m_\alpha n_\alpha \left ( \frac{\partial \uvec_\alpha}{\partial t} + \uvec_\alpha \cdot \nabla \uvec_\alpha \right ) \right \}\\
    & =\uvec_\alpha \cdot \left [ m_\alpha \hat{S}_\alpha \uvec_\alpha  + m_\alpha n_\alpha \left ( \frac{\partial \uvec_\alpha}{\partial t} + \uvec_\alpha \cdot \nabla \uvec_\alpha \right ) \right ] \\
    & =2m_\alpha \hat{S}_\alpha K_\alpha + m_\alpha n_\alpha \left ( \frac{\partial K_\alpha}{\partial t} + \uvec_\alpha \cdot \nabla K_\alpha \right) \\
    & =m_\alpha \hat{S}_\alpha K_\alpha + \left [ \frac{\partial m_\alpha n_\alpha}{\partial t} + \nabla \cdot \left (m_\alpha n_\alpha \uvec_\alpha \right ) \right] K_\alpha + m_\alpha n_\alpha \left ( \frac{\partial K_\alpha}{\partial t} + \uvec_\alpha \cdot \nabla K_\alpha \right) \\
    & =m_\alpha \hat{S}_\alpha K_\alpha + \frac{\partial m_\alpha n_\alpha K_\alpha}{\partial t} + \nabla \cdot ( m_\alpha n_\alpha K \uvec_\alpha ).
\end{align}
Thus, the equation for the turbulent kinetic energy is
\begin{multline}
\frac{\partial m_\alpha n_\alpha K_\alpha}{\partial t} + \nabla \cdot ( m_\alpha n_\alpha K \uvec_\alpha ) - Z_\alpha e n_\alpha \Evec \cdot \uvec_\alpha =\\
-\nabla \cdot ( \uvec_\alpha p_\alpha ) + \nabla \cdot (\uvec_\alpha \cdot \tvec_\alpha ) + p_\alpha \nabla \cdot \uvec_\alpha - \tvec_\alpha : \nabla \uvec_\alpha + \uvec_\alpha \cdot \Rvec_\alpha + \uvec_\alpha \cdot \hat{\Mvec}_\alpha - m_\alpha K_\alpha \hat{S}_\alpha .
\end{multline}
Subtracting the above equation from \cref{eq:energy_derv_6} leads to 
\begin{equation}
\frac{\partial}{\partial t} \left ( \frac{3}{2} p_\alpha \right ) + \nabla \cdot \left ( \frac{3}{2} p_\alpha \uvec_\alpha \right ) = -p_\alpha \nabla \cdot \uvec_\alpha + \tvec_\alpha : \nabla \uvec_\alpha - \nabla \cdot \qvec_\alpha + Q_{\alpha} + \hat{Q}_\alpha - \uvec_\alpha \cdot \hat{\Mvec}_\alpha + m_\alpha K_\alpha \hat{S}_\alpha .
\end{equation}

%--------------------------------------------
\subsection{Summary}
%--------------------------------------------
To summarize, we have,
\begin{itemize}
    \item Particle density
\begin{equation}
\label{eq:cons_mass}
    \frac{\partial n_\alpha}{\partial t} + \nabla \cdot \left ( n_\alpha \uvec_\alpha \right ) = \hat{S}_\alpha,
\end{equation}

    \item Momentum
\begin{equation}
\label{eq:cons_mom}
    \frac{\partial m_\alpha n_\alpha \uvec_\alpha}{\partial t} + \nabla \cdot \left ( m_\alpha n_\alpha \uvec_\alpha \uvec_\alpha \right) - Z_\alpha e n_\alpha ( \Evec + \uvec_\alpha \times \Bvec ) = - \nabla p_\alpha + \nabla \cdot \tvec_\alpha + \Rvec_\alpha + \hat{\Mvec}_\alpha,
\end{equation}

    \item Total Energy
\begin{multline}
\label{eq:cons_te}
\frac{\partial m_\alpha n_\alpha E_\alpha}{\partial t} + \nabla \cdot (m_\alpha n_\alpha E_\alpha \uvec_\alpha + p_\alpha \uvec_\alpha ) - Z_\alpha e n_\alpha \Evec \cdot \uvec_\alpha = \nabla \cdot ( \uvec_\alpha \cdot \tvec_\alpha ) - \nabla \cdot \qvec_\alpha \\
+ \uvec_\alpha \cdot \Rvec_\alpha + Q_{\alpha} + \hat{Q}_\alpha, 
\end{multline}
    
    \item Kinetic Energy
\begin{multline}
\label{eq:cons_ke}
\frac{\partial m_\alpha n_\alpha K_\alpha}{\partial t} + \nabla \cdot ( m_\alpha n_\alpha K \uvec_\alpha ) - Z_\alpha e n_\alpha \Evec \cdot \uvec_\alpha =\\
-\nabla \cdot ( \uvec_\alpha p_\alpha ) + \nabla \cdot (\uvec_\alpha \cdot \tvec_\alpha ) + p_\alpha \nabla \cdot \uvec_\alpha - \tvec_\alpha : \nabla \uvec_\alpha + \uvec_\alpha \cdot \Rvec_\alpha + \uvec_\alpha \cdot \hat{\Mvec}_\alpha - m_\alpha K_\alpha \hat{S}_\alpha .
\end{multline}    
    
    \item Internal Energy
\begin{equation}
\label{eq:cons_ie}
    \frac{\partial}{\partial t} \left ( \frac{3}{2} p_\alpha \right ) + \nabla \cdot \left ( \frac{3}{2} p_\alpha \uvec_\alpha \right ) = -p_\alpha \nabla \cdot \uvec_\alpha + \tvec_\alpha : \nabla \uvec_\alpha - \nabla \cdot \qvec_\alpha + Q_{\alpha} + \hat{Q}_\alpha - \uvec_\alpha \cdot \hat{\Mvec}_\alpha + m_\alpha K_\alpha \hat{S}_\alpha .  
\end{equation}

\end{itemize}

%-------------------------------------------------------------------------------
\chapter{Transport coefficients}
%-------------------------------------------------------------------------------
Collison integral

\begin{equation}
    \Omega_{\alpha \beta}^{(lk)} = \sqrt{ \frac{k_B T}{2 \pi M_{\alpha \beta}} } \int_0^\infty e^{-g^2} g^{2k+3} \phi_{\alpha \beta}^{(l)} \, dg.
\end{equation}
In the above $M_{\alpha \beta}$ is the reduced mass, given by
\begin{equation}
    M_{\alpha \beta} = \frac{M_\alpha M_\beta}{M_\alpha + M_\beta},
\end{equation}
and $\phi^{(l)}_{\alpha \beta}$ is the collision cross section for a given velocity, and is computed as
\begin{equation}
    \phi_{\alpha \beta}^{(l)} = 2 \pi \int_0^\infty \left ( 1 - \cos^l \chi_{\alpha \beta} \right ) b \, db.
\end{equation}
The scattering angle $\chi_{\alpha \beta}$ is given by
\begin{equation}
    \chi_{\alpha \beta} = \pi - 2 \int_{r_{\alpha \beta}^{\text{min}}}^\infty \frac{b}{r^2 \left [ 1 - \frac{b^2}{r^2} - \frac{V_{\alpha \beta (r)}}{g^2 k_B T} \right ]^{1/2} } \, dr.
\end{equation}

For a Coulombic interaction between ions, we can define the natural scale fore the cross-sectional area as
\begin{equation}
    \phi^{(0)}_{\alpha \beta} = \frac{ \pi \left (Z_\alpha Z_\beta e^2 \right)^2}{ \left(2 k_B T\right)^2}.
\end{equation}
Given this definition, we express the collision integral as
\begin{equation}
    \Omega_{\alpha \beta} = \sqrt{ \frac{\pi }{M_{\alpha \beta}}} \frac{( Z_\alpha Z_\beta e^2)^2 }{(2 k_B T )^{3/2}} \mathcal{F}^{lk}_{\alpha \beta},
\end{equation}
where
\begin{equation}
    \mathcal{F}^{(lk)}_{\alpha \beta} = \frac{1}{2 \phi_0} \int_0^\infty e^{-g^2} g^{2k+3} \phi_{\alpha \beta}^{(l)} \, dg
\end{equation}
We note that $\mathcal{F}^{(lk)}_{\alpha \beta} = 4 \mathcal{K}_{lk}(g_{\alpha \beta})$, where $\mathcal{K}_{lk}(g_{\alpha \beta})$ is the notation from the Stanton-Murillo paper.

%%-------------------------------------------------------------------------------
%\section{Chapman-Enskog Method}
%%-------------------------------------------------------------------------------
%Equations \eqref{eq:cons_mass}, \eqref{eq:mom_derv_5}, and \eqref{eq:energy_derv_5} have many unclosed terms. To %simplify matters a bit, we'll look at the case where there is only one species, there is no external source %$S_\alpha$, and there is no external electromagnetic force. In that case, the fluid equations look as follows
%\begin{equation}
%\label{eq:cons_mass_simple}
%\frac{\partial n}{\partial t} + \nabla \cdot \left ( n \uvec \right ) = 0,
%\end{equation}
%\begin{equation}
%\label{eq:cons_mom_simple}
%\frac{\partial m n \uvec}{\partial t} + \nabla \cdot \left ( m n \uvec \uvec \right) = \nabla \cdot %\boldsymbol{\sigma},
%\end{equation}
%\begin{equation}
%\label{eq:cons_ene_simple}
%\frac{\partial m n E}{\partial t} + \nabla \cdot (m n E \uvec ) = - \nabla \cdot \qvec + \nabla \cdot ( %\boldsymbol{\sigma} \cdot \uvec ).
%\end{equation}
%For the above, the unclosed terms are $\sigma$ and $\qvec$. An approach to obtain approximations to such values is %called the Chapman-Enskog method.

%The kinetic equation for the distribution function, in this case where there is one species and no external sources %and forces, is given by
%\begin{equation}
%    \frac{\partial f}{\partial t} + \vvec \cdot \nabla f = C.
%\end{equation}

