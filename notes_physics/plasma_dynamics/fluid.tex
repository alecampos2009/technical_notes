%###############################################################################
%
%
\part{Fluid description}
%
%
%###############################################################################

%-------------------------------------------------------------------------------
\chapter{Magnetohydrodynamics}
%-------------------------------------------------------------------------------
%--------------------------------------------
\section{Two-fluid equations}
%--------------------------------------------
\label{sec:two_fluid_equations}
The starting point are the multi-fluid conservation laws \cref{eq:cons_mass,eq:cons_mom,eq:cons_ie} and the Maxwell equations \cref{eq:maxwell_1,eq:maxwell_2,eq:maxwell_3,eq:maxwell_4}. We assume there are two species: electrons and ions. Additionally, we assume no sources. Thus, the starting governing equations are
\begin{equation}
\label{eq:twof_ni}
    \frac{\partial n_i}{\partial t} + \nabla \cdot \left ( n_i \uvec_i \right ) = 0,
\end{equation}
\begin{equation}
\label{eq:twof_ne}
    \frac{\partial n_e}{\partial t} + \nabla \cdot \left ( n_e \uvec_e \right ) = 0,
\end{equation}

\begin{equation}
\label{eq:twof_ui}
    \frac{\partial m_i n_i \uvec_i}{\partial t} + \nabla \cdot \left ( m_i n_i \uvec_i \uvec_i \right) - Z e n_i ( \Evec + \uvec_i \times \Bvec ) = - \nabla p_i + \nabla \cdot \tvec_i + \Rvec_i,
\end{equation}
\begin{equation}
\label{eq:twof_ue}
    \frac{\partial m_e n_e \uvec_e}{\partial t} + \nabla \cdot \left ( m_e n_e \uvec_e \uvec_e \right) + e n_e ( \Evec + \uvec_e \times \Bvec ) = - \nabla p_e + \nabla \cdot \tvec_e + \Rvec_e,
\end{equation}

\begin{equation}
\label{eq:twof_iei}
    \frac{\partial}{\partial t} \left ( \frac{3}{2} p_i \right ) + \nabla \cdot \left ( \frac{3}{2} p_i \uvec_i \right ) = -p_i \nabla \cdot \uvec_i + \tvec_i : \nabla \uvec_i - \nabla \cdot \qvec_i + Q_i,
\end{equation}
\begin{equation}
\label{eq:twof_iee}
    \frac{\partial}{\partial t} \left ( \frac{3}{2} p_e \right ) + \nabla \cdot \left ( \frac{3}{2} p_e \uvec_e \right ) = -p_e \nabla \cdot \uvec_e + \tvec_e : \nabla \uvec_e - \nabla \cdot \qvec_e + Q_e,
\end{equation}

\begin{equation}
\label{eq:twof_maxwell_3}
\nabla \cdot \Evec = \frac{\rho_e}{\epsilon_0} 
\end{equation}
\begin{equation}
\label{eq:twof_maxwell_4}
\nabla \cdot \Bvec = 0.
\end{equation}
\begin{equation}
\label{eq:twof_maxwell_1}
\nabla \times \Evec = -\frac{ \partial \Bvec}{\partial t}
\end{equation}
\begin{equation}
\label{eq:twof_maxwell_2}
\nabla \times \Bvec = \mu_0 \Jvec + \mu_0 \epsilon_0 \frac{\partial \Evec}{\partial t}
\end{equation}
\begin{equation}
\label{eq:twof_curr_density}
    \Jvec = e (Z n_i \uvec_i - n_e \uvec_e)
\end{equation}
\begin{equation}
\label{eq:twof_mass_density}
    \rho_e = e (Z n_i - n_e) 
\end{equation}
These equations correspond to eq.\@ (2.22) in Freidberg's Ideal MHD book, but for ions that are not singly charged. For the sections below, however, we'll assume singly-charged ions.

%--------------------------------------------
\section{Low-frequency, long-wavelength, asymptotic expansions}
%--------------------------------------------
Two assumptions:
\begin{enumerate}
\item Transform full Maxwell's equations to low-frequency pre-Maxwell's equations. Formally achieved with $\epsilon_0 \to 0$. This has two consequences:
\begin{itemize}
\item $\epsilon_0 \partial \Evec / \partial t \to 0$ \\
For this to be achieved it is required that $w/k \ll c$ and $V_{Ti}, V_{Te} \ll c$.
\item $\epsilon_0 \nabla \cdot \Evec \to 0$\\
For this to be achieved it is required that $w \ll w_{pe}$ and $a \gg \lambda_{D}$.
\end{itemize}
\item Neglect electron inertia in the electron momentum equations. Formally achieved with $m_e \to 0$.
\end{enumerate}

Due to the first assumption, the Maxwell equations \cref{eq:twof_maxwell_1,eq:twof_maxwell_2,eq:twof_maxwell_3,eq:twof_maxwell_4,eq:twof_curr_density,eq:twof_mass_density} are now written as
\begin{equation}
\label{eq:preMax_gauss}
n_i - n_e = 0
\end{equation}
\begin{equation}
\label{eq:preMax_Bdiv}
\nabla \cdot \Bvec = 0.
\end{equation}
\begin{equation}
\label{eq:preMax_faradays}
\nabla \times \Evec = -\frac{ \partial \Bvec}{\partial t}
\end{equation}
\begin{equation}
\label{eq:preMax_amperes}
\nabla \times \Bvec = \mu_0 \Jvec 
\end{equation}

%--------------------------------------------
\section{Single-fluid equations}
%--------------------------------------------
We define single-fluid variables as
\begin{equation}
    \rho = m_i n_i + m_e n_e = m_i n
\end{equation}
\begin{equation}
    \vvec = \frac{m_i n_i \uvec_i + m_e n_e \uvec_e}{m_i n_i + m_e n_e} = \uvec_i
\end{equation}
\begin{equation}
    p = p_i + p_e = n (T_i + T_e)
\end{equation}
\begin{equation}
    T = \frac{T_i + T_e}{2}.
\end{equation}

The two conservation of mass equations \cref{eq:twof_ni,eq:twof_ne} will lead to two single-fluid equations. The first is obtained by multiplying \cref{eq:twof_ni} by $m_i$, and the second is obtained by multiplying the ion and electron mass equations by $e$ and then subtracting. The results are
\begin{equation}
    \frac{\partial \rho}{\partial t} + \nabla \cdot (\rho \vvec) = 0,
\end{equation}
\begin{equation}
    \nabla \cdot \Jvec = 0.
\end{equation}
Note that the second equation above is superfluous since it also follows from taking the divergence of \cref{eq:preMax_amperes}

The two conservation of momentum equations will also lead to two single-fluid equations. The first is obtained by adding the ion and electron conservation of momentum equations to obtain
\begin{equation}
\label{eq:generalized_MHD_vel}
    \rho \left (\frac{\partial \vvec}{\partial t} + \vvec \cdot \nabla \uvec \right) - \Jvec \times \Bvec + \nabla p = \nabla \cdot \tvec_i + \nabla \cdot \tvec_e.
\end{equation}
For the second equation, we use $m_e \to 0$ and quasineutrality in the electron momentum equation to obtain
\begin{equation}
    e n ( \Evec + \uvec_e \times \Bvec ) = - \nabla p_e + \nabla \cdot \tvec_e + \Rvec_e,
\end{equation}
Assuming quasi-neutrality, the definition of the current given in \cref{eq:twof_curr_density} is now
\begin{equation}
\label{eq:current_def_quasineutral}
    \Jvec = e n (\uvec_i - \uvec_e),
\end{equation}
which is also written as
\begin{equation}
    \Jvec = e n (\vvec - \uvec_e).
\end{equation}
Using the above in the electron continuity equation gives
\begin{equation}
\label{eq:generalized_MHD_ohms}
    \Evec + \vvec \times \Bvec = \frac{1}{en} ( \Jvec \times \Bvec - \nabla p_e + \nabla \cdot \tvec_e + \Rvec_e).
\end{equation}

The two conservation of energy equations will also lead to two single-fluid equations. Each is evaluated using the single-fluid variables. As part of this derivation, we first rewrite the ion and electron internal energy \cref{eq:twof_iei,eq:twof_iee} as
\begin{equation}
\label{eq:twof_ie}
    \frac{1}{\gamma-1} \left ( \frac{\partial p_i}{\partial t} + \uvec_i \cdot \nabla p_i + \gamma p_i \nabla \cdot \uvec_i \right ) = \tvec_i : \nabla \uvec_i - \nabla \cdot \qvec_i + Q_i,
\end{equation}
\begin{equation}
    \frac{1}{\gamma-1} \left ( \frac{\partial p_e}{\partial t} + \uvec_e \cdot \nabla p_e + \gamma p_e \nabla \cdot \uvec_e \right ) = \tvec_e : \nabla \uvec_e - \nabla \cdot \qvec_e + Q_e,
\end{equation}
where we have used $\gamma = 5/3$ (the ratio of specific heats for monoatomic systems). We then note that
\begin{equation}
    \nabla \cdot \vvec = -\frac{1}{\rho} \frac{\partial \rho}{\partial t} - \frac{1}{\rho} \nabla \rho \cdot \vvec = - \frac{\partial \ln{\rho} }{\partial t} -  \nabla \ln{\rho} \cdot \vvec,
\end{equation}
and thus
\begin{equation}
    \gamma \nabla \cdot \vvec = -\frac{1}{\rho^\gamma} \frac{\partial \rho^\gamma}{\partial t} - \frac{1}{\rho^\gamma} \nabla \rho^\gamma \cdot \vvec.
\end{equation}
The result above allows us to write
\begin{align}
    \frac{\partial p_\alpha}{\partial t} + \vvec \cdot \nabla p_\alpha + \gamma p_\alpha \nabla \cdot \vvec 
    &=  \frac{\partial p_\alpha}{\partial t} - p_\alpha \frac{1}{\rho^\gamma} \frac{\partial \rho^\gamma}{\partial t} + \vvec \cdot \nabla p_\alpha - p_\alpha \frac{1}{\rho^\gamma} \nabla \rho^\gamma \cdot \vvec \nonumber \\
    & = \rho^\gamma \left [ \frac{\partial}{\partial t} \left ( \frac{p_\alpha}{\rho^\gamma} \right) + \vvec \cdot \nabla \left (\frac{p_\alpha}{\rho^\gamma} \right) \right ].
\end{align}
Thus, the ion energy equation becomes
\begin{equation}
\label{eq:generalized_MHD_pi}
   \frac{\partial}{\partial t} \left ( \frac{p_i}{\rho^\gamma} \right) + \vvec \cdot \nabla \left (\frac{p_i}{\rho^\gamma} \right) = \frac{\gamma - 1}{\rho^\gamma} \left( \tvec_i : \nabla \vvec - \nabla \cdot \qvec_i + Q_i \right),
\end{equation}
and the electron energy equation becomes
\begin{equation}
\label{eq:generalized_MHD_pe}
    \frac{\partial}{\partial t} \left ( \frac{p_e}{\rho^\gamma} \right) + \vvec \cdot \nabla \left (\frac{p_e}{\rho^\gamma} \right) = \frac{\gamma - 1}{\rho^\gamma} \left[ \tvec_e : \nabla \left (\vvec -\frac{\Jvec}{e n} \right) - \nabla \cdot \qvec_e + Q_e \right] + \frac{1}{e n} \Jvec \cdot \nabla \left ( \frac{ p_e}{\rho^\gamma} \right).
\end{equation}

%--------------------------------------------
\section{Resisitive MHD}
%--------------------------------------------
The electron collision term is modeled as
\begin{equation}
    \Rvec_e = m_e n_e \nu_{ei} \left ( \uvec_i - \uvec_e \right ).
\end{equation}
Assuming quasi-neutrality, the expression for current in \cref{eq:current_def_quasineutral} can be used to obtain
\begin{equation}
\label{eq:elec_coll_current}
    \Rvec_e = \frac{m_e \nu_{ei}}{e} \Jvec.
\end{equation}
Neglecting all terms on the right-hand side of \cref{eq:generalized_MHD_ohms} except for the electron collision term, we have
\begin{equation}
    \Evec + \vvec \times \Bvec = \frac{1}{en} \Rvec_e.
\end{equation}
Using \cref{eq:elec_coll_current} in the above, we have
\begin{equation}
    \Evec + \vvec \times \Bvec = \frac{m_e \nu_{ei}}{e^2n} \Jvec,
\end{equation}
which we re-write as 
\begin{equation}
    \Evec + \vvec \times \Bvec = \eta \Jvec,
\end{equation}
where
\begin{equation}
\label{eq:resistivity}
    \eta = \frac{m_e \nu_{ei}}{e^2 n}
\end{equation}
is the resistivity.

%--------------------------------------------
\section{Ideal MHD}
%--------------------------------------------
The ideal MHD equations are obtained by neglecting the right-hand sides of \cref{eq:generalized_MHD_vel,eq:generalized_MHD_ohms,eq:generalized_MHD_pi,eq:generalized_MHD_pe}. Summing the two pressure equations, the resulting equations would be
\begin{equation}
    \frac{\partial \rho}{\partial t} + \nabla \cdot (\rho \vvec) = 0,
\end{equation}
\begin{equation}
    \rho \left (\frac{\partial \vvec}{\partial t} + \vvec \cdot \nabla \vvec \right ) = - \nabla p  + \Jvec \times \Bvec
\end{equation}
\begin{equation}
    \Evec + \vvec \times \Bvec = 0,
\end{equation}
\begin{equation}
    \frac{\partial}{\partial t} \left ( \frac{p}{\rho^\gamma} \right ) + \vvec \cdot \nabla \left (\frac{p}{\rho^\gamma} \right ) = 0,
\end{equation}
\begin{equation}
\nabla \cdot \Bvec = 0.
\end{equation}
\begin{equation}
\nabla \times \Evec = -\frac{ \partial \Bvec}{\partial t},
\end{equation}
\begin{equation}
\nabla \times \Bvec = \mu_0 \Jvec ,
\end{equation}

Given the vector identity
\begin{equation}
    \frac{1}{2} \nabla \left ( B^2 \right ) = \Bvec \times \left (\nabla \times \Bvec \right ) + \left ( \Bvec \cdot \nabla \right ) \Bvec,
\end{equation}
we can use Ampere's law to re-write the $\Jvec \times \Bvec$ term in the velocity equation as
\begin{equation}
    \Jvec \times \Bvec = \frac{1}{\mu_0} \left ( \nabla \times \Bvec \right ) \times \Bvec = \frac{1}{\mu_0} \left [ \left ( \Bvec \cdot \nabla \right ) \Bvec - \frac{1}{2} \nabla \left ( B^2 \right ) \right ].
\end{equation}
Similarly, given the vector identity
\begin{equation}
    \nabla \times \left ( \Bvec \times \vvec \right ) = \left (\vvec \cdot \nabla \right ) \Bvec - \left ( \Bvec \cdot \nabla \right ) \vvec + \Bvec \left ( \nabla \cdot \vvec \right ) - \vvec \left ( \nabla \cdot \Bvec \right ),
\end{equation}
we can use Ohm's law to re-write the $\nabla \times \Evec$ term in Faraday's law as
\begin{equation}
    \nabla \times \Evec = \nabla \times (-\vvec \times \Bvec) = \left (\vvec \cdot \nabla \right ) \Bvec - \left ( \Bvec \cdot \nabla \right ) \vvec + \Bvec \left ( \nabla \cdot \vvec \right ).
\end{equation}
Thus, the ideal MHD equations can be summarized as follows
\begin{equation}
    \frac{\partial \rho}{\partial t} + \nabla \cdot (\rho \vvec) = 0,
\end{equation}
\begin{equation}
    \nabla \cdot \Bvec = 0.
    \end{equation}
\begin{equation}
    \rho \left (\frac{\partial \vvec}{\partial t} + \vvec \cdot \nabla \vvec \right ) = - \nabla p  + \frac{1}{\mu_0} \left [ \left ( \Bvec \cdot \nabla \right ) \Bvec - \frac{1}{2} \nabla \left ( B^2 \right ) \right ]
\end{equation}
\begin{equation}
    \frac{\partial \Bvec}{\partial t} + \left ( \vvec \cdot \nabla \right ) \Bvec = \left ( \Bvec \cdot \nabla \right ) \vvec - \Bvec (\nabla \cdot \vvec)
\end{equation}
\begin{equation}
    \frac{\partial}{\partial t} \left ( \frac{p}{\rho^\gamma} \right ) + \vvec \cdot \nabla \left (\frac{p}{\rho^\gamma} \right ) = 0,
\end{equation}
If we assume incompressibility, then the above simplifies to
\begin{equation}
    \nabla \cdot \vvec = 0,
\end{equation}
\begin{equation}
    \nabla \cdot \Bvec = 0.
    \end{equation}
\begin{equation}
    \rho \left (\frac{\partial \vvec}{\partial t} + \vvec \cdot \nabla \vvec \right ) = - \nabla p  + \frac{1}{\mu_0} \left [ \left ( \Bvec \cdot \nabla \right ) \Bvec - \frac{1}{2} \nabla \left ( B^2 \right ) \right ]
\end{equation}
\begin{equation}
    \frac{\partial \Bvec}{\partial t} + \left ( \vvec \cdot \nabla \right ) \Bvec = \left ( \Bvec \cdot \nabla \right ) \vvec
\end{equation}

%-------------------------------------------------------------------------------
\chapter{Laser-plasma interactions}
%-------------------------------------------------------------------------------

%--------------------------------------------
\section{Governing equations}
%--------------------------------------------
We'll start with the multi-fluid conservation laws described in \cref{sec:two_fluid_equations}. We'll assume no shear stresses, collisions, and homentropic flows. Thus, the governing equations are
\begin{equation}
    \label{eq:pwaves_ion_density}
    \frac{\partial n_i}{\partial t} + \nabla \cdot \left (n_i \uvec_i \right ) = 0,
\end{equation}
\begin{equation}
    \label{eq:pwaves_electron_density}
    \frac{\partial n_e}{\partial t} + \nabla \cdot \left (n_e \uvec_e \right ) = 0,
\end{equation}
\begin{equation}
    \label{eq:pwaves_ion_momentum}
    \frac{\partial m_i n_i \uvec_i}{\partial t} + \nabla \cdot \left ( m_i n_i \uvec_i \uvec_i \right ) - Ze n_i \left ( \Evec + \uvec_i \times \Bvec \right ) = -\nabla p_i,
\end{equation}
\begin{equation}
    \label{eq:pwaves_electron_momentum}
    \frac{\partial m_e n_e \uvec_e}{\partial t} + \nabla \cdot \left ( m_e n_e \uvec_e \uvec_e \right ) + e n_e \left ( \Evec + \uvec_e \times \Bvec \right ) = -\nabla p_e.
\end{equation}
\begin{equation}
    p_i = C_i n_i^{\gamma_i},
\end{equation}
\begin{equation}
    p_e = C_e n_e^{\gamma_e},
\end{equation}
\begin{equation}
    \label{eq:pwaves_maxwell_3}
    \nabla \cdot \Evec = \frac{\rho_e}{\epsilon_0} 
\end{equation}
\begin{equation}
    \label{eq:p_waves_maxwell_4}
    \nabla \cdot \Bvec = 0.
\end{equation}
\begin{equation}
    \label{eq:p_waves_maxwell_1}
    \nabla \times \Evec = -\frac{ \partial \Bvec}{\partial t}
\end{equation}
\begin{equation}
    \label{eq:p_waves_maxwell_2}
    \nabla \times \Bvec = \mu_0 \Jvec + \mu_0 \epsilon_0 \frac{\partial \Evec}{\partial t}
\end{equation}
\begin{equation}
    \label{eq:p_waves_curr_density}
    \Jvec = e (Z n_i \uvec_i - n_e \uvec_e)
\end{equation}
\begin{equation}
    \label{eq:p_waves_mass_density}
    \rho_e = e (Z n_i - n_e) 
\end{equation}

%--------------------------------------------
\section{Electron-plasma and ion-acoustic waves}
%--------------------------------------------
%--------------------------------------------
\subsection{Linearization}
%--------------------------------------------
\label{sec:p_waves_linearization}
The following decompositions will be used in the derivation of electron-plasma and ion-acoustic waves:
\begin{align}
    n_i &= n_{i0} + n_{i1}, \nonumber \\
    n_e &= n_{e0} + n_{e1}, \nonumber \\
    p_i &= p_{i0} + p_{i1}, \nonumber \\
    p_e &= p_{e0} + p_{e1}, \nonumber \\
    \uvec_i &= \uvec_{i0} + \uvec_{i1}, \nonumber \\
    \uvec_e &= \uvec_{e0} + \uvec_{e1}, \nonumber \\
    \Evec &= \Evec_0 + \Evec_1, \nonumber \\
    \Bvec &= \Bvec_0 + \Bvec_1.
\end{align}
For these decompositions, we'll assume
\begin{enumerate}
    \item Terms with a subscript 1 are small and thus products of two small quantities can be neglected. \label{it:p_waves_assumption_1}
    \item $\uvec_{i0}$, $\uvec_{e0}$, $\Evec_0$, and $\Bvec_0$ are zero. \label{it:p_waves_assumption_2}
    \item $n_{i0}$, $n_{e0}$, $p_{i0}$, and $p_{e0}$ are uniform in space and time. \label{it:p_waves_assumption_3}
\end{enumerate}

Using the variable decompositions in the electron density equation, we have
\begin{equation*}
    \frac{\partial n_{e0} + n_{e1}}{\partial t} + \nabla \cdot \left [ \left ( n_{e0} + n_{e1} \right )  \left ( \uvec_{e0} + \uvec_{e1} \right ) \right ] = 0.
\end{equation*}
Using assumptions in \cref{it:p_waves_assumption_1,it:p_waves_assumption_2,it:p_waves_assumption_3}, the above simplifies to
\begin{equation}
    \label{eq:p_waves_e_den_linearized}
    \frac{\partial n_{e1}}{\partial t} + \nabla \cdot \left (n_{e0} \uvec_{e1} \right ) = 0.
\end{equation}
Using the variable decompositions in the ion density equation, we have
\begin{equation*}
    \frac{\partial n_{i0} + n_{i1}}{\partial t} + \nabla \cdot \left [ \left ( n_{i0} + n_{i1} \right )  \left ( \uvec_{i0} + \uvec_{i1} \right ) \right ] = 0.
\end{equation*}
Given the assumptions in \cref{it:p_waves_assumption_1,it:p_waves_assumption_2,it:p_waves_assumption_3}, the above simplifies to
\begin{equation}
    \label{eq:p_waves_i_den_linearized}
    \frac{\partial n_{i1}}{\partial t} + \nabla \cdot \left (n_{i0} \uvec_{i1} \right ) = 0.
\end{equation}

Using the variable decompositions in the electron momentum equation, we have 
\begin{multline*}
    \frac{\partial}{\partial t} \left [ m_e \left ( n_{e0} + n_{e1} \right ) \left ( \uvec_{e0} + \uvec_{e1} \right ) \right ] + \nabla \cdot \left [ m_e \left ( n_{e0} + n_{e1} \right ) \left ( \uvec_{e0} + \uvec_{e1} \right ) \left ( \uvec_{e0} + \uvec_{e1} \right ) \right ] \\
    + e \left ( n_{e0} + n_{e1} \right ) \left [ \left ( \Evec_0 + \Evec_1 \right ) + \left ( \uvec_{e0} + \uvec_{e1} \right ) \times \left ( \Bvec_0 + \Bvec_1 \right ) \right ] = -\nabla \left ( p_{e0} + p_{e1} \right ).
\end{multline*}
Given the assumptions in \cref{it:p_waves_assumption_1,it:p_waves_assumption_2,it:p_waves_assumption_3}, the above simplifies to
\begin{equation}
    \label{eq:p_waves_e_mom_linearized}
    \frac{\partial n_{e0} \uvec_{e1}}{\partial t} + \frac{e n_{e0}}{m_e} \Evec_1 = - \frac{1}{m_e} \nabla p_{e1} .
\end{equation}
Using the variable decompositions in the ion momentum equation, we have 
\begin{multline*}
    \frac{\partial}{\partial t} \left [ m_i \left ( n_{i0} + n_{i1} \right ) \left ( \uvec_{i0} + \uvec_{i1} \right ) \right ] + \nabla \cdot \left [ m_i \left ( n_{i0} + n_{i1} \right ) \left ( \uvec_{i0} + \uvec_{i1} \right ) \left ( \uvec_{i0} + \uvec_{i1} \right ) \right ] \\
    - Z e \left ( n_{i0} + n_{i1} \right ) \left [ \left ( \Evec_0 + \Evec_1 \right ) + \left ( \uvec_{i0} + \uvec_{i1} \right ) \times \left ( \Bvec_0 + \Bvec_1 \right ) \right ] = -\nabla \left ( p_{i0} + p_{i1} \right ).
\end{multline*}
Given the assumptions in \cref{it:p_waves_assumption_1,it:p_waves_assumption_2,it:p_waves_assumption_3}, the above simplifies to
\begin{equation}
    \label{eq:p_waves_i_mom_linearized}
    \frac{\partial n_{i0} \uvec_{i1}}{\partial t} - \frac{Z e n_{i0}}{m_i} \Evec_1 = - \frac{1}{m_i}\nabla p_{i1}.
\end{equation}

We'll often need to take the gradient of the ion and electron pressure. We begin by showing
\begin{equation*}
    \nabla p_\alpha = C_\alpha \gamma_\alpha n_\alpha^{\gamma_\alpha-1} \nabla n_\alpha = C_\alpha \gamma_\alpha \frac{n_\alpha^{\gamma_\alpha}}{n_\alpha} \nabla n_\alpha = \gamma_\alpha \frac{p_\alpha}{n_\alpha} \nabla n_\alpha,
\end{equation*}
or
\begin{equation*}
    n_\alpha \nabla p_\alpha = \gamma_\alpha p_\alpha \nabla n_\alpha,
\end{equation*}
where $\alpha = i,e$. Using the variable decompositions, we have
\begin{equation*}
    \left ( n_{\alpha 0} + n_{\alpha 1} \right ) \nabla \left ( p_{\alpha 0} + p_{\alpha 1} \right ) = \gamma_\alpha \left ( p_{\alpha 0} + p_{\alpha 1} \right ) \nabla \left ( n_{\alpha 0} + n_{\alpha 1} \right ).
\end{equation*}
Given the assumptions in \cref{it:p_waves_assumption_1,it:p_waves_assumption_3}, the above simplifies to
\begin{equation}
    \label{eq:p_waves_alpha_pressure_linearized}
    n_{\alpha 0} \nabla p_{\alpha 1} = \gamma_\alpha p_{\alpha 0} \nabla n_{\alpha 1}.
\end{equation}
Thus, for electrons we have
\begin{equation}
    \label{eq:p_waves_e_pressure_linearized}
    n_{e0} \nabla p_{e1} = \gamma_e p_{e0} \nabla n_{e1},
\end{equation}
and for ions
\begin{equation}
    \label{eq:p_waves_i_pressure_linearized}
    n_{i0} \nabla p_{i1} = \gamma_i p_{i0} \nabla n_{i1}.
\end{equation}

An alternate derivation to obtain the previous results is to use an equation of state for the large pressure and density of the following form
\begin{equation}
    \label{eq:p_waves_eos}
    p_{\alpha 0} = n_{\alpha 0} k_B T_\alpha,
\end{equation}
where $\alpha = i,e$. Note that the temperature is one whole quantity, that is, it is not split into large and small terms. Let's assume that the large quantities also satisfy isentropic flow, that is $p_{\alpha 0} = C n^{\gamma_\alpha}_{\alpha 0}$. Thus, we can show that
\begin{align*}
    p_{\alpha 1} &= p_{\alpha} - p_{\alpha 0} \\
    &= C n_\alpha^{\gamma_\alpha} - p_{\alpha 0} \\
    &= C (n_{\alpha 0} + n_{\alpha 1})^{\gamma_\alpha} - p_{\alpha 0} \\
    &= C n_{\alpha 0}^{\gamma_\alpha} \left ( 1 + \frac{n_{\alpha 1}}{n_{\alpha 0}} \right )^{\gamma_\alpha} - p_{\alpha 0} \\
    &= p_{\alpha 0} \left ( 1 + \frac{n_{\alpha 1}}{n_{\alpha 0}} \right )^{\gamma_\alpha} - p_{\alpha 0}.
\end{align*} 
Since $n_{\alpha 1} / n_{\alpha 0}$ is small, we can use the binomial series to proceed as follows
\begin{align*}
    p_{\alpha 1} &= p_{\alpha 0} \left ( 1 + \gamma_\alpha \frac{n_{\alpha 1}}{n_{\alpha 0}} \right ) - p_{\alpha 0} \\
    &= \gamma_\alpha \frac{p_{\alpha 0}}{n_{\alpha 0}} n_{\alpha 1}.
\end{align*}
With $p_{\alpha 0}$ and $n_{\alpha 0}$ constant, this then gives \cref{eq:p_waves_alpha_pressure_linearized}. Finally, using \cref{eq:p_waves_eos} in the above we get
\begin{equation}
    p_{\alpha 1} = \gamma_\alpha k_B T_\alpha n_{\alpha 1},
\end{equation}
which can be interpreted as the equation of state for the small quantities.

%--------------------------------------------
\subsection{Electron Plasma Waves}
%--------------------------------------------
On top of the assumptions in \cref{sec:p_waves_linearization}, we'll assume 
\begin{enumerate}
    \item Quasi-neutrality for the base flow, $Zn_{i0} = n_{e0}$.
    \item Uniform ion density, $n_{i1} = 0$.
\end{enumerate}

Combining \Cref{eq:p_waves_e_mom_linearized} with \cref{eq:p_waves_e_pressure_linearized} gives
\begin{equation}
    \label{eq:ep_waves_mom_linearized}
    \frac{\partial n_{e0} \uvec_{e1}}{\partial t} + \frac{e n_{e0}}{m_e} \Evec_1 = - \frac{\gamma_e p_{e0}}{n_{e0} m_e} \nabla n_{e1}.
\end{equation}
Taking the time derivative of \cref{eq:p_waves_e_den_linearized} and using \cref{eq:ep_waves_mom_linearized} leads to the the wave equation for electron density
\begin{equation}
    \label{eq:ep_waves_den_combined}
    \frac{\partial^2 n_{e1}}{\partial t^2} - \frac{e n_{e0}}{m_e} \nabla \cdot \Evec_1 = \frac{\gamma_e p_{e0}}{n_{e0} m_e} \nabla^2 n_{e1}.
\end{equation}
For electron plasma waves, we'll assume that $n_{i}$ varies in space and time so slowly that it can be assumed to be constant. That is, we assume $n_{i1} = 0$. Thus, Gauss's law now takes the form
\begin{equation*}
    \nabla \cdot \Evec_1 = \frac{e}{\epsilon_0} \left ( Z n_{i0} - n_{e0} - n_{e1} \right ).
\end{equation*}
Using the quasi-neutrality assumption ($Zn_{i0} = n_{e0}$)
\begin{equation}
    \label{eq:ep_waves_efield_divergence}
    \nabla \cdot \Evec_1 = -\frac{e}{\epsilon_0} n_{e1}.
\end{equation}
Plugging the above in the electron wave equation we obtain
\begin{equation*}
    \frac{\partial^2 n_{e1}}{\partial t^2} + \frac{e^2 n_{e0}}{m_e \epsilon_0} n_{e1} = \frac{\gamma_e p_{e0}}{n_{e0} m_e} \nabla^2 n_{e1},
\end{equation*}
or
\begin{equation}
    \frac{\partial^2 n_{e1}}{\partial t^2} + w^2_{pe} n_{e1} - \frac{\gamma_e p_{e0}}{n_{e0} m_e} \nabla^2 n_{e1} = 0.
\end{equation}

Assuming a mode of the form $n_{e1} = \hat{n}_{e1} \exp [ i \left ( \kvec_e \cdot \xvec - w t \right ) ]$, where $\kvec_e$ is the wave vector of the electron-plasma wave, gives the following dispersion relation
\begin{equation}
    w^2 - w^2_{pe} - \frac{\gamma_e p_{e0}}{n_{e0} m_e}  k_e^2 = 0.
\end{equation}
If we define the thermal velocity without the factor of two, that is, $v_{T\alpha} = \sqrt{k_BT_\alpha / m_\alpha}$, then we have
\begin{equation}
    w^2 - w^2_{pe} - \gamma_e k_e^2 v^2_{T_{e0}} = 0.
\end{equation}

%--------------------------------------------
\subsection{Ion Acoustic Waves}
%--------------------------------------------
On top of the assumptions in \cref{sec:p_waves_linearization}, we'll assume 
\begin{enumerate}
    \item Quasi-neutrality for the base flow, $Zn_{i0} = n_{e0}$.
    \item Approximate quasi-neutrality for the fluctuations, $Z n_{i1} \approx n_{e1}$.
    \item Negligible electron mass, $m_e \to 0$.
\end{enumerate}

Combining \cref{eq:p_waves_i_mom_linearized} with \cref{eq:p_waves_i_pressure_linearized} gives
\begin{equation}
    \label{eq:ia_waves_mom_linearized}
    \frac{\partial n_{i0} \uvec_{i1}}{\partial t} - \frac{Z e n_{i0}}{m_i} \Evec_1 = - \frac{\gamma_i p_{i0}}{n_{i0} m_i} \nabla n_{i1}.
\end{equation}

Taking the time derivative of \cref{eq:p_waves_i_den_linearized} and using \cref{eq:ia_waves_mom_linearized} leads to the the wave equation for ion density
\begin{equation}
    \label{eq:ia_waves_den_combined}
    \frac{\partial^2 n_{i1}}{\partial t^2} + \frac{Z e n_{i0}}{m_i} \nabla \cdot \Evec_1 = \frac{\gamma_i p_{i0}}{n_{i0} m_i} \nabla^2 n_{i1}.
\end{equation}
For this case, we assume that the mass of the electron, which is significantly smaller than that of the ions, is negligible. Thus, \cref{eq:p_waves_e_mom_linearized} simplifies to 
\begin{equation}
    \label{eq:ia_waves_E}
    e n_{e0} \Evec_1 = - \frac{\gamma_e p_{e0}}{n_{e0}} \nabla n_{e1}.
\end{equation}
Plugging in the above in the ion wave equation we obtain
\begin{equation*}
    \frac{\partial^2 n_{i1}}{\partial t^2} = \frac{Z n_{i0}}{ n_{e0}} \frac{\gamma_e p_{e0}}{n_{e0} m_i} \nabla^2 n_{e1} + \frac{\gamma_i p_{i0}}{n_{i0} m_i} \nabla^2 n_{i1}.
\end{equation*}
Due to quasi-neutrality, we have $Z n_{i0} = n_{e0}$ and $Z n_{i1} \approx n_{e1}$, which gives
\begin{equation*}
    \frac{\partial^2 n_{i1}}{\partial t^2} = \frac{1}{m_i} \left ( \frac{Z \gamma_e p_{e0}}{n_{e0}} + \frac{\gamma_i p_{i0}}{n_{i0}} \right ) \nabla^2 n_{i1}.
\end{equation*}
Since $p_{i0}/n_{i0} = k_B T_i$ and $p_{e0}/n_{e0} = k_B T_e$, we finally have
\begin{equation}
    \frac{\partial^2 n_{i1}}{\partial t^2} - \left ( \frac{Z \gamma_e k_B T_e + \gamma_i k_B T_i}{m_i} \right ) \nabla^2 n_{i1} = 0.
\end{equation}

Assuming a mode of the form $n_{i1} = \hat{n}_{i1} \exp [ i \left ( \kvec_i \cdot \xvec - w t \right ) ]$, where $\kvec_i$ is the wave vector of the ion-acoustic wave, we obtain the following dispersion relation
\begin{equation}
    w^2 - k_i^2 v_s^2 = 0,
\end{equation}
where
\begin{equation}
    v_s = \sqrt{ \frac{Z \gamma_e k_B T_e + \gamma_i k_B T_i }{m_i} }.
\end{equation}

%--------------------------------------------
\section{Longitudinal and transverse waves}
%--------------------------------------------
We begin with the Helmholtz decomposition
\begin{equation}
    \Fvec = \Fvec_l + \Fvec_t,
\end{equation}
where $\Fvec_l$ is the longitudinal component and $\Fvec_t$ the transverse component. These are defined by
\begin{align}
    \nabla \times \Fvec_l &= 0, \label{eq:ltw_plasma_long_def}\\
    \nabla \cdot \Fvec_t &= 0. \label{eq:ltw_plasma_tran_def}
\end{align}
Assume the vector functions under consideration are of the following form
\begin{equation}
    \label{eq:em_general_wave_form}
    \Fvec(\xvec,t) = \hat{\Fvec} \exp \left [i\left ( \kvec \cdot \xvec - wt \right ) \right ],
\end{equation}
\begin{equation}
    \Fvec_l(\xvec,t) = \hat{\Fvec}_l \exp \left [i\left ( \kvec \cdot \xvec - wt \right ) \right ],
\end{equation}
\begin{equation}
    \Fvec_t(\xvec,t) = \hat{\Fvec}_t \exp \left [i\left ( \kvec \cdot \xvec - wt \right ) \right ].
\end{equation}
These are the so-called plane waves. Note that $\hat{\Fvec}$, $\hat{\Fvec}_l$, and $\hat{\Fvec}_t$ are complex vectors also, where the real and complex components point in the same direction. Also note that different waves will wave different wave vectors and frequencies. For example, we'll use $\kvec_e$ for electron-plasma waves, $\kvec_i$ for ion-acoustic waves, $\kvec_L$ for laser waves and $\kvec_s$ for scattered waves. The same applies for the frequencies $w_e$, $w_i$, $w_L$, $w_s$.

\Cref{eq:ltw_plasma_long_def,eq:ltw_plasma_tran_def} now translate to
\begin{align}
    \kvec \times \hat{\Fvec}_l &= 0 \\
    \kvec \cdot \hat{\Fvec}_t &= 0
\end{align}
The first says $\hat{\Fvec}_l$ is parallel to $\kvec$ and the second says $\hat{\Fvec}_t$ is orthogonal to $\kvec$. Thus, $\hat{\Fvec}_l \cdot \hat{\Fvec}_t = 0$. We will often have situations where $\nabla \times \Fvec = \nabla \cdot \Fvec = 0$, which by its own does not imply $\Fvec = 0$. However, due to the wave form of $\Fvec$ (\cref{eq:em_general_wave_form}), the previous expressions translate to $\kvec \times \hat{\Fvec} = \kvec \cdot \hat{\Fvec} =  0$. The latter equality states that $\kvec$ and $\hat{\Fvec}$ are orthogonal, that is, the angle between them is $90^\circ$. The former equality leads to $|\hat{\Fvec}| \sin(90^\circ) = 0$, which in turn means $\hat{\Fvec} = 0$. As a result, the entire vector $\Fvec$ is equal to zero. In other words,
\begin{equation}
    \label{eq:ltw_general_null_vector}
    \nabla \times \Fvec = \nabla \cdot \Fvec = 0 \to \Fvec = 0.
\end{equation}

%--------------------------------------------
\subsection{Electron-plasma and ion-acoustic waves}
%--------------------------------------------
For both electron-plasma and ion-acoustic waves we can assume the magnetic field does not change. Thus, Faraday's law gives
\begin{equation}
    \nabla \times \Evec = \nabla \times \Evec_t = 0.
\end{equation}
By definition, $\nabla \cdot \Evec_t = 0$. Thus, using \cref{eq:ltw_general_null_vector} we get $\Evec_t = 0$, that is, $\Evec = \Evec_l$. 

For electron-plasma waves, we can write \cref{eq:ep_waves_mom_linearized} in spectral form to obtain
\begin{equation}
-i w n_{e0} \hat{\uvec}_{e1} + \frac{e n_{e0}}{m_e} \hat{\Evec}_{1,l} = -\kvec_e \frac{\gamma_e p_{e0}}{n_{e0} m_e} \hat{n}_{e1}. 
\end{equation}
Since the second term on the left-hand side and the term on the right-hand side point along $\kvec_e$, $\hat{\uvec}_{e1}$ also points along $\kvec_e$, that is, $\hat{\uvec}_{e1} = \hat{\uvec}_{e1,l}$.

For ion-acoustic waves, we can write \cref{eq:ia_waves_mom_linearized} in spectral to obtain
\begin{equation}
-i w n_{i0} \hat{\uvec}_{i1} - \frac{Z e n_{i0}}{m_i} \hat{\Evec}_{1,l} = -\kvec_i \frac{\gamma_i p_{i0}}{n_{i0} m_i} \hat{n}_{i1}. 
\end{equation}
Since the second term on the left-hand side and the term on the right-hand side point along $\kvec_i$, $\hat{\uvec}_{i1}$ also points along $\kvec_i$, that is, $\hat{\uvec}_{i1} = \hat{\uvec}_{i1,l}$. Finally, we note that the electric field being purely longitudinal is in agreement with \cref{eq:ia_waves_E}.

%--------------------------------------------
\section{Electromagnetic waves in plasmas}
%--------------------------------------------
\label{sec:electromagnetic_waves_plasmas}
In the Appendix we analyzed electromagnetic waves in vacuum, that is, for cases where $\rho_e = \Jvec = 0$. In this section we relax both of these assumptions. Consider the electric and magnetic fields as well as the scalar and vector potentials, which satisfy
\begin{equation}
    \label{eq:emp_general_E_potential}
    \Evec = -\nabla \phi - \frac{\partial \Avec}{\partial t},
\end{equation}
\begin{equation}
    \label{eq:emp_general_B_potential}
    \Bvec = \nabla \times \Avec.
\end{equation}
For the above, we choose $\nabla \cdot \Avec = 0$. Using the fact that the magnetic field is solenoidal, we have
\begin{equation*}
    \nabla \cdot \Bvec = \nabla \cdot \Bvec_l + \nabla \cdot \Bvec_t = \nabla \cdot \Bvec_l = 0.
\end{equation*}
However, by definition, $\nabla \times \Bvec_l = 0$ as well. Thus, using \cref{eq:ltw_general_null_vector}, we have $\Bvec_l = 0$. The same argument applies to the vector potential, and thus $\Avec_l = 0$. For the electric field, we have
\begin{equation*}
    \nabla \cdot \Evec = \nabla \cdot \Evec_l + \nabla \cdot \Evec_t = \nabla \cdot \Evec_l = \nabla \cdot \left ( -\nabla \phi \right ),
\end{equation*}
where we used \cref{eq:emp_general_E_potential} for the last equality. In other words, we have
\begin{equation*}
    \nabla \cdot \left ( \Evec_l + \nabla \phi \right ) = 0.
\end{equation*}
By definition, we also have
\begin{equation*}
    \nabla \times \left ( \Evec_l + \nabla \phi \right ) = 0.
\end{equation*}
Thus, using \cref{eq:ltw_general_null_vector}, we have $\Evec_l = -\nabla \phi$. A similar argument can be used to show $\Evec_t = -\partial \Avec / \partial t$. Our goal in this section will be to determine equations for $\Evec_l$, $\Evec_t$ and $\Bvec$.

We'll begin with the conservation of charge equation
\begin{equation*}
    \frac{\partial \rho_e}{\partial t} + \nabla \cdot \Jvec = 0,
\end{equation*}
which we re-write as
\begin{equation*}
    \frac{\partial \rho_e}{\partial t} + \nabla \cdot \Jvec_l = 0,
\end{equation*}
Using Poisson's equation $\nabla^2\phi = -\rho_e / \epsilon_0$ in the above, we get
\begin{equation*}
    \frac{\partial}{\partial t} \left ( -\epsilon_0 \nabla^2 \phi \right ) + \nabla \cdot \Jvec_l = 0,
\end{equation*}
or
\begin{equation*}
    \nabla \cdot \left ( \frac{\partial \nabla \phi}{\partial t} - \frac{1}{\epsilon_0} \Jvec_l \right ) = 0.
\end{equation*}
However, by definition, we also have
\begin{equation*}
    \nabla \times \left ( \frac{\partial \nabla \phi}{\partial t} - \frac{1}{\epsilon_0} \Jvec_l \right ) = 0.
\end{equation*}
Using \cref{eq:ltw_general_null_vector}, we conclude
\begin{equation}
    \label{eq:emp_general_longitudinal_J}
    \frac{\partial \nabla \phi}{\partial t} = \frac{1}{\epsilon_0} \Jvec_l.
\end{equation}
This gives the equation for $\Evec_l$, namely,
\begin{equation}
    \label{eq:emp_general_long_E}
    \frac{\partial^2 \Evec_l}{\partial t^2} + \frac{1}{\epsilon_0} \frac{\partial \Jvec_l}{\partial t} = 0.
\end{equation}

Both $\Evec_t$ and $\Bvec$ can be extracted from $\Avec$, so now we proceed to find an equation for the transverse vector potential. Ampere's law with Maxwell's correction gives
\begin{equation*}
    \nabla \times \left ( \nabla \times \Avec \right ) = \mu_0 \Jvec + \mu_0 \epsilon_0 \frac{\partial \Evec}{\partial t}.
\end{equation*}
The above is re-written as
\begin{equation*}
    \nabla \left ( \nabla \cdot \Avec \right ) - \nabla^2 \Avec = \mu_0 \Jvec + \mu_0 \epsilon_0 \left ( -\frac{\partial \nabla \phi}{\partial t} - \frac{\partial^2 \Avec}{\partial t^2} \right ),
\end{equation*}
which gives
\begin{equation*}
    \frac{\partial^2 \Avec}{\partial t} - \frac{1}{\mu_0 \epsilon_0} \nabla^2 \Avec = \frac{1}{\epsilon_0} \Jvec - \frac{\partial \nabla \phi}{\partial t},
\end{equation*}
or
\begin{equation*}
    \frac{\partial^2 \Avec}{\partial t} - c_0^2 \nabla^2 \Avec = \frac{1}{\epsilon_0} \Jvec - \frac{\partial \nabla \phi}{\partial t},
\end{equation*}
where $c_0 = 1/\sqrt{\mu_0 \epsilon_0}$. Expanding the current density as $\Jvec = \Jvec_l + \Jvec_t$, and using \cref{eq:emp_general_longitudinal_J}, we get
\begin{equation}
    \label{eq:emp_general_trans_vec_pot}
    \frac{\partial^2 \Avec}{\partial t} - c_0^2 \nabla^2 \Avec = \frac{1}{\epsilon_0} \Jvec_t.
\end{equation}
Using the functional form in \cref{eq:em_general_wave_form} for $\Avec$ and $\Jvec_t$ gives
\begin{equation}
    -w^2 \hat{\Avec} + k^2 c_0^2 \hat{\Avec}= \frac{1}{\epsilon_0} \hat{\Jvec}_t.
\end{equation}
That is, $\Avec$ and $\Jvec_t$ point in the same direction.

Taking the time derivative of \cref{eq:emp_general_trans_vec_pot} gives the equation for $\Evec_t$, that is
\begin{equation}
    \label{eq:emp_general_trans_E}
    \frac{\partial^2 \Evec_t}{\partial t^2} - c_0^2 \nabla^2 \Evec_t + \frac{1}{\epsilon_0} \frac{\partial \Jvec_t}{\partial t} = 0.
\end{equation}
Taking the curl of \cref{eq:emp_general_trans_vec_pot} gives the equation for $\Bvec$, that is
\begin{equation}
    \label{eq:emp_general_trans_B}
    \frac{\partial^2 \Bvec}{\partial t^2} - c_0^2 \nabla^2 \Bvec - \frac{1}{\epsilon_0} \nabla \times \Jvec_t = 0.
\end{equation}
Using the functional form in \cref{eq:em_general_wave_form} for $\Evec_t$, $\Bvec$ and $\Jvec_t$ gives
\begin{equation}
    -w^2 \hat{\Evec}_t + k^2 c_0^2 \hat{\Evec}_t - \frac{iw}{\epsilon_0} \hat{\Jvec}_t = 0,
\end{equation}
\begin{equation}
    -w^2 \Bvec + k^2 c_0^2 \Bvec - \frac{i}{\epsilon_0} \kvec \times \Jvec_t = 0.
\end{equation}
That is, $\Evec_t$ points in the same direction as $\Jvec_t$, which as shown before points in the same direction as $\Avec$. Additionally, $\Bvec$ points in the direction of $\kvec \times \Jvec_t$, that is, it is orthogonal to $\Evec_t$.

We briefly note that taking the curl of \cref{eq:p_waves_maxwell_1}, and using \cref{eq:p_waves_maxwell_2}, gives the wave equation for the total electric field $\Evec$, that is 
\begin{equation}
    \frac{\partial^2 \Evec}{\partial t^2} - c_0^2 \nabla^2 \Evec + c_0^2 \nabla (\nabla \cdot \Evec) + \frac{1}{\epsilon_0} \frac{\partial \Jvec}{\partial t} = 0.
\end{equation}
The above can be considered as the sum of the following three equations
\begin{align*}
    \frac{\partial^2 \Evec_l}{\partial t^2} + \frac{1}{\epsilon_0} \frac{\partial \Jvec_l}{\partial t} &= 0, \\
    \frac{\partial^2 \Evec_t}{\partial t^2} - c_0^2 \nabla^2 \Evec_t + \frac{1}{\epsilon_0} \frac{\partial \Jvec_t}{\partial t} &= 0, \\
    -c_0^2 \nabla^2 \Evec_l + c_0^2 \nabla (\nabla \cdot \Evec_l) &= 0.
\end{align*}
The first is the equation for the longitudinal electric field, that is \cref{eq:emp_general_long_E}. The second is the equation for the transverse electric field, that is \cref{eq:emp_general_trans_E}. The third equation above follows from the vector identity $\nabla \times (\nabla \times \Fvec) = -\nabla^2 \Fvec + \nabla ( \nabla \cdot \Fvec )$ and the fact that $\nabla \times \Evec_l = 0$.

It will often be the case that transverse waves will oscillate at such a fast rate that the ions, which have a large inertia, will be unable to react quickly enough. Thus, we can assume $\uvec_{i,t} = 0$. Given the definition of the current density in \cref{eq:p_waves_curr_density}, the transverse current density is expressed as $\Jvec_t = e \left (Z n_i \uvec_{i,t} - n_e \uvec_{e,t} \right )$, which now simplifies to 
\begin{equation}
    \label{eq:emp_transverse_current}
    \Jvec_t = -e n_e \uvec_{e,t}.
\end{equation}
Thus, the transverse electron velocity $\uvec_{e,t}$ points in the same direction as $\Jvec_t$, which is the same direction as $\Evec_t$ and $\Avec$. The next section focuses on deriving an expression for $\uvec_{e,t}$. 

We begin with \cref{eq:pwaves_electron_momentum}, the electron momentum equation, which, due to the electron continuity equation, can be written as
\begin{equation*}
    m_e n_e \frac{\partial\uvec_e}{\partial t} + m_e n_e \uvec_e \cdot \nabla \uvec_e + e n_e \left ( \Evec + \uvec_e \times \Bvec \right ) = -\nabla p_e,
\end{equation*}
or
\begin{equation*}
    \frac{\partial\uvec_e}{\partial t} + \uvec_e \cdot \nabla \uvec_e + \frac{e}{m_e} \left ( \Evec + \uvec_e \times \Bvec \right ) = -\frac{1}{n_e m_e}\nabla p_e,
\end{equation*}
Using the scalar and vector potentials we have
\begin{equation*}
    \frac{\partial\uvec_e}{\partial t} + \uvec_e \cdot \nabla \uvec_e +\frac{e}{m_e} \left [ -\nabla \phi - \frac{\partial \Avec}{\partial t} + \uvec_e \times \left ( \nabla \times \Avec \right ) \right ] = -\frac{1}{n_e m_e} \nabla p_e.
\end{equation*}
Using the vector identity $\nabla \left ( F^2 / 2 \right ) = \Fvec \times \left ( \nabla \times \Fvec \right ) + \Fvec \cdot \nabla \Fvec$, we write the above as
\begin{equation}
    \frac{\partial\uvec_e}{\partial t} - \uvec_e \times \left ( \nabla \times \uvec_e \right ) + \nabla \left (\frac{u_e^2}{2} \right ) + \frac{e}{m_e} \left [ -\nabla \phi - \frac{\partial \Avec}{\partial t} + \uvec_e \times \left ( \nabla \times \Avec \right ) \right ] = -\frac{1}{n_e m_e} \nabla p_e,
\end{equation}
which is equivalent to 
\begin{equation}
    \label{eq:emp_electron_momentum}
    \frac{\partial\uvec_e}{\partial t} - \uvec_e \times \left ( \nabla \times \uvec_{e,t} \right ) + \nabla \left (\frac{u_e^2}{2} \right ) + \frac{e}{m_e} \left [ -\nabla \phi - \frac{\partial \Avec}{\partial t} + \uvec_e \times \left ( \nabla \times \Avec \right ) \right ] = -\frac{1}{n_e m_e} \nabla p_e.
\end{equation}
We'll now introduce a more specific coordinate system. We'll be dealing with at most three waves at a time, a laser wave, a scattered wave, and a plasma wave (either electron-plasma or ion-acoustic wave). We'll assume all three of these waves lie on a so-called base plane. That is, $\kvec_e$ (or $\kvec_i$), $\kvec_L$, and $\kvec_s$ are all on this plane. We now choose the main transverse direction, that is, the direction of $\uvec_{e,t}$, $\Jvec_t$, $\Evec_t$, and $\Avec$ to be the direction orthogonal to this plane, so that these vectors are orthogonal to any $\kvec$. As an aside, we note that the longitudinal and transverse components of the electron velocity can belong to different waves. That is
\begin{align}
    \uvec_{e,l} &= \hat{\uvec}_{e,l} \exp \left [ i (\kvec_p \cdot \xvec - w_pt) \right ] \\
    \uvec_{e,t} &= \hat{\uvec}_{e,t} \exp \left [ i (\kvec_q \cdot \xvec - w_qt) \right ].
\end{align}
Note that since all wave vectors point along the base plane, $u_e^2$, $\phi$ and $p_e$ will only vary along that plane, and thus their gradients will be confined to that plane. Thus, the component of \cref{eq:emp_electron_momentum} along the main transverse direction is
\begin{equation}
    \label{eq:emp_electron_momentum_transverse}
    \frac{\partial\uvec_{e,t}}{\partial t} - \uvec_{e,l} \times \left ( \nabla \times \uvec_{e,t} \right ) + \frac{e}{m_e} \left [ - \frac{\partial \Avec}{\partial t} + \uvec_{e,l} \times \left ( \nabla \times \Avec \right ) \right ] = 0.
\end{equation}
Using $c = w/k$, we show the following scalings 
\begin{align}
    \frac{1}{c^2} \frac{\partial \uvec_{e,t}}{\partial t} &= -\frac{iw \uvec_{e,t}}{c^2} \sim i \frac{\uvec_{e,t}}{c} k , \nonumber \\
    \frac{1}{c^2} \uvec_{e,l} \times \left (\nabla \times \uvec_{e,t} \right ) & = \frac{i \uvec_{e,l} \times \left ( \kvec \times \uvec_{e,t} \right ) }{c^2} \sim i \frac{\uvec_{e,l}}{c} \frac{\uvec_{e,t}}{c} k , \nonumber \\
    \frac{1}{c^2} \frac{\partial \Avec}{\partial t} &= -\frac{iw \Avec}{c^2} \sim i \frac{\Avec}{c} k , \nonumber \\
    \frac{1}{c^2} \uvec_{e,l} \times \left ( \nabla \times \Avec \right ) &= \frac{i \uvec_{e,l} \times \left ( \kvec \times \Avec \right )}{c^2} \sim i \frac{\uvec_{e,l}}{c} \frac{\Avec}{c} k .
\end{align}
Thus, assuming $\uvec_{e,l} \ll c$, the terms involving the double cross product are smaller than those involving the time derivative. As a result, \cref{eq:emp_electron_momentum_transverse} becomes
\begin{equation}
    \frac{\partial\uvec_{e,t}}{\partial t} - \frac{e}{m_e} \frac{\partial \Avec}{\partial t} = 0.
\end{equation}
The above is equivalent to 
\begin{equation}
    -i w \uvec_{e,t} +i w \frac{e \Avec}{m_e} = 0,
\end{equation}
which upon re-arranging gives
\begin{equation}
    \label{eq:emp_transverse_velocity}
    \uvec_{e,t} = \frac{e\Avec}{m_e}.
\end{equation}

Using both the transverse current given by \cref{eq:emp_transverse_current} and the transverse velocity given by \cref{eq:emp_transverse_velocity}, \cref{eq:emp_general_trans_vec_pot} can be re-written as
\begin{equation*}
    \frac{\partial^2 \Avec}{\partial t} - c_0^2 \nabla^2 \Avec = -\frac{e n_e}{\epsilon_0} \uvec_{e,t} = -\frac{e^2 n_e}{\epsilon_0 m_e} \Avec.
\end{equation*}
We now use the decomposition $n_e = n_{e0} + n_{e1}$, where $n_{e0}$ is time independent. The above becomes
\begin{equation}
    \label{eq:emp_general_trans_vec_pot_complete}
    \frac{\partial^2 \Avec}{\partial t} + w_{pe}^2 \Avec - c_0^2 \nabla^2 \Avec = -\frac{e^2 n_{e1}}{\epsilon_0 m_e} \Avec,
\end{equation}
where $w_{pe}^2 = e^2 n_{e0} / m_e \epsilon_0$.

%--------------------------------------------
\section{Electromagnetic waves in a stable plasma}
%--------------------------------------------
We start with \cref{eq:emp_general_trans_vec_pot_complete}, but focus on the stable-plasma case, that is, $n_{e1} = 0$. Thus, we have
\begin{equation}
    \label{eq:temp_general_trans_vec_pot_complete}
    \frac{\partial^2 \Avec}{\partial t} + w_{pe}^2 \Avec - c_0^2 \nabla^2 \Avec = 0.
\end{equation}
We note that $n_{e0}$ is only time independent, that is, it is still allowed to vary across space. As a result, $w_{pe}^2$ is also allowed to vary across space. We'll now use the functional form in \cref{eq:em_general_wave_form} for $\Avec$. However, we'll express the vector potential as $\Avec = \tilde{\Avec} \exp(-iwt)$, where $\tilde{\Avec} = \hat{\Avec} \exp(i\kvec \cdot \xvec)$. Thus, \cref{eq:temp_general_trans_vec_pot_complete} gives
\begin{equation}
    -w^2 \Avec + w_{pe}^2 \Avec - c_0^2 \nabla^2 \Avec = 0.
\end{equation}
We re-write the above as
\begin{equation}
    \frac{w^2}{c_0^2} \Avec - \frac{w^2}{c_0^2} \frac{w_{pe}^2}{w^2} \Avec + \nabla^2 \Avec = 0.
\end{equation}
Defining $\epsilon = 1 - w_{pe}^2 / w^2$, we ultimately get
\begin{equation}
    \label{eq:temp_general_trans_vec_pot_complete_spec}
    \frac{w^2}{c_0^2} \epsilon \Avec + \nabla^2 \Avec = 0.
\end{equation}

Taking the time derivative of \cref{eq:temp_general_trans_vec_pot_complete} gives the equation for $\Evec_t$, that is
\begin{equation}
    \label{eq:temp_general_trans_E_complete}
    \frac{\partial^2 \Evec_t}{\partial t^2} + w_{pe}^2 \Evec_t - c_0^2 \nabla^2 \Evec_t = 0.
\end{equation}
Using $\Evec_t = \tilde{\Evec}_t \exp(-iwt)$, where $\tilde{\Evec}_t = \hat{\Evec}_t \exp(i \kvec \cdot \xvec)$, we get
\begin{equation}
    -w^2 \Evec_t + w_{pe}^2 \Evec_t - c_0^2 \nabla^2 \Evec_t = 0.
\end{equation}
We re-write the above as 
\begin{equation}
    \frac{w^2}{c_0^2} \Evec_t - \frac{w^2}{c_0^2} \frac{w_{pe}^2}{w^2} \Evec_t + \nabla^2 \Evec_t = 0,
\end{equation}
which becomes
\begin{equation}
    \frac{w^2}{c_0^2} \epsilon \Evec_t + \nabla^2 \Evec_t = 0.
\end{equation}

Taking the curl of \cref{eq:emp_general_trans_vec_pot_complete} gives the equation for $\Bvec$, that is
\begin{equation}
    \label{eq:emp_general_trans_B_complete}
    \frac{\partial^2 \Bvec}{\partial t^2} + w_{pe}^2 \Bvec - c_0^2 \nabla^2 \Bvec + \nabla w_{pe}^2 \times \Avec = 0.
\end{equation}
Using $\Bvec = \tilde{\Bvec} \exp(-iwt)$, where $\tilde{\Bvec} = \hat{\Bvec} \exp(i \kvec \cdot \xvec)$, we get
\begin{equation}
    -w^2 \Bvec + w_{pe}^2 \Bvec - c_0^2 \nabla^2 \Bvec + \nabla w_{pe}^2 \times \Avec = 0.
\end{equation}
We re-write the above as
\begin{equation}
    \frac{w^2}{c_0^2} \Bvec - \frac{w^2}{c_0^2} \frac{w_{pe}^2}{w^2} \Bvec + \nabla^2 \Bvec - \frac{w^2}{c_0^2} \nabla \left ( \frac{w_{pe}^2}{w^2} \right ) \times \Avec = 0,
\end{equation}
which becomes
\begin{equation}
    \frac{w^2}{c_0^2} \epsilon \Bvec + \nabla^2 \Bvec + \frac{w^2}{c_0^2} \nabla \epsilon \times \Avec = 0.
\end{equation}
Using \cref{eq:temp_general_trans_vec_pot_complete_spec} we get 
\begin{equation}
    \frac{w^2}{c_0^2} \epsilon \Bvec + \nabla^2 \Bvec - \frac{1}{\epsilon} \nabla \epsilon \times \nabla^2 \Avec = 0.
\end{equation}
The vector identity $\nabla \times \left ( \nabla \times \Fvec \right ) = \nabla \left ( \nabla \cdot \Fvec \right ) - \nabla^2 \Fvec$ gives $\nabla^2 \Avec = - \nabla \times \left ( \nabla \times \Avec \right ) = - \nabla \times \Bvec$. Thus, we finally get
\begin{equation}
    \frac{w^2}{c_0^2} \epsilon \Bvec + \nabla^2 \Bvec + \frac{1}{\epsilon} \nabla \epsilon \times \left ( \nabla \times \Bvec \right ) = 0.
\end{equation}

%-------------------------------------------------------------------------------
\section{Stimulated Raman and Brillouin instabilities}
%-------------------------------------------------------------------------------
%-------------------------------------------------------------------------------
\subsection{Linearization}
%-------------------------------------------------------------------------------
The following decompositions will be used in the derivation of stimulated Raman and Brillouin instabilities:
\begin{align}
    n_i &= n_{i0} + n_{i1}, \nonumber \\
    n_e &= n_{e0} + n_{e1}, \nonumber \\
    p_i &= p_{i0} + p_{i1}, \nonumber \\
    p_e &= p_{e0} + p_{e1}, \nonumber \\
    \uvec_{i,l} &= \uvec_{i0,l} + \uvec_{i1,l}, \nonumber \\
    \uvec_{e,l} &= \uvec_{e0,l} + \uvec_{e1,l}, \nonumber \\
    \Evec_l &= \Evec_{0,l} + \Evec_{1,l}, \nonumber \\
    \Avec &= \Avec_L + \Avec_s.
\end{align}
For these decompositions, we'll assume
\begin{enumerate}
    \item Terms with a subscript 1 are small and thus products of two small quantities can be neglected. \label{it:p_instabilities_assumption_1}
    \item $\uvec_{i0,l}$, $\uvec_{e0,l}$, and $\Evec_{0,l}$,are zero. \label{it:p_instabilities_assumption_2}
    \item $n_{i0}$, $n_{e0}$, $p_{i0}$, and $p_{e0}$ are uniform in space and time. \label{it:p_instabilities_assumption_3}
\end{enumerate}

Thus, unlike the previous section, we do not assume the plasma is stable, that is, we assume fluctuations such as $n_{e1}$ are small but non zero. $\Avec_L$ is the vector potential associated with the laser light, and $\Avec_s$ is the potential associated with the scattered light. For linearization purposes, we'll assume $\Avec_s$ is small. 

Using the decomposition for $\Avec$, \cref{eq:emp_general_trans_vec_pot_complete} is written as 
\begin{equation}
    \frac{\partial^2 \Avec_L}{\partial t} + \frac{\partial^2 \Avec_s}{\partial t} + w_{pe}^2 \Avec_L + w_{pe}^2 \Avec_s - c_0^2 \nabla^2 \Avec_L - c_0^2 \nabla^2 \Avec_s = -\frac{e^2 n_{e1}}{\epsilon_0 m_e} \Avec_L -\frac{e^2 n_{e1}}{\epsilon_0 m_e} \Avec_s,
\end{equation}
Dropping products of small quantities we have
\begin{equation}
    \label{eq:srs_temp1}
    \frac{\partial^2 \Avec_L}{\partial t} + \frac{\partial^2 \Avec_s}{\partial t} + w_{pe}^2 \Avec_L + w_{pe}^2 \Avec_s - c_0^2 \nabla^2 \Avec_L - c_0^2 \nabla^2 \Avec_s = -\frac{e^2 n_{e1}}{\epsilon_0 m_e} \Avec_L.
\end{equation}
We'll assume the laser light is stable, that is, it satisfies \cref{eq:temp_general_trans_vec_pot_complete}, which we re-write below
\begin{equation}
    \label{eq:srs_temp2}
    \frac{\partial^2 \Avec_L}{\partial t} + w_{pe}^2 \Avec_L - c_0^2 \nabla^2 \Avec_L = 0.
\end{equation}
Thus, \cref{eq:srs_temp1} becomes
\begin{equation}
    \frac{\partial^2 \Avec_s}{\partial t} + w_{pe}^2 \Avec_s - c_0^2 \nabla^2 \Avec_s = -\frac{e^2 n_{e1}}{\epsilon_0 m_e} \Avec_L.
\end{equation}
The above shows that the fluctuating $n_{e1}$ couples with the laser light to serve as a source for the scattered light.

The electron density equation is now written as
\begin{equation*}
    \frac{\partial n_{e0} + n_{e1}}{\partial t } + \nabla \cdot \left [ \left ( n_{e0} + n_{e1} \right )\left ( \uvec_{e,t} + \uvec_{e0,l} + \uvec_{e1,l} \right ) \right ] = 0,
\end{equation*}
which, since $\uvec_{e,t}$ is transverse, can be written as
\begin{equation*}
    \frac{\partial n_{e0} + n_{e1}}{\partial t } + \uvec_{e,t} \cdot \nabla \left (n_{e0} + n_{e1} \right ) + \nabla \cdot \left [ \left ( n_{e0} + n_{e1} \right )\left ( \uvec_{e0,l} + \uvec_{e1,l} \right ) \right ] = 0.
\end{equation*}
Given the assumptions in \cref{it:p_instabilities_assumption_1,it:p_instabilities_assumption_2,it:p_instabilities_assumption_3}, the above simplifies to
\begin{equation*}
    \frac{\partial n_{e1}}{\partial t } + \uvec_{e,t} \cdot \nabla n_{e1} + \nabla \cdot \left ( n_{e0} \uvec_{e1,l} \right ) = 0.
\end{equation*}
Since $\uvec_{e,t}$ and $\nabla n_{e1}$ are orthogonal, we finally have
\begin{equation}
    \label{eq:p_instabilities_e_den_linearized}
    \frac{\partial n_{e1}}{\partial t} + \nabla \cdot \left ( n_{e0} \uvec_{e1,l} \right ) = 0.
\end{equation}

The ion density equation is now written as
\begin{equation*}
    \frac{\partial n_{i0} + n_{i1}}{\partial t } + \nabla \cdot \left [ \left ( n_{i0} + n_{i1} \right )\left ( \uvec_{i,t} + \uvec_{i0,l} + \uvec_{i1,l} \right ) \right ] = 0,
\end{equation*}
As stated in \cref{sec:electromagnetic_waves_plasmas}, it is often the case that transverse waves oscillate at such a fast rate that the ions, which have large inertia, are unable to react on comparable time scales. Thus, we can assume $\uvec_{i,t} = 0$,
\begin{equation*}
    \frac{\partial n_{i0} + n_{i1}}{\partial t } + \nabla \cdot \left [ \left ( n_{i0} + n_{i1} \right )\left ( \uvec_{i0,l} + \uvec_{i1,l} \right ) \right ] = 0.
\end{equation*}
Given the assumptions in \cref{it:p_instabilities_assumption_1,it:p_instabilities_assumption_2,it:p_instabilities_assumption_3}, the above simplifies to
\begin{equation}
    \label{eq:p_instabilities_i_den_linearized}
    \frac{\partial n_{i1}}{\partial t } + \nabla \cdot \left ( n_{i0} \uvec_{i1,l} \right ) = 0.
\end{equation}

Consider the electron momentum equation. Subtracting \cref{eq:emp_electron_momentum_transverse} from \cref{eq:emp_electron_momentum} gives
\begin{equation}
    \label{eq:emp_electron_momentum_longitudinal}
    \frac{\partial\uvec_{e,l}}{\partial t} - \uvec_{e,t} \times \left ( \nabla \times \uvec_{e,t} \right ) + \nabla \left (\frac{u_e^2}{2} \right ) + \frac{e}{m_e} \left [ -\nabla \phi +\uvec_{e,t} \times \left ( \nabla \times \Avec \right ) \right ] = -\frac{1}{n_e m_e} \nabla p_e.
\end{equation}
Since $\uvec_{e,t} = e \Avec / m_e$, the above simplifies to
\begin{equation*}
    \frac{\partial\uvec_{e,l}}{\partial t} + \nabla \left (\frac{u_e^2}{2} \right ) - \frac{e}{m_e} \nabla \phi = -\frac{1}{n_e m_e} \nabla p_e,
\end{equation*}
or 
\begin{equation*}
    \frac{\partial\uvec_{e,l}}{\partial t} + \nabla \left (\frac{u_e^2}{2} \right ) + \frac{e}{m_e} \Evec_l = -\frac{1}{n_e m_e} \nabla p_e.
\end{equation*}
Since $\uvec_{e,l}$ and $\uvec_{e,t}$ are orthogonal $u_e^2 = \uvec_e \cdot \uvec_e = u_{e,l}^2 + u_{e,t}^2$. The electron momentum equation is then 
\begin{equation*}
    \frac{\partial\uvec_{e,l}}{\partial t} + \nabla \left (\frac{u_{e,l}^2 + u_{e,t}^2}{2} \right ) + \frac{e}{m_e} \Evec_l = -\frac{1}{n_e m_e} \nabla p_e.
\end{equation*}
Given the assumptions in \cref{it:p_instabilities_assumption_1,it:p_instabilities_assumption_2,it:p_instabilities_assumption_3}, the above simplifies to
\begin{equation*}
    \frac{\partial n_{e0} \uvec_{e1,l}}{\partial t} + n_{e0} \nabla \left (\frac{u_{e,t}^2}{2} \right ) + \frac{e n_{e0}}{m_e} \Evec_{1,l} = -\frac{1}{m_e} \nabla p_{e1} .
\end{equation*}
For the transverse electron velocity we have
\begin{equation*}
    u_{e,t}^2 = \left ( \frac{e \Avec}{m_e} \right ) \cdot \left ( \frac{e \Avec}{m_e} \right ) = \frac{e^2}{m_e^2} \left ( \Avec_L \cdot \Avec_ L + 2 \Avec_L \cdot \Avec_s + \Avec_s \cdot \Avec_s \right ).
\end{equation*}
Since the product of small quantities can be neglected, the $\Avec_s \cdot \Avec_s$ term is dropped. We'll also ignore the $\Avec_L \cdot \Avec_L$ term, given that $\Avec_L$ is stable and thus its magnitude does not play a critical role in the growth of the instabilities. Thus, the electron momentum equation becomes
\begin{equation}
    \label{eq:p_instabilities_e_mom_linearized}
    \frac{\partial n_{e0} \uvec_{e1,l}}{\partial t} + \frac{e^2 n_{e0}}{m_e^2} \nabla \left (\Avec_L \cdot \Avec_s \right ) + \frac{e n_{e0}}{m_e} \Evec_{1,l} = -\frac{1}{m_e} \nabla p_{e1}.
\end{equation}

Consider now the ion momentum equation, given by \cref{eq:pwaves_ion_momentum}, which we re-write below as 
\begin{equation*}
    \frac{\partial n_i \uvec_i}{\partial t} + \nabla \cdot \left ( n_i \uvec_i \uvec_i \right ) - \frac{Ze n_i}{m_i} \left ( \Evec + \uvec_i \times \Bvec \right ) = -\frac{1}{m_i} \nabla p_i,
\end{equation*}
The longitudinal component of the above is
\begin{equation*}
    \frac{\partial n_i \uvec_{i,l}}{\partial t} + \left [ \nabla \cdot \left (n_i \uvec_i \uvec_i \right ) \right ]_l - \frac{Z e n_i}{m_i} \left ( \Evec_l + \uvec_{i,t} \times \Bvec \right ) = -\frac{1}{m_i} \nabla p_i,
\end{equation*}
where $[\cdot]_l$ denotes longitudinal component. Since $\uvec_{i,t} = 0$, we have
\begin{equation*}
    \frac{\partial n_i \uvec_{i,l}}{\partial t} + \left [ \nabla \cdot \left (n_i \uvec_i \uvec_i \right ) \right ]_l - \frac{Z e n_i}{m_i} \Evec_l = -\frac{1}{m_i} \nabla p_i.
\end{equation*}
Using the variable decompositions, we have
\begin{multline*}
    \frac{\partial}{\partial t} \left [ \left ( n_{i0} + n_{i1} \right ) \left ( \uvec_{i0,l} + \uvec_{i1,l} \right ) \right ] \\
    + \left \{ \nabla \cdot \left [ \left (n_{i0} + n_{i1} \right ) \left ( \uvec_{i,t} + \uvec_{i0,l} + \uvec_{i1,l} \right ) \left ( \uvec_{i,t} + \uvec_{i0,l} + \uvec_{i1,l} \right ) \right ] \right \}_l \\
    - \frac{Z e}{m_i} \left ( n_{i0} + n_{i1} \right ) \left ( \Evec_{0,l} + \Evec_{1,l} \right ) = - \frac{1}{m_i} \nabla \left ( p_{i0} + p_{i1} \right ).
\end{multline*}
Given the assumptions in \cref{it:p_instabilities_assumption_1,it:p_instabilities_assumption_2,it:p_instabilities_assumption_3}, the above simplifies to
\begin{equation}
    \label{eq:p_instabilities_i_mom_linearized}
    \frac{\partial n_{i0} \uvec_{i1,l}}{\partial t} - \frac{Z e n_{i0}}{m_i} \Evec_{1,l} = - \frac{1}{m_i} \nabla p_{i1}.
\end{equation}

%-------------------------------------------------------------------------------
\subsection{Stimulated Raman Scattering}
%-------------------------------------------------------------------------------
We employ the same assumptions as for the electron-plasma waves, that is 
\begin{enumerate}
    \item Quasi-neutrality for the base flow, $Zn_{i0} = n_{e0}$.
    \item Uniform ion density, $n_{i1} = 0$.
\end{enumerate}

Combining \cref{eq:p_instabilities_e_mom_linearized} with \cref{eq:p_waves_e_pressure_linearized} gives 
\begin{equation}
    \label{eq:srs_mom_linearized}
    \frac{\partial n_{e0} \uvec_{e1,l}}{\partial t} + \frac{e^2 n_{e0}}{m_e^2} \nabla \left (\Avec_L \cdot \Avec_s \right ) + \frac{e n_{e0}}{m_e} \Evec_{1,l} = -\frac{\gamma_e p_{e0}}{n_{e0} m_e} \nabla n_{e1}.
\end{equation}
Taking the time derivative of \cref{eq:p_instabilities_e_den_linearized} and using \cref{eq:srs_mom_linearized} leads to the wave equation for electron density
\begin{equation*}
    \frac{\partial^2 n_{e1}}{\partial t^2} - \frac{e^2 n_{e0}}{m_e^2} \nabla^2 \left (\Avec_L \cdot \Avec_s \right ) - \frac{e n_{e0}}{m_e} \nabla \cdot \Evec_{1,l} = \frac{\gamma_e p_{e0}}{n_{e0} m_e} \nabla^2 n_{e1}.
\end{equation*}
As before, using \cref{eq:ep_waves_efield_divergence} we obtain
\begin{equation*}
    \frac{\partial^2 n_{e1}}{\partial t^2} - \frac{e^2 n_{e0}}{m_e^2} \nabla^2 \left (\Avec_L \cdot \Avec_s \right ) + \frac{e^2 n_{e0}}{m_e \epsilon_0} n_{e1} = \frac{\gamma_e p_{e0}}{n_{e0} m_e} \nabla^2 n_{e1}.
\end{equation*}
or
\begin{equation}
    \frac{\partial^2 n_{e1}}{\partial t^2} + w_{pe}^2 n_{e1} - \frac{\gamma_e p_{e0}}{n_{e0} m_e} \nabla^2 n_{e1} =   \frac{e^2 n_{e0}}{m_e^2} \nabla^2 \left (\Avec_L \cdot \Avec_s \right ).
\end{equation}
Thus, the scattered laser light $\Avec_s$ couples with the laser light to serve as a source for the electron-plasma wave.

%-------------------------------------------------------------------------------
\subsection{Stimulated Brillouin Scattering}
%-------------------------------------------------------------------------------
We employ the same assumptions as for the ion-acoustic waves, that is
\begin{enumerate}
    \item Quasi-neutrality for the base flow, $Zn_{i0} = n_{e0}$.
    \item Approximate quasi-neutrality for the fluctuations, $Z n_{i1} \approx n_{e1}$.
    \item Negligible electron mass, $m_e \to 0$.
\end{enumerate}
Combining \cref{eq:p_instabilities_i_mom_linearized} with \cref{eq:p_waves_i_pressure_linearized} gives
\begin{equation}
    \label{eq:sbs_mom_linearized}
    \frac{\partial n_{i0} \uvec_{i1,l}}{\partial t} - \frac{Z e n_{i0}}{m_i} \Evec_{1,l} = - \frac{\gamma_i p_{i0}}{n_{i0} m_i} \nabla n_{i1}.
\end{equation}
Taking the time derivative of \cref{eq:p_instabilities_i_den_linearized} and using \cref{eq:sbs_mom_linearized} leads to the wave equation for ion density
\begin{equation}
    \frac{\partial^2 n_{i1}}{\partial t^2} + \frac{Z e n_{i0}}{m_i} \nabla \cdot \Evec_{1,l} = \frac{\gamma_i p_{i0}}{n_{i0} m_i} \nabla^2 n_{i1}.
\end{equation}
For this case, we assume that the mass of the electron, which is significantly smaller than that of the ions, is negligible. Thus, \cref{eq:p_instabilities_e_mom_linearized} simplifies to 
\begin{equation}
    \frac{e^2 n_{e0}}{m_e} \nabla \left (\Avec_L \cdot \Avec_s \right ) + e n_{e0} \Evec_{1,l} = -\frac{\gamma_e p_{e0}}{n_{e0}} \nabla n_{e1}.
\end{equation}
Plugging in the above in the ion wave equation we obtain
\begin{equation}
    \frac{\partial^2 n_{i1}}{\partial t^2} = \frac{Z n_{i0}}{n_{e0}} \frac{\gamma_e p_{e0}}{n_{e0} m_i} \nabla^2 n_{e1} + \frac{\gamma_i p_{i0}}{n_{i0} m_i} \nabla^2 n_{i1} + \frac{Z e^2 n_{i0}}{m_i m_e} \nabla^2 \left (\Avec_L \cdot \Avec_s \right ).
\end{equation}
Due to quasi-neutrality, we have $Zn_{i0} = n_{e0}$ and $Zn_{i1} \approx n_{e1}$, which gives
\begin{equation}
    \frac{\partial^2 n_{i1}}{\partial t^2} = \frac{1}{m_i} \left ( \frac{Z \gamma_e p_{e0}}{n_{e0}} + \frac{\gamma_i p_{i0}}{n_{i0}} \right ) \nabla^2 n_{i1} + \frac{Z e^2 n_{i0}}{m_i m_e} \nabla^2 \left (\Avec_L \cdot \Avec_s \right ).
\end{equation}
Since $p_{i0}/n_{i0} = k_B T_{i0}$ and $p_{e0} / n_{e0} = k_B T_{e0}$, we finally have
\begin{equation}
    \frac{\partial^2 n_{i1}}{\partial t^2} - \left ( \frac{Z \gamma_e k_B T_{e0} + \gamma_i k_B T_{i0}}{m_i} \right ) \nabla^2 n_{i1} = \frac{Z e^2 n_{i0}}{m_i m_e} \nabla^2 \left (\Avec_L \cdot \Avec_s \right ).
\end{equation}
Thus, the scattered laser light $\Avec_s$ couples with the laser light to serve as a source for the ion-acoustic wave.

%-------------------------------------------------------------------------------
\chapter{Instabilities}
%-------------------------------------------------------------------------------

%-------------------------------------------------------------------------------
\section{Linear stability analysis}
%-------------------------------------------------------------------------------
Consider the following system of PDEs for a two-dimensional problem
\begin{equation}
    \frac{\partial w}{\partial t} = \frac{\partial \phi}{\partial y}\frac{\partial w}{\partial x} - \frac{\partial \phi}{\partial x} \frac{\partial w}{\partial y},
\end{equation}
\begin{equation}
    w = \nabla^2 \phi .
\end{equation}
In the above, $\phi = \phi(x,y,t)$ is the electrostatic potential and $w = w(x,y,t)$ the vorticity.

The analysis begins by splitting the variables into equilibrium and fluctuating components, namely
\begin{equation}
    \phi = \phi_0 + \phi_1,
\end{equation}
\begin{equation}
    w = w_0 + w_1.
\end{equation}
For the above, $\phi_0 = \phi_0(x)$, $w_0 = w_0(x)$, and $\phi_1 = \phi_1(x,y,t)$, $w_1 = w_1(x,y,t)$. We now introduce a Fourier series decomposition for the fluctuating variables, and focus on a single Fourier mode as follows 
\begin{equation}
    \phi_1(x,y,t) = F(x,k_y,t) e^{ik_y y} = \tilde{\phi}(x,k_y) e^{\gamma t + i k_y y},
\end{equation}
\begin{equation}
    w_1(x,y,t) = G(x,k_y,t) e^{ik_y y} = \tilde{w}(x,k_y) e^{\gamma t + i k_y y}.
\end{equation}
In the above, $\gamma$, which can be complex, is the growth rate factor, and $\tilde{\phi} = \tilde{\phi}(x, k_y)$, $\tilde{w} = \tilde{w}(x, k_y)$ are the remaining part of the Fourier coefficient.

We then plug in the decompositions for $\phi$ and $w$ in the governing PDEs. Collecting the lowest order terms leads to equations for the equilibrium solution. For example, the Poisson equation to lowest order is
\begin{equation}
    \label{eq:lin_stab_w0_phi0}
    w_0 = \nabla^2 \phi_0 = \frac{\partial^2 \phi_0}{\partial x^2}.
\end{equation}
Combining terms up to next order gives
\begin{equation}
    \label{eq:lin_stab_pde_mid_order}
    \frac{\partial w_1}{\partial t} = \frac{\partial \phi_0}{\partial y}\frac{\partial w_1}{\partial x} + \frac{\partial \phi_1}{\partial y}\frac{\partial w_0}{\partial x} - \frac{\partial \phi_0}{\partial x} \frac{\partial w_1}{\partial y} - \frac{\partial \phi_1}{\partial x} \frac{\partial w_0}{\partial y},
\end{equation}
\begin{equation}
    w_1 = \nabla^2 \phi_1.
\end{equation}
Using the expression for $\phi_1$ in the Poisson equation above leads to 
\begin{equation}
    w_1 = \left ( \frac{\partial^2 \tilde{\phi}}{\partial x^2} - k_y^2 \tilde{\phi} \right ) e^{\gamma t + i k_y y},
\end{equation}
or 
\begin{equation}
    \tilde{w} = \frac{\partial^2 \tilde{\phi}}{\partial x^2} - k_y^2 \tilde{\phi} .
\end{equation}
We'll now evaluate each of the terms in \cref{eq:lin_stab_pde_mid_order}.
\begin{align}
    \frac{\partial w_1}{\partial t} &= \gamma \left ( \frac{\partial^2 \tilde{\phi}}{\partial x^2} - k_y^2 \tilde{\phi} \right ) e^{\gamma t + i k_y y}, \nonumber \\
    \frac{\partial \phi_0}{\partial y}\frac{\partial w_1}{\partial x} &= 0, \nonumber \\
    \frac{\partial \phi_1}{\partial y}\frac{\partial w_0}{\partial x} &= i k_y \tilde{\phi} e^{\gamma t + i k_y y} \frac{\partial^3 \phi_0}{\partial x^3}, \nonumber \\
    \frac{\partial \phi_0}{\partial x} \frac{\partial w_1}{\partial y} &= \frac{\partial \phi_0}{\partial x} i k_y \left ( \frac{\partial^2 \tilde{\phi}}{\partial x^2} - k_y^2 \tilde{\phi} \right ) e^{\gamma t + i k_y y},\nonumber \\
    \frac{\partial \phi_1}{\partial x} \frac{\partial w_0}{\partial y} &= 0 . \nonumber \\
\end{align}
Combining all of the above, we obtain
\begin{equation}
    \gamma \left ( \frac{\partial^2 \tilde{\phi}}{\partial x^2} - k_y^2 \tilde{\phi} \right ) = ik_y \tilde{\phi} \frac{\partial^3 \phi_0}{\partial x^3} - \frac{\partial \phi_0}{\partial x} i k_y \left ( \frac{\partial^2 \tilde{\phi}}{\partial x^2} - k_y^2 \tilde{\phi} \right ).
\end{equation}
This is re-written as 
\begin{equation}
    \gamma \left ( \frac{\partial^2 \tilde{\phi}}{\partial x^2} - k_y^2 \tilde{\phi} \right ) = -i k_y \left [ \frac{\partial \phi_0}{\partial x} \left ( \frac{\partial^2 \tilde{\phi}}{\partial x^2} - k_y^2 \tilde{\phi} \right ) - \frac{\partial^3 \phi_0}{\partial x^3} \tilde{\phi} \right ].
\end{equation}

%-------------------------------------------------------------------------------
\chapter{Simple models}
%-------------------------------------------------------------------------------
%--------------------------------------------
\section{Hasegawa-Mima}
%--------------------------------------------
References for this model can be found in \cite{hasegawa1977,horton1994}.

%--------------------------------------------
\subsection{Assumptions}
%--------------------------------------------
\begin{enumerate}
    \item Singly-charged ions. \label{it:hm_single_charge_ions}
    \item No shear stresses, collisions, or sources. \label{it:hm_no_shear_source_coll}
    \item Cold ion approximation, i.e. $T_e \gg T_i$ and thus $\nabla p_i \approx 0$, \cite{hasegawa1977}. \label{it:hm_cold}
    \item Isothermal electron fluid, i.e.\@ $T_e$ is constant. \label{it:hm_isothermal_electron}
    \item Electrostatic field, i.e. $\Evec = - \nabla \phi$. \label{it:hm_electrostatic}
    \item Magnetic field is constant.
    \item Neglect parallel ion velocity, i.e. $u_{i,||} \approx 0$, \cite{hasegawa1977}. \label{it:hm_par_ion}
    \item Quasi-neutrality, i.e. $n_i \approx n_e$. \label{it:hm_quasineutrality}
    \item Adiabatic electrons, i.e. $n_e = n_0 \exp (e\phi/T_e)$, where $n_0 = n_0(x_1)$. \label{it:hm_adiabatic}
\end{enumerate}

%--------------------------------------------
\subsection{Derivation}
%--------------------------------------------
Using the assumptions in \cref{it:hm_single_charge_ions,it:hm_no_shear_source_coll}, the momentum \cref{eq:cons_mom} for ions becomes
\begin{equation}
    m_i n_i \left (\frac{\partial \uvec_i}{\partial t} + \uvec_i \cdot \nabla \uvec_i \right ) = e n_i (\Evec + \uvec_i \times \Bvec) - \nabla p_i.
\end{equation}
Using the assumptions in \cref{it:hm_cold,it:hm_electrostatic}, the above becomes
\begin{equation}
    \frac{\partial \uvec_i}{\partial t} + \uvec_i \cdot \nabla \uvec_i = -\frac{e}{m_i} \nabla \phi + \frac{e}{m_i}\uvec_i \times \Bvec.
\end{equation}
Introduce the coordinate system $\evec_1$, $\evec_2$, $\evec_3$, and assume $\Bvec$ points in the $\evec_3$ direction. Defining the perpendicular velocity as $\uvec_{i,\perp} = [u_{i,1}, u_{i,2}, 0]^T$ and the perpendicular gradient as $\nabla_\perp = [\partial_1, \partial_2, 0]^T$, we have
\begin{equation}
    \frac{\partial \uvec_{i,\perp}}{\partial t} + \uvec_i \cdot \nabla \uvec_{i,\perp} = -\frac{e}{m_i} \nabla_\perp \phi + \frac{e}{m_i}\uvec_i \times \Bvec.
\end{equation}
Using the assumption in \cref{it:hm_par_ion} and noting that $\uvec_{i} \times \Bvec = \uvec_{i,\perp} \times \Bvec$, we obtain
\begin{equation}
\label{eq:hm_momentum_perpendicular}
    \frac{\partial \uvec_{i,\perp}}{\partial t} + \uvec_{i,\perp} \cdot \nabla_\perp \uvec_{i,\perp} = -\frac{e}{m_i} \nabla_\perp \phi + \frac{e}{m_i} \uvec_{i,\perp} \times \Bvec.
\end{equation}

We now introduce the scalings for a characteristic frequency $w$ and length scale $r$
\begin{equation}
    \frac{w}{w_{c,i}} \sim \epsilon \qquad \frac{r_s}{r} \sim \epsilon,
\end{equation}
where $w_{c,i} = eB/m_i$ is the cyclotron frequency, $r_s = v_s/w_{c,i}$ is a reference length scale, and $v_s = \sqrt{T_e/m_i}$ a reference velocity scale. Given these variables, we assume
\begin{equation}
    \frac{\partial \uvec_{i,\perp}}{\partial t} \sim \uvec_{i,\perp} w \qquad \nabla_\perp \uvec_{i,\perp} \sim \frac{\uvec_{i,\perp}}{r} \qquad \Evec \sim \uvec_{i,\perp} B.
\end{equation}
Finally, we introduce the decomposition $\uvec_{i,\perp} = \uvec_{i,\perp}^{(0)} + \uvec_{i,\perp}^{(1)}$, where $\uvec_{i,\perp}^{(0)} \sim v_s$ and $\uvec_{i,\perp}^{(1)} \sim \epsilon v_s$. We use this decomposition in \cref{eq:hm_momentum_perpendicular} and then divide the PDE by $w_{c,i} v_s$. The order of each element in the resulting equation is as follows
\begin{enumerate}
    \item $\frac{\partial \uvec_{i,\perp}^{(0)}}{\partial t} \sim \epsilon$.
    \item $\frac{\partial \uvec_{i,\perp}^{(1)}}{\partial t} \sim \epsilon^2$.
    \item $\uvec_{i,\perp}^{(0)} \cdot \nabla_\perp \uvec_{i,\perp}^{(0)} \sim \epsilon$.
    \item $\uvec_{i,\perp}^{(0)} \cdot \nabla_\perp \uvec_{i,\perp}^{(1)} \sim \epsilon^2$.
    \item $\uvec_{i,\perp}^{(1)} \cdot \nabla_\perp \uvec_{i,\perp}^{(0)} \sim \epsilon^2$.
    \item $\uvec_{i,\perp}^{(1)} \cdot \nabla_\perp \uvec_{i,\perp}^{(1)} \sim \epsilon^3$.
    \item $-\frac{e}{m_i} \nabla_\perp \phi \sim 1$.
    \item $\frac{e}{m_i} \uvec_{i,\perp}^{(0)} \times \Bvec \sim 1$.
    \item $\frac{e}{m_i} \uvec_{i,\perp}^{(1)} \times \Bvec \sim \epsilon$.
\end{enumerate}
Combining the first order terms we obtain
\begin{equation}
    0 = - \nabla_\perp \phi + \uvec_{i,\perp}^{(0)} \times \Bvec,
\end{equation}
which, upon crossing by $\Bvec$, gives
\begin{equation}
\label{eq:hm_e_cross_b_drift}
    \uvec_{i,\perp}^{(0)} = -\nabla_\perp \phi \times \frac{\bvec}{B}.
\end{equation}
Combining the terms of order $\epsilon$ we obtain
\begin{equation}
    \frac{\partial \uvec_{i,\perp}^{(0)}}{\partial t} + \uvec_{i,\perp}^{(0)} \cdot \nabla_\perp \uvec_{i,\perp}^{(0)} = \frac{e}{m_i} \uvec_{i,\perp}^{(1)} \times \Bvec,
\end{equation}
which, upon crossing by $\Bvec$, gives
\begin{equation}
\label{eq:hm_polarization_drift}
    \uvec_{i,\perp}^{(1)} = -\frac{1}{w_{c,i} B} \left [ \frac{\partial \nabla_\perp \phi}{\partial t} + \uvec_{i,\perp}^{(0)} \cdot \nabla_\perp \left ( \nabla_\perp \phi \right ) \right ].
\end{equation}
The above is the polarization drift, cf.\@ \cref{eq:particle_polarization_drift}. The velocity given by \cref{eq:hm_e_cross_b_drift} is referred to as the $E \times B$ drift, and the velocity given by \cref{eq:hm_polarization_drift} as the polarization drift.

Using the assumption in \cref{it:hm_no_shear_source_coll}, the continuity equation for ions is
\begin{equation}
    \frac{\partial n_i}{\partial t} + \nabla \cdot (n_i \uvec_i) = 0.
\end{equation}
Using the assumption in \cref{it:hm_par_ion} the above becomes
\begin{equation}
    \frac{\partial n_i}{\partial t} + \nabla_\perp \cdot (n_i \uvec_{i,\perp}) = 0,
\end{equation}
or
\begin{equation}
    \frac{\partial n_i}{\partial t} +  \uvec_{i,\perp} \cdot \nabla_\perp n_i + n_i \nabla_\perp \cdot \uvec_{i,\perp} = 0,
\end{equation}
One of the main components of the derivation of the Hasegawa-Mima equation is the assumption that advection is governed by the lowest-order velocity only; that is, by $\uvec^{(0)}_{i,\perp}$ and not $\uvec^{(1)}_{i,\perp}$. Thus, the above is written as
\begin{equation}
    \frac{\partial n_i}{\partial t} +  \uvec^{(0)}_{i,\perp} \cdot \nabla_\perp n_i + n_i \nabla_\perp \cdot \uvec_{i,\perp} = 0.
\end{equation}
We note that $\uvec^{(0)}_{i,\perp}$ is divergence free, and thus we have
\begin{equation}
    \frac{\partial n_i}{\partial t} +  \uvec^{(0)}_{i,\perp} \cdot \nabla_\perp n_i + n_i \nabla_\perp \cdot \uvec^{(1)}_{i,\perp} = 0.
\end{equation}
We divide by $n_i$ to express the density in terms of its logarithm 
\begin{equation}
    \label{eq:intermediate_hm_1}
    \frac{\partial \ln n_i}{\partial t} + \uvec^{(0)}_{i,\perp} \cdot \nabla_\perp \ln n_i + \nabla_\perp \cdot \uvec^{(1)}_{i,\perp} = 0.
\end{equation}

We now use the assumptions in \cref{it:hm_adiabatic,it:hm_quasineutrality} to obtain
\begin{equation}
    \ln n_i = \ln \left [ n_0 \exp \left ( \frac{e \phi}{T_e} \right ) \right ] = \ln n_0 + \frac{e \phi}{T_e}.
\end{equation}
Taking into account the fact that $n_0$ is time independent, the continuity equation becomes
\begin{equation}
    \frac{\partial}{\partial t} \left ( \frac{e\phi}{T_e} \right ) + \uvec^{(0)}_{i,\perp} \cdot \nabla_\perp \left [ \ln n_0 + \frac{e \phi}{T_e} \right ] +  \nabla_\perp \cdot \uvec^{(1)}_{i,\perp} = 0.
\end{equation}
Since $\uvec^{(0)}_{i,\perp}$ and $\nabla_\perp \phi$ are orthogonal, the above simplifies to
\begin{equation}
    \frac{\partial}{\partial t} \left ( \frac{e\phi}{T_e} \right ) + \uvec^{(0)}_{i,\perp} \cdot \nabla_\perp \ln n_0 +  \nabla_\perp \cdot \uvec^{(1)}_{i,\perp} = 0,
\end{equation}
which we re-write as
\begin{equation}
    \label{eq:intermediate_hm_2}
    \frac{\partial}{\partial t} \left ( \frac{e\phi}{T_e} \right ) + \uvec^{(0)}_{i,\perp} \cdot \nabla_\perp \ln \left ( \frac{n_0}{w_{c,i}} \right ) +  \nabla_\perp \cdot \uvec^{(1)}_{i,\perp} = 0.
\end{equation}

Given the definition of the polarization drift, we have
\begin{equation}
    \nabla_\perp \cdot \uvec_{i,\perp}^{(1)} = -\frac{1}{w_{c,i} B} \left \{ \frac{\partial \nabla_\perp^2 \phi}{\partial t} + \nabla_\perp \cdot \left [ \uvec_{i,\perp}^{(0)} \cdot \nabla_\perp \left ( \nabla_\perp \phi \right ) \right ] \right \}.
\end{equation}
The second term above is best computed using tensor notation, and we'll use $u_j$ to denote the components of $\uvec_{i,\perp}^{(0)}$. Thus,
\begin{equation}
    \frac{\partial}{\partial x_i} \left [ \left ( u_j \frac{\partial}{\partial x_j} \right) \frac{\partial \phi}{\partial x_i} \right ] = \frac{\partial u_j}{\partial x_i} \frac{\partial^2 \phi}{\partial x_j \partial x_i} + u_j \frac{\partial}{\partial x_j} \left (\frac{\partial^2 \phi}{\partial x_i \partial x_i} \right ).
\end{equation}
Using the definition of $\uvec_{i,\perp}^{(0)}$, the first term on the right-hand side above can be expressed as
\begin{align}
    \frac{\partial u_j}{\partial x_i} \frac{\partial^2 \phi}{\partial x_j \partial x_i} &= -\frac{1}{B^2} \epsilon_{jpq} \frac{\partial^2 \phi}{\partial x_p \partial x_i} B_q \frac{\partial^2 \phi}{\partial x_j \partial x_i} \nonumber \\
    &= -\frac{1}{B^2} \epsilon_{qjp} \frac{\partial^2 \phi}{\partial x_j \partial x_i} \frac{\partial^2 \phi}{\partial x_p \partial x_i} B_q  \nonumber \\
    &= -\frac{1}{B^2} \epsilon_{qjp} \left ( \frac{\partial^2 \phi}{\partial x_j \partial x_1} \frac{\partial^2 \phi}{\partial x_p \partial x_1} + \frac{\partial^2 \phi}{\partial x_j \partial x_2} \frac{\partial^2 \phi}{\partial x_p \partial x_2}  \right ) B_q.
\end{align}
Since $\epsilon_{qjp} \partial_j a \partial_p a \to \nabla a \times \nabla a = 0$ for any scalar $a$, the term above is identically zero. Thus, we have
\begin{equation}
    \label{eq:hm_divergence_polarization}
    \nabla_\perp \cdot \uvec_{i,\perp}^{(1)} = -\frac{1}{w_{c,i} B} \left [ \frac{\partial \nabla_\perp^2 \phi}{\partial t} + \uvec_{i,\perp}^{(0)} \cdot \nabla_\perp \left ( \nabla_\perp^2 \phi \right ) \right ],
\end{equation}
and \cref{eq:intermediate_hm_2} becomes
\begin{equation}
   \frac{\partial}{\partial t}\left ( \frac{1}{w_{c,i} B}\nabla_\perp^2 \phi - \frac{e \phi}{T_e} \right ) + \uvec_{i,\perp}^{(0)} \cdot \nabla_\perp \left [ \frac{1}{w_{c,i} B} \nabla_\perp^2 \phi - \ln \left( \frac{n_0}{w_{c,i}} \right) \right] = 0.
\end{equation}
Plugging in for $\uvec_{i,\perp}^{(0)}$,
\begin{equation}
\label{eq:intermediate_hm_3}
   \frac{\partial}{\partial t}\left ( \frac{1}{w_{c,i} B}\nabla_\perp^2 \phi - \frac{e \phi}{T_e} \right ) - \left( \nabla_\perp \phi \times \frac{\bvec}{B} \right) \cdot \nabla_\perp \left [ \frac{1}{w_{c,i} B} \nabla_\perp^2 \phi - \ln \left ( \frac{n_0}{w_{c,i}} \right) \right ] = 0.
\end{equation}
We now introduce the normalizations
\begin{equation}
    \phi(t,\xvec) = \frac{T_e}{e} \hat{\phi}(\hat{t},\hat{\xvec}) \qquad n_0(x_1) = \hat{n}_0(\hat{x}_1),
\end{equation}
where $\hat{t} = t w_{c,i}$ and $\hat{\xvec} = \xvec / r_s$. Neglecting the hat notation for the sake of simplicity, \cref{eq:intermediate_hm_3} finally becomes
\begin{equation}
    \label{eq:hasegawa_mima}
   \frac{\partial}{\partial t}\left ( \nabla_\perp^2 \phi - \phi \right ) - \left( \nabla_\perp \phi \times \bvec \right) \cdot \nabla_\perp \left [ \nabla_\perp^2 \phi - \ln \left( \frac{n_0}{w_{c,i}} \right) \right ] = 0.
\end{equation}

Using the following expansion
\begin{equation}
    \left ( \nabla_\perp \phi \times \bvec \right ) \cdot \nabla_\perp = \frac{\partial \phi}{\partial x_2} \frac{\partial}{\partial x_1} - \frac{\partial \phi}{\partial x_1} \frac{\partial}{\partial x_2},
\end{equation}
The Hasegawa-Mima equation can be written as
\begin{equation}
   \frac{\partial}{\partial t}\left ( \nabla_\perp^2 \phi - \phi \right ) - \frac{\partial \phi}{\partial x_2} \frac{\partial \nabla_\perp^2 \phi}{\partial x_1} + \frac{\partial \phi}{\partial x_1} \frac{\partial \nabla_\perp^2 \phi}{\partial x_2} + \beta \frac{\partial \phi}{\partial x_2} = 0,
\end{equation}
where 
\begin{equation}
    \beta = \frac{\partial}{\partial x_1} \ln \left ( \frac{n_0}{w_{c,i}} \right ).
\end{equation}

%--------------------------------------------
\subsection{Spectral space}
%--------------------------------------------
In this section we derive the equation for the Fourier coefficient $\hat{\phi}_\nvec = \hat{\phi}(t)_\nvec$, which relates to the potential through the following
\begin{equation}
    \phi(t,x) = \sum_{\nvec=-\infty}^\infty \hat{\phi}_\nvec(t) e^{i \kvec_\nvec \cdot \xvec},
\end{equation}
\begin{equation}
    \hat{\phi}_\nvec(t)= \frac{1}{L^2} \int_{L^2} \phi(t, \xvec) e^{-i \kvec_\nvec \cdot \xvec} \, d\xvec .
\end{equation}
We introduce the operator $\mathcal{F} \{ \}_\nvec$, which is defined by
\begin{equation}
    \mathcal{F} \{ \phi(t,\xvec) \}_\nvec = \frac{1}{L^2} \int_{L^2} \phi(t, \xvec) e^{-i \kvec_\nvec \cdot \xvec} \, d\xvec .
\end{equation}
The equation for $\hat{\phi}_\nvec$ is obtained by applying this operator to \cref{eq:hasegawa_mima}. Thus, the time derivative term in that equation becomes 
\begin{equation}
    \mathcal{F} \left \{ \frac{\partial}{\partial t} \left ( \nabla_\perp^2 - \phi \right ) \right \}_\nvec = \frac{\partial}{\partial t} \mathcal{F} \left \{ \nabla_\perp^2 \phi - \phi \right \}_\nvec = \frac{\partial}{\partial t} \left ( -k_\nvec^2 \hat{\phi}_\nvec - \hat{\phi}_\nvec \right ) = - \left ( 1 + k^2_\nvec \right ) \frac{\partial \hat{\phi}_\nvec}{\partial t}.
\end{equation}
We assume $\nabla \ln (n_o / w_{ci})$ is constant in space. Thus, the term containing the inhomogeneity becomes 
\begin{multline}
    \mathcal{F} \left \{ \left ( \nabla_\perp \phi \times \bvec \right ) \cdot \nabla_\perp \ln \left ( \frac{n_o}{w_{ci}} \right ) \right \}_\nvec \\
    = \left ( \mathcal{F} \left \{ \nabla_\perp \phi \right \}_\nvec \times \bvec \right ) \cdot \nabla_\perp \ln \left ( \frac{n_o}{w_{ci}} \right ) = i \left ( \kvec_\nvec \times \bvec \right ) \cdot \nabla_\perp \ln \left ( \frac{n_o}{w_{ci}} \right ) \hat{\phi}_\nvec.
\end{multline}
The remaining term is computed as follows
\begin{align}
    \mathcal{F} \left \{ - \left (\nabla_\perp \phi \times \bvec \right ) \right . & \left . \cdot \nabla^3_\perp \phi \right \}_\nvec \nonumber \\
    &= \mathcal{F} \left \{ -\sum_{\nvec'=-\infty}^\infty \sum_{\nvec''=-\infty}^\infty \left [ \hat{\phi}_{\nvec'} i \left ( \kvec_{\nvec'} \times \bvec \right ) e^{i \kvec_{\nvec'} \cdot \xvec} \right ] \cdot \left [ \hat{\phi}_{\nvec''} \left ( -i k^2_{\nvec''} \kvec_{\nvec''} \right ) e^{i \kvec_{\nvec''} \cdot \xvec} \right ] \right \}_\nvec \nonumber \\
    &= -\sum_{\nvec'=-\infty}^\infty \sum_{\nvec''=-\infty}^\infty \left ( \kvec_{\nvec'} \times \bvec \right ) \cdot \kvec_{\nvec''} k^2_{\nvec''} \hat{\phi}_{\nvec'} \hat{\phi}_{\nvec''} \mathcal{F} \left \{ e^{i \kvec_{\nvec'} \cdot \xvec} e^{i \kvec_{\nvec''} \cdot \xvec} \right \} \nonumber \\
    &= \sum_{\nvec'=-\infty}^\infty \sum_{\nvec''=-\infty}^\infty \left ( \kvec_{\nvec'} \times \kvec_{\nvec''} \right ) \cdot \bvec k^2_{\nvec''} \hat{\phi}_{\nvec'} \hat{\phi}_{\nvec''} \delta_{\nvec, \nvec' + \nvec''}
\end{align}
Since $\nvec'$ and $\nvec''$ are just symbolic variables for the summation, we can write the above as follows
\begin{align}
    \mathcal{F} \left \{ - \left (\nabla_\perp \phi \times \bvec \right ) \right . & \left . \cdot \nabla^3_\perp \phi \right \}_\nvec \nonumber \\
    =& \frac{1}{2} \sum_{\nvec'=-\infty}^\infty \sum_{\nvec''=-\infty}^\infty \left ( \kvec_{\nvec'} \times \kvec_{\nvec''} \right ) \cdot \bvec k^2_{\nvec''} \hat{\phi}_{\nvec'} \hat{\phi}_{\nvec''} \delta_{\nvec, \nvec' + \nvec''} \nonumber \\
    &+ \frac{1}{2} \sum_{\nvec''=-\infty}^\infty \sum_{\nvec'=-\infty}^\infty \left ( \kvec_{\nvec''} \times \kvec_{\nvec'} \right ) \cdot \bvec k^2_{\nvec'} \hat{\phi}_{\nvec''} \hat{\phi}_{\nvec'} \delta_{\nvec, \nvec'' + \nvec'} \nonumber \\
    =& \sum_{\nvec'=-\infty}^\infty \sum_{\nvec''=-\infty}^\infty \frac{1}{2} \left ( \kvec_{\nvec'} \times \kvec_{\nvec''} \right ) \cdot \bvec \left ( k^2_{\nvec''} - k^2_{\nvec'} \right ) \hat{\phi}_{\nvec'} \hat{\phi}_{\nvec''} \delta_{\nvec, \nvec' + \nvec''}
\end{align}
Thus, we finally have
\begin{equation}
    \mathcal{F} \left \{ - \left (\nabla_\perp \phi \times \bvec \right ) \cdot \nabla^3_\perp \phi \right \}_\nvec = \sum_{\nvec = \nvec' + \nvec''} \left ( \kvec_{\nvec'} \times \kvec_{\nvec''} \right ) \cdot \bvec \left ( k^2_{\nvec''} - k^2_{\nvec'} \right ) \hat{\phi}_{\nvec'} \hat{\phi}_{\nvec''}
\end{equation}
Combining the results above, we obtain
\begin{equation}
    \frac{\partial \hat{\phi}_\nvec}{\partial t} + i w_\nvec \hat{\phi}_\nvec = \sum_{\nvec = \nvec' + \nvec''} \Lambda^\nvec_{\nvec', \nvec''} \hat{\phi}_{\nvec'} \hat{\phi}_{\nvec''},
\end{equation}
where
\begin{equation}
    w_\nvec = - \frac{ \left (\kvec_\nvec \times \bvec \right ) }{1 + k^2_\nvec} \cdot \nabla_\perp \ln \left ( \frac{n_o}{w_{ci}} \right ),
\end{equation}
and
\begin{equation}
    \Lambda^\nvec_{\nvec', \nvec''} = \frac{1}{2} \frac{  \left ( \kvec_{\nvec'} \times \kvec_{\nvec''} \right ) \cdot \bvec \left ( k^2_{\nvec''} - k^2_{\nvec'} \right ) }{1 + k^2_\nvec}.
\end{equation}
Note that $w_\nvec$ can also be written as
\begin{equation}
    w_\nvec = -\frac{k_{2,\nvec} \evec_1 - k_{1,\nvec} \evec_2}{1 + k^2_\nvec} \cdot \beta \evec_1 = -\frac{k_{2,\nvec} \beta}{1 + k^2_\nvec}.
\end{equation}

%--------------------------------------------
\section{Hasegawa-Wakatani}
%--------------------------------------------
References for this model can be found in \cite{wakatani1984,hasegawa1987}.

%--------------------------------------------
\subsection{Assumptions}
%--------------------------------------------
\begin{enumerate}
    \item Singly-charged ions. \label{it:hw_single_charge_ions}
    \item No shear stresses in the electron momentum equation, no collisions in the ion momentum equation, no sources. \label{it:hw_no_shear_source_coll}
    \item Cold ion approximation, i.e. $T_e \gg T_i$ and thus $\nabla p_i \approx 0$. \label{it:hw_cold}
    \item Isothermal electron fluid, i.e.\@ $T_e$ is constant. \label{it:hw_isothermal_electron}
    \item Electrostatic field, i.e. $\Evec = -\nabla \phi$. \label{it:hw_electrostatic}
    \item Magnetic field is constant.
    \item Neglect parallel ion velocity, i.e. $u_{i,||} \approx 0$. \label{it:hw_par_ion}
    \item Quasi-neutrality, i.e. $n_i \approx n_e$. \label{it:hw_quasineutrality}
    \item $n_i = n_0 + n'$, where $n_0 = n_0(x_1)$ and $n'$ is smaller than $n_0$. \label{it:hw_ion_density}
    \item Perpendicular components of the ion shear-stress term are modeled as $(\nabla \cdot \tvec_i)_\perp/(m_i n_i) = \nu \nabla_\perp^2 \uvec_{i,\perp}$. \label{it:hw_shear_stress}
    \item Assume infinitesimally small electron mass, i.e.\@ $m_e \to 0$. \label{it:hw_small_electron_mass}
\end{enumerate}

%--------------------------------------------
\subsection{Derivation}
%--------------------------------------------
Using the assumptions in \cref{it:hw_single_charge_ions,it:hw_no_shear_source_coll}, the momentum \cref{eq:cons_mom} for ions becomes
\begin{equation}
    m_i n_i \left (\frac{\partial \uvec_i}{\partial t} + \uvec_i \cdot \nabla \uvec_i \right ) = e n_i (\Evec + \uvec_i \times \Bvec) - \nabla p_i + \nabla \cdot \tvec.
\end{equation}
Using the assumptions in \cref{it:hw_cold,it:hw_electrostatic}, the above becomes
\begin{equation}
    \frac{\partial \uvec_i}{\partial t} + \uvec_i \cdot \nabla \uvec_i = -\frac{e}{m_i} \nabla \phi + \frac{e}{m_i}\uvec_i \times \Bvec + \frac{\nabla \cdot \tvec_i}{m_i n_i}.
\end{equation}
As before, introduce the coordinate system $\evec_1$, $\evec_2$, $\evec_3$, and assume $\Bvec$ points in the $\evec_3$ direction. Defining the perpendicular velocity as $\uvec_{i,\perp} = [u_{i,1}, u_{i,2}, 0]^T$, the perpendicular gradient as $\nabla_\perp = [\partial_1, \partial_2, 0]^T$, and the perpendicular shear stress as $(\nabla \cdot \tvec_i)_\perp = [(\nabla \cdot \tvec)_1, (\nabla \cdot \tvec)_2, 0]^T$, we have
\begin{equation}
    \frac{\partial \uvec_{i,\perp}}{\partial t} + \uvec_i \cdot \nabla \uvec_{i,\perp} = -\frac{e}{m_i} \nabla_\perp \phi + \frac{e}{m_i}\uvec_i \times \Bvec + \frac{(\nabla \cdot \tvec_i)_\perp}{m_i n_i}.
\end{equation}
Using the assumption in \cref{it:hw_par_ion} and noting that $\uvec_{i} \times \Bvec = \uvec_{i,\perp} \times \Bvec$, we obtain
\begin{equation}
    \frac{\partial \uvec_{i,\perp}}{\partial t} + \uvec_{i,\perp} \cdot \nabla_\perp \uvec_{i,\perp} = -\frac{e}{m_i} \nabla_\perp \phi + \frac{e}{m_i} \uvec_{i,\perp} \times \Bvec + \frac{(\nabla \cdot \tvec_i)_\perp}{m_i n_i}.
\end{equation}
Finally, using the assumption in \cref{it:hw_shear_stress}, we obtain
\begin{equation}
    \label{eq:hw_momentum_perpendicular}
        \frac{\partial \uvec_{i,\perp}}{\partial t} + \uvec_{i,\perp} \cdot \nabla_\perp \uvec_{i,\perp} = -\frac{e}{m_i} \nabla_\perp \phi + \frac{e}{m_i} \uvec_{i,\perp} \times \Bvec + \nu \nabla_\perp^2 \uvec_{i,\perp}.
\end{equation}
The same scaling analysis performed for the derivation of the Hasegawa Mima equation is now applied. The only new term in \cref{eq:hw_momentum_perpendicular} is the viscous term. We note that the kinematic viscosity $\nu$ scales as
\begin{equation}
    \nu \sim r^2 w.
\end{equation}
Thus, the viscous stress term leads to the following scalings
\begin{enumerate}
    \item $\nu \nabla^2_\perp \uvec^{(0)}_{i,\perp} \sim \epsilon$
    \item $\nu \nabla^2_\perp \uvec^{(1)}_{i,\perp} \sim \epsilon^2$
\end{enumerate}
As before, the first order terms lead to the $E \times B$ drift
\begin{equation}
    \label{eq:hw_e_cross_b_drift}
    \uvec^{(0)}_{i,\perp} = -\nabla_\perp \phi \times \frac{\bvec}{B}.
\end{equation}
However, the equation for terms of order $\epsilon$ now contains the viscous term as shown below
\begin{equation}
    \frac{\partial \uvec_{i,\perp}^{(0)}}{\partial t} + \uvec_{i,\perp}^{(0)} \cdot \nabla_\perp \uvec_{i,\perp}^{(0)} = \frac{e}{m_i} \uvec_{i,\perp}^{(1)} \times \Bvec + \nu \nabla_\perp^2 \uvec^{(0)}_{i,\perp}.
\end{equation}
Upon crossing by $\Bvec$, the above gives
\begin{equation}
\label{eq:hw_polarization_drift}
    \uvec_{i,\perp}^{(1)} = -\frac{1}{w_{c,i} B} \left [ \frac{\partial \nabla_\perp \phi}{\partial t} + \uvec_{i,\perp}^{(0)} \cdot \nabla_\perp \left ( \nabla_\perp \phi \right ) \right ] + \frac{\nu}{w_{c,i}B} \nabla^2_\perp (\nabla_\perp \phi).
\end{equation}
That is, an additional viscous term appears in the polarization drift.

As shown in the derivation of the Hasegawa-Mima equation, the continuity equation for ions can be expressed in the form of \cref{eq:intermediate_hm_1}, which is repeated below for convenience
\begin{equation}
    \frac{\partial \ln n_i}{\partial t} + \uvec^{(0)}_{i,\perp} \cdot \nabla_\perp \ln n_i + \nabla_\perp \cdot \uvec^{(1)}_{i,\perp} = 0.
\end{equation}
Given the assumption in \cref{it:hw_ion_density}, the natural logarithm of density is re-written as follows
\begin{equation}
    \ln n_i = \ln \left ( n_0 + n' \right ) = \ln \left [ n_0 \left ( 1 + \frac{n'}{n_0} \right ) \right ] = \ln n_0 + \ln \left (1 + \frac{n'}{n_0} \right ).
\end{equation}
We now introduce $n = n' / n_0$, which is small due to the assumption in \cref{it:hw_ion_density}. Thus, a Taylor series expansion would allow us to write
\begin{equation}
    \label{eq:hw_ion_density}
    \ln n_i = \ln n_0 + n.
\end{equation}
Since $n_0$ is time independent (assumption in \cref{it:hw_ion_density}), the continuity equation becomes
\begin{equation}
    \label{eq:hw_intermediate_ion_1}
    \frac{\partial n}{\partial t} + \uvec^{(0)}_{i,\perp} \cdot \nabla_\perp \left ( \ln n_0 + n \right ) + \nabla_\perp \cdot \uvec^{(1)}_{i,\perp} = 0.
\end{equation}
Using the same derivation for \cref{eq:hm_divergence_polarization}, we now have
\begin{equation}
    \nabla_\perp \cdot \uvec_{i,\perp}^{(1)} = -\frac{1}{w_{c,i} B} \left [ \frac{\partial \nabla_\perp^2 \phi}{\partial t} + \uvec_{i,\perp}^{(0)} \cdot \nabla_\perp \left ( \nabla_\perp^2 \phi \right ) \right ] + \frac{\nu}{w_{c,i} B} \nabla^4_\perp \phi.
\end{equation}
Plugging this in \cref{eq:hw_intermediate_ion_1}, we obtain
\begin{equation}
    \frac{\partial}{\partial t} \left ( \frac{1}{w_{c,i} B} \nabla^2_\perp \phi - n \right ) + \uvec^{(0)}_{i,\perp} \cdot \nabla_\perp \left ( \frac{1}{w_{c,i} B} \nabla^2_\perp \phi - n - \ln n_0 \right ) - \frac{\nu}{w_{c,i} B} \nabla^4_\perp \phi = 0.
\end{equation}
Plugging in for $\uvec^{(0)}_{i,\perp}$,
\begin{equation}
    \label{eq:hw_ion_continuity}
    \frac{\partial}{\partial t} \left ( \frac{1}{w_{c,i} B} \nabla^2_\perp \phi - n \right ) - \left (\nabla_\perp \phi \times \frac{\bvec}{B} \right ) \cdot \nabla_\perp \left ( \frac{1}{w_{c,i} B} \nabla^2_\perp \phi - n - \ln n_0 \right ) - \frac{\nu}{w_{c,i} B} \nabla^4_\perp \phi = 0.
\end{equation}

Using the assumptions in \cref{it:hw_single_charge_ions,it:hw_no_shear_source_coll}, the momentum \cref{eq:cons_mom} for electrons becomes
\begin{equation}
    m_e n_e \left (\frac{\partial \uvec_e}{\partial t} + \uvec_e \cdot \nabla \uvec_e \right ) = -e n_e (\Evec + \uvec_e \times \Bvec) - \nabla p_e + \Rvec_e.
\end{equation}
Using the assumptions in \cref{it:hw_small_electron_mass,it:hw_isothermal_electron,it:hw_electrostatic}, the above simplifies to
\begin{equation}
    0 = -e n_e (-\nabla \phi + \uvec_e \times \Bvec) - T_e \nabla n_e + \Rvec_e.
\end{equation}
Dividing by $-e n_e$, we get
\begin{equation}
    \label{eq:hw_electron_momentum_no_inertia}
    0 = -\nabla \phi + \uvec_e \times \Bvec + \frac{T_e}{e} \nabla \ln n_e - \frac{1}{e n_e}\Rvec_e.
\end{equation}
We now focus on the component of the equation above that is parallel to $\Bvec$, that is 
\begin{equation}
    0 = -\nabla_{||} \phi + \frac{T_e}{e} \nabla_{||} \ln n_e - \frac{1}{e n_e} R_{e,||}.
\end{equation}
Using quasineutrality to replace $n_e$ by $n_i$, and plugging in \cref{eq:hw_ion_density} for $n_i$ gives 
\begin{equation}
    0 = -\nabla_{||} \phi + \frac{T_e}{e} \nabla_{||} (\ln n_0 + n) - \frac{1}{e n_e} R_{e,||}.
\end{equation}
We note that the gradient of $\ln n_0$ above is zero since $n_0$ does not vary along the direction of the magnetic field. The definition of the electron collision term is given by \cref{eq:elec_coll_current}, that is, $\Rvec_e = (m_e \nu_{ei}/e) \Jvec$. Using the definition of the resistivity given by \cref{eq:resistivity} ($\eta = m_e \nu_{ei}/e^2 n_e$), we get $\Rvec_e = e n_e \eta \Jvec$. Thus, we now have
\begin{equation}
    0 = -\nabla_{||} \phi + \frac{T_e}{e} \nabla_{||} n - \eta J_{||},
\end{equation}
which, upon re-arranging, gives
\begin{equation}
    \label{eq:hw_electron_momentum}
    J_{||} = \frac{T_e}{e \eta} \nabla_{||} \left ( n - \frac{e\phi}{T_e} \right ).
\end{equation}
The perpendicular component of \cref{eq:hw_electron_momentum_no_inertia} is as follows
\begin{equation}
    0 = -\nabla_\perp \phi + \uvec_e \times \Bvec + \frac{T_e}{e} \nabla_\perp \ln n_e - \frac{1}{e n_e}\Rvec_{e,\perp}.
\end{equation}
Since $\uvec_e \times \Bvec = \uvec_{e,\perp} \times \Bvec$ we have
\begin{equation}
    0 = -\nabla_\perp \phi + \uvec_{e,\perp} \times \Bvec + \frac{T_e}{e} \nabla_\perp \ln n_e - \frac{1}{e n_e}\Rvec_{e,\perp}.
\end{equation}
Again, using the definition of the electron collision term, we get
\begin{equation}
    0 = -\nabla_\perp \phi + \uvec_{e,\perp} \times \Bvec + \frac{T_e}{e} \nabla_\perp \ln n_e - \eta \Jvec_\perp.
\end{equation}
Crossing the above by $\Bvec$ gives 
\begin{equation}
    \uvec_{e,\perp} = -\nabla_\perp \phi \times \frac{\bvec}{B} - \frac{T_e}{e n_e B} {\bvec \times \nabla_\perp n_e} + \frac{\eta}{B} \bvec \times \Jvec_\perp.
\end{equation}
Typically the last term on the right-hand side above is significantly smaller, and thus it can be neglected. The electron velocity is thus
\begin{equation}
    \label{eq:hw_elec_vel}
    \uvec_{e,\perp} = -\nabla_\perp \phi \times \frac{\bvec}{B} - \frac{T_e}{e n_e B} {\bvec \times \nabla_\perp n_e}.
\end{equation}

The continuity equation for electrons is as follows
\begin{equation}
    \frac{\partial n_e}{\partial t} + \nabla \cdot (n_e \uvec_e) = 0.
\end{equation}
We split the convection term in the above into the perpendicular and parallel components
\begin{equation}
    \frac{\partial n_e}{\partial t} + \nabla_\perp \cdot (n_e \uvec_{e,\perp}) + \nabla_{||} \cdot (n_e u_{e,||}) = 0.
\end{equation}
Given the assumption in \cref{it:hw_par_ion}, we have $J_{||} = e n_e (u_{i,||} - u_{e,||}) = -en_e u_{e,||}$. Thus, the above becomes
\begin{equation}
    \frac{\partial n_e}{\partial t} + \nabla_\perp \cdot (n_e \uvec_{e,\perp}) = \frac{1}{e} \nabla_{||} J_{||}.
\end{equation}
Using the identity $\nabla \cdot (\Avec \times \Bvec) = \Bvec \cdot (\nabla \times \Avec) - \Avec \cdot ( \nabla \times \Bvec)$ we show that 
\begin{equation}
    \nabla_\perp \cdot (\bvec \times \nabla_\perp n_e) = \nabla_\perp n_e \cdot (\nabla_\perp \times \bvec) - \bvec \cdot (\nabla_\perp \times \nabla_\perp n_e) = 0.
\end{equation}
As a result, the second term on the right-hand side of \cref{eq:hw_elec_vel} does not contribute to $\nabla_\perp \cdot (n_e \uvec_{e,\perp})$. The electron continuity equation then becomes
\begin{equation}
    \frac{\partial n_e}{\partial t} - \left ( \nabla_\perp \phi \times \frac{\bvec}{B} \right ) \cdot \nabla_\perp n_e = \frac{1}{e} \nabla_{||} J_{||}.
\end{equation}
Dividing by $n_e$ and using quasi-neutrality
\begin{equation}
    \frac{\partial \ln n_i}{\partial t} - \left ( \nabla_\perp \phi \times \frac{\bvec}{B} \right ) \cdot \nabla_\perp \ln n_i = \frac{1}{e n_i} \nabla_{||} J_{||}.
\end{equation}
Using the expression for $\ln n_i$ in \cref{eq:hw_ion_density}, we get
\begin{equation}
    \frac{\partial n}{\partial t} - \left ( \nabla_\perp \phi \times \frac{\bvec}{B} \right ) \cdot \nabla_\perp (\ln n_0 + n ) = \frac{1}{e n_i} \nabla_{||} J_{||}.
\end{equation}
Finally, using the assumption in \cref{it:hw_ion_density}, we neglect $n'$ in the denominator of the right-hand side, and obtain
\begin{equation}
    \label{eq:hw_electron_continuity}
    \frac{\partial n}{\partial t} - \left ( \nabla_\perp \phi \times \frac{\bvec}{B} \right ) \cdot \nabla_\perp \left ( n + \ln n_0 \right ) = \frac{1}{en_0} \nabla_{||} J_{||}.
\end{equation}

Combining \cref{eq:hw_electron_momentum,eq:hw_ion_continuity,eq:hw_electron_continuity} leads to the dimensional form of the Hasegawa-Wakatani model
\begin{multline}
    \frac{\partial}{\partial t} \left ( \frac{1}{w_{c,i} B} \nabla^2_\perp \phi \right ) - \left (\nabla_\perp \phi \times \frac{\bvec}{B} \right ) \cdot \nabla_\perp \left ( \frac{1}{w_{c,i} B} \nabla^2_\perp \phi \right ) \\
    = \frac{T_e}{e^2 n_0 \eta} \nabla_{||}^2 \left (n - \frac{e \phi}{T_e} \right ) + \frac{\nu}{w_{c,i} B} \nabla^4_\perp \phi.
\end{multline}
\begin{equation}
    \frac{\partial n}{\partial t} - \left ( \nabla_\perp \phi \times \frac{\bvec}{B} \right ) \cdot \nabla_\perp \left ( n + \ln n_0 \right ) = \frac{T_e}{e^2 n_0 \eta} \nabla_{||}^2 \left (n - \frac{e \phi}{T_e} \right ).
\end{equation}
It is quite common to replace the parallel-gradient operator $\nabla_{||}$ by a coefficient, say $1/l^2$.

We now introduce the following non-dimensionalization
\begin{align}
    \phi(t,x_1,x_2,x_3) &= \frac{T_e}{e} \hat{\phi}(\hat{t},\hat{x}_1,\hat{x}_2, x_3) \\
    n_0(x_1) &= \hat{n}_0(\hat{x}_1) \\
    n(t,x_1,x_2,x_3) &= \hat{n}(\hat{t}, \hat{x}_1, \hat{x}_2, x_3),
\end{align}
where $\hat{t} = t w_{c,i}$, $\hat{x}_1 = x_1 / r_s$, and $\hat{x}_2 = x_2 / r_s$. Neglecting the hat notation for the sake of simplicity, the Hasegawa-Wakatani model in non-dimensional form is written as
\begin{equation}
    \label{eq:hw_potential}
    \frac{\partial \nabla^2_\perp \phi}{\partial t} - \left (\nabla_\perp \phi \times \bvec \right ) \cdot \nabla_\perp \left ( \nabla^2_\perp \phi \right ) = c_1 ( \phi - n) + c_2 \nabla^4_\perp \phi,
\end{equation}
\begin{equation}
    \label{eq:hw_density}
    \frac{\partial n}{\partial t} - \left ( \nabla_\perp \phi \times \bvec \right ) \cdot \nabla_\perp \left ( n + \ln n_0 \right ) = c_1 (\phi - n),
\end{equation}
where
\begin{equation}
    c_1 = -\frac{T_e}{e^2 n_0 \eta w_{c,i}} \nabla^2_{||} \qquad c_2 = \frac{\nu}{w_{c,i} r_s^2}.
\end{equation}
If $\nabla_{||}$ is replaced by $1/l^2$, then the $c_1$ operator is simply a coefficient.

%--------------------------------------------
\subsection{Relationship to other models}
%--------------------------------------------
The Hasegawa-Wakatani \cref{eq:hw_potential,eq:hw_density} contain two limits. Assuming $c_1$ is a coefficient rather than an operator, one of the limits is obtained by letting $c_1 = 0$. Then, the $\phi$ and $n$ equations are decoupled, and the $\phi$ equation corresponds to the third (and only non-zero) component of the 2D Navier-Stokes equations for vorticity
\begin{equation}
    \frac{\partial \wvec}{\partial t} + \uvec \cdot \nabla_\perp \wvec = \nu \nabla^2_\perp \wvec,
\end{equation}
where
\begin{equation}
    \uvec = \nabla_\perp \times (-\phi \bvec) = - \nabla_\perp \phi \times \bvec,
\end{equation}
and
\begin{equation}
    \wvec = \nabla_\perp \times \uvec = \nabla^2_\perp \phi \bvec.
\end{equation}
If, on the other hand, $c_1 \to \infty$, then dividing \cref{eq:hw_density} by $c_1$ shows that $n = \phi$. Subtracting \cref{eq:hw_density} from \cref{eq:hw_potential} and assuming $c_2=0$ one obtains 
\begin{equation}
    \frac{\partial}{\partial t} \left ( \nabla^2_\perp \phi - \phi \right ) - \left (\nabla_\perp \phi \times \bvec \right ) \cdot \nabla_\perp \left ( \nabla^2_\perp \phi - \ln n_0 \right ) = 0.
\end{equation}
The above is the Hasegawa-Mima equation.