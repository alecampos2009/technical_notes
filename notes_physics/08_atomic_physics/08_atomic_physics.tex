\documentclass[a4paper,11pt]{report}
\usepackage{fullpage}

\usepackage{"../../info/packages"}
\usepackage{"../../info/nomenclature"}
\usepackage{fullpage}


\title{Atomic Physics}
\author{Alejandro Campos}

\begin{document}
\maketitle
\tableofcontents

%----------------------------------------------------------------------------------------------------------------------
\chapter{EOS}
%----------------------------------------------------------------------------------------------------------------------

%----------------------------------------------------------------------------------------------------------------------
\chapter{Ionization}
%----------------------------------------------------------------------------------------------------------------------

%----------------------------------------------------------------------------------------------------------------------
\chapter{Opacities}
%----------------------------------------------------------------------------------------------------------------------
Matter can affect the number and energy of photons through multiple processes:
\begin{itemize}
    \item Absorption 
    \begin{itemize}
        \item Bound-bound excitation (stimulated absorption)
        \item Bound-free ionization
        \item Free-free photo-absorption (inverse breemstrahlung)
    \end{itemize}
    \item Emission 
    \begin{itemize}
        \item Bound-bound de-excitation (stimulated emission, spontaneous emission)
        \item Bound-free recombination
        \item Free-free photo-emission (breemstrahlung)
    \end{itemize}
    \item Scattering
    \begin{itemize}
        \item Thompson scattering
        \item Compton scattering
        \item Rayleigh scattering
    \end{itemize}
    \item Reflection
    \item Transmission (nothing happens)
\end{itemize}

Each of the above would attenuate the beam of photons passing through the material. The cross-sections of each process, which would depend on the incoming photon frequency $\nu$, can be added up to obtain a total cross section $\sigma = \sigma(\nu)$. From the definition of a cross section, $\sigma$ can be used to determine how many photons keep their course as they traverse through the material and how many do not. Consider the material under consideration to have the shape of a thick slab, which starts at $x=0$ and continues on for a definite length along $x>0$. We want to know how $I(x)$, the number of photons crossing the slab at any location $x$, decreases as we travel along the $x$ direction. Let's focus on an infinitesimal thin lamina within the slab, of width $dx$ and located at some arbitrary location $x$. The number of target particles in that lamina will be $n dx$, where $n$ is the number volume density of particles in the target material. Then, the number of incident photons after crossing the lamina would be
\begin{equation}
    I(x+dx) = I(x) -\sigma I(x) n dx.
\end{equation}
This leads to the ODE $dI(x)/dx = -\sigma I(x) n$, which has as solution
\begin{equation}
    I(x) = I(0) \exp(-\sigma n x).
\end{equation}
The attenuation coefficient is defined as $\sigma n$ [1/cm]. The mass attenuation coefficient, also referred to as opacity, is then given by $\kappa = \sigma n / \rho$ [cm\textsuperscript{2}/g]. $\Lambda = 1 / \sigma n$ is referred to as the attenuation length. 

\end{document}