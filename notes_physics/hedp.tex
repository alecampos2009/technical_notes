\documentclass[a4paper,11pt]{report}
\usepackage{fullpage}

\usepackage{"../info/packages"}
\usepackage{"../info/nomenclature"}

\usepackage{scalerel}
\usepackage{subfiles}

\newcommand{\xvecdot}{\dot{\xvec}}
\newcommand{\xdot}{\dot{x}}
\newcommand{\qvecdot}{\dot{\qvec}}
\newcommand{\qdot}{\dot{q}}

\setlength{\cellspacetoplimit}{3pt}
\setlength{\cellspacebottomlimit}{3pt}

\title{High-Energy-Density Physics}
\author{Alejandro Campos}

\begin{document}
\maketitle
\tableofcontents

\subfile{./introduction/introduction}

\subfile{./hydrodynamics/hydrodynamics}

\subfile{./magnetic_fields/magnetic_fields}

\subfile{./radiation_transfer/radiation_transfer}

\subfile{./nuclear_physics/nuclear_physics}

%%%%%%%%%%%%%%%%%%%%%%%%%%%%%%%%%%%%%%%%%%%%%%%%%%%%%%%%%%%%%%%%%%%%%%%%%
\part{Atomic Physics}
%%%%%%%%%%%%%%%%%%%%%%%%%%%%%%%%%%%%%%%%%%%%%%%%%%%%%%%%%%%%%%%%%%%%%%%%%

%########################################################################
\chapter{Equation of state}
%########################################################################

%########################################################################
\chapter{Opacities}
%########################################################################
There are multiple processes that control the behavior of photons:
\begin{itemize}
    \item Absorption 
    \begin{itemize}
        \item Bound-bound excitation (\textbf{stimulated absorption})
        \item Bound-free ionization (\textbf{photoionization})
        \item Free-free photo-absorption (\textbf{inverse breemstrahlung})
    \end{itemize}
    \item Emission 
    \begin{itemize}
        \item Bound-bound de-excitation (\textbf{stimulated emission}, \textbf{spontaneous emission})
        \item Free-bound recombination
        \item Free-free photo-emission (\textbf{breemstrahlung})
    \end{itemize}
    \item Scattering
    \begin{itemize}
        \item \textbf{Rayleigh scattering}: elastic scattering of a photon from an atom or molecule whose size is less than that of the wavelength of the photon. 
        \item \textbf{Mie scattering}: same as Rayleigh scattering but for cases where the sizes of the atoms or molecules are comparable to the wavelength of the incoming photon.
        \item \textbf{Raman scattering}: inelastic scattering of a photon from a molecule. The interaction changes the molecule's vibrational, rotational, or electron energy.
        \item \textbf{Brillouin scattering}: inelastic scattering of a photon caused by its interaction with material waves in a medium (i.e. mass oscillation modes, charge displacement modes, magnetic spin oscillation modes). 
        \item \textbf{Compton scattering}: inelastic scattering of a photon from a charged particle. 
        \item \textbf{Thomson scattering}: low-energy limit of Compton scattering. The photon energy and the particle's kinetic energy do not change as a result of the scattering. Can be explained with classical electrodynamics.
    \end{itemize}
    \item Reflection
    \item Transmission (nothing happens)
    \item \textbf{Pair production/annihilation}
\end{itemize}

Each of the above would alter the beam of photons passing through the material (except for transmission). The cross-sections of each process, which would depend on the incoming photon frequency $\nu$, can be added up to obtain a total cross section $\sigma = \sigma(\nu)$. From the definition of a cross section, $\sigma$ can be used to determine how many photons keep their course as they traverse through the material and how many do not. Consider the material under consideration to have the shape of a thick slab, which starts at $x=0$ and continues on for a definite length along $x>0$. We want to know how $I(x)$, the number of photons crossing the slab at any location $x$, decreases as we travel along the $x$ direction. Let's focus on an infinitesimal thin lamina within the slab, of width $dx$ and located at some arbitrary location $x$. The number of target particles in that lamina will be $n dx$, where $n$ is the number volume density of particles in the target material. Then, the number of incident photons after crossing the lamina would be
\begin{equation}
    I(x+dx) = I(x) -\sigma I(x) n dx.
\end{equation}
This leads to the ODE $dI(x)/dx = -\sigma I(x) n$, which has as solution
\begin{equation}
    I(x) = I(0) \exp(-\sigma n x).
\end{equation}
The attenuation coefficient is defined as $\sigma n$ [1/cm]. The mass attenuation coefficient, also referred to as opacity, is then given by $\kappa = \sigma n / \rho$ [cm\textsuperscript{2}/g]. $\Lambda = 1 / \sigma n$ is referred to as the attenuation length. 

%%%%%%%%%%%%%%%%%%%%%%%%%%%%%%%%%%%%%%%%%%%%%%%%%%%%%%%%%%%%%%%%%%%%%%%%%
\part{Laser Physics}
%%%%%%%%%%%%%%%%%%%%%%%%%%%%%%%%%%%%%%%%%%%%%%%%%%%%%%%%%%%%%%%%%%%%%%%%%

%########################################################################
\chapter{Some notes on lasers}
%########################################################################
Consider a wave that depends on time $t$ and single spatial dimension $x$, which is orthogonal to the direction of propagation. 

A wave is spatially coherent at a given time $t$, position $x$, and separation distance $L$ if the phase difference between the points $x$ and $x+L$ at time $t$ is the same as that at a later time $t+dt$. You can keep on picking larger and larger values of $L$ until this is not the case, this $L$ would be the spatial coherence length $L_c$ = $L_c(t,x)$.

A wave is temporally coherent at a given time $t$, position $x$, and separation time $\tau$ if the phase difference between times $t$ and $t+\tau$ at position $x$ is the same as that between times $t+dt$ and $t+dt+\tau$. You can keep on picking larger and larger values of $\tau$ until this is no longer the case, this $\tau$ would be the temporal coherence length $\tau_c = \tau_c(t,x)$.

\begin{figure}
    \centering
    \includegraphics[width=.7\textwidth]{../images/temp_cohe_1.pdf}
    \caption{}
    \label{fig:laser_temp_cohe}
\end{figure}

\begin{figure}
    \centering
    \begin{subfigure}{.7\textwidth}
      \centering
      \includegraphics[width=\textwidth]{../images/temp_cohe_2.pdf}
      \caption{}
      \label{fig:laser_temp_cohe_1}
    \end{subfigure}
    \begin{subfigure}{.7\textwidth}
        \centering
        \includegraphics[width=\textwidth]{../images/temp_cohe_3.pdf}
        \caption{}
        \label{fig:laser_temp_cohe_2}
      \end{subfigure}
    \caption{Depiction of temporal coherence for a light pulse.}
    \label{fig:laser_temp_cohe_12}
\end{figure}

A depiction of temporal coherence is provided in \cref{fig:laser_temp_cohe}, which shows a light pulse as a function of time. An arbitrary time $t_*$ is chosen along the light wave, and the wave's phase at that time is $\theta_1$. At the time $t_* + \tau$ the wave has a different phase $\theta_2$, and thus the phase difference between times $t_*$ and $t_*+\tau$ is $\theta_2 - \theta_1$. This is depicted by the orange dots. The green dots are used to highlight the phase difference between times $t_*+dt$ and $t_*+dt+\tau$. In this case, the phases are respectively $\theta_1 + d\theta$ and $\theta_2 + d\theta$, and hence the phase difference is still maintained at $\theta_2 - \theta_1$.

Rather than picking the time $t_*$ to be at an arbitrary location along the wave, in \cref{fig:laser_temp_cohe_1} we choose $t_*$ to be at the beginning of the wave, and also choose $\tau$ to be just a bit smaller than the entire duration of the pulse. For this case, the phase difference between $t_*$ and $t_*+\tau$ is $d\theta$, which is equal to the phase difference between $t_*+dt$ and $t_*+dt+\tau$. On the other hand, in \cref{fig:laser_temp_cohe_2} we have chosen $\tau$ to be just a bit larger so as to be equal to the duration of the wave. For this case the phase difference between $t_*$ and $t_*+\tau$ is zero, which is not the same as the phase difference between $t_*+dt$ and $t_*+dt+\tau$, namely $d\theta$. Thus, the maximum value of $\tau$ that maintains a constant phase difference has been reached, and therefore by definition this value is referred to as the temporal coherence length $\tau_c$ at time $t_*$ (other locations of $t_*$, such as that in \cref{fig:laser_temp_cohe}, have a different $\tau_c$).

%########################################################################
\chapter{Longitudinal and transverse waves}
%########################################################################

%------------------------------------------------------------------------
\section{Definitions}
%------------------------------------------------------------------------
The Helmholtz decomposition for a function $\Fvec = \Fvec(\xvec,t)$ is of the following form
\begin{equation}
    \Fvec = \Fvec_l + \Fvec_t,
\end{equation}
where $\Fvec_l = \Fvec_l(\xvec,t)$ is the longitudinal component and $\Fvec_t = \Fvec_t(\xvec,t)$ the transverse component. These are defined by
\begin{align}
    \nabla \times \Fvec_l &= 0, \label{eq:ltw_plasma_long_def}\\
    \nabla \cdot \Fvec_t &= 0. \label{eq:ltw_plasma_tran_def}
\end{align}
We'll assume that any vector function $\Gvec = \Gvec(\xvec,t)$ can be expressed as the real component of
\begin{equation}
    \label{eq:em_more_general_wave_form}
    \Gvec = \hat{\Gvec} \exp \left [ i \left ( \int_0^x k_x(x') \, dx' + \int_0^y k_y(y') \, dy' + \int_0^z k_z(z') \, dz'  - wt \right ) \right ].
\end{equation}
For the above, $w$ a frequency constant in time and space, and $k_x = k_x(x)$, $k_y = k_y(y)$, and $k_z = k_z(z)$ form the wave vector $\kvec = [k_x, k_y, k_z]$. $\hat{\Gvec} = \hat{\Gvec}(\xvec,t)$ is a complex vector where the real and complex components point in the same direction. Additionally, we enforce the constraint that if $\nabla \times \Gvec = 0$, then $\nabla \times \hat{\Gvec} = 0$, and similarly, if $\nabla \cdot \Gvec = 0$, then $\nabla \cdot \hat{\Gvec} = 0$. We'll often use a different subscript in the wave vectors and frequencies of different waves. For example, we'll use $\kvec_e$ for electron-plasma waves, $\kvec_i$ for ion-acoustic waves, $\kvec_L$ for laser waves, and $\kvec_s$ for scattered waves. Similarly for the frequencies $w_e$, $w_i$, $w_L$, $w_s$.

We note that
\begin{multline}
    \nabla \exp \left [ i \left ( \int_0^x k_x(x') \, dx' + \int_0^y k_y(y') \, dy' + \int_0^z k_z(z') \, dz'  - wt \right ) \right ] \\
    = i \kvec \exp \left [ i \left ( \int_0^x k_x(x') \, dx' + \int_0^y k_y(y') \, dy' + \int_0^z k_z(z') \, dz'  - wt \right ) \right ]
\end{multline}
Given the identity $\nabla \times (\Avec f) = (\nabla \times \Avec ) f - \Avec \times (\nabla f)$, we can show that
\begin{align}
    \label{eq:ltw_no_curl}
    \nabla \times \Gvec &= \nabla \times \left [ \hat{\Gvec} \exp (...) \right ] \nonumber \\
    &= \left ( \nabla \times \hat{\Gvec} \right ) \exp (...) - \hat{\Gvec} \times \left [ \nabla \exp (...) \right ] \nonumber \\
    &= \left ( \nabla \times \hat{\Gvec} \right ) \exp (...) - \hat{\Gvec} \times \left [ i\kvec \exp (...) \right ] \nonumber \\
    &= \left ( \nabla \times \hat{\Gvec} \right ) \exp (...) + i \kvec \times \Gvec.
\end{align}
Given the identity $\nabla \cdot (\Avec f) = ( \nabla \cdot \Avec) f + \Avec \cdot (\nabla f)$, we can show that
\begin{align}
    \label{eq:ltw_no_div}
    \nabla \cdot \Gvec &= \nabla \cdot \left [ \hat{\Gvec} \exp (...) \right ] \nonumber \\
    &= \left ( \nabla \cdot \hat{\Gvec} \right ) \exp (...) + \hat{\Gvec} \cdot \left [ \nabla \exp (...) \right ] \nonumber \\
    &= \left ( \nabla \cdot \hat{\Gvec} \right ) \exp (...) + \hat{\Gvec} \cdot \left [ i \kvec \exp (...) \right ] \nonumber \\
    &= \left ( \nabla \cdot \hat{\Gvec} \right ) \exp (...) + i \kvec \cdot \Gvec.
\end{align}
By definition, $\Fvec_l$ has no curl and $\Fvec_t$ has no divergence. As mentioned earlier, we then require $\hat{\Fvec}_l$ to have no curl and $\hat{\Fvec}_t$ to have no divergence. Using this in \cref{eq:ltw_no_curl,eq:ltw_no_div} allow us to write
\begin{align}
    \kvec \times \Fvec_l = 0 \label{eq:ltw_plasma_long_wavevector} \\
    \kvec \cdot \Fvec_t = 0 \label{eq:ltw_plasma_trans_wavevector}.
\end{align}

The first expression above says $\Fvec_l$ is parallel to $\kvec$ and the second says $\Fvec_t$ is orthogonal to $\kvec$. Thus, $\Fvec_l \cdot \Fvec_t = 0$. We will often have situations where $\nabla \times \Fvec = \nabla \cdot \Fvec = 0$, which by its own does not imply $\Fvec = 0$. However, using \cref{eq:ltw_no_curl,eq:ltw_no_div}, this translates to to $\kvec \times \Fvec = \kvec \cdot \Fvec =  0$. The latter equality states that $\kvec$ and $\Fvec$ are orthogonal, that is, the angle between them is $90^\circ$. The former equality leads to $|\Fvec| \sin(90^\circ) = 0$, which in turn means $\Fvec = 0$. To summarize,
\begin{equation}
    \label{eq:ltw_general_null_vector}
    \nabla \times \Fvec = \nabla \cdot \Fvec = 0 \to \Fvec = 0.
\end{equation}

For some cases we'll further restrict $\hat{\Gvec}$ in \cref{eq:em_more_general_wave_form} such that $\hat{\Gvec} = \hat{\Gvec} (\xvec)$, that is, the time dependence of the wave is fully captured by the $\exp(-iwt)$ term. For this case, we'll often re-write the expression for $\Gvec$ as
\begin{equation}
    \label{eq:em_semi_general_wave_form}
    \Gvec = \tilde{\Gvec} \exp (-iwt),
\end{equation}
where $\tilde{\Gvec} = \tilde{\Gvec}(\xvec)$ is given by
\begin{equation}
    \tilde{\Gvec} = \hat{\Gvec} \exp \left [ i \left ( \int_0^x k_x(x') \, dx' + \int_0^y k_y(y') \, dy' + \int_0^z k_z(z') \, dz' \right ) \right ].
\end{equation}
Finally, a further simplification occurs when $\kvec$ and $\hat{\Gvec}$ are assumed to be constant in space. For this case, the expression for $\Gvec$ becomes
\begin{equation}
    \label{eq:em_general_wave_form}
    \Gvec = \hat{\Gvec} \exp \left [i\left ( \kvec \cdot \xvec - wt \right ) \right ].
\end{equation}
These are the so-called plane waves. We end with the cautionary note that the second gradients of $\Gvec$ in \cref{eq:em_more_general_wave_form} are not necessarily the same as those of $\Gvec$ in \cref{eq:em_general_wave_form}.

%------------------------------------------------------------------------
\section{Electron-plasma and ion-acoustic waves}
%------------------------------------------------------------------------
For both electron-plasma and ion-acoustic waves we can assume the magnetic field does not change. Thus, Faraday's law gives
\begin{equation}
    \nabla \times \Evec = \nabla \times \Evec_t = 0.
\end{equation}
By definition, $\nabla \cdot \Evec_t = 0$. Thus, using \cref{eq:ltw_general_null_vector} we get $\Evec_t = 0$, that is, $\Evec = \Evec_l$. 

For electron-plasma waves, we can use \cref{eq:em_general_wave_form} to write \cref{eq:ep_waves_mom_linearized} in spectral form and thus obtain
\begin{equation}
-i w n_{e0} \hat{\uvec}_{e1} + \frac{e n_{e0}}{m_e} \hat{\Evec}_{1,l} = -\kvec_e \frac{\gamma_e p_{e0}}{n_{e0} m_e} \hat{n}_{e1}. 
\end{equation}
Since the second term on the left-hand side and the term on the right-hand side point along $\kvec_e$, $\hat{\uvec}_{e1}$ also points along $\kvec_e$, that is, $\hat{\uvec}_{e1} = \hat{\uvec}_{e1,l}$.

For ion-acoustic waves, we can use \cref{eq:em_general_wave_form} to write \cref{eq:ia_waves_mom_linearized} in spectral form and thus obtain
\begin{equation}
-i w n_{i0} \hat{\uvec}_{i1} - \frac{Z e n_{i0}}{m_i} \hat{\Evec}_{1,l} = -\kvec_i \frac{\gamma_i p_{i0}}{n_{i0} m_i} \hat{n}_{i1}. 
\end{equation}
Since the second term on the left-hand side and the term on the right-hand side point along $\kvec_i$, $\hat{\uvec}_{i1}$ also points along $\kvec_i$, that is, $\hat{\uvec}_{i1} = \hat{\uvec}_{i1,l}$. Finally, we note that the electric field being purely longitudinal is in agreement with \cref{eq:ia_waves_E}.

%########################################################################
\chapter{Electromagnetic waves in plasmas}
%########################################################################

\label{sec:electromagnetic_waves_plasmas}
In introductory electrodynamics, one typically studies electromagnetic waves in vacuum, that is, for cases where $\rho_e = \Jvec = 0$. In this section we relax both of these assumptions. Consider the electric and magnetic fields as well as the scalar and vector potentials, which satisfy
\begin{equation}
    \label{eq:emp_general_E_potential}
    \Evec = -\nabla \phi - \frac{\partial \Avec}{\partial t},
\end{equation}
\begin{equation}
    \label{eq:emp_general_B_potential}
    \Bvec = \nabla \times \Avec.
\end{equation}
For the above, we choose $\nabla \cdot \Avec = 0$. Using the fact that the magnetic field is solenoidal, we have
\begin{equation*}
    \nabla \cdot \Bvec = \nabla \cdot \Bvec_l + \nabla \cdot \Bvec_t = \nabla \cdot \Bvec_l = 0.
\end{equation*}
However, by definition, $\nabla \times \Bvec_l = 0$ as well. Thus, using \cref{eq:ltw_general_null_vector}, we have $\Bvec_l = 0$. The same argument applies to the vector potential, and thus $\Avec_l = 0$. For the electric field, we have
\begin{equation*}
    \nabla \cdot \Evec = \nabla \cdot \Evec_l + \nabla \cdot \Evec_t = \nabla \cdot \Evec_l .
\end{equation*}
Taking the divergence of \cref{eq:emp_general_E_potential}, we get
\begin{equation}
    \nabla \cdot \Evec = \nabla \cdot \left ( -\nabla \phi \right ).
\end{equation}
Combining the last two equations gives
\begin{equation*}
    \nabla \cdot \left ( \Evec_l + \nabla \phi \right ) = 0.
\end{equation*}
By definition, we also have
\begin{equation*}
    \nabla \times \left ( \Evec_l + \nabla \phi \right ) = 0.
\end{equation*}
Thus, using \cref{eq:ltw_general_null_vector}, we have $\Evec_l = -\nabla \phi$. A similar argument can be used to show $\Evec_t = -\partial \Avec / \partial t$. Our goal in this section will be to determine equations for $\Evec_l$, $\Evec_t$ and $\Bvec$.

We'll begin with the conservation of charge equation
\begin{equation*}
    \frac{\partial \rho_e}{\partial t} + \nabla \cdot \Jvec = 0,
\end{equation*}
which we re-write as
\begin{equation*}
    \frac{\partial \rho_e}{\partial t} + \nabla \cdot \Jvec_l = 0,
\end{equation*}
Using Poisson's equation $\nabla^2\phi = -\rho_e / \epsilon_0$ in the above, we get
\begin{equation*}
    \frac{\partial}{\partial t} \left ( -\epsilon_0 \nabla^2 \phi \right ) + \nabla \cdot \Jvec_l = 0,
\end{equation*}
or
\begin{equation*}
    \nabla \cdot \left ( \frac{\partial \nabla \phi}{\partial t} - \frac{1}{\epsilon_0} \Jvec_l \right ) = 0.
\end{equation*}
However, by definition, we also have
\begin{equation*}
    \nabla \times \left ( \frac{\partial \nabla \phi}{\partial t} - \frac{1}{\epsilon_0} \Jvec_l \right ) = 0.
\end{equation*}
Using \cref{eq:ltw_general_null_vector}, we conclude
\begin{equation}
    \label{eq:emp_general_longitudinal_J}
    \frac{\partial \nabla \phi}{\partial t} = \frac{1}{\epsilon_0} \Jvec_l.
\end{equation}
This gives the equation for $\Evec_l$, namely,
\begin{equation}
    \label{eq:emp_general_long_E}
    \frac{\partial^2 \Evec_l}{\partial t^2} + \frac{1}{\epsilon_0} \frac{\partial \Jvec_l}{\partial t} = 0.
\end{equation}

Both $\Evec_t$ and $\Bvec$ can be extracted from $\Avec$, so now we proceed to find an equation for the transverse vector potential. Ampere's law with Maxwell's correction gives
\begin{equation*}
    \nabla \times \left ( \nabla \times \Avec \right ) = \mu_0 \Jvec + \mu_0 \epsilon_0 \frac{\partial \Evec}{\partial t}.
\end{equation*}
The above is re-written as
\begin{equation*}
    \nabla \left ( \nabla \cdot \Avec \right ) - \nabla^2 \Avec = \mu_0 \Jvec + \mu_0 \epsilon_0 \left ( -\frac{\partial \nabla \phi}{\partial t} - \frac{\partial^2 \Avec}{\partial t^2} \right ),
\end{equation*}
which gives
\begin{equation*}
    \frac{\partial^2 \Avec}{\partial t} - \frac{1}{\mu_0 \epsilon_0} \nabla^2 \Avec = \frac{1}{\epsilon_0} \Jvec - \frac{\partial \nabla \phi}{\partial t},
\end{equation*}
or
\begin{equation*}
    \frac{\partial^2 \Avec}{\partial t} - c_0^2 \nabla^2 \Avec = \frac{1}{\epsilon_0} \Jvec - \frac{\partial \nabla \phi}{\partial t},
\end{equation*}
where $c_0 = 1/\sqrt{\mu_0 \epsilon_0}$. Expanding the current density as $\Jvec = \Jvec_l + \Jvec_t$, and using \cref{eq:emp_general_longitudinal_J}, we get
\begin{equation}
    \label{eq:emp_general_trans_vec_pot}
    \frac{\partial^2 \Avec}{\partial t} - c_0^2 \nabla^2 \Avec = \frac{1}{\epsilon_0} \Jvec_t.
\end{equation}
Using the functional form in \cref{eq:em_general_wave_form} for $\Avec$ and $\Jvec_t$ gives
\begin{equation}
    -w^2 \hat{\Avec} + k^2 c_0^2 \hat{\Avec}= \frac{1}{\epsilon_0} \hat{\Jvec}_t.
\end{equation}
That is, $\Avec$ and $\Jvec_t$ point in the same direction.

Taking the time derivative of \cref{eq:emp_general_trans_vec_pot} gives the equation for $\Evec_t$, that is
\begin{equation}
    \label{eq:emp_general_trans_E}
    \frac{\partial^2 \Evec_t}{\partial t^2} - c_0^2 \nabla^2 \Evec_t + \frac{1}{\epsilon_0} \frac{\partial \Jvec_t}{\partial t} = 0.
\end{equation}
Taking the curl of \cref{eq:emp_general_trans_vec_pot} gives the equation for $\Bvec$, that is
\begin{equation}
    \label{eq:emp_general_trans_B}
    \frac{\partial^2 \Bvec}{\partial t^2} - c_0^2 \nabla^2 \Bvec - \frac{1}{\epsilon_0} \nabla \times \Jvec_t = 0.
\end{equation}
Using the functional form in \cref{eq:em_general_wave_form} for $\Evec_t$, $\Bvec$ and $\Jvec_t$ gives
\begin{equation}
    -w^2 \hat{\Evec}_t + k^2 c_0^2 \hat{\Evec}_t - \frac{iw}{\epsilon_0} \hat{\Jvec}_t = 0,
\end{equation}
\begin{equation}
    -w^2 \Bvec + k^2 c_0^2 \Bvec - \frac{i}{\epsilon_0} \kvec \times \Jvec_t = 0.
\end{equation}
That is, $\Evec_t$ points in the same direction as $\Jvec_t$, which as shown before points in the same direction as $\Avec$. Additionally, $\Bvec$ points in the direction of $\kvec \times \Jvec_t$, that is, it is orthogonal to $\Evec_t$.

We briefly note that taking the curl of \cref{eq:p_waves_maxwell_1}, and using \cref{eq:p_waves_maxwell_2}, gives the wave equation for the total electric field $\Evec$, that is 
\begin{equation}
    \frac{\partial^2 \Evec}{\partial t^2} - c_0^2 \nabla^2 \Evec + c_0^2 \nabla (\nabla \cdot \Evec) + \frac{1}{\epsilon_0} \frac{\partial \Jvec}{\partial t} = 0.
\end{equation}
The above can be considered as the sum of the following three equations
\begin{align*}
    \frac{\partial^2 \Evec_l}{\partial t^2} + \frac{1}{\epsilon_0} \frac{\partial \Jvec_l}{\partial t} &= 0, \\
    \frac{\partial^2 \Evec_t}{\partial t^2} - c_0^2 \nabla^2 \Evec_t + \frac{1}{\epsilon_0} \frac{\partial \Jvec_t}{\partial t} &= 0, \\
    -c_0^2 \nabla^2 \Evec_l + c_0^2 \nabla (\nabla \cdot \Evec_l) &= 0.
\end{align*}
The first is the equation for the longitudinal electric field, that is \cref{eq:emp_general_long_E}. The second is the equation for the transverse electric field, that is \cref{eq:emp_general_trans_E}. The third equation above follows from the vector identity $\nabla \times (\nabla \times \Fvec) = -\nabla^2 \Fvec + \nabla ( \nabla \cdot \Fvec )$ and the fact that $\nabla \times \Evec_l = 0$.

It will often be the case that transverse waves will oscillate at such a fast rate that the ions, which have a large inertia, will be unable to react quickly enough. Thus, we can assume $\uvec_{i,t} = 0$. Given the definition of the current density in \cref{eq:p_waves_curr_density}, the transverse current density is expressed as $\Jvec_t = e \left (Z n_i \uvec_{i,t} - n_e \uvec_{e,t} \right )$, which now simplifies to 
\begin{equation}
    \label{eq:emp_transverse_current}
    \Jvec_t = -e n_e \uvec_{e,t}.
\end{equation}
Thus, the transverse electron velocity $\uvec_{e,t}$ points in the same direction as $\Jvec_t$, which is the same direction as $\Evec_t$ and $\Avec$. The next section focuses on deriving an expression for $\uvec_{e,t}$. 

We begin with \cref{eq:pwaves_electron_momentum}, the electron momentum equation, which, due to the electron continuity equation, can be written as
\begin{equation*}
    m_e n_e \frac{\partial\uvec_e}{\partial t} + m_e n_e \uvec_e \cdot \nabla \uvec_e + e n_e \left ( \Evec + \uvec_e \times \Bvec \right ) = -\nabla p_e,
\end{equation*}
or
\begin{equation*}
    \frac{\partial\uvec_e}{\partial t} + \uvec_e \cdot \nabla \uvec_e + \frac{e}{m_e} \left ( \Evec + \uvec_e \times \Bvec \right ) = -\frac{1}{n_e m_e}\nabla p_e,
\end{equation*}
Using the scalar and vector potentials we have
\begin{equation*}
    \frac{\partial\uvec_e}{\partial t} + \uvec_e \cdot \nabla \uvec_e +\frac{e}{m_e} \left [ -\nabla \phi - \frac{\partial \Avec}{\partial t} + \uvec_e \times \left ( \nabla \times \Avec \right ) \right ] = -\frac{1}{n_e m_e} \nabla p_e.
\end{equation*}
Using the vector identity $\nabla \left ( F^2 / 2 \right ) = \Fvec \times \left ( \nabla \times \Fvec \right ) + \Fvec \cdot \nabla \Fvec$, we write the above as
\begin{equation}
    \frac{\partial\uvec_e}{\partial t} - \uvec_e \times \left ( \nabla \times \uvec_e \right ) + \nabla \left (\frac{u_e^2}{2} \right ) + \frac{e}{m_e} \left [ -\nabla \phi - \frac{\partial \Avec}{\partial t} + \uvec_e \times \left ( \nabla \times \Avec \right ) \right ] = -\frac{1}{n_e m_e} \nabla p_e,
\end{equation}
which is equivalent to 
\begin{equation}
    \label{eq:emp_electron_momentum}
    \frac{\partial\uvec_e}{\partial t} - \uvec_e \times \left ( \nabla \times \uvec_{e,t} \right ) + \nabla \left (\frac{u_e^2}{2} \right ) + \frac{e}{m_e} \left [ -\nabla \phi - \frac{\partial \Avec}{\partial t} + \uvec_e \times \left ( \nabla \times \Avec \right ) \right ] = -\frac{1}{n_e m_e} \nabla p_e.
\end{equation}
We'll now introduce a more specific coordinate system. We'll be dealing with at most three waves at a time: a laser wave, a scattered wave, and a plasma wave (either electron-plasma or ion-acoustic wave). We'll assume all three of these waves lie on a so-called base plane. That is, $\kvec_e$ (or $\kvec_i$), $\kvec_L$, and $\kvec_s$ all point along this plane. We now choose the main transverse direction, that is, the direction of $\uvec_{e,t}$, $\Jvec_t$, $\Evec_t$, and $\Avec$ to be the direction orthogonal to this plane, so that these vectors are orthogonal to any $\kvec$. As an aside, we note that the longitudinal and transverse components of the electron velocity can belong to different waves. That is
\begin{align}
    \uvec_{e,l} &= \hat{\uvec}_{e,l} \exp \left [ i (\kvec_p \cdot \xvec - w_pt) \right ] \\
    \uvec_{e,t} &= \hat{\uvec}_{e,t} \exp \left [ i (\kvec_q \cdot \xvec - w_qt) \right ].
\end{align}
The vectors $\nabla \left ( u_e^2 / 2 \right )$, $\nabla \phi$ and $\nabla p_e$ are all by definition longitudinal. As \cref{eq:ltw_plasma_long_wavevector} states, longitudinal vectors point along their wave vectors. Since we chose all wave vectors to be confined to the base plane, $\nabla \left ( u_e^2 / 2 \right )$, $\nabla \phi$ and $\nabla p_e$ do not have a component along the main transverse direction. As a result, the component of \cref{eq:emp_electron_momentum} along the main transverse direction simplifies to
\begin{equation}
    \label{eq:emp_electron_momentum_transverse}
    \frac{\partial\uvec_{e,t}}{\partial t} - \uvec_{e,l} \times \left ( \nabla \times \uvec_{e,t} \right ) + \frac{e}{m_e} \left [ - \frac{\partial \Avec}{\partial t} + \uvec_{e,l} \times \left ( \nabla \times \Avec \right ) \right ] = 0.
\end{equation}
Using $c = w/k$, we show the following scalings 
\begin{align}
    \frac{1}{c^2} \frac{\partial \uvec_{e,t}}{\partial t} &= -\frac{iw \uvec_{e,t}}{c^2} \sim i \frac{\uvec_{e,t}}{c} k , \nonumber \\
    \frac{1}{c^2} \uvec_{e,l} \times \left (\nabla \times \uvec_{e,t} \right ) & = \frac{i \uvec_{e,l} \times \left ( \kvec \times \uvec_{e,t} \right ) }{c^2} \sim i \frac{\uvec_{e,l}}{c} \frac{\uvec_{e,t}}{c} k , \nonumber \\
    \frac{1}{c^2} \frac{\partial \Avec}{\partial t} &= -\frac{iw \Avec}{c^2} \sim i \frac{\Avec}{c} k , \nonumber \\
    \frac{1}{c^2} \uvec_{e,l} \times \left ( \nabla \times \Avec \right ) &= \frac{i \uvec_{e,l} \times \left ( \kvec \times \Avec \right )}{c^2} \sim i \frac{\uvec_{e,l}}{c} \frac{\Avec}{c} k .
\end{align}
Thus, assuming $\uvec_{e,l} \ll c$, the terms involving the double cross product are smaller than those involving the time derivative. As a result, \cref{eq:emp_electron_momentum_transverse} becomes
\begin{equation}
    \frac{\partial\uvec_{e,t}}{\partial t} - \frac{e}{m_e} \frac{\partial \Avec}{\partial t} = 0.
\end{equation}
Using \cref{eq:em_semi_general_wave_form}, the above is equivalent to 
\begin{equation}
    -i w \uvec_{e,t} +i w \frac{e \Avec}{m_e} = 0,
\end{equation}
which upon re-arranging gives
\begin{equation}
    \label{eq:emp_transverse_velocity}
    \uvec_{e,t} = \frac{e\Avec}{m_e}.
\end{equation}

Using both the transverse current given by \cref{eq:emp_transverse_current} and the transverse velocity given by \cref{eq:emp_transverse_velocity}, \cref{eq:emp_general_trans_vec_pot} can be re-written as
\begin{equation*}
    \frac{\partial^2 \Avec}{\partial t} - c_0^2 \nabla^2 \Avec = -\frac{e n_e}{\epsilon_0} \uvec_{e,t} = -\frac{e^2 n_e}{\epsilon_0 m_e} \Avec.
\end{equation*}
We now use the decomposition $n_e = n_{e0} + n_{e1}$, where $n_{e0}$ is time independent. The above becomes
\begin{equation}
    \label{eq:emp_general_trans_vec_pot_complete}
    \frac{\partial^2 \Avec}{\partial t} + w_{pe}^2 \Avec - c_0^2 \nabla^2 \Avec = -\frac{e^2 n_{e1}}{\epsilon_0 m_e} \Avec,
\end{equation}
where $w_{pe}^2 = e^2 n_{e0} / m_e \epsilon_0$.

%########################################################################
\chapter{Electromagnetic waves in a stable plasma}
%########################################################################

%------------------------------------------------------------------------
\section{The vector potential}
%------------------------------------------------------------------------
\label{sec:semp_vector_potential}
We start with \cref{eq:emp_general_trans_vec_pot_complete}, but focus on the stable-plasma case, that is, $n_{e1} = 0$. Thus, we have
\begin{equation}
    \label{eq:semp_general_trans_vec_pot_complete}
    \frac{\partial^2 \Avec}{\partial t} + w_{pe}^2 \Avec - c_0^2 \nabla^2 \Avec = 0.
\end{equation}
We note that $n_{e0}$ is only time independent, that is, it is still allowed to vary across space. As a result, $w_{pe}^2$ is also allowed to vary across space. Using \cref{eq:em_semi_general_wave_form} for the vector potential, \cref{eq:semp_general_trans_vec_pot_complete} becomes
\begin{equation*}
    -w^2 \Avec + w_{pe}^2 \Avec - c_0^2 \nabla^2 \Avec = 0.
\end{equation*}
We re-write the above as
\begin{equation*}
    \frac{w^2}{c_0^2} \Avec - \frac{w^2}{c_0^2} \frac{w_{pe}^2}{w^2} \Avec + \nabla^2 \Avec = 0.
\end{equation*}
Defining $\epsilon = 1 - w_{pe}^2 / w^2$, we ultimately get
\begin{equation}
    \label{eq:semp_general_trans_vec_pot_complete_spec}
    \frac{w^2}{c_0^2} \epsilon \Avec + \nabla^2 \Avec = 0.
\end{equation}

We now consider the case of a \textit{uniform} stable plasma, that is, a plasma where $n_{e0}$ is uniform across space, and thus $w_{pe}$ and $\epsilon$ are also uniform across space. Using the standard plane-wave expression $\Avec = \hat{\Avec} \exp[i (\kvec \cdot \xvec - wt)]$ in \cref{eq:semp_general_trans_vec_pot_complete_spec} gives the following dispersion relation
\begin{equation}
    \label{eq:semp_uniform_dispersion_relation}
    \frac{w^2}{c_0^2} \epsilon = k^2.
\end{equation}
We expand the above to obtain
\begin{equation*}
    w^2 - w_{pe}^2 = c_0^2 k^2.
\end{equation*}
Taking the derivative $\partial / \partial k$ on both sides we get
\begin{equation*}
    2w \frac{\partial w}{\partial k} = 2 c_0^2 k,
\end{equation*}
which in turn gives the following expressions for the group velocity $v_g$
\begin{equation}
    v_g = \frac{c_0^2 k}{w}.
\end{equation}
Using \cref{eq:semp_uniform_dispersion_relation}, we can also write the above as
\begin{equation}
    v_g = \frac{c_0^2 k}{w} = c_0^2 \frac{\sqrt{\epsilon}}{c_0} = c_0 \sqrt{\epsilon}.
\end{equation}

%------------------------------------------------------------------------
\section{The electric field}
%------------------------------------------------------------------------
Taking the time derivative of \cref{eq:semp_general_trans_vec_pot_complete} gives the equation for $\Evec_t$, that is
\begin{equation}
    \label{eq:semp_general_trans_E_complete}
    \frac{\partial^2 \Evec_t}{\partial t^2} + w_{pe}^2 \Evec_t - c_0^2 \nabla^2 \Evec_t = 0.
\end{equation}
Using \cref{eq:em_semi_general_wave_form} for the electric field, the time derivative in \cref{eq:semp_general_trans_E_complete} evaluates such that 
\begin{equation*}
    -w^2 \Evec_t + w_{pe}^2 \Evec_t - c_0^2 \nabla^2 \Evec_t = 0.
\end{equation*}
We re-write the above as 
\begin{equation*}
    \frac{w^2}{c_0^2} \Evec_t - \frac{w^2}{c_0^2} \frac{w_{pe}^2}{w^2} \Evec_t + \nabla^2 \Evec_t = 0,
\end{equation*}
which becomes
\begin{equation}
    \label{eq:semp_general_trans_E_complete_spec}
    \frac{w^2}{c_0^2} \epsilon \Evec_t + \nabla^2 \Evec_t = 0.
\end{equation}

%------------------------------------------------------------------------
\section{The magnetic field}
%------------------------------------------------------------------------
Taking the curl of \cref{eq:semp_general_trans_vec_pot_complete} gives the equation for $\Bvec$, that is
\begin{equation}
    \label{eq:semp_general_trans_B_complete}
    \frac{\partial^2 \Bvec}{\partial t^2} + w_{pe}^2 \Bvec - c_0^2 \nabla^2 \Bvec + \nabla w_{pe}^2 \times \Avec = 0.
\end{equation}
Using \cref{eq:em_semi_general_wave_form} for the magnetic field, the time derivative in \cref{eq:semp_general_trans_B_complete} evaluates such that 
\begin{equation*}
    -w^2 \Bvec + w_{pe}^2 \Bvec - c_0^2 \nabla^2 \Bvec + \nabla w_{pe}^2 \times \Avec = 0.
\end{equation*}
We re-write the above as
\begin{equation*}
    \frac{w^2}{c_0^2} \Bvec - \frac{w^2}{c_0^2} \frac{w_{pe}^2}{w^2} \Bvec + \nabla^2 \Bvec - \frac{w^2}{c_0^2} \nabla \left ( \frac{w_{pe}^2}{w^2} \right ) \times \Avec = 0,
\end{equation*}
which becomes
\begin{equation*}
    \frac{w^2}{c_0^2} \epsilon \Bvec + \nabla^2 \Bvec + \frac{w^2}{c_0^2} \nabla \epsilon \times \Avec = 0.
\end{equation*}
Using \cref{eq:semp_general_trans_vec_pot_complete_spec} we get 
\begin{equation*}
    \frac{w^2}{c_0^2} \epsilon \Bvec + \nabla^2 \Bvec - \frac{1}{\epsilon} \nabla \epsilon \times \nabla^2 \Avec = 0.
\end{equation*}
The vector identity $\nabla \times \left ( \nabla \times \Fvec \right ) = \nabla \left ( \nabla \cdot \Fvec \right ) - \nabla^2 \Fvec$ gives $\nabla \times \left ( \nabla \times \Avec \right ) = -\nabla^2 \Avec $, or
\begin{equation}
    \label{eq:semp_b_to_a_vec_identity}
    \nabla \times \Bvec = -\nabla^2 \Avec.
\end{equation}
Thus, we finally get
\begin{equation}
    \label{eq:semp_general_trans_B_complete_spec}
    \frac{w^2}{c_0^2} \epsilon \Bvec + \nabla^2 \Bvec + \frac{1}{\epsilon} \nabla \epsilon \times \left ( \nabla \times \Bvec \right ) = 0.
\end{equation}

As a side note, we can use the expressions above to write Ampere's law in a new form. We combine the vector identity \cref{eq:semp_b_to_a_vec_identity} with \cref{eq:semp_general_trans_vec_pot_complete_spec} to obtain
\begin{equation*}
    \nabla \times \Bvec = \frac{w^2}{c_0^2} \epsilon \Avec.
\end{equation*}
Using \cref{eq:em_semi_general_wave_form} for the vector potential, the expression $\Evec_t = -\partial \Avec / \partial t$ gives
\begin{equation*}
    \Evec_t = iw \Avec.
\end{equation*}
Thus, the curl of $\Bvec$ can be expressed as
\begin{equation}
    \nabla \times \Bvec = -i \frac{w}{c_0^2} \epsilon \Evec_t.
\end{equation}

As mentioned in \cref{sec:semp_vector_potential}, for a uniform stable plasma we have $\epsilon$ equal to a constant. Thus, \cref{eq:semp_general_trans_B_complete_spec} becomes identical to \cref{eq:semp_general_trans_E_complete_spec}, that is, the wave forms of $\Evec_t$ and $\Bvec$ are the same. 

%########################################################################
\chapter{Stimulated Raman and Brillouin instabilities}
%########################################################################

%------------------------------------------------------------------------
\section{Linearization}
%------------------------------------------------------------------------
The following decompositions will be used in the derivation of stimulated Raman and Brillouin instabilities:
\begin{align}
    n_i &= n_{i0} + n_{i1}, \nonumber \\
    n_e &= n_{e0} + n_{e1}, \nonumber \\
    p_i &= p_{i0} + p_{i1}, \nonumber \\
    p_e &= p_{e0} + p_{e1}, \nonumber \\
    \uvec_{i,l} &= \uvec_{i0,l} + \uvec_{i1,l}, \nonumber \\
    \uvec_{e,l} &= \uvec_{e0,l} + \uvec_{e1,l}, \nonumber \\
    \Evec_l &= \Evec_{0,l} + \Evec_{1,l}, \nonumber \\
    \Avec &= \Avec_L + \Avec_s.
\end{align}
For these decompositions, we'll assume
\begin{enumerate}
    \item Terms with a subscript 1 are small and thus products of two small quantities can be neglected. \label{it:p_instabilities_assumption_1}
    \item $\uvec_{i0,l}$, $\uvec_{e0,l}$, and $\Evec_{0,l}$,are zero. \label{it:p_instabilities_assumption_2}
    \item $n_{i0}$, $n_{e0}$, $p_{i0}$, and $p_{e0}$ are uniform in space and time. \label{it:p_instabilities_assumption_3}
\end{enumerate}

Thus, unlike the previous section, we do not assume the plasma is stable, that is, we assume fluctuations such as $n_{e1}$ are small but non zero. $\Avec_L$ is the vector potential associated with the laser light, and $\Avec_s$ is the potential associated with the scattered light. For linearization purposes, we'll assume $\Avec_s$ is small. 

Using the decomposition for $\Avec$, \cref{eq:emp_general_trans_vec_pot_complete} is written as 
\begin{equation}
    \frac{\partial^2 \Avec_L}{\partial t} + \frac{\partial^2 \Avec_s}{\partial t} + w_{pe}^2 \Avec_L + w_{pe}^2 \Avec_s - c_0^2 \nabla^2 \Avec_L - c_0^2 \nabla^2 \Avec_s = -\frac{e^2 n_{e1}}{\epsilon_0 m_e} \Avec_L -\frac{e^2 n_{e1}}{\epsilon_0 m_e} \Avec_s,
\end{equation}
Dropping products of small quantities we have
\begin{equation}
    \label{eq:srs_temp1}
    \frac{\partial^2 \Avec_L}{\partial t} + \frac{\partial^2 \Avec_s}{\partial t} + w_{pe}^2 \Avec_L + w_{pe}^2 \Avec_s - c_0^2 \nabla^2 \Avec_L - c_0^2 \nabla^2 \Avec_s = -\frac{e^2 n_{e1}}{\epsilon_0 m_e} \Avec_L.
\end{equation}
We'll assume the laser light is stable, that is, it satisfies \cref{eq:semp_general_trans_vec_pot_complete}, which we re-write below
\begin{equation}
    \label{eq:srs_temp2}
    \frac{\partial^2 \Avec_L}{\partial t} + w_{pe}^2 \Avec_L - c_0^2 \nabla^2 \Avec_L = 0.
\end{equation}
Thus, \cref{eq:srs_temp1} becomes
\begin{equation}
    \frac{\partial^2 \Avec_s}{\partial t} + w_{pe}^2 \Avec_s - c_0^2 \nabla^2 \Avec_s = -\frac{e^2 n_{e1}}{\epsilon_0 m_e} \Avec_L.
\end{equation}
The above shows that the fluctuating $n_{e1}$ couples with the laser light to serve as a source for the scattered light.

The electron density equation is now written as
\begin{equation*}
    \frac{\partial n_{e0} + n_{e1}}{\partial t } + \nabla \cdot \left [ \left ( n_{e0} + n_{e1} \right )\left ( \uvec_{e,t} + \uvec_{e0,l} + \uvec_{e1,l} \right ) \right ] = 0,
\end{equation*}
which, since $\uvec_{e,t}$ is transverse, can be written as
\begin{equation*}
    \frac{\partial n_{e0} + n_{e1}}{\partial t } + \uvec_{e,t} \cdot \nabla \left (n_{e0} + n_{e1} \right ) + \nabla \cdot \left [ \left ( n_{e0} + n_{e1} \right )\left ( \uvec_{e0,l} + \uvec_{e1,l} \right ) \right ] = 0.
\end{equation*}
Given the assumptions in \cref{it:p_instabilities_assumption_1,it:p_instabilities_assumption_2,it:p_instabilities_assumption_3}, the above simplifies to
\begin{equation*}
    \frac{\partial n_{e1}}{\partial t } + \uvec_{e,t} \cdot \nabla n_{e1} + \nabla \cdot \left ( n_{e0} \uvec_{e1,l} \right ) = 0.
\end{equation*}
Since $\uvec_{e,t}$ and $\nabla n_{e1}$ are orthogonal, we finally have
\begin{equation}
    \label{eq:p_instabilities_e_den_linearized}
    \frac{\partial n_{e1}}{\partial t} + \nabla \cdot \left ( n_{e0} \uvec_{e1,l} \right ) = 0.
\end{equation}

The ion density equation is now written as
\begin{equation*}
    \frac{\partial n_{i0} + n_{i1}}{\partial t } + \nabla \cdot \left [ \left ( n_{i0} + n_{i1} \right )\left ( \uvec_{i,t} + \uvec_{i0,l} + \uvec_{i1,l} \right ) \right ] = 0,
\end{equation*}
As stated in \cref{sec:electromagnetic_waves_plasmas}, it is often the case that transverse waves oscillate at such a fast rate that the ions, which have large inertia, are unable to react on comparable time scales. Thus, we can assume $\uvec_{i,t} = 0$,
\begin{equation*}
    \frac{\partial n_{i0} + n_{i1}}{\partial t } + \nabla \cdot \left [ \left ( n_{i0} + n_{i1} \right )\left ( \uvec_{i0,l} + \uvec_{i1,l} \right ) \right ] = 0.
\end{equation*}
Given the assumptions in \cref{it:p_instabilities_assumption_1,it:p_instabilities_assumption_2,it:p_instabilities_assumption_3}, the above simplifies to
\begin{equation}
    \label{eq:p_instabilities_i_den_linearized}
    \frac{\partial n_{i1}}{\partial t } + \nabla \cdot \left ( n_{i0} \uvec_{i1,l} \right ) = 0.
\end{equation}

Consider the electron momentum equation. Subtracting \cref{eq:emp_electron_momentum_transverse} from \cref{eq:emp_electron_momentum} gives
\begin{equation}
    \label{eq:emp_electron_momentum_longitudinal}
    \frac{\partial\uvec_{e,l}}{\partial t} - \uvec_{e,t} \times \left ( \nabla \times \uvec_{e,t} \right ) + \nabla \left (\frac{u_e^2}{2} \right ) + \frac{e}{m_e} \left [ -\nabla \phi +\uvec_{e,t} \times \left ( \nabla \times \Avec \right ) \right ] = -\frac{1}{n_e m_e} \nabla p_e.
\end{equation}
Since $\uvec_{e,t} = e \Avec / m_e$, the above simplifies to
\begin{equation*}
    \frac{\partial\uvec_{e,l}}{\partial t} + \nabla \left (\frac{u_e^2}{2} \right ) - \frac{e}{m_e} \nabla \phi = -\frac{1}{n_e m_e} \nabla p_e,
\end{equation*}
or 
\begin{equation*}
    \frac{\partial\uvec_{e,l}}{\partial t} + \nabla \left (\frac{u_e^2}{2} \right ) + \frac{e}{m_e} \Evec_l = -\frac{1}{n_e m_e} \nabla p_e.
\end{equation*}
Since $\uvec_{e,l}$ and $\uvec_{e,t}$ are orthogonal $u_e^2 = \uvec_e \cdot \uvec_e = u_{e,l}^2 + u_{e,t}^2$. The electron momentum equation is then 
\begin{equation*}
    \frac{\partial\uvec_{e,l}}{\partial t} + \nabla \left (\frac{u_{e,l}^2 + u_{e,t}^2}{2} \right ) + \frac{e}{m_e} \Evec_l = -\frac{1}{n_e m_e} \nabla p_e.
\end{equation*}
Given the assumptions in \cref{it:p_instabilities_assumption_1,it:p_instabilities_assumption_2,it:p_instabilities_assumption_3}, the above simplifies to
\begin{equation*}
    \frac{\partial n_{e0} \uvec_{e1,l}}{\partial t} + n_{e0} \nabla \left (\frac{u_{e,t}^2}{2} \right ) + \frac{e n_{e0}}{m_e} \Evec_{1,l} = -\frac{1}{m_e} \nabla p_{e1} .
\end{equation*}
For the transverse electron velocity we have
\begin{equation*}
    u_{e,t}^2 = \left ( \frac{e \Avec}{m_e} \right ) \cdot \left ( \frac{e \Avec}{m_e} \right ) = \frac{e^2}{m_e^2} \left ( \Avec_L \cdot \Avec_ L + 2 \Avec_L \cdot \Avec_s + \Avec_s \cdot \Avec_s \right ).
\end{equation*}
Since the product of small quantities can be neglected, the $\Avec_s \cdot \Avec_s$ term is dropped. We'll also ignore the $\Avec_L \cdot \Avec_L$ term, given that $\Avec_L$ is stable and thus its magnitude does not play a critical role in the growth of the instabilities. Thus, the electron momentum equation becomes
\begin{equation}
    \label{eq:p_instabilities_e_mom_linearized}
    \frac{\partial n_{e0} \uvec_{e1,l}}{\partial t} + \frac{e^2 n_{e0}}{m_e^2} \nabla \left (\Avec_L \cdot \Avec_s \right ) + \frac{e n_{e0}}{m_e} \Evec_{1,l} = -\frac{1}{m_e} \nabla p_{e1}.
\end{equation}

Consider now the ion momentum equation, given by \cref{eq:pwaves_ion_momentum}, which we re-write below as 
\begin{equation*}
    \frac{\partial n_i \uvec_i}{\partial t} + \nabla \cdot \left ( n_i \uvec_i \uvec_i \right ) - \frac{Ze n_i}{m_i} \left ( \Evec + \uvec_i \times \Bvec \right ) = -\frac{1}{m_i} \nabla p_i,
\end{equation*}
The longitudinal component of the above is
\begin{equation*}
    \frac{\partial n_i \uvec_{i,l}}{\partial t} + \left [ \nabla \cdot \left (n_i \uvec_i \uvec_i \right ) \right ]_l - \frac{Z e n_i}{m_i} \left ( \Evec_l + \uvec_{i,t} \times \Bvec \right ) = -\frac{1}{m_i} \nabla p_i,
\end{equation*}
where $[\cdot]_l$ denotes longitudinal component. Since $\uvec_{i,t} = 0$, we have
\begin{equation*}
    \frac{\partial n_i \uvec_{i,l}}{\partial t} + \left [ \nabla \cdot \left (n_i \uvec_i \uvec_i \right ) \right ]_l - \frac{Z e n_i}{m_i} \Evec_l = -\frac{1}{m_i} \nabla p_i.
\end{equation*}
Using the variable decompositions, we have
\begin{multline*}
    \frac{\partial}{\partial t} \left [ \left ( n_{i0} + n_{i1} \right ) \left ( \uvec_{i0,l} + \uvec_{i1,l} \right ) \right ] \\
    + \left \{ \nabla \cdot \left [ \left (n_{i0} + n_{i1} \right ) \left ( \uvec_{i,t} + \uvec_{i0,l} + \uvec_{i1,l} \right ) \left ( \uvec_{i,t} + \uvec_{i0,l} + \uvec_{i1,l} \right ) \right ] \right \}_l \\
    - \frac{Z e}{m_i} \left ( n_{i0} + n_{i1} \right ) \left ( \Evec_{0,l} + \Evec_{1,l} \right ) = - \frac{1}{m_i} \nabla \left ( p_{i0} + p_{i1} \right ).
\end{multline*}
Given the assumptions in \cref{it:p_instabilities_assumption_1,it:p_instabilities_assumption_2,it:p_instabilities_assumption_3}, the above simplifies to
\begin{equation}
    \label{eq:p_instabilities_i_mom_linearized}
    \frac{\partial n_{i0} \uvec_{i1,l}}{\partial t} - \frac{Z e n_{i0}}{m_i} \Evec_{1,l} = - \frac{1}{m_i} \nabla p_{i1}.
\end{equation}

%------------------------------------------------------------------------
\section{Stimulated Raman Scattering}
%------------------------------------------------------------------------
We employ the same assumptions as for the electron-plasma waves, that is 
\begin{enumerate}
    \item Quasi-neutrality for the base flow, $Zn_{i0} = n_{e0}$.
    \item Uniform ion density, $n_{i1} = 0$.
\end{enumerate}

Combining \cref{eq:p_instabilities_e_mom_linearized} with \cref{eq:p_waves_pressure_gradient} gives 
\begin{equation}
    \label{eq:srs_mom_linearized}
    \frac{\partial n_{e0} \uvec_{e1,l}}{\partial t} + \frac{e^2 n_{e0}}{m_e^2} \nabla \left (\Avec_L \cdot \Avec_s \right ) + \frac{e n_{e0}}{m_e} \Evec_{1,l} = -\frac{\gamma_e k_B T_e}{m_e} \nabla n_{e1}.
\end{equation}
Taking the time derivative of \cref{eq:p_instabilities_e_den_linearized} and using \cref{eq:srs_mom_linearized} leads to the wave equation for electron density
\begin{equation*}
    \frac{\partial^2 n_{e1}}{\partial t^2} - \frac{e^2 n_{e0}}{m_e^2} \nabla^2 \left (\Avec_L \cdot \Avec_s \right ) - \frac{e n_{e0}}{m_e} \nabla \cdot \Evec_{1,l} = \frac{\gamma_e k_B T_e}{m_e} \nabla^2 n_{e1}.
\end{equation*}
As before, using \cref{eq:ep_waves_efield_divergence} we obtain
\begin{equation*}
    \frac{\partial^2 n_{e1}}{\partial t^2} - \frac{e^2 n_{e0}}{m_e^2} \nabla^2 \left (\Avec_L \cdot \Avec_s \right ) + \frac{e^2 n_{e0}}{m_e \epsilon_0} n_{e1} = \frac{\gamma_e k_B T_e}{m_e} \nabla^2 n_{e1}.
\end{equation*}
or
\begin{equation}
    \frac{\partial^2 n_{e1}}{\partial t^2} + w_{pe}^2 n_{e1} - \frac{\gamma_e k_B T_e}{m_e} \nabla^2 n_{e1} =   \frac{e^2 n_{e0}}{m_e^2} \nabla^2 \left (\Avec_L \cdot \Avec_s \right ).
\end{equation}
Thus, the scattered laser light $\Avec_s$ couples with the laser light to serve as a source for the electron-plasma wave.

%------------------------------------------------------------------------
\section{Stimulated Brillouin Scattering}
%------------------------------------------------------------------------
We employ the same assumptions as for the ion-acoustic waves, that is
\begin{enumerate}
    \item Quasi-neutrality for the base flow, $Zn_{i0} = n_{e0}$.
    \item Approximate quasi-neutrality for the fluctuations, $Z n_{i1} \approx n_{e1}$.
    \item Negligible electron mass, $m_e \to 0$.
\end{enumerate}
Combining \cref{eq:p_instabilities_i_mom_linearized} with \cref{eq:p_waves_pressure_gradient} gives
\begin{equation}
    \label{eq:sbs_mom_linearized}
    \frac{\partial n_{i0} \uvec_{i1,l}}{\partial t} - \frac{Z e n_{i0}}{m_i} \Evec_{1,l} = - \frac{\gamma_i k_B T_i}{m_i} \nabla n_{i1}.
\end{equation}
Taking the time derivative of \cref{eq:p_instabilities_i_den_linearized} and using \cref{eq:sbs_mom_linearized} leads to the wave equation for ion density
\begin{equation}
    \frac{\partial^2 n_{i1}}{\partial t^2} + \frac{Z e n_{i0}}{m_i} \nabla \cdot \Evec_{1,l} = \frac{\gamma_i k_B T_i}{m_i} \nabla^2 n_{i1}.
\end{equation}
For this case, we assume that the mass of the electron, which is significantly smaller than that of the ions, is negligible. Thus, \cref{eq:p_instabilities_e_mom_linearized} simplifies to 
\begin{equation}
    \frac{e^2 n_{e0}}{m_e} \nabla \left (\Avec_L \cdot \Avec_s \right ) + e n_{e0} \Evec_{1,l} = -\gamma_e k_B T_e \nabla n_{e1}.
\end{equation}
Using the above in the ion wave equation we obtain
\begin{equation}
    \frac{\partial^2 n_{i1}}{\partial t^2} = \frac{Z n_{i0}}{n_{e0}} \frac{\gamma_e k_B T_e}{m_i} \nabla^2 n_{e1} + \frac{\gamma_i k_B T_i}{m_i} \nabla^2 n_{i1} + \frac{Z e^2 n_{i0}}{m_i m_e} \nabla^2 \left (\Avec_L \cdot \Avec_s \right ).
\end{equation}
Due to quasi-neutrality, we have $Zn_{i0} = n_{e0}$ and $Zn_{i1} \approx n_{e1}$, which gives
\begin{equation}
    \frac{\partial^2 n_{i1}}{\partial t^2} = \left ( \frac{Z \gamma_e k_B T_e}{m_i} + \frac{\gamma_i k_B T_i}{m_i} \right ) \nabla^2 n_{i1} + \frac{Z e^2 n_{i0}}{m_i m_e} \nabla^2 \left (\Avec_L \cdot \Avec_s \right ).
\end{equation}
We now use the velocity defined in \cref{eq:ia_waves_vs}, which allows us to write the equation for $n_{i1}$ as
\begin{equation}
    \frac{\partial^2 n_{i1}}{\partial t^2} - v_s^2 \nabla^2 n_{i1} = \frac{Z e^2 n_{i0}}{m_i m_e} \nabla^2 \left (\Avec_L \cdot \Avec_s \right ).
\end{equation}
Thus, the scattered laser light $\Avec_s$ couples with the laser light to serve as a source for the ion-acoustic wave.

%%%%%%%%%%%%%%%%%%%%%%%%%%%%%%%%%%%%%%%%%%%%%%%%%%%%%%%%%%%%%%%%%%%%%%%%%
\appendix
%%%%%%%%%%%%%%%%%%%%%%%%%%%%%%%%%%%%%%%%%%%%%%%%%%%%%%%%%%%%%%%%%%%%%%%%%

%########################################################################
\chapter{Multi-component fluid flows}
%########################################################################
Multi-component fluid flows are governed by the multi-component Navier-Stokes equations. These consist of the traditional Navier-Stokes equations augmented with multiple scalars, where each scalar is the species mass fraction of a component in the system. The species mass fraction of component $\alpha$ is given by $Y_\alpha = m_\alpha/m$, where $m_\alpha = m_\alpha(t,\xvec)$ is the mass of the $\alpha$ component, and $m = m(t,\xvec)$ is the total mass. The species mass fractions are no longer passive scalars. For ideal gases, for example, they affect the value of $R$, which in turns affects the pressure, density, and temperature of the whole mixture. They also affect the enthalpy diffusion, which then alters the heat flux and internal energy. 

%------------------------------------------------------------------------
\section{Relevant fractions}
%------------------------------------------------------------------------
\begin{itemize}
    \item Mole fraction ($X_\alpha$): number of moles $o_\alpha$ of the $\alpha$ species over the total number of moles $o$.
    \begin{equation}
        X_\alpha = \frac{o_\alpha}{o}.
    \end{equation}
    \item Mass fraction ($Y_\alpha$): mass $m_\alpha$ of the $\alpha$ species over the total mass $m$. 
    \begin{equation}
        Y_\alpha = \frac{m_\alpha}{m}.
    \end{equation}
    Using $m_\alpha = o_\alpha M_\alpha$, where $M_\alpha$ is the molar mass of species $\alpha$, we can introduce two relationships between $X_\alpha$ and $Y_\alpha$:
    \begin{equation}
    \label{eq:y_intermsof_x}
        Y_\alpha = \frac{o_\alpha M_\alpha}{\sum_\beta o_\beta M_\beta} = \frac{X_\alpha M_\alpha}{\sum_\beta X_\beta M_\beta}
    \end{equation}
    \begin{equation}
    \label{eq:x_intermsof_y}
        X_\alpha = \frac{m_\alpha / M_\alpha}{\sum_\beta m_\beta / M_\beta} = \frac{Y_\alpha / M_\alpha}{ \sum_\beta Y_\beta / M_\beta}
    \end{equation}
    \item Volume fraction ($V_\alpha$): volume $v^{(\alpha)}$ of the $\alpha$ species over the total volume $v$.
    \begin{equation}
        V_\alpha = \frac{v_\alpha}{v}.
    \end{equation}
    If we define the species density as $\rho_\alpha = m_\alpha/v_\alpha$, then
    \begin{equation}
    \label{eq:relation_volume_density_mass}
        V_\alpha = \frac{v_\alpha}{v} \frac{m}{m} \frac{m_\alpha}{m_\alpha} = \frac{\rho Y_\alpha}{\rho_\alpha}.
    \end{equation}
\end{itemize}

%------------------------------------------------------------------------
\section{Equations of state}
%------------------------------------------------------------------------
The main thermodynamic variables are $\rho_\alpha$, $p_\alpha$, $T_\alpha$, and $e_\alpha$, for a total of $4N$ variables, where $N$ is the number of species.

The thermal equation of state is expressed as
\begin{equation}
    p_\alpha = \phi_\alpha (\rho_\alpha, T_\alpha)
\end{equation}
and the caloric equation of state as
\begin{equation}
    e_\alpha = \psi_\alpha (\rho_\alpha, T_\alpha)
\end{equation}
We now assume an equilibrium state in which all of the species have the same pressure and temperature. Thus, the total number of unknowns is reduced to $2N+2$. The thermal and caloric equations of state take the form
\begin{equation}
\label{eq:eos_thermal_multicomponent}
    p = \phi_\alpha (\rho_\alpha, T),
\end{equation}
\begin{equation}
\label{eq:eos_caloric_multicomponent}
    e_\alpha = \psi_\alpha (\rho_\alpha, T).
\end{equation}
We need two more equations to provide closure. These are
\begin{equation}
\label{eq:from_rho_alpha_to_rho}
   \frac{1}{\rho} = \sum_\alpha \frac{Y_\alpha}{\rho_\alpha},
\end{equation}
and
\begin{equation}
\label{eq:from_e_alpha_to_e}
    e = \sum_\alpha Y_\alpha e_\alpha.
\end{equation}
For the above, $\rho$, $Y_\alpha$, and $e$ are known variables provided by the fluid transport equations. Thus, \cref{eq:eos_thermal_multicomponent,eq:eos_caloric_multicomponent,eq:from_rho_alpha_to_rho,eq:from_e_alpha_to_e} constitute $2N+2$ equations for the $2N+2$ unknowns. \Cref{eq:from_rho_alpha_to_rho} is derived as follows
\begin{equation}
    \frac{1}{\rho} = \frac{v}{m} = \sum_\alpha \frac{v_\alpha}{m} = \sum_\alpha \frac{m_\alpha / m}{m_\alpha / v_\alpha} = \sum_\alpha \frac{Y_\alpha}{\rho_\alpha},
\end{equation}
and \cref{eq:from_e_alpha_to_e} follows from the fact that the total internal energy $m e$ is simply the sum of the internal energies $m_\alpha e_\alpha$ of all the species. 

An additional thermodynamic variable, the species enthalpy $h_\alpha$, is simply defined in terms of the others according to 
\begin{equation}
\label{eq:species_enthalpy}
    h_\alpha = e_\alpha + p / \rho_\alpha.
\end{equation}
For the entire system, we can show that
\begin{equation}
\label{eq:from_h_alpha_to_h}
     h = e + \frac{p}{\rho} = \sum Y_\alpha e_\alpha + \frac{Y_\alpha}{\rho_\alpha} p = \sum_\alpha Y_\alpha h_\alpha.
\end{equation}

In the absence of chemical reactions, condensation, or other processes, $Y_i$ is independent of temperature. For these cases, differentiating \cref{eq:from_e_alpha_to_e} and \cref{eq:from_h_alpha_to_h} gives
\begin{equation}
    C_v = \sum_\alpha Y_\alpha C_{v,\alpha},
\end{equation}
\begin{equation}
    C_p = \sum_\alpha Y_\alpha C_{p,\alpha}.
\end{equation}

%---------------------------------
\subsection{Ideal gasses}
%---------------------------------
For an ideal gas, \cref{eq:eos_thermal_multicomponent} and \cref{eq:eos_caloric_multicomponent} are given by
\begin{equation}
\label{eq:eos_thermal_multicomponent_ideal}
p = \rho_\alpha R_\alpha T,
\end{equation}
and
\begin{equation} 
\label{eq:eos_caloric_multicomponent_ideal}
e_\alpha = C_{v,\alpha} T .
\end{equation}
In the above, $R_\alpha = R_u / M_u$ is the ideal gas constant of each species and $C_{v,\alpha}$ is the specific heat at constant temperature for each species. For the species enthalpy 
\begin{equation}
    h_\alpha = e_\alpha + \frac{p}{\rho_\alpha} = e_\alpha + R_\alpha T = C_{v,\alpha} T + \left ( C_{p,\alpha} - C_{v,\alpha} \right ) T = C_{p,\alpha} T.
\end{equation}

The thermal equation of state for the entire mixture can be derived from the thermal equation of state for each species. We first multiply \cref{eq:eos_thermal_multicomponent_ideal} by $Y_\alpha$, divide by $\rho_\alpha$, and sum over all $\alpha$. That is
\begin{equation}
\sum_\alpha p \frac{Y_\alpha}{\rho_\alpha}= \sum_\alpha Y_\alpha R_\alpha T.
\end{equation}
The above can be expressed as
\begin{equation}
p = \rho \sum_\alpha \frac{Y_\alpha}{M_\alpha} R_u T.
\end{equation}
We now define the the average molar mass of the mixture as $M = \sum_\alpha X_\alpha M_\alpha$. Using \cref{eq:y_intermsof_x}, this can be rewritten as 
\begin{equation}
    \frac{1}{M} = \sum_\alpha \frac{Y_\alpha}{M_\alpha}
\end{equation}
Thus, the thermal equation of state for the entire mixture can be expressed as
\begin{equation}
    p = \rho R T
\end{equation}
where
\begin{equation}
R = \frac{R_u}{M}
\end{equation}
Using the above, we can easily show that
\begin{equation}
    R = \sum_\alpha Y_\alpha \frac{R_u}{M_\alpha} = \sum_\alpha Y_\alpha R_\alpha = \sum_\alpha Y_\alpha \left ( C_{p,\alpha} - C_{v,\alpha} \right ) = C_p - C_v.
\end{equation}

%------------------------------------------------------------------------
\section{Summary of multi-component Navier-Stokes equations for ideal gas}
%------------------------------------------------------------------------
%---------------------------------
\subsection{Conservation equations}
%---------------------------------
\begin{equation}
\frac{\partial \rho}{\partial t} + \frac{\partial \rho u_i}{\partial x_i} = 0
\end{equation}
\begin{equation}
\frac{\partial \rho u_i}{\partial t} + \frac{\partial \rho u_i u_j}{\partial x_j} = - \frac{\partial p}{\partial x_i} + \frac{\partial t_{ij}}{\partial x_j} + \rho f_i
\end{equation}
\begin{equation}
\frac{\partial \rho E}{\partial t} + \frac{\partial}{\partial x_i} \left [ \rho \left ( E + \frac{p}{\rho} \right ) u_i \right ] = \frac{\partial u_i t_{ij}}{\partial x_j} + \rho f_i u_i -  \frac{\partial q_i}{\partial x_i}
\end{equation}
\begin{equation}
\frac{\partial\rho Y_\alpha}{\partial t}+\frac{\partial \rho Y_\alpha u_i}{\partial x_i} = -\frac{\partial J_{\alpha,i}}{\partial x_i} + w_\alpha
\end{equation}

%---------------------------------
\subsection{Transport models}
%---------------------------------
Shear stress, heat flux, and diffusive species flux:
\begin{equation}
t_{ij} = 2\mu S_{ij}^*
\end{equation}
\begin{equation}
q_i = -\kappa \frac{\partial T}{\partial x_i}  + \sum_{\alpha=1} h_\alpha J_{\alpha,i}
\end{equation}
\begin{equation}
J_{\alpha,i} = -\rho \left ( D_\alpha \frac{\partial Y_\alpha}{\partial x_i} - Y_\alpha \sum_\beta D_\beta \frac{\partial Y_\beta}{\partial x_i} \right )
\end{equation}

Transport coefficients:
\begin{equation}
\mu = \mu_0 \left ( \frac{T}{T_0} \right )^n
\end{equation}
\begin{equation}
\kappa = \frac{\mu C_p}{Pr}
\end{equation}
\begin{equation}
D_\alpha = \frac{\mu}{\rho Sc_\alpha}
\end{equation}

%---------------------------------
\subsection{Equation of state}
%---------------------------------
Perfect gas:
\begin{equation}
    p = \rho R T 
\end{equation}
\begin{equation}
    R = \frac{R_u}{M}
\end{equation}
\begin{equation}
    \frac{1}{M} = \sum_\alpha \frac{Y_\alpha}{M_\alpha}
\end{equation}

\begin{equation}
    e = C_v T
\end{equation}
\begin{equation}
    h_\alpha = C_{p,\alpha} T
\end{equation}
\begin{equation}
    C_v = \sum_\alpha Y_\alpha C_{v,\alpha}
\end{equation}
\begin{equation}
    C_p = \sum_\alpha Y_\alpha C_{p,\alpha}
\end{equation}

%---------------------------------
\subsection{Additional relations}
%---------------------------------
\begin{equation}
E = e + K
\end{equation}
\begin{equation}
K = \frac{1}{2} u_i u_i
\end{equation}
\begin{equation}
    S^*_{ij} = \frac{1}{2} \left ( \frac{\partial u_i}{\partial x_j} + \frac{\partial u_j}{\partial x_i} \right ) - \frac{1}{3} \frac{\partial u_k}{\partial x_k} \delta_{ij}
\end{equation}

%########################################################################
\chapter{Lagrangian and Eulerian PDFs}
%########################################################################

%------------------------------------------------------------------------
\section{Eulerian PDF}
%------------------------------------------------------------------------
Consider an Eulerian velocity field $\uvec = \uvec(\xvec,t)$. The Eulerian PDF $f = f(\Vvec; \xvec,t)$ gives the probability that the velocity field will have a value of $\Vvec$ at location $\xvec$ and at time $t$. We'll also introduce the fine-grained Eulerian PDF $f' = f'(\Vvec;\xvec,t)$, which is defined as 
\begin{equation}
    f'(\Vvec; \xvec, t) = \delta(\uvec(\xvec,t) - \Vvec).
\end{equation}
Note: a delta function of a 3D argument means the following $\delta(\avec) = \delta(a_1) \delta(a_2) \delta(a_3) $. The Eulerian PDF can be obtained from the fine-grained Eulerian PDF using 
\begin{equation}
    \label{eq:fine_eul_pdf}
    f(\Vvec; \xvec, t)=\langle f'(\Vvec; \xvec, t) \rangle.
\end{equation}
The proof is as follows,
\begin{align}
    \langle f'(\Vvec; \xvec, t) \rangle &= \langle \delta( \uvec(\xvec,t) - \Vvec) \rangle \nonumber \\
    &= \int \delta( \Vvec' - \Vvec) f(\Vvec';\xvec,t) d\Vvec' \nonumber \\
    &= f(\Vvec; \xvec, t).
\end{align}

%------------------------------------------------------------------------
\section{Lagrangian PDF}
%------------------------------------------------------------------------
Consider a Lagrangian particle with velocity $\uvec^+ = \uvec^+(t,\yvec)$ and position $\xvec^+(t,\yvec)$. The Lagrangian PDF $f_L = f_L(\Vvec, \xvec; t | \yvec)$ gives the probability that the particle that started at location $\yvec$ at the reference time $t_0$ will have a velocity $\Vvec$ and position $\xvec$ at time $t$. We'll also introduce the fine-grained Eulerian PDF $f'_L = f'_L(\Vvec, \xvec; t | \yvec)$, which is defined as 
\begin{equation}
    f'_L(\Vvec, \xvec; t | \yvec) = \delta(\uvec^+(t,\yvec) - \Vvec) \delta(\xvec^+(t,\yvec) - \xvec).
\end{equation}
Note: a delta function of a 3D argument means the following $\delta(\avec) = \delta(a_1) \delta(a_2) \delta(a_3) $. The Lagrangian PDF can be obtained from the fine-grained Lagrangian PDF using
\begin{equation}
    \label{eq:fine_lag_pdf}
    f_L(\Vvec, \xvec; t | \yvec) = \langle f'_L(\Vvec, \xvec; t | \yvec) \rangle.
\end{equation} 
The proof is as follows,
\begin{align}
    \langle f'_L(\Vvec, \xvec; t | \yvec) \rangle &= \langle \delta( \uvec^+(t,\yvec) - \Vvec) \delta(\xvec^+(t,\yvec) - \xvec) \rangle \nonumber \\
    &= \int \delta( \Vvec' - \Vvec) \delta ( \xvec' - \xvec) f(\Vvec', \xvec';t | \yvec) d\Vvec' d\xvec' \nonumber \\
    &= f_L(\Vvec, \xvec; t | \yvec).
\end{align}

%------------------------------------------------------------------------
\section{Relation between Lagrangian and Eulerian PDFs}
%------------------------------------------------------------------------
As a quick side note, we mention that the inverse of $\xvec^+$ is $\yvec^+ = \yvec^+(t,\zvec)$, which gives the initial location of a fluid particle that at time $t$ is located at position $\zvec$. Thus, $\xvec^+(t,\yvec^+(t,\zvec)) = \zvec$.

We begin as follows
\begin{align}
\int f'_L(\Vvec,\xvec;t|\yvec) \, d\yvec &= \int \delta(\uvec^+(t,\yvec) - \Vvec) \delta(\xvec^+(t,\yvec) - \xvec) \, d\yvec \nonumber \\
&= \int \delta(\uvec(\xvec^+(t,\yvec),t) - \Vvec) \delta(\xvec^+(t,\yvec) - \xvec) \, d\yvec \nonumber \\
&= \int \delta(\uvec(\xvec^+(t,\yvec),t) - \Vvec) \delta(\xvec^+(t,\yvec) - \xvec) | \det D \xvec^+ | \, d\yvec,
\end{align}
where we have introduced $| \det D \xvec^+ |$, which is the absolute value of the determinant of the Jacobean $\partial \xvec^+/\partial \yvec$, and is equal to one for incompressible flows. Using integration by substitution we obtain
\begin{equation}
\int f'_L(\Vvec,\xvec;t|\yvec) \, d\yvec = \int \delta(\uvec(\zvec,t) - \Vvec) \delta(\zvec - \xvec) \, d\zvec = \delta(\uvec(\xvec,t) - \Vvec)
\end{equation}
Given the definition of $f'(\Vvec; \xvec, t)$, we have
\begin{equation}
    \label{eq:fine_eul_lag_pdf}
    \int f'_L(\Vvec,\xvec;t|\yvec) \, d\yvec = f'(\Vvec; \xvec, t).
\end{equation}
Taking the expectation of the above we obtain
\begin{equation}
    \label{eq:eul_lag_pdf}
    \int f_L(\Vvec,\xvec;t|\yvec) \, d\yvec = f(\Vvec;\xvec,t).
\end{equation}

A summary of all of the relations derived thus far is given by the following graph
\setlength{\unitlength}{1cm}
\begin{center}
    \begin{picture}(12,2.5)(0,0)
        \put(0.5,0){Eulerian PDF}
        \put(8,0){Lagrangian PDF}
            \put(8.0,0.1){\vector(-1,0){5.0}}
            \put(4.5,0.2){\cref{eq:eul_lag_pdf}}
        \put(-0.5,2){Eulerian fine-grained PDF}
            \put(1.5,1.9){\vector(0,-1){1.5}}
            \put(1.6,1.2){\cref{eq:fine_eul_pdf}}
        \put(7,2){Lagrangian fine-grained PDF}
            \put(9.0,1.9){\vector(0,-1){1.5}}
            \put(9.1,1.2){\cref{eq:fine_lag_pdf}}
            \put(7.0,2.1){\vector(-1,0){3.0}}
            \put(4.7,2.2){\cref{eq:fine_eul_lag_pdf}}
    \end{picture}
\end{center}

%------------------------------------------------------------------------
\section{Evolution equation for fine-grained Eulerian PDF}
%------------------------------------------------------------------------

%------------------------------------------------------------------------
\section{Evolution equation for fine-grained Lagrangian PDF}
%------------------------------------------------------------------------

\end{document}
