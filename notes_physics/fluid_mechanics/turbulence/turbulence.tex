% FINISHED UPDATING NOTATION UP TO SECTION 2.3.5 
% (Internal energy for FAVRE averaging)

\documentclass[oneside,a4paper,11pt]{report}

\usepackage{"../../../info/packages"}
\usepackage{"../../../info/nomenclature"}
\usepackage{fullpage}
\usepackage{array}
\usepackage{cellspace}

\setlength{\cellspacetoplimit}{2pt}
\setlength{\cellspacebottomlimit}{2pt}

% RANS variables
\newcommand{\rhoavg}{\overline{\rho}}

\newcommand{\aavg}{\overline{a}}
\newcommand{\eavg}{\langle e \rangle}
\newcommand{\havg}{\langle h \rangle}
\newcommand{\pavg}{\overline{p}}
\newcommand{\qavg}{\overline{q}}
\newcommand{\tavg}{\overline{t}}
\newcommand{\uavg}{\overline{u}}
\newcommand{\vavg}{\overline{v}}
\newcommand{\yavg}{\overline{Y}}

\newcommand{\Javg}{\overline{J}}
\newcommand{\Kavg}{\overline{K}}
\newcommand{\Savg}{\overline{S}}
\newcommand{\Wavg}{\overline{W}}

\newcommand{\afluc}{a'}
\newcommand{\pfluc}{p'}
\newcommand{\sfluc}{s'}
\newcommand{\tfluc}{t'}
\newcommand{\ufluc}{u'}
\newcommand{\vfluc}{v'}
\newcommand{\wfluc}{w'}
\newcommand{\yfluc}{y'}

\newcommand{\rs}{\tau}          % Reynolds stresses
\newcommand{\ars}{a}            % Anisotropic Reynolds stresses
\newcommand{\nars}{b}           % Non-dimensional anisotropic Reynolds stress
\newcommand{\tpvc}{\mathcal{R}} % Two-point velocity correlation
\newcommand{\redi}{\phi}        % Redistribution
\newcommand{\est}{E}            % Energy spectrum tensor

\newcommand{\Stau}{S^*}
\newcommand{\Wtau}{W^*}
\newcommand{\Sdev}{\hat{S}}

%--------old variables-------------
\newcommand{\uiavg}{\langle U_i \rangle}
\newcommand{\ujavg}{\langle U_j \rangle}
\newcommand{\ukavg}{\langle U_k \rangle}
\newcommand{\ulavg}{\langle U_l \rangle}
\newcommand{\upavg}{\langle U_p \rangle}
\newcommand{\uqavg}{\langle U_q \rangle}
\newcommand{\uifluc}{u_i}
\newcommand{\ujfluc}{u_j}
\newcommand{\ukfluc}{u_k}
\newcommand{\umfluc}{u_m}
\newcommand{\unfluc}{u_n}
\newcommand{\upfluc}{u_p}
\newcommand{\uqfluc}{u_q}
\newcommand{\urfluc}{u_r}
\newcommand{\utfluc}{u_t}
\newcommand{\usfluc}{u_s}


\newcommand{\wiavg}{\langle W_i \rangle}
\newcommand{\wjavg}{\langle W_j \rangle}
\newcommand{\wkavg}{\langle W_k \rangle}
\newcommand{\wlavg}{\langle W_l \rangle}
\newcommand{\wpavg}{\langle W_p \rangle}
\newcommand{\wqavg}{\langle W_q \rangle}
\newcommand{\wifluc}{w_i}
\newcommand{\wjfluc}{w_j}
\newcommand{\wkfluc}{w_k}
\newcommand{\wmfluc}{w_m}
\newcommand{\wnfluc}{w_n}
\newcommand{\wpfluc}{w_p}
\newcommand{\wqfluc}{w_q}
\newcommand{\wrfluc}{w_r}
\newcommand{\wtfluc}{w_t}
\newcommand{\wsfluc}{w_s}

\newcommand{\phiavg}{\overline{\Phi}}
\newcommand{\phifluc}{\phi'}

% Favre-averaged variables
\newcommand{\aavgf}{\widetilde{a}}
\newcommand{\bavgf}{\widetilde{b}}
\newcommand{\cavgf}{\widetilde{c}}
\newcommand{\eavgf}{\widetilde{e}}
\newcommand{\favgf}{\widetilde{f}}
\newcommand{\uavgf}{\widetilde{u}}

\newcommand{\Eavgf}{\widetilde{E}}
\newcommand{\Tavgf}{\widetilde{T}}
\newcommand{\Yavgf}{\widetilde{Y}}
\newcommand{\Savgf}{\widetilde{S}}

\newcommand{\aflucf}{a''}
\newcommand{\bflucf}{b''}
\newcommand{\eflucf}{e''}
\newcommand{\fflucf}{f''}
\newcommand{\cflucf}{c''}
\newcommand{\Yflucf}{Y''}

\newcommand{\Tflucf}{T''}
%--------old variables-------------
\newcommand{\ukavgf}{\widetilde{U}_k}

\newcommand{\Uavgf}{\widetilde{U}}
\newcommand{\Favgf}{\widetilde{F}}

\newcommand{\ukflucf}{u_k''}
\newcommand{\uflucf}{u''}

% Filtered variables
\newcommand{\ures}{\overline{U}}
\newcommand{\usgs}{u'}
\newcommand{\pres}{\overline{P}}

\title{Classical Turbulence \thanks{Disclaimer: a large chunk of the work in this document is not original; instead, parts of this document are personal notes obtained from books and papers I've read, principal among which are Pope (2000), etc.}}
\date{\today}
\author{Alejandro Campos}

\begin{document}
\maketitle
\tableofcontents

%%%%%%%%%%%%%%%%%%%%%%%%%%%%%%%%%%%%%%%%%%%%%%%%%%%%%%%%%%%%%%%%%%%%%%%%%
%                                                                       %
%                                                                       %
%                                                                       %
%                                                                       %
\part{Introductory Concepts}
%                                                                       %
%                                                                       %
%                                                                       %
%                                                                       %
%%%%%%%%%%%%%%%%%%%%%%%%%%%%%%%%%%%%%%%%%%%%%%%%%%%%%%%%%%%%%%%%%%%%%%%%%

%###############################################################################
%
\chapter{Governing Equations for Fluid Dynamics}
%
%###############################################################################
Before we derive the conservation equations for turbulent quantities, we first briefly summarize the conservation equations for the instantaneous fluid variables.
\begin{itemize}
\item Conservation of mass
\begin{equation}
    \label{eq:cons_mass}
    \frac{\partial \rho}{\partial t} + \frac{ \partial \rho u_i}{ \partial x_i} = 0.
\end{equation}
\item Conservation of momentum
\begin{equation}
    \label{eq:cons_momentum}
    \frac{\partial \rho u_i}{\partial t}+\frac{\partial \rho u_i u_j}{\partial x_j} =-\frac{\partial p}{\partial x_i} + \frac{\partial t_{ij}}{\partial x_j} + \rho f_i.
\end{equation}
\item Conservation of total energy
\begin{equation}
    \label{eq:cons_energy}
    \frac{\partial \rho E}{\partial t} + \frac{\partial}{\partial x_j} \left [ \rho \left ( E + \frac{p}{\rho} \right ) u_j \right ] = \frac{\partial u_i t_{ij}}{\partial x_j} + \rho f_j u_j - \frac{\partial q_j}{\partial x_j}.
\end{equation}
\item Conservation of scalar
\begin{equation}
    \label{eq:cons_scalar}
    \frac{\partial \rho Y}{\partial t} + \frac{\partial \rho Y u_j}{\partial x_j} = -\frac{\partial J_j}{\partial x_j} + w
\end{equation}
\end{itemize}
In the above, $\rho$ is the density, $u_i$ the velocity, $Y$ the scalar, $J_j$ the diffusive scalar flux, $w$ the scalar source, $p$ the pressure, $t_{ij}$ the shear-stress tensor, $f_i$ a body force, $E$ the total energy, and $q$ the heat-conduction flux. The diffusive scalar flux, the shear-stress tensor, and the heat-conduction flux are given by
\begin{equation}
\label{eq:shear_stress}
    t_{ij} = 2\mu S_{ij}^*
\end{equation}
\begin{equation}
    q_i = -\kappa \frac{\partial T}{\partial x_i}
\end{equation}
\begin{equation}
    \label{eq:diffusive_scalar_flux}
    J_i = -\rho D \frac{\partial Y}{\partial x_i} 
\end{equation}
where $S_{ij}^*$ is the deviatoric rate-of-strain tensor, and $T$ the temperature. The transport coefficients are the diffusivity $D$, the viscosity $\mu$ and the thermal conductivity $\mu$, which are obtained from
\begin{equation}
    \mu = \mu_0 \left ( \frac{T}{T_0} \right )^n
\end{equation}
\begin{equation}
    \kappa = \frac{\mu C_p}{Pr}
\end{equation}
\begin{equation}
    D = \frac{\mu}{\rho Sc}
\end{equation}
In the above, $\mu_0$ is a reference viscosity, $T_0$ a reference temperature, $n$ the power-law exponent, $Sc$ the Schmidt number, $Pr$ the Prandlt number, and $C_p$ the heat capacity at constant pressure. A perfect gas is assumed, i.e.,
\begin{equation}
    p = \rho R T,
\end{equation}
\begin{equation} 
    e = C_v T ,
\end{equation}
\begin{equation}
    R = \frac{R_u}{M}.
\end{equation}
To obtain closure, the following is still needed
\begin{equation}
     E = e + K ,
\end{equation}
\begin{equation}
    K = \frac{1}{2} u_i u_i ,
\end{equation}
\begin{equation}
    S^*_{ij} = \frac{1}{2} \left ( \frac{\partial u_i}{\partial x_j} + \frac{\partial u_j}{\partial x_i} \right ) - \frac{1}{3} \frac{\partial u_k}{\partial x_k} \delta_{ij} 
\end{equation}

From the governing equations, additional transport PDEs can be derived.
\begin{itemize}
\item Conservation of kinetic energy
\begin{equation}
    \label{eq:cons_kinetic_energy}
    \frac{\partial \rho K}{\partial t} + \frac{\partial \rho K u_j}{\partial x_j} = - \frac{\partial u_j p}{\partial x_j} + \frac{\partial u_i t_{ij}}{\partial x_j} + p \frac{\partial u_i}{\partial x_i} - t_{ij} \frac{\partial u_i}{\partial x_j} + \rho f_j u_j .
\end{equation}
\item Conservation of internal energy
\begin{equation}
    \label{eq:cons_internal_energy}
    \frac{ \partial \rho e}{\partial t} + \frac{\partial \rho e u_j}{\partial x_j} = -\frac{\partial q_j}{\partial x_j} - p \frac{\partial u_i}{\partial x_i} + t_{ij} \frac{\partial u_i}{\partial x_j}.
\end{equation}
\item Conservation of squared scalar
\begin{equation}
    \label{eq:cons_squared_scalar}
    \frac{\partial \rho Y^2}{\partial t} + \frac{\partial \rho Y^2 u_j}{\partial x_j} = -2\frac{\partial Y J_j}{\partial x_j} + 2J_j \frac{\partial Y}{\partial x_j} + 2 Y w.
\end{equation}
\end{itemize}

%###############################################################################
%
\chapter{Governing Statistical Equations for Turbulent Flows}
%
%###############################################################################

%-------------------------------------------------------------------------------
\section{Decompositions}
%-------------------------------------------------------------------------------

We introduce two main decompositions for any instantaneous flow variable  $a_i$:
\begin{align}
\text{Reynolds decomposition: } & a_i = \aavg_i + \afluc_i \\
\text{Favre decomposition: } & a_i = \aavgf_i + \aflucf_i
\end{align}
For the above, we have
\begin{equation} 
\aavg_i = \int A_i f(\Avec) d\Avec \qquad \text{and} \qquad \aavgf_i = \frac{\overline{ \rho a_i }}{\rhoavg},
\end{equation}
where $A_i$ is the sample space variable corresponding to $a_i$.

%-------------------------------------------------------------------------------
\section{Favre-averaged Navier-Stokes equations}
%-------------------------------------------------------------------------------
Using the definition of Favre-averaging, we note the following expressions involving density 
\begin{align}
    \overline{\rho a} & = \rhoavg \widetilde{a} \label{eq:favre_rho_a} , \\
    \overline{\rho ab} & = \rhoavg \widetilde{ab} \label{eq:favre_rho_ab} , \\
    \overline{\rho abc} & = \rhoavg \widetilde{abc} \label{eq:favre_rho_abc} .
\end{align}
We also note that, for the decomposition $a = \aavgf + \aflucf$, we have
\begin{equation}
    \widetilde{\aflucf} = \widetilde{a - \aavgf} = 0.
\end{equation}
Thus, expressions that hold for Reynolds averaging also hold for the Favre averaging, that is
\begin{align}
    \widetilde{ab} &= \aavgf \bavgf + \widetilde{a''b''} \label{eq:favre_ab}, \\
    \widetilde{abc} & = \aavgf \bavgf \cavgf + \aavgf \widetilde{\bflucf \cflucf} + \bavgf \widetilde{\aflucf \cflucf} + \cavgf \widetilde{\aflucf \bflucf} + \widetilde{\aflucf \bflucf \cflucf}. \label{eq:favre_abc}
\end{align}

Taking the average of the conservation of mass equation leads to
\begin{equation}
\frac{\partial \rhoavg}{\partial t} + \frac{\partial \rhoavg \uavgf_j}{\partial x_j} = 0.
\end{equation}

Before taking the average of the conservation of momentum equation, we make use of \cref{eq:favre_rho_ab,eq:favre_ab} to show
\begin{equation}
\label{eq:mom_flux_expansion}
    \overline{\rho u_i u_j} = \rhoavg \widetilde{u_i u_j} = \rhoavg \left ( \uavgf_i \uavgf_j + \widetilde{\uflucf_i \uflucf_j} \right).
\end{equation}
The second term on the right-hand side is the Favre stress tensor $\rs_{ij} = \widetilde{\uflucf_i \uflucf_j}$. The average of the conservation of momentum equation can now be expressed as 
\begin{equation}
\frac{\partial \rhoavg \uavgf_i}{\partial t} + \frac{\partial \rhoavg \uavgf_i \uavgf_j}{\partial x_j} = -\frac{\partial \pavg}{\partial x_i} + \frac{\partial \tavg_{ij}}{\partial x_j} - \frac{\partial \rhoavg \rs_{ij} }{\partial x_j} + \rhoavg \favgf_i,
\end{equation}

Before taking the average of the conservation of energy equations, we use \cref{eq:favre_rho_ab,eq:favre_rho_abc,eq:favre_ab,eq:favre_abc} to obtain the following expansions
\begin{equation}
\label{eq:intener_flux_exansion}
    \overline{ \rho e u_j } = \rhoavg \widetilde{e u_j} = \rhoavg \left ( \eavgf \uavgf_j + \widetilde{\eflucf \uflucf_j} \right)
\end{equation}
\begin{equation}
\label{eq:tke_expansion}
    \overline{ \rho \frac{1}{2} u_i u_i } = \rhoavg \frac{1}{2} \widetilde{u_i u_i} = \rhoavg \left ( \frac{1}{2} \uavgf_i \uavgf_i +  \frac{1}{2} \widetilde{\uflucf_i \uflucf_i} \right)
\end{equation}
\begin{equation}
\label{eq:tke_flux_expansion}
    \overline{ \rho \frac{1}{2} u_i u_i u_j } = \rhoavg \frac{1}{2} \widetilde{u_i u_i u_j} = \rhoavg \left ( \frac{1}{2} \uavgf_i \uavgf_i \uavgf_j + \frac{1}{2} \widetilde{\uflucf_i \uflucf_i} \uavgf_j + \uavgf_i \widetilde{\uflucf_i \uflucf_j} +  \frac{1}{2} \widetilde{\uflucf_i \uflucf_i \uflucf_j} \right)
\end{equation}
We simplify the notation above by introducing
\begin{equation}
    \mathcal{K} = \frac{1}{2} \uavgf_i \uavgf_i \qquad k = \frac{1}{2} \widetilde{\uflucf_i \uflucf_i}
\end{equation}
Thus, we have for the energy density
\begin{equation}
    \overline{ \rho E } = \overline{ \rho e } + \overline{ \rho \frac{1}{2} u_i u_i } = \rhoavg \left ( \eavgf +  \mathcal{K} + k \right ),
\end{equation}
and for the energy flux
\begin{equation}
    \overline{ \rho E u_j } = \overline{ \rho e u_j } + \overline{ \rho \frac{1}{2} u_i u_i u_j } = \rhoavg \left(\eavgf + \mathcal{K} + k \right) \uavgf_j + \rhoavg \widetilde{\eflucf \uflucf_j} + \rhoavg \uavgf_i \rs_{ij} + \rhoavg \frac{1}{2} \widetilde{\uflucf_i \uflucf_i \uflucf_j} .
\end{equation}
The average of the conservation of energy \cref{eq:cons_energy} can now be expressed as
\begin{multline}
\label{eq:favre_energy_intermediate}
\frac{\partial \rhoavg \Eavgf}{\partial t} + \frac{\partial \rhoavg \Eavgf \uavgf_j }{\partial x_j} = -\frac{\partial \rhoavg \widetilde{ \eflucf u''_j } }{\partial x_j} - \frac{\partial \rhoavg \uavgf_i \rs_{ij} }{\partial x_j} - \frac{ \partial }{ \partial x_j } \left ( \rhoavg \frac{1}{2} \widetilde{ \uflucf_i \uflucf_i \uflucf_j } \right ) \\
- \frac{ \partial \overline{ u_j p } }{\partial x_j} + \frac{ \partial \overline{ u_i t_{ij} } }{\partial x_j} - \frac{\partial \qavg_j }{\partial x_j} + \rhoavg \favgf_i \uavgf_i + \rhoavg \widetilde{ \fflucf_i \uflucf_i }
\end{multline}
The term $\overline{u_j p}$ is treated as follows
\begin{align}
\overline{ u_j p } &= \uavgf_j \pavg  + \overline{ \uflucf_j p } \nonumber \\ 
&= \rhoavg \frac { \uavgf_j \pavg}{\rhoavg} + R \rhoavg \widetilde{ \uflucf_j T } \nonumber \\
&= \rhoavg \frac{ \uavgf_j \pavg }{\rhoavg} + C_p \rhoavg \widetilde{ \uflucf_j \Tflucf } - \rhoavg \widetilde{ \uflucf_j \eflucf } \label{eq:puj_expansion} ,
\end{align}
where we have used $p = \rho R T$, $R = C_p - C_v$ and $e = C_v T$. For $\overline{ u_i t_{ij} }$ we have
\begin{equation}
\label{eq:tuj_expansion}
\overline{ u_i t_{ij} } = \uavgf_i \tavg_{ij} + \overline{ \uflucf_i t_{ij} }. 
\end{equation}
These expansions are then plugged into \cref{eq:favre_energy_intermediate}. 

Before taking the average of the conservation equation for a scalar, we make use of \cref{eq:favre_rho_ab,eq:favre_ab} to show
\begin{equation}
\label{eq:scalar_flux_expansion}
    \overline{ \rho Y u_j } = \rhoavg \widetilde{Y u_j} = \rhoavg \left ( \Yavgf \uavgf_j + \widetilde{\Yflucf \uflucf_j} \right).
\end{equation}
The second term on the right-hand side above is the turbulent scalar flux. The average of the conservation equation for a scalar can now be expressed as
\begin{equation}
    \frac{\partial \rho \Yavgf}{\partial t} + \frac{\partial \rhoavg \Yavgf \uavgf_j}{\partial x_j} = -\frac{ \partial \Javg_j }{ \partial x_j } - \frac{ \partial \rhoavg \widetilde{\Yflucf \uflucf_j} }{ \partial x_j } + \Wavg.
\end{equation}

To summarize, the Favre-averaged governing equations are
\begin{equation}
\label{eq:favre_rho}
\frac{\partial \rhoavg}{\partial t} + \frac{\partial \rhoavg \uavgf_j}{\partial x_j} = 0,
\end{equation}
\begin{equation}
\label{eq:favre_vel}
\frac{\partial \rhoavg \uavgf_i}{\partial t} + \frac{\partial \rhoavg \uavgf_i \uavgf_j}{\partial x_j} = -\frac{\partial \pavg}{\partial x_i} + \frac{\partial \tavg_{ij}}{\partial x_j} - \frac{\partial \rhoavg \rs_{ij} }{\partial x_j} + \rhoavg \favgf_i,
\end{equation}
\begin{multline}
\label{eq:favre_energy}
\frac{\partial \rhoavg \Eavgf}{\partial t} + \frac{\partial}{\partial x_j} \left [ \rhoavg \left ( \Eavgf + \frac{ \pavg}{ \rhoavg} \right ) \uavgf_j \right ] = - \frac{\partial}{\partial x_j} \left [ \uavgf_i \left ( \rhoavg \rs_{ij} - \tavg_{ij} \right ) \right ] \\
- \frac{ \partial }{\partial x_j} \left ( \rhoavg \frac{1}{2} \widetilde{ \uflucf_i \uflucf_i \uflucf_j } - \overline{ \uflucf_i t_{ij} } \right ) - \frac{\partial \qavg_j}{\partial x_j} - \frac{\partial}{\partial x_j} \left ( C_p \rhoavg \widetilde{ \uflucf_j \Tflucf } \right ) + \rhoavg \favgf_i \uavgf_i + \rhoavg \widetilde{ \fflucf_i \uflucf_i }.
\end{multline}
\begin{equation}
    \label{eq:favre_scalar}
    \frac{\partial \rhoavg \Yavgf}{\partial t} + \frac{\partial \rhoavg \Yavgf \uavgf_j}{\partial x_j} = -\frac{\partial \Javg_j}{\partial x_j} - \frac{\partial \rhoavg \widetilde{\Yflucf \uflucf_j}}{\partial x_j} + \Wavg.    
\end{equation}
The averages of the diffusive scalar flux, the shear-stress tensor, and the heat-conduction flux are approximated as
\begin{equation}
    \label{eq:shear_stress_model}
    \tavg_{ij} \approx 2 \mu \Savgf^*_{ij},
\end{equation}
\begin{equation}
    \qavg_i \approx -\kappa \frac{ \partial \Tavgf }{ \partial x_i }.
\end{equation}
\begin{equation}
    \Javg_i \approx - \rhoavg D \frac{\partial \Yavgf}{\partial x_i},
\end{equation}
The transport coefficients are computed using
\begin{equation}
    \mu = \mu_0 \left ( \frac{\Tavgf}{T_0} \right )^n
\end{equation}
\begin{equation}
    \kappa = \frac{\mu C_p}{Pr}
\end{equation}
\begin{equation}
    D = \frac{\mu}{\rhoavg Sc}
\end{equation}
The averaged equations for a perfect gas are
\begin{equation}
    \pavg = \rhoavg R \Tavgf
\end{equation}
\begin{equation}
    \eavgf = C_v \Tavgf
\end{equation}
\begin{equation}    
    R = \frac{R_u}{M}
\end{equation}
Additional exact relations are below
\begin{equation} 
    \Eavgf = \eavgf + \mathcal{K} + k 
\end{equation}
\begin{equation}    
    \mathcal{K} = \frac{1}{2} \uavgf_i \uavgf_i
\end{equation}
\begin{equation}    
    k = \frac{1}{2} \widetilde{\uflucf_i \uflucf_i}
\end{equation}
\begin{equation}
    \Savgf^*_{ij} = \frac{1}{2} \left ( \frac{\partial \uavgf_i}{\partial x_j} + \frac{\partial \uavgf_j}{\partial x_i} \right ) - \frac{1}{3} \frac{\partial \uavgf_k}{\partial x_k} \delta_{ij}
\end{equation}
Finally, the following models can be used to provide closure
\begin{equation}
    \rs_{ij} = ...
\end{equation}
\begin{equation}
- \rhoavg \frac{1}{2} \widetilde{ \uflucf_i \uflucf_i \uflucf_j} +  \overline{ \uflucf_i t_{ij} } \approx \left(\mu + \frac{\mu_t}{\sigma_k}\right)\frac{\partial k}{\partial x_j},
\end{equation}
\begin{equation}
C_p \rhoavg \widetilde{ \uflucf_j \Tflucf } \approx -\kappa_t \frac{\partial \Tavgf}{\partial x_j} \qquad \text{where} \qquad \kappa_T = \frac{\mu_T C_p}{Pr_T}.
\end{equation}

%-------------------------------------------------------------------------------
\section{Additional Favre-averaged equations}
%-------------------------------------------------------------------------------

%--------------------------------------------
\subsection{Vorticity}
%--------------------------------------------

%--------------------------------------------
\subsection{Fluctuating velocity}
%--------------------------------------------

To derive an equation for the favre-fluctuating velocity, we will first focus on the left-hand side of \cref{eq:cons_momentum} only. Plugging in the decomposition $u_i = \uavgf_i + \uflucf_i$ in the left-hand side of \cref{eq:cons_momentum}, one obtains
\begin{equation}
\label{eq:deriv_uflucf}
    \frac{\partial \rho \uavgf_i}{\partial t}  + \frac{\partial \rho \uflucf_i}{\partial t} + \frac{ \partial \rho \uavgf_i u_j}{\partial x_j} + \frac{\partial \rho \uflucf_i u_j}{\partial x_j}.
\end{equation}
We will now look at the first and third terms above. Thus,
\begin{align}
    \frac{\partial \rho \uavgf_i}{\partial t} + \frac{ \partial \rho \uavgf_i u_j}{\partial x_j} &= \frac{\partial \rho}{\partial t} \uavgf_i + \rho \frac{\partial \uavgf_i}{\partial t} + \frac{\partial \rho u_j}{\partial x_j} \uavgf_i + \rho u_j \frac{\partial \uavgf_i}{\partial x_j} \nonumber \\
    &= \rho \frac{\partial \uavgf_i}{\partial t} + \rho \uavgf_j \frac{\partial \uavgf_i}{\partial x_j} + \rho \uflucf_j \frac{\partial \uavgf_i}{\partial x_j}.
\end{align}
The fourth term of \cref{eq:deriv_uflucf} is decomposed as
\begin{align}
    \frac{\partial \rho \uflucf_i u_j}{\partial x_j} &= \frac{\partial \rho \uflucf_i \uavgf_j}{\partial x_j} + \frac{\partial \rho \uflucf_i \uflucf_j}{\partial x_j} \nonumber \\
    &= \frac{\partial \rho \uflucf_i}{\partial x_j} \uavgf_j + \rho \uflucf_i \frac{\partial \uavgf_j}{\partial x_j} + \frac{\partial \rho \uflucf_i \uflucf_j}{\partial x_j}.
\end{align}
Combining all of the results above, one obtains
\begin{equation}
\label{eq:vel_fluc_favre}
    \frac{\partial \rho \uflucf_i}{\partial t} + \rho \frac{\partial \uavgf_i}{\partial t} + \rho \uavgf_j \frac{\partial \uavgf_i}{\partial x_j} + \rho \uflucf_j \frac{\partial \uavgf_i}{\partial x_j} + \frac{\partial \rho \uflucf_i}{\partial x_j} \uavgf_j + \rho \uflucf_i \frac{\partial \uavgf_j}{\partial x_j} + \frac{\partial \rho \uflucf_i \uflucf_j}{\partial x_j} = -\frac{\partial p}{\partial x_i} + \frac{\partial t_{ij}}{\partial x_j} + \rho f_i.
\end{equation}

%--------------------------------------------
\subsection{Kinetic energy of the mean flow}
%--------------------------------------------
The equation for the kinetic energy of the mean flow is obtained by multiplying \cref{eq:favre_vel}---the Favre-averaged equation for velocity---by $\uavgf_i$, which leads to
\begin{equation}
\label{eq:ke_favreflow}
\frac{ \partial \rhoavg \mathcal{K} }{ \partial t} + \frac{ \partial \rhoavg \mathcal{K} \uavgf_j }{ \partial x_j} = \rhoavg \rs_{ij} \frac{ \partial \uavgf_i }{ \partial x_j } + \pavg \frac{ \partial \uavgf_i }{ \partial x_i } - \tavg_{ij]} \frac{ \partial \uavgf_i }{ \partial x_j } - \frac{\partial}{\partial x_j} \left ( \uavgf_i \rhoavg \rs_{ij} + \uavgf_j \pavg - \uavgf_i \tavg_{ij} \right ) + \rhoavg \favgf_i \uavgf_i
\end{equation}

%--------------------------------------------
\subsection{Turbulent kinetic energy}
%--------------------------------------------
To obtain the governing equation for the turbulent kinetic energy one subtracts \cref{eq:ke_favreflow} from the Reynolds average of \cref{eq:cons_kinetic_energy}, which is the equation for conservation of kinetic energy. We first regroup the shear stress terms of \cref{eq:cons_kinetic_energy}  
\begin{equation}
\label{eq:stress_tke_inter_T}
     \frac{\partial u_i t_{ij} }{\partial x_j} - t_{ij} \frac{\partial u_i}{\partial x_j} =  u_i \frac{\partial t_{ij} }{\partial x_j}
\end{equation}
and \cref{eq:ke_favreflow}
\begin{equation}
     -\tavg_{ij} \frac{ \partial \uavgf_i}{\partial x_j} + \frac{ \partial \uavgf_i \tavg_{ij} }{\partial x_j} = \uavgf_i \frac{ \partial \tavg_{ij} }{\partial x_j}.
\end{equation}
For the average of the right-hand side of \cref{eq:stress_tke_inter_T}, we will use the decompositions $u_i = \uavg_i + \ufluc_i$ and $t_{ij} = \tavg_{ij} + \tfluc_{ij}$, instead of the Favre decomposition and \cref{eq:tuj_expansion}. This thus allows us to write 
\begin{align}
\overline{ u_i \frac{\partial t_{ij} }{\partial x_j} } - \uavgf_i \frac{ \partial \tavg_{ij} }{\partial x_j} & = \uavg_i \frac{\partial \tavg_{ij} }{\partial x_j} + \overline{ \ufluc_i \frac{\partial \tfluc_{ij}}{\partial x_j} } - \uavgf_i \frac{\partial \tavg_{ij} }{\partial x_j} \nonumber \\
& = \overline{ \uflucf_i } \frac{\partial \tavg_{ij} }{\partial x_j} + \overline{ \ufluc_i \frac{\partial \tfluc_{ij} }{\partial x_j} } \nonumber \\
& = \overline{ \uflucf_i } \frac{\partial \tavg_{ij} }{\partial x_j} + \frac{\partial \overline{ \ufluc_i \tfluc_{ij} }}{\partial x_j} - \overline{ \tfluc_{ij} \frac{\partial \ufluc_i}{\partial x_j} }.
\end{align} 

The same procedure is followed for the pressure terms. We first regroup the pressure terms of \cref{eq:cons_kinetic_energy}
\begin{equation}
\label{eq:stress_tke_inter_p}
      -\frac{\partial u_i p }{\partial x_i} + p \frac{\partial u_i}{\partial x_i} =  -u_i \frac{\partial p }{\partial x_i}   
\end{equation}
and \cref{eq:ke_ransflow}
\begin{equation}
     \pavg \frac{ \partial \uavgf_i}{\partial x_i} - \frac{ \partial \uavgf_i \pavg }{\partial x_i} = -\uavgf_i \frac{ \partial \pavg}{\partial x_i}.    
\end{equation}
Again, the Reynolds decomposition of velocity is used for the average of the right-hand side of \cref{eq:stress_tke_inter_p}. This allows us to write
\begin{align}
-\overline{ u_i \frac{\partial p }{\partial x_i} } + \uavgf_i \frac{ \partial \pavg }{\partial x_i} & = -\uavg_i \frac{\partial \pavg}{\partial x_i} - \overline{ \ufluc_i \frac{\partial \pfluc}{\partial x_i} } + \uavgf_i \frac{\partial \pavg}{\partial x_i} \nonumber \\
& = - \overline{ \uflucf_i } \frac{\partial \pavg }{\partial x_i} - \overline{ \ufluc_i \frac{\partial \pfluc}{\partial x_i} } \nonumber \\
& = -\overline{ \uflucf_i } \frac{\partial \pavg }{\partial x_i} - \frac{\partial \overline{ \ufluc_i \pfluc } }{\partial x_i} + \overline{ \pfluc \frac{\partial \ufluc_i}{\partial x_i} }.
\end{align} 

The overall equation for the turbulent kinetic energy is shown below
\begin{multline}
\label{eq:ke_favre_conservation}
\frac{ \partial \rhoavg k }{ \partial t}  + \frac{ \partial \rhoavg k \uavgf_j }{ \partial x_j } = - \overline{ \uflucf_i } \frac{ \partial \pavg }{ \partial x_i } + \overline{ \uflucf_i } \frac{ \partial \tavg_{ij} }{ \partial x_j } \\
- \rhoavg \rs_{ij} \frac{ \partial \uavgf_i }{ \partial x_j}  + \overline{ \pfluc \frac{ \partial \ufluc_i } { \partial x_i } } - \overline{ \tfluc_{ij} \frac{ \partial \ufluc_i }{ \partial x_j } } - \frac{\partial}{ \partial x_j} \left ( \frac{1}{2} \rhoavg \widetilde{ \uflucf_j \uflucf_i \uflucf_i } + \overline{ \ufluc_j \pfluc }  - \overline{ \ufluc_i \tfluc_{ij} } \right ) + \rhoavg \widetilde{ \fflucf_i \uflucf_i }.
\end{multline}

We define the dissipation $\epsilon$ as
\begin{equation}
\rhoavg \epsilon = \overline{ \tfluc_{ij} \frac{ \partial \ufluc_i }{\partial x_j } } ,
\end{equation}
Since $\tfluc_{ij} = t_{ij} - \tavg_{ij}$, the above becomes
\begin{equation}
    \rhoavg \epsilon = \overline{ t_{ij} \frac{\partial \ufluc_i}{\partial x_j} } = \overline{ 2 \mu \frac{\partial \ufluc_i}{\partial x_j} } \Savg^*_{ij} + \overline{ 2 \mu \sfluc^*_{ij} \frac{\partial \ufluc_i}{\partial x_j} }.
\end{equation}
Neglecting the molecular viscosity fluctuations, as stated in Sagaut \& Cambon (Homogeneous compressible dynamics) gives
\begin{equation}
    \rhoavg \epsilon = 2 \overline{ \mu } \; \overline{ \sfluc^*_{ij} \frac{\partial \ufluc_i}{\partial x_j} }.
\end{equation}
Expanding $\sfluc^*_{ij}$,
\begin{equation}
    \rhoavg \epsilon = 2 \overline{ \mu } \; \overline{ \sfluc_{ij} \sfluc_{ij} - \frac{1}{3} \left ( \frac{ \partial \ufluc_k }{ \partial x_k } \right )^2 }.
\end{equation}
The above agrees with Sarkar et al, JFM, 1991. \cite{Sarkar1991}. Using the identities 
\begin{equation}
\sfluc_{ij} \sfluc_{ij} = \frac{1}{2} \wfluc_i \wfluc_i + \frac{ \partial \ufluc_i }{ \partial x_j } \frac{ \partial \ufluc_j }{ \partial x_i}, 
\end{equation}
and
\begin{equation}
\frac{ \partial \ufluc_i }{ \partial x_j } \frac{ \partial \ufluc_j }{ \partial x_i } = \frac{ \partial^2 \ufluc_i \ufluc_i }{ \partial x_i \partial x_j}  -2 \frac{ \partial }{ \partial x_j } \left ( \frac{ \partial \ufluc_i }{ \partial x_i } \ufluc_j \right ) + \left ( \frac{ \partial \ufluc_k }{ \partial x_k } \right )^2 ,
\end{equation}
and neglecting the first two terms in the equation above due to an order of magnitude estimate for high turbulence Reynolds number as suggested by Sarkar et al, we express the dissipation as
\begin{equation}
\rhoavg \epsilon = \overline{ \mu } \overline{ \wfluc_i \wfluc_i } + \frac{4}{3} \overline{ \mu } \overline{ \left ( \frac{ \partial \ufluc_k }{ \partial x_k } \right )^2 }.
\end{equation}
The first component above is referred to as the solenoidal dissipation, while the second component is the compressible dissipation.

%--------------------------------------------
\subsection{Internal Energy}
%--------------------------------------------
We take the Reynolds average of \cref{eq:cons_internal_energy}, the conservation form of the internal energy equation. For this, we will use
\begin{equation}
    \overline{ t_{ij} \frac{\partial u_i}{\partial x_j} } = \tavg_{ij} \frac{\partial \uavg_i}{\partial x_j} + \overline{ \tfluc_{ij} \frac{\partial \ufluc_i}{\partial x_j} },
\end{equation}
and
\begin{equation}
    \overline{ -p \frac{\partial u_i}{\partial x_i} } = -\pavg \frac{\partial \uavg_i}{\partial x_i} - \overline{ \pfluc \frac{\partial \ufluc_i}{\partial x_i} }.
\end{equation}
For the first term on the right-hand side of each of the two equations above, we expand $\uavg_i$ as $\uavg_i = \uavgf_i + \overline{ \uflucf_i }$. Thus, we now have 
\begin{align}
    \overline{ t_{ij} \frac{\partial u_i}{\partial x_j} } &=  \tavg_{ij}  \frac{\partial \uavgf_i}{\partial x_j} + \tavg_{ij}  \frac{\partial \overline{ \uflucf_i } }{\partial x_j} + \overline{ \tfluc_{ij} \frac{\partial \ufluc_i}{\partial x_j} } \nonumber \\
    &= \tavg_{ij} \frac{\partial \uavgf_i}{\partial x_j} - \overline{ \uflucf_i } \frac{\partial \tavg_{ij} }{\partial x_j} + \overline{ \tfluc_{ij} \frac{\partial \ufluc_i}{\partial x_j} } + \frac{\partial \overline{ \uflucf_i } \tavg_{ij} }{\partial x_j},
\end{align}
and
\begin{align}
    \overline{ -p \frac{\partial u_i}{\partial x_i} } &= -\pavg \frac{\partial \uavgf_i}{\partial x_i} - \pavg \frac{\partial \overline{ \uflucf_i } }{\partial x_i} - \overline{ \pfluc \frac{\partial \ufluc_i}{\partial x_i} } \nonumber \\
    &= -\pavg \frac{\partial \uavgf_i}{\partial x_i} + \overline{ \uflucf_i } \frac{\partial \pavg}{\partial x_i} - \overline{ \pfluc \frac{\partial \ufluc_i}{\partial x_i} } - \frac{\partial \overline{ \uflucf_i } \pavg}{\partial x_i}.
\end{align}
Thus, using the above, the average of the internal energy equation becomes
\begin{multline}
    \frac{\partial \rhoavg \eavgf}{\partial t} + \frac{\partial \rhoavg \eavgf \uavgf_j}{\partial x_j} = -\pavg \frac{\partial \uavgf_i}{\partial x_i} + \tavg_{ij} \frac{\partial \uavgf_i}{\partial x_j} + \overline{ \uflucf_i } \frac{\partial \pavg}{\partial x_i} - \overline{ \uflucf_i } \frac{\partial \tavg_{ij} }{\partial x_j} \\
    - \overline{ \pfluc \frac{\partial \ufluc_i}{\partial x_i} } + \overline{ \tfluc_{ij} \frac{\partial \ufluc_i}{\partial x_j} } - \frac{\partial}{\partial x_j} \left ( \rhoavg \widetilde{ \eflucf \uflucf_j }  +  \overline{ \uflucf_j } \pavg - \overline{ \uflucf_i } \tavg_{ij} + \qavg_j \right)
\end{multline}
Note that we can write
\begin{align}
    \rhoavg \widetilde{ \eflucf \uflucf_j } + \overline{ \uflucf_j } \pavg &= \rhoavg \widetilde{ \eflucf \uflucf_j } + \overline{ \pfluc \uflucf_j } + \overline{ \uflucf_j } \pavg - \overline{ \pfluc \uflucf_j } \nonumber \\ 
    &= \rhoavg \widetilde{ \eflucf \uflucf_j } + \overline{ p \uflucf_j } - \overline{ \pfluc \uflucf_j } \nonumber \\
    &=  \rhoavg \widetilde{ e \uflucf_j} + \overline{ p \uflucf_j } - \overline{ \pfluc \uflucf_j } \nonumber \\
    &=\overline{ \rho e\uflucf_j + p\uflucf_j} - \overline{ \pfluc \uflucf_j } \nonumber \\
    &= \rhoavg \widetilde{ h \uflucf_j} - \overline{ \pfluc \uflucf_j } \nonumber \\
    &= \rhoavg \widetilde{ h'' \uflucf_j } - \overline{ \pfluc \uflucf_j } ,
\end{align}
so as to be in agreement with the equations in Lele 1994.

%--------------------------------------------
\subsection{Summary of conservation equations for $\mathcal{K}$, $k$, $\Uavgf$}
%--------------------------------------------
\begin{equation}
\frac{ \partial \rhoavg \mathcal{K} }{ \partial t} + \frac{ \partial \rhoavg \mathcal{K} \uavgf_j }{ \partial x_j} = \rhoavg \rs_{ij} \frac{ \partial \uavgf_i }{ \partial x_j } + \pavg \frac{ \partial \uavgf_i }{ \partial x_i } - \langle T_{ij} \rangle \frac{ \partial \uavgf_i }{ \partial x_j } - \frac{\partial}{\partial x_j} \left ( \uavgf_i \rhoavg \rs_{ij} + \uavgf_j \pavg - \uavgf_i \langle T_{ij} \rangle \right ) + \rhoavg \tilde{F}_i \uavgf_i
\end{equation}
\begin{multline}
\frac{ \partial \rhoavg k }{ \partial t}  + \frac{ \partial \rhoavg k \uavgf_j }{ \partial x_j } = - \langle \uflucf_i \rangle \frac{ \partial \pavg }{ \partial x_i } + \langle \uflucf_i \rangle \frac{ \partial \langle T_{ij} \rangle }{ \partial x_j } \\
- \rhoavg \rs_{ij} \frac{ \partial \uavgf_i }{ \partial x_j}  + \left < p \frac{ \partial \uifluc } { \partial x_i } \right > - \left <  t_{ij} \frac{ \partial u_i }{ \partial x_j } \right > - \frac{\partial}{ \partial x_j} \left ( \frac{1}{2} \langle \rho \uflucf_j \uflucf_i \uflucf_i \rangle + \langle \ujfluc p \rangle  - \langle \uifluc t_{ij} \rangle \right ) + \langle \rho f_i'' \uflucf_i \rangle
\end{multline}
\begin{multline}
    \frac{\partial \rhoavg \Uavgf}{\partial t} + \frac{\partial \rhoavg \Uavgf \uavgf_j}{\partial x_j} = -\pavg \frac{\partial \uavgf_i}{\partial x_i} + \langle T_{ij} \rangle \frac{\partial \uavgf_i}{\partial x_j} +\langle \uflucf_i \rangle \frac{\partial \pavg}{\partial x_i} - \langle \uflucf_i \rangle \frac{\partial \langle T_{ij} \rangle}{\partial x_j} \\
    - \left < \pfluc \frac{\partial \uifluc}{\partial x_i} \right > + \left < t_{ij} \frac{\partial \uifluc}{\partial x_j} \right > - \frac{\partial}{\partial x_j} \left ( \langle \rho u'' \uflucf_j \rangle +  \langle \uflucf_j \rangle \pavg - \langle \uflucf_i \rangle \langle T_{ij} \rangle + \lbra q_j \rbra \right)
\end{multline}
Note that if you add up all three equations, one obtains a conservation equation for $\Eavgf$. After this addition, the only terms on the right-hand side that remain are the transport and forcing terms.
%--------------------------------------------
\subsection{Reynolds stresses}
%--------------------------------------------

%--------------------------------------------
\subsection{Two-point velocity correlations}
%--------------------------------------------

%--------------------------------------------
\subsection{Squared averaged scalar}
%--------------------------------------------
The equation for the squared averaged scalar is obtained by multiplying \cref{eq:favre_scalar}---the Favre-averaged equation for the scalar---by $\Yavgf$, which leads to
\begin{equation}
    \label{eq:squared_averaged_scalar_favre}
    \frac{\partial \rhoavg \Yavgf^2}{\partial t} + \frac{\partial \rhoavg \Yavgf^2 \uavgf_j}{\partial x_j} = 2 \langle J_j \rangle \frac{\partial \Yavgf}{\partial x_j} + 2 \rhoavg \widetilde{\Yflucf \uflucf_j} \frac{\partial \Yavgf}{\partial x_j} - 2\frac{\partial}{\partial x_j} \left ( \Yavgf \langle J_j \rangle + \Yavgf \rhoavg \widetilde{\Yflucf \uflucf_j} \right ) + 2 \Yavgf \langle W \rangle.
\end{equation}

%--------------------------------------------
\subsection{Scalar variance}
%--------------------------------------------
To obtain the governing equation for the scalar variance, one subtracts \cref{eq:squared_averaged_scalar_favre} from the Reynolds average of \cref{eq:cons_squared_scalar}. We first regroup the diffusive scalar flux terms of \cref{eq:cons_squared_scalar}
\begin{equation}
    \label{eq:diffusive_scalar_var_inter}
    -\frac{\partial Y J_j}{\partial x_j} + J_j \frac{\partial Y}{\partial x_j} = -Y \frac{\partial J_j}{\partial x_j},
\end{equation}
and \cref{eq:squared_averaged_scalar_favre}
\begin{equation}
    \langle J_j \rangle \frac{\partial \Yavgf}{\partial x_j} - \frac{\partial \Yavgf \langle J_j \rangle}{\partial x_j} = -\Yavgf \frac{\partial \langle J_j \rangle}{\partial x_j}.
\end{equation}
For the average of the right-hand side of \cref{eq:diffusive_scalar_var_inter}, we will use the decomposition $Y = \yavg + \yfluc$ and $J_i = \langle J_i \rangle + j_i$, instead of the traditional Favre decompositions. This thus allows us to write
\begin{align}
    -\left < Y \frac{\partial J_j}{\partial x_j} \right > + \Yavgf \frac{\partial \langle J_j \rangle}{\partial x_j} &= -\yavg \frac{\partial \langle J_j \rangle}{\partial x_j} - \left < \yfluc \frac{\partial j_j}{\partial x_j} \right > + \Yavgf \frac{\partial \langle J_j \rangle}{\partial x_j} \nonumber \\
    &= -\langle \Yflucf \rangle \frac{\partial \langle J_j \rangle}{\partial x_j} - \left < \yfluc \frac{\partial j_j}{\partial x_j} \right > \nonumber \\
    &= -\langle \Yflucf \rangle \frac{\partial \langle J_j \rangle}{\partial x_j} - \frac{\partial \langle \yfluc j_j \rangle}{\partial x_j} + \left < j_j \frac{\partial \yfluc}{\partial x_j} \right >.
\end{align}
Using the above, the overall equation for the scalar variance is written as
\begin{multline}
    \label{eq:scalar_variance_favre}
    \frac{\partial \rhoavg \widetilde{\Yflucf \Yflucf}}{\partial t} + \frac{\partial \rhoavg \widetilde{\Yflucf \Yflucf} \uavgf_j }{\partial x_j} = -2 \langle \Yflucf \rangle \frac{\partial \langle J_j \rangle }{\partial x_j} \\
    -2 \rhoavg \widetilde{\Yflucf \uflucf_j} \frac{\partial \Yavgf}{\partial x_j} + 2 \left < j_j \frac{\partial y}{\partial x_j} \right > - 2 \frac{\partial}{\partial x_j} \left ( \frac{1}{2} \rhoavg \widetilde{\Yflucf \Yflucf \uflucf_j} +  \langle \yfluc j_j \rangle \right) + 2 \langle \Yflucf \rangle \langle W \rangle + 2 \langle \Yflucf w \rangle.
\end{multline}

As was done for the TKE, one can define a scalar dissipation $\epsilon_Y$ as 
\begin{equation}
    \rhoavg \epsilon_Y = -2\lbra j_j \frac{\partial y}{\partial x_j} \rbra.
\end{equation}
Since $j_j = J_j - \lbra J_j \rbra$, the above becomes
\begin{equation}
    \rhoavg \epsilon_Y = -2\lbra J_j \frac{\partial y}{\partial x_j} \rbra = 2 \lbra \rho D \frac{\partial y}{\partial x_j} \rbra \frac{\partial \lbra Y \rbra}{\partial x_j} + 2 \lbra \rho D \frac{\partial y}{\partial x_j} \frac{\partial y}{\partial x_j} \rbra.
\end{equation}


%-------------------------------------------------------------------------------
\section{Reynolds-averaged Navier-Stokes equations}
%-------------------------------------------------------------------------------
Assuming a constant density, the Favre average is equivalent to the Reynolds average. Thus, for incompressible flow the continuity \cref{eq:favre_rho} becomes
\begin{equation}
\label{eq:rans_rho}
\frac{\partial \uiavg}{\partial x_i} = 0.
\end{equation}
The conservation of momentum \cref{eq:favre_vel}, along with the definition of the shear stress tensor in \cref{eq:shear_stress}, leads to 
\begin{equation}
\label{eq:rans_vel}
\frac{\partial \uiavg}{\partial t} + \ujavg \frac{\partial \uiavg}{\partial x_j} = -\frac{1}{\rho} \frac{\partial \pavg}{\partial x_i} + \nu \frac{ \partial^2 \uavgf_i }{ \partial x_j \partial x_j} - \frac{\partial \rs_{ij}}{\partial x_j} + \langle F_i \rangle,
\end{equation}
where $\rs_{ij} = \langle \uifluc \ujfluc \rangle$.
Taking the divergence of the momentum equation, we obtain the equation for the averaged pressure
\begin{equation}
\label{eq:rans_pressure}
\frac{1}{\rho} \frac{\partial^2 \pavg}{\partial x_j \partial x_j} = - \frac{\partial \uiavg}{\partial x_j} \frac{\partial \ujavg}{\partial x_i} - \frac{\partial \rs_{ij}}{\partial x_i \partial x_j} + \frac{ \partial \langle F_i \rangle}{ \partial x_i}.
\end{equation}
Taking the Reynolds average of the scalar transport \cref{eq:cons_scalar} leads to
\begin{equation}
\label{eq:rans_scalar}
    \frac{\partial \rho \phiavg}{\partial t} + \frac{\partial \rho \ujavg \phiavg}{\partial x_j} = -\frac{\partial \langle J_j \rangle}{\partial x_j} -\frac{\partial \rho \langle \ujfluc \phifluc \rangle }{\partial x_j}
\end{equation}

%-------------------------------------------------------------------------------
\section{Additional Reynolds-averaged equations}
%-------------------------------------------------------------------------------

%--------------------------------------------
\subsection{Vorticity}
%--------------------------------------------
The equation for the mean vorticity $\wiavg$, also denoted by $\Omega_i$ is as follows
\begin{equation}
\frac{\partial \wiavg}{\partial t} + \ujavg \frac{\partial \wiavg}{\partial x_j} = \wjavg \frac{\partial \uiavg}{\partial x_j} + \nu \frac{\partial^2 \wiavg}{\partial x_j \partial x_j} + \frac{\partial}{\partial x_j} [ \langle \uifluc w_j \rangle - \langle \ujfluc w_i \rangle ].
\end{equation}

%--------------------------------------------
\subsection{Fluctuating velocity and pressure}
%--------------------------------------------
\Cref{eq:vel_fluc_favre} for incompressible flows becomes
\begin{equation}
    \frac{\partial \uifluc}{\partial t} + \frac{\partial \uiavg}{\partial t} + \ujavg \frac{\partial \uiavg}{\partial x_j} + \ujfluc \frac{\partial \uiavg}{\partial x_j} + \frac{\partial \uifluc}{\partial x_j} \ujavg + \frac{\partial \uifluc \ujfluc}{\partial x_j} = -\frac{1}{\rho}\frac{\partial P}{\partial x_i} + \nu \frac{\partial \uiavg}{\partial x_j \partial x_j} + F_i.
\end{equation}
We now subtract \cref{eq:rans_vel} from the above, to obtain
\begin{equation}
\label{eq:vel_fluc}
    \frac{\partial \uifluc}{\partial t} + \ujavg \frac{\partial \uifluc}{\partial x_j} = -\ujfluc \frac{\partial \uiavg}{\partial x_j} - \frac{\partial \uifluc \ujfluc}{\partial x_j} - \frac{1}{\rho}\frac{\partial p}{\partial x_i} + \nu \frac{\partial \uifluc}{\partial x_j \partial x_j} + \frac{\partial \tau_{ij}}{\partial x_j} + f_i.
\end{equation}
Assuming incompressible flow, and taking the divergence of \cref{eq:cons_momentum}, one obtains the equation for pressure,
\begin{equation}
\frac{1}{\rho} \frac{\partial^2 P}{\partial x_j \partial x_j} = -\frac{\partial U_i}{\partial x_j}\frac{\partial U_j}{\partial x_i} + \frac{\partial F}{\partial x_i}.
\end{equation}
Subtracting \cref{eq:rans_pressure} from the above leads to
\begin{equation}
\label{press_fluc}
\frac{1}{\rho} \frac{\partial^2 \pfluc}{\partial x_j \partial x_j} = -2 \frac{\partial \uiavg}{\partial x_j} \frac{\partial \ujfluc}{\partial x_i} - \frac{\partial^2 \uifluc \ujfluc}{\partial x_i \partial x_j} + \frac{\partial^2 \rs_{ij}}{\partial x_i \partial x_j} + \frac{\partial f_i}{\partial x_i}.
\end{equation}

%--------------------------------------------
\subsection{Kinetic energy of the mean flow}
%--------------------------------------------
Assuming incompressible flow, \cref{eq:ke_favreflow} becomes
\begin{equation}
\label{eq:ke_ransflow}
\frac{\partial \bar{K} }{\partial t} + \ujavg \frac{ \partial \bar{K} }{\partial x_j} = \rs_{ij} \frac{\partial \uiavg}{\partial x_j} - \frac{ \langle T_{ij} \rangle }{ \rho } \frac{ \partial \uiavg }{\partial x_j} -  \frac{\partial}{\partial x_j} \left ( \uiavg \rs_{ij} + \frac{ \ujavg \pavg }{\rho} - \frac{ \uiavg \langle T_{ij} \rangle }{\rho} \right ) + \langle F_i \rangle \uiavg.
\end{equation}

%--------------------------------------------
\subsection{Turbulent kinetic energy}
%--------------------------------------------
Assuming incompressible flow, \cref{eq:ke_favre_conservation} becomes  
\begin{equation}
\label{eq:tke_ransflow}
\frac{\partial k}{\partial t} + \ujavg \frac{\partial k}{\partial x_j} = -\rs_{ij} \frac{ \partial \uiavg }{ \partial x_j} - \left < \frac{ t_{ij} }{\rho} \frac{ \partial u_i }{ \partial x_j} \right > - \frac{\partial}{\partial x_j} \left ( \frac{1}{2} \langle \ujfluc \uifluc \uifluc \rangle + \frac{ \langle \ujfluc p \rangle }{ \rho } - \frac{ \langle \uifluc t_{ij} \rangle }{ \rho } \right ) + \langle f_i \uifluc \rangle .
\end{equation}
The production term can be rewritten as $-\rs_{ij} S_{ij}$, that is, production of TKE is influenced by $S_{ij}$ only. The second term is referred to as the dissipation $\epsilon$, which can be rewritten as
\begin{equation}
\epsilon = \left < \frac{ t_{ij} }{ \rho } s_{ij} \right > = 2 \nu \langle s_{ij} s_{ij} \rangle.
\end{equation}

%--------------------------------------------
\subsection{Reynolds-stresses}
%--------------------------------------------

The transport equation for the Reynolds stresses is obtained by multiplying equation (\ref{eq:vel_fluc}) by $\ujfluc$, adding to it the same product but with indices $i$ and $j$ switched, and then performing an average of the resulting equation. Thus, one obtains
\begin{equation}
\label{rs_transport}
\frac{\partial \rs_{ij}}{\partial t} + \ukavg \frac{\partial \rs_{ij}}{\partial x_k} = \mathcal{P}_{ij} + \Pi_{ij} - \epsilon_{ij} - \frac{\partial}{\partial x_k} \left ( \mathcal{T}^{(u)}_{kij} + \mathcal{T}^{(\nu)}_{kij} \right ) + \langle \ujfluc f_i \rangle + \langle \uifluc f_j \rangle.
\end{equation}
The production is given by
\begin{equation}
\mathcal{P}_{ij} = -\rs_{ik} \frac{\partial \ujavg}{\partial x_k} - \rs_{jk} \frac{\partial \uiavg}{\partial x_k},
\end{equation}
the velocity-pressure-gradient tensor by
\begin{equation}
\Pi_{ij} = -\frac{1}{\rho} \left \langle \uifluc \frac{\partial \pfluc}{\partial x_j} + \ujfluc \frac{\partial \pfluc}{\partial x_i} \right \rangle,
\end{equation}
the dissipation tensor by
\begin{equation}
\epsilon_{ij} = 2 \nu \left \langle \frac{\partial \uifluc}{\partial x_k}\frac{\partial \ujfluc}{\partial x_k} \right \rangle,
\end{equation}
the turbulent transport by
\begin{equation}
\mathcal{T}^{(u)}_{kij} = \langle \uifluc \ujfluc \ukfluc \rangle,
\end{equation}
and the viscous diffusion by
\begin{equation}
\mathcal{T}^{(\nu)}_{kij} = -\nu \frac{\partial \rs_{ij}}{\partial x_k}.
\end{equation}

The velocity-pressure-gradient tensor has two common decompositions. One of these is the following
\begin{equation}
\Pi_{ij} = \redi_{ij} - \frac{\partial}{\partial x_k} \mathcal{T}^{(p)}_{kij},
\end{equation}
where $\redi_{ij}$ is the pressure-rate-of-strain tensor and $\mathcal{T}^{(p)}_{kij}$ is the pressure transport tensor. These are defined as follows
\begin{equation}
\redi_{ij} = \left \langle \frac{\pfluc}{\rho} \left ( \frac{\partial \uifluc}{\partial x_j} + \frac{\partial \ujfluc}{\partial x_i} \right ) \right \rangle \qquad \text{and} \qquad \mathcal{T}^{(p)}_{kij} = \frac{1}{\rho} \langle \uifluc \pfluc \rangle \delta_{jk} + \frac{1}{\rho} \langle \ujfluc \pfluc \rangle \delta_{ik}.
\end{equation}
An alternate decomposition is as follows
\begin{equation}
\Pi_{ij} = \hat{\redi}_{ij} - \frac{\partial}{\partial x_k} \mathcal{T}^{(p')}_{kij},
\end{equation}
where $\hat{\redi}_{ij}$ is the anisotropic velocity-pressure-gradient tensor and
\begin{equation}
- \frac{\partial}{\partial x_k} \mathcal{T}^{(p')}_{kij} = \frac{1}{3} \Pi_{kk} \delta_{ij} \qquad
\mathcal{T}^{(p')}_{kij} =  \frac{2}{3} \delta_{ij} \mathcal{T}^{(p)}_k \qquad
\mathcal{T}^{(p)}_{k} = \frac{1}{\rho} \langle \ukfluc \pfluc \rangle.
\end{equation}
We label $\mathcal{T}^{(p')}_{kij}$ as the isotropic pressure transport.

Assuming no body forces, we can decompose the fluctuating pressure $p$ into three contributions, namely
\begin{equation}
p = p^{(r)} + p^{(s)} + p^{(h)},
\end{equation}
where $p^{(r)}$ is the rapid pressure, $p^{(s)}$ is the slow pressure, and $p^{(h)}$ is the harmonic pressure (also referred to as Stokes pressure). Thus, $p^{(r)}$ satisfies
\begin{equation}
\frac{1}{\rho} \frac{\partial^2 \pfluc^{(r)}}{\partial x_j \partial x_j} = -2 \frac{\partial \uiavg}{\partial x_j} \frac{\partial \ujfluc}{\partial x_i},
\end{equation}
$p^{(s)}$ satisfies
\begin{equation}
\frac{1}{\rho} \frac{\partial^2 \pfluc^{(s)}}{\partial x_j \partial x_j} = - \frac{\partial^2 \uifluc \ujfluc}{\partial x_i \partial x_j} + \frac{\partial^2 \rs_{ij}}{\partial x_i \partial x_j},
\end{equation}
and $p^{(h)}$ satisfies
\begin{equation}
\frac{1}{\rho} \frac{\partial^2 \pfluc^{(h)}}{\partial x_j \partial x_j} = 0.
\end{equation}
The rapid pressure is the only non-zero pressure in homogeneous RDT, the slow pressure is the only non-zero contribution in decaying homogeneous turbulence. The boundary conditions are specified on $p^{(h)}$ so that the total pressure satisfies the required boundary conditions.

Using the pressure decompositions defined above we can correspondingly decompose the redistribution tensors. For example,
\begin{equation}
\redi_{ij} = \redi_{ij}^{(r)} + \redi_{ij}^{(s)} + \redi_{ij}^{(h)},
\end{equation}
where, instead of using $p$, $p^{(r)}$ is used in the definition of $\redi^{(r)}_{ij}$, $p^{(s)}$ in the definition of $\redi^{(s)}_{ij}$, and $p^{(h)}$ in the definition of $\redi^{(h)}_{ij}$. Similarly
\begin{equation}
\Pi_{ij} = \Pi^{(r)}_{ij} + \Pi^{(s)}_{ij} + \Pi^{(h)}_{ij},
\end{equation}
where the different $\Pi_{ij}$'s are defined using their corresponding pressures.

%--------------------------------------------
\subsection{Anisotropic Reynolds-stresses}
%--------------------------------------------

The reynolds stresses can be decomposed into an isotropic and an anisotropic component as follows
\begin{equation}
\rs_{ij} = \frac{2}{3} k \delta_{ij} + \ars_{ij}.
\end{equation}
A non-dimensional anisotropy is defined as follows
\begin{equation}
b_{ij} = \frac{a_{ij}}{2k} = \frac{\rs_{ij}}{2k} - \frac{1}{3} \delta_{ij}.
\end{equation}
Taking the material derivative of $b_{ij}$ one obtains 
\[ \frac{\bar{D} b_{ij}}{\bar{D}t} = \frac{1}{2k} \frac{\bar{D}\rs_{ij} }{\bar{D}t} - \frac{\rs_{ij}}{2k^2} \frac{\bar{D} k}{\bar{D}t} \]
Thus, the transport equation for the non-dimensional anisotropy without body forces can be expressed as follows
\begin{multline}
\label{anisotropy_eq}
2k \left( \frac{\partial b_{ij}}{\partial t} + \ukavg \frac{\partial b_{ij}}{\partial x_k} \right ) =\\ \mathcal{P}_{ij} + \redi_{ij} - \epsilon_{ij} - \frac{\partial}{\partial x_k} \left ( \mathcal{T}^{(u)}_{kij} + \mathcal{T}^{(p)}_{kij} + \mathcal{T}^{(\nu)}_{kij} \right ) - \frac{\rs_{ij}}{k} \left [ \mathcal{P} - \epsilon - \frac{\partial}{\partial x_k} \left ( \mathcal{T}^{(u)}_{k} + \mathcal{T}^{(p)}_{k} + \mathcal{T}^{(\nu)}_{k} \right ) \right ].
\end{multline}

%--------------------------------------------
\subsection{Two-point velocity correlations}
%--------------------------------------------

%--------------------------------------------
\subsection{Squared averaged scalar}
%--------------------------------------------
Assuming incompressible flow for \cref{eq:squared_averaged_scalar_favre} leads to
\begin{equation}
    \frac{\partial \yavg ^2}{\partial t} + \ujavg \frac{\partial \yavg^2}{\partial x_j} = 2\frac{\langle J_j \rangle}{\rho} \frac{\partial \yavg}{\partial x_j} + 2\langle \yfluc \ujfluc \rangle \frac{\partial \yavg}{\partial x_j} - 2\frac{\partial}{\partial x_j} \left ( \frac{\yavg \langle J_j \rangle}{\rho} + \yavg \langle \yfluc \ujfluc \rangle \right )
\end{equation}

%--------------------------------------------
\subsection{Scalar variance}
%--------------------------------------------
Assuming incompressible flow for \cref{eq:scalar_variance_favre} leads to
\begin{equation}
    \label{eq:scalar_variance_rans}
    \frac{\partial \langle \yfluc \yfluc \rangle}{\partial t} + \ujavg \frac{\partial  \langle \yfluc \yfluc \rangle }{\partial x_j} = -2 \langle \yfluc \ujfluc \rangle \frac{\partial \yavg}{\partial x_j} + 2 \left < \frac{j_j}{\rho} \frac{\partial y}{\partial x_j} \right > - 2 \frac{\partial}{\partial x_j} \left ( \frac{1}{2} \langle \yfluc \yfluc \ujfluc \rangle +  \frac{ \langle \yfluc j_j \rangle}{\rho} \right).
\end{equation}

%%%%%%%%%%%%%%%%%%%%%%%%%%%%%%%%%%%%%%%%%%%%%%%%%%%%%%%%%%%%%%%%%%%%%%%%%
%                                                                       %
%                                                                       %
%                                                                       %
%                                                                       %
\part{Homogeneous Turbulence}
%
%                                                                       %
%                                                                       %
%                                                                       %
%                                                                       %
%%%%%%%%%%%%%%%%%%%%%%%%%%%%%%%%%%%%%%%%%%%%%%%%%%%%%%%%%%%%%%%%%%%%%%%%%

%###############################################################################
%
\chapter{Introduction}
%
%###############################################################################

%-------------------------------------------------------------------------------
\section{Evolution of the mean variables}
%-------------------------------------------------------------------------------
\label{sec:homo_mean_incomp}
In this section we derive the time evolutions of $\uiavg$ and $\pavg$. We know that for homogeneous turbulence, gradients of turbulent statistics for homogeneous turbulence vanish. The incompressible conservation of TKE \cref{eq:tke_ransflow} for homogeneous flows thus becomes
\begin{equation}
\frac{\partial k}{\partial t} = -\rs_{ij} \frac{\partial \uiavg}{\partial x_j} - \epsilon + \langle f_i \uifluc \rangle.
\end{equation}
This shows that the derivative of mean velocity is a function of time only, which we label as $G_{ij} = G_{ij}(t)$. Thus, we express the mean velocity as $\uiavg = G_{ij} x_j + C_i$, or 
\begin{equation}
\label{eq:ui_homo_incomp}
\uiavg = G_{ij} x_j ,
\end{equation}
after neglecting the constant of integration.

The transport equation for velocity \cref{eq:rans_vel2} then simplifies to the following for homogeneous turbulence
\begin{equation}
\frac{\partial \uiavg}{\partial t} + \ukavg G_{ik} = -\frac{1}{\rho} \frac{\partial \pavg}{\partial x_i} + \langle F_i \rangle.
\end{equation}
Differentiating the above by $x_j$ shows that $G_{ij}$ evolves according to
\begin{equation}
\label{eq:gij_homo_incomp}
\frac{d G_{ij} }{d t} + G_{kj}G_{ik} = -\frac{1}{\rho} \frac{\partial ^2 \pavg}{\partial x_i \partial x_j} + \frac{\partial \langle F_i \rangle}{\partial x_j} .
\end{equation}
Given no body forces, \cref{eq:gij_homo_incomp} shows that the second gradient of $\pavg$ is a function of time only, which we label as $C_{ij} = C_{ij}(t)$. We note that $C_{ij}$ is symmetric. We then express the pressure as $\pavg = C + C_i x_i + \frac{1}{2} C_{ij} x_ix_j$, or
\begin{equation}
    \pavg = \frac{1}{2} C_{ij} x_i x_j,
\end{equation}
after neglecting the constants of integration. 

Using the definition of $S_{ij}$ one can show that
\begin{align}
\label{eq:sij_homo_incomp}
\frac{d S_{ij}}{dt} &= \frac{1}{2} \left ( \frac{d G_{ij}}{dt} + \frac{d G_{ji}}{dt} \right ) \nonumber \\
& = \frac{1}{2} \left ( -G_{kj}G_{ik} - G_{ki}G_{jk} - \frac{2}{\rho} \frac{\partial^2 \pavg}{\partial x_i \partial x_j}  + \frac{\partial \langle F_i \rangle}{\partial x_j} + \frac{ \partial \langle F_j \rangle}{\partial x_i} \right ) \nonumber \\
& = -S_{ik}S_{jk} - W_{ik}W_{kj} - \frac{1}{\rho} \frac{\partial^2 \pavg}{\partial x_i \partial x_j} + F^{(s)}_{ij},
\end{align}
where
\begin{equation}
F^{(s)}_{ij} = \frac{1}{2} \left ( \frac{\partial \langle F_i \rangle}{\partial x_j} + \frac{ \partial \langle F_j \rangle}{\partial x_i} \right ).
\end{equation}
Following the same approach for $W_{ij}$ one obtains
\begin{equation}
\label{eq:wij_homo_incomp}
\frac{dW_{ij}}{dt} = -S_{kj}W_{ik} - W_{kj}S_{ik} + F^{(w)}_{ij},
\end{equation}
where
\begin{equation}
F^{(w)}_{ij} = \frac{1}{2} \left ( \frac{\partial \langle F_i \rangle}{\partial x_j} - \frac{ \partial \langle F_j \rangle}{\partial x_i} \right ).
\end{equation}

To know the evolution of the mean velocity and pressure for the case of incompressible homogeneous turbulence, one needs to know the evolution of $G_{ij}$ and $C_{ij}$. To do so, one can specify an evolution of $S_{ij}$, and use \cref{eq:wij_homo_incomp} to compute $W_{ij}$, to thus obtain the evolution of $G_{ij}$. $C_{ij}$ then follows from either \cref{eq:gij_homo_incomp} or \cref{eq:sij_homo_incomp}.


%-------------------------------------------------------------------------------
\section{Evolution of the fluctuating variables}
%-------------------------------------------------------------------------------

%--------------------------------------------
\subsection{Governing Equations}
%--------------------------------------------
Just as in section \ref{sec:homo_mean_incomp} our goal was to find evolution equations for the mean variables $\uiavg$ and $\pavg$, in this section we now seek to find the equations for the fluctuations $\uifluc = \uifluc(t,\xvec)$ and $\pfluc= p(t,\xvec)$. Under the assumption of homogeneous turbulence, the PDEs for these fluctuating variables become 
\begin{equation}
\label{fluc_u_homo}
\frac{\partial \uifluc}{\partial t} + G_{jk} x_k \frac{\partial \uifluc}{\partial x_j} = -\ujfluc G_{ij} - \frac{\partial \uifluc \ujfluc}{\partial x_j} - \frac{1}{\rho} \frac{\partial \pfluc}{\partial x_i} + \nu \frac{\partial^2 \uifluc}{\partial x_j \partial x_j},
\end{equation}
\begin{equation}
\label{fluc_p_homo}
\frac{1}{\rho} \frac{\partial^2 \pfluc}{\partial x_j \partial x_j} = -2 G_{ij} \frac{\partial \ujfluc}{\partial x_i} - \frac{\partial^2 \uifluc \ujfluc}{\partial x_i \partial x_j}.
\end{equation}

%--------------------------------------------
\subsection{Rogallo Transformation}
%--------------------------------------------
It is convenient to use the method of change of variables to transform the PDEs above to eliminate the convective term on the equation for the fluctuating velocity. Given functions $\urog_i = \urog_i(t,\xrogvec)$ and $\prog = \prog(t,\xrogvec)$ suppose the velocity and pressure fields can be re-expressed as follows
\begin{align}
\uifluc &= \urog_i(t,\fvec) \\
p &= \prog(t,\fvec),
\end{align}
for some function $\fvec = \bold{f}(t,\xvec)$. Thus, derivatives in the original frame convert to the following in the transformed frame
\begin{align}
    \frac{\partial }{\partial t} &= \left [ \frac{\partial }{\partial t } \right ]_{\xrogvec = \bold{f}} + \left [ \frac{\partial }{\partial \xrog_k}\right ]_{\xrogvec = \fvec} \frac{\partial f_k }{\partial t} \\
    \frac{\partial }{\partial x_i} &= \left [ \frac{\partial }{\partial \xrog_k} \right ]_{\xrogvec = \fvec} \frac{\partial f_k}{\partial x_i} \\
    \frac{\partial^2 }{\partial x_i \partial x_j} &= \left [ \frac{\partial^2 }{\partial \xrog_p \partial \xrog_k} \right ]_{\xrogvec = \fvec} \frac{\partial f_p}{\partial x_i} \frac{\partial f_k}{\partial x_j} + \left [ \frac{\partial  }{\partial \xrog_k} \right]_{\xrogvec = \fvec } \frac{\partial^2 f_k }{\partial x_i \partial x_j} .
\end{align}

Lets now pick a specific form for the function $\fvec$, namely $\fvec = \mathbf{A}\xvec$, where $\mathbf{A} = \mathbf{A}(t)$ is a square matrix. In tensor notation this is equivalent to
\begin{equation}
f_i = A_{in} x_n.
\end{equation}
The first two terms in the PDE for the fluctuating velocity thus become
\begin{equation}
\frac{\partial \uifluc}{\partial t} + G_{jk}x_k\frac{\partial u_i }{\partial x_j} = \left [ \frac{\partial \urog_i }{\partial t } \right ]_{\xrogvec = \mathbf{A}\xvec} + \left [ \frac{\partial \urog_i }{\partial \xrog_k}\right ]_{\xrogvec = \mathbf{A} \xvec} \frac{d A_{kn} }{d t} x_n + G_{jn}x_n \left [ \frac{\partial \urog_i }{\partial \xrog_k} \right ]_{\xrogvec = \mathbf{A} \xvec} A_{kj}
\end{equation}
If we choose $A_{kn}$ to satisfy
\begin{equation}
\label{rogallo_trans}
\frac{d A_{kn}}{dt} + G_{jn}A_{kj} = 0,
\end{equation}
with initial condition $A_{ij} = \delta_{ij}$ at $t = 0$, we then obtain 
\begin{equation}
\frac{\partial \uifluc }{\partial t} + G_{jk}x_k\frac{\partial u_i}{\partial x_j} = \left [ \frac{\partial \urog_i }{\partial t } \right ]_{\xrogvec = \mathbf{A} \xvec}.
\end{equation}

Using the above we can re-write the equation for the fluctuating velocity (\ref{fluc_u_homo}) as follows
\begin{equation}
\left [ \frac{\partial \urog_i }{\partial t } \right ]_{\xrogvec = \mathbf{A} \xvec} = -[\urog_j]_{\xrogvec = \mathbf{A} \xvec} G_{ij} - \left [ \frac{\partial \urog_i \urog_j}{\partial \xrog_k} \right ]_{\xrogvec = \mathbf{A}\xvec} A_{kj} - \left [ \frac{1}{\rho} \frac{\partial \prog }{\partial \xrog_k} \right ]_{\xrogvec = \mathbf{A}\xvec } A_{ki} + \nu \left [ \frac{ \partial^2 \urog_i}{\partial \xrog_p \partial \xrog_k} \right ]_{\xrogvec = \mathbf{A} \xvec} A_{pj} A_{kj},
\end{equation}
which is equivalent to
\begin{equation}
\label{eq:u_lin_tran}
\frac{\partial \urog_i }{\partial t } = -\urog_j G_{ij} - \frac{\partial \urog_i \urog_j}{\partial \xrog_k} A_{kj} - \frac{1}{\rho} \frac{\partial \prog }{\partial \xrog_k} A_{ki} + \nu \frac{ \partial^2 \urog_i}{\partial \xrog_p \partial \xrog_k} A_{pj} A_{kj}.
\end{equation}
A similar analysis for the fluctuating pressure equation leads to
\begin{equation}
\label{p_lin_tran}
\frac{1}{\rho} \frac{\partial^2 \prog }{\partial \xrog_m \partial \xrog_l} A_{mk} A_{lk} = -2 G_{ij} \frac{\partial \urog_j }{\partial \xrog_t} A_{ti} - \frac{\partial^2 \urog_i \urog_j}{\partial \xrog_p \partial \xrog_k} A_{pi}A_{kj}.
\end{equation}

%-------------------------------------------------------------------------------
\section{The Reynolds-stress evolution equation}
%-------------------------------------------------------------------------------

The transport equation for the Reynolds stresses simplifies to the following evolution equation
\begin{equation}
\frac{d \rs_{ij}}{dt} = \mathcal{P}_{ij} +\redi_{ij} - \epsilon_{ij},
\end{equation}
where $\redi_{ij} = \redi^{(r)}_{ij} + \redi^{(s)}_{ij}$, that is, $\redi^{(h)} = 0$. The fluctuating pressure satisfies the following Poisson equation
\begin{equation}
\frac{\partial^2 \pfluc}{\partial x_j \partial x_j} = -2 \rho \frac{\partial \uiavg}{\partial x_j} \frac{\partial \ujfluc}{\partial x_i}  - \rho \frac{\partial^2 \uifluc \ujfluc}{\partial x_i \partial x_j},
\end{equation}
where the first term on the right hand side corresponds to the rapid pressure whereas the second corresponds to the slow pressure.

The solution to the PDE above is obtained using Green's function, that is, for 
\begin{equation}
\frac{\partial^2 f(\xvec)}{\partial x_k \partial x_k} = S(\xvec) \qquad (\xvec \in \Rthree)
\end{equation}
the solution is of the following form, 
\begin{equation}
f(\xvec) = \int_\Rthree S(\yvec) G(\xvec - \yvec) d\yvec
\end{equation}
where 
\begin{equation}
G(\rvec) = \frac{-1}{4\pi ||\rvec||_2} \quad \text{for } \Rthree.
\end{equation}
Since the deformation $\partial \uiavg / \partial x_j$ is constant and $\partial u_j(\xvec) / \partial x_i$ as a function of $\yvec$ becomes $\partial u_j(\yvec) / \partial y_i$, we obtain,
\begin{equation}
\label{pfluc_homo}
p(\xvec) =  -2 \rho \frac{\partial \uiavg}{\partial x_j} \int_\Rthree \frac{\partial u_j(\yvec)}{\partial y_i} G(\xvec-\yvec) d\yvec - \rho \int_\Rthree \frac{\partial^2 \uifluc \ujfluc}{\partial x_i \partial x_j} G(\xvec - \yvec) d\yvec 
\end{equation}

%--------------------------------------------
\subsection{The rapid redistribution}
%--------------------------------------------

Using equation (\ref{pfluc_homo}) we have
\begin{equation}
\left \langle \frac{p^{(r)}(\xvec)}{\rho}\frac{\partial u_i(\xvec)}{\partial x_j} \right \rangle= -2\frac{\partial \ukavg}{\partial x_l} \int_\Rthree \left \langle \frac{\partial u_i(\xvec)}{\partial x_j} \frac{\partial u_l(\yvec)}{\partial y_k} \right \rangle G(\xvec-\yvec) d\yvec.
\end{equation}
Since the two-point correlation is a function of $\xvec - \yvec$ and not of $\xvec$ and $\yvec$ independently, we can rewrite the expectation inside the integral as follows,
\begin{equation}
\left \langle \frac{\partial u_i(\xvec)}{\partial x_j} \frac{\partial u_l(\yvec)}{\partial y_k} \right \rangle = \frac{\partial}{\partial x_j} \frac{\partial}{\partial y_k} \langle u_i(\xvec)u_l(\yvec) \rangle = \frac{\partial}{\partial x_j} \frac{\partial}{\partial y_k} \tpvc_{il}(\xvec - \yvec) = \left [-\frac{\partial}{\partial r_j} \frac{\partial}{\partial r_k} \tpvc_{il}(\rvec) \right]_{\rvec = \xvec-\yvec}
\end{equation}
where we have used the chain rule to obtain the last equality.
Hence the expectation of the pressure-velocity gradient has the following form,
\begin{equation}
\left \langle \frac{p^{(r)}(\xvec)}{\rho}\frac{\partial u_i(\xvec)}{\partial x_j} \right \rangle= 2\frac{\partial \ukavg}{\partial x_l} \int_\Rthree \left [\frac{\partial^2 \tpvc_{il}(\rvec)}{\partial r_j \partial r_k} \right]_{\rvec = \xvec-\yvec} G(\xvec-\yvec) d\yvec.
\end{equation}
We then proceed by integration by substitution, with the substitution $\yvec = \xvec + \rvec$.
\begin{equation}
\left \langle \frac{p^{(r)}(\xvec)}{\rho}\frac{\partial u_i(\xvec)}{\partial x_j} \right \rangle = 2\frac{\partial \ukavg}{\partial x_l} \int_\Rthree\frac{\partial^2 \tpvc_{il}(\rvec)}{\partial r_j \partial r_k} G(\rvec) d\rvec.
\end{equation}
Using the definition of the Green's function, we rewrite the above as,
\begin{equation}
\left \langle \frac{p^{(r)}(\xvec)}{\rho}\frac{\partial u_i(\xvec)}{\partial x_j} \right \rangle = 2 \frac{\partial \ukavg}{\partial x_l} \int_\Rthree \frac{-1}{4\pi||\rvec||_2} \frac{\partial^2 \tpvc_{il}(\rvec)}{\partial r_j \partial r_k} d\rvec = 2 \frac{\partial \ukavg}{\partial x_l} M_{iljk}
\end{equation}
where we have defined the tensor 
\begin{equation}
M_{iljk} = \frac{-1}{4\pi}\int_\Rthree \frac{1}{||\rvec||_2} \frac{\partial^2 \tpvc_{il}(\rvec)}{\partial r_j \partial r_k} d\rvec 
\end{equation}
The rapid pressure-rate-of-strain term can now be written in the succinct notation as follows,
\begin{equation}
\redi^{(r)}_{ij} = 2\frac{\partial \ukavg}{\partial x_l} (M_{iljk} + M_{jlik}).
\end{equation}

%###############################################################################
%
\chapter{Spectral Analysis}
%
%###############################################################################

%-------------------------------------------------------------------------------
\section{Definitions}
%-------------------------------------------------------------------------------
One can use spectral tools such as the Fourier transform to analyze the fluid equations from a different perspective. In this section we provide definitions for various spectral quantities required for this spectral approach. For sake of notational simplicity, we refer to either the Reynolds-fluctuating or the Favre-fluctuating velocity simply as $u_i$ (no apostrophe or double apostrophe).

%--------------------------------------------
\subsection{Three-dimensional spectral analysis}
%--------------------------------------------
We begin with the more general three-dimensional case.

\subsubsection{Velocity}
The fluctuating velocity is expressed either as a Fourier series (denoted here as the discrete approach) or as an inverse Fourier transform (denoted as the continuous approach). Thus, for fluctuating velocity we have 
\begin{itemize}
\item Discrete
\begin{align}
\ujfluc(\xvec,t) &= \sum_{\kvec} \hat{u}_j(\kvec,t) e^{i \kvec \cdot \xvec}, \\
\hat{u}_j(\kvec,t) &= \frac{1}{L^3} \int_{\mathbb{L}^3} \ujfluc(\xvec,t) e^{-i \kvec \cdot \xvec} d\xvec.
\end{align}
The above represents the entire flow as a juxtaposition of cubic domains 
\begin{equation}
\mathbb{L}^3 = \{ (x,y,z): 0< x < L, 0 < y < L, 0< z< L \}
\end{equation}
The velocity field is periodic across the cubes, i.e. $\uifluc(\xvec + NL,t) = \uifluc(\xvec,t)$ for all integer vectors $N$. Note that $\uifluc(\xvec,t)$ corresponds to the velocity of the original homogeneous flow within a cube of equal dimensions. This representation maintains spatial homogeneity over $\Rthree$. The effects of the artificially imposed periodicity decreases as the ratio of $L$ over the integral length scale approaches infinity.
\item Continuous
\begin{align}
\ujfluc(\xvec,t) &= \int_\Rthree \hat{u}_j(\kvec,t) e^{i\kvec \cdot \xvec} d\kvec, \\
\hat{u}_j(\kvec,t) &= \frac{1}{(2\pi)^3} \int_\Rthree \ujfluc(\xvec,t) e^{-i \kvec \cdot \xvec} d\xvec.
\end{align}
Write something here about why we can use the Fourier transform...generalized functions.
\end{itemize}

\subsubsection{Spectra}
Using the above representations, the two-point velocity correlation can be used to define the discrete ($\hat{\tpvc}_{ij}$) and continuous ($\est_{ij}$) velocity spectrum tensors.
\begin{itemize}
\item Discrete
\begin{align}
\label{eq:tpvc_discrete}
\tpvc_{ij}(\rvec,t) & = \sum_{\kvec} \hat{\tpvc}_{ij}(\kvec,t) e^{i \kvec \cdot \rvec}, \\
\label{espec_discrete}
\hat{\tpvc}_{ij}(\kvec,t) & = \frac{1}{L^3} \int_{\mathbb{L}^3} \tpvc_{ij}(\rvec,t) e^{-i \kvec \cdot \rvec} d\rvec.
\end{align}
\item Continuous
\begin{align}
\label{eq:tpvc_continuous}
\tpvc_{ij}(\rvec,t) &= \int_\Rthree E_{ij}(\kvec,t) e^{i \kvec \cdot \rvec} d\kvec, \\
\label{espec_continuous}
E_{ij}(\kvec,t) & = \frac{1}{(2\pi)^3} \int_\Rthree \tpvc_{ij}(\rvec,t) e^{-i\kvec \cdot \rvec} d\rvec.
\end{align}
\end{itemize}

\subsubsection{Additional identities}
\begin{itemize}
\item Discrete
\begin{equation} 
\label{fromu_toE_disc}
\langle \hat{u}_i^*(\kvec,t) \hat{u}_j(\kvec',t) \rangle = \hat{\tpvc}_{ij}(\kvec',t) \delta_{\kvec',\kvec}.
\end{equation}
\item Continuous
\begin{equation} 
\label{fromu_toE_cont}
\langle \hat{u}_i^*(\kvec,t) \hat{u}_j(\kvec',t) \rangle = E_{ij}(\kvec',t)\delta(\kvec' - \kvec).
\end{equation}
\end{itemize}

%--------------------------------------------
\subsection{One-dimensional spectral analysis}
%--------------------------------------------
All of the relations above apply for transforms along one direction only. These are shown below.

\subsubsection{Velocity}
\begin{itemize}
\item Discrete
\begin{align}
\ujfluc(\evec_1 x_1,t) &= \sum_{\kappa_1} \hat{u}^{(\evec_1)}_j(\kappa_1,t) e^{i \kappa_1 x_1}\\
\hat{u}^{(\evec_1)}_j(\kappa_1,t) &= \frac{1}{L} \int_0^L \ujfluc(\evec_1 x_1,t) e^{-i \kappa_1 x_1} \, dx_1,
\end{align}
\item Continuous
\begin{align}
\ujfluc(\evec_1 x_1,t) &= \int_{-\infty}^{\infty} \hat{u}^{(\evec_1)}_j(\kappa_1,t) e^{i \kappa_1 x_1} \, d\kappa_1 \\
\hat{u}^{(\evec_1)}_j(\kappa_1,t) &= \frac{1}{2\pi} \int_{-\infty}^{\infty} \ujfluc(\evec_1 x_1,t) e^{-i \kappa_1 x_1} \, dx_1.
\end{align}
\end{itemize}

\subsubsection{Spectra}
\begin{itemize}
\item Discrete
\begin{align}
\tpvc_{ij}(\evec_1 r_1, t) &= \sum_{\kappa_1} \hat{\tpvc}_{ij}^{(\evec_1)}(\kappa_1,t) e^{i \kappa_1 x_1} \\
\hat{\tpvc}_{ij}^{(\evec_1)}(\kappa_1,t) &= \frac{1}{L} \int_0^L \tpvc_{ij} (\evec_1 r_1, t) e^{-i\kappa_1 r_1} \, dr_1,
\end{align}
\item Continuous
\begin{align}
\tpvc_{ij}(\evec_1 r_1, t) &= \int_{-\infty}^{\infty} \est_{ij}^{(\evec_1)}(\kappa_1,t) e^{i \kappa_1 x_1} \, d\kappa_1 \\
\est_{ij}^{(\evec_1)}(\kappa_1,t) &=  \frac{1}{2 \pi} \int_{-\infty}^{\infty} \tpvc_{ij}(\evec_1 r_1,t) e^{-i \kappa_1 r_1} \, d r_1
\end{align}
\end{itemize}

\subsubsection{Additional identities}
\begin{itemize}
\item Discrete
\begin{equation} 
\langle \hat{u}^{(\evec_1)*}_i(\kappa_1,t) \hat{u}^{(\evec_1)}_j(\kappa'_1,t) \rangle = \hat{\tpvc}^{(\evec_1)}_{ij}(\kappa'_1,t) \delta_{\kappa'_1,\kappa_1}.
\end{equation}
\item Continuous
\begin{equation} 
\langle \hat{u}^{(\evec_1)*}_i(\kappa_1,t) \hat{u}^{(\evec_1)}_j(\kappa'_1,t) \rangle = E^{(\evec_1)}_{ij}(\kappa'_1,t)\delta(\kappa'_1 - \kappa_1).
\end{equation}
\end{itemize}

The relationship between one- and three-dimensional spectra are given by
\begin{itemize}
\item Discrete
\begin{equation}
\hat{\tpvc}_{ij}^{(\evec_1)}(\kappa_1,t) = \sum_{\kappa_2} \sum_{\kappa_3} \hat{\tpvc}_{ij} (\kvec)
\end{equation}
\item Continuous
\begin{equation}
\est_{ij}^{(1d)}(\kappa_1,t) = \iint_{-\infty}^{\infty} \est_{ij}(\kvec,t) \, d\kappa_2 \, d\kappa_3
\end{equation}
\end{itemize}

%--------------------------------------------
\subsection{Kinetic Energy spectrum}
%--------------------------------------------

The energy spectrum is defined as 
\begin{equation}
\est( \kappa, t ) = \oint \frac{1}{2} \est_{ii}( \kvec, t ) \, dS(\kappa) = \int_0^{2\pi} \int_0^\pi \frac{1}{2} \est_{ii}(\kvec, t) \kappa^2 \sin(\theta) \, d\theta d\phi
\end{equation}

A relationship can be obtained between the discrete and continuous approaches. If we denote $\kvec_D$ and $\kvec_C$ as the wave vectors of the discrete and continuous approach, this relationship is as follows
\begin{equation}
\label{eq:vst_discrete_continuous}
\est_{ij}(\kvec_C,t) = \sum_{\kvec_D} \delta(\kvec_C - \kvec_D) \hat{\tpvc}_{ij}(\kvec_D,t).
\end{equation}

An alternative relationship between the discrete and continuous approaches that is more amenable for numerical computations can be established. We note that the TKE can be computed from either the discrete or continuous approach as follows
\begin{equation}
    \label{eq:vst_discrete_continuous_temp1}
    k = \sum_{\kvec} \frac{1}{2} \hat{\tpvc}_{ii}(\kvec,t) = \int_0^\infty \est(\kappa,t) \, d\kappa.
\end{equation}
We now discretize the magnitude of the wave vector, that is, $\kappa = n \Delta \kappa$, where $n$ is an integer and $\Delta \kappa$ is the width of the discrete chunks of $\kappa$. We introduce the set $S(\kappa)$ as all the wavevectors $\kvec$ whose magnitude falls within $[\kappa, \kappa + \Delta \kappa]$. Using the above, we re-write \cref{eq:vst_discrete_continuous_temp1} as
\begin{equation}
    \sum_{\kappa} \left ( \sum_{\kvec \in S(\kappa)} \frac{1}{2} \hat{\tpvc}_{ii}(\kvec,t) \right) = \sum_{\kappa} \est(\kappa,t) \Delta \kappa.
\end{equation}
Thus, we approximate the energy spectrum as
\begin{equation}
    \label{eq:num_comp_spectra}
    E(\kappa,t) = \frac{1}{\Delta \kappa} \sum_{\kvec \in S(\kappa)} \frac{1}{2} \hat{\tpvc}_{ii}(\kvec,t) .
\end{equation}


%--------------------------------------------
\subsection{Spectral representation of statistics}
%--------------------------------------------
Using expressions from the previous section we obtain
\begin{itemize}
\item Discrete
\begin{equation}
    \langle \uifluc \ujfluc \rangle = \sum_{\kvec} \hat{\tpvc}_{ij}(\kvec,t)
\end{equation} 
\begin{equation}
    \label{eq:discrete_k_spectral}
    k = \sum_{\kvec} \frac{1}{2} \hat{\tpvc}_{ii}(\kvec,t) 
\end{equation}
\begin{equation}
    \left < \frac{ \partial \uifluc }{ \partial x_k} \frac{ \partial \ujfluc }{ \partial x_l } \right > = \sum_{\kvec} \kappa_k \kappa_l \hat{\tpvc}_{ij}(\kvec,t)
\end{equation}
\begin{equation}
    \langle s^2 \rangle = \langle 2 s_{ij} s_{ij} \rangle = \sum_{\kvec} \kappa_i \kappa_i \hat{\tpvc}_{jj}(\kvec,t)
\end{equation}

\item Continuous
\begin{equation}
    \label{eq:continuous_uiuj_spectral}
    \langle \uifluc \ujfluc \rangle = \int_\Rthree E_{ij}(\kvec,t) d\kvec
\end{equation} 
\begin{equation}
    k =\int_0^\infty \est(\kappa,t) \, d\kappa
\end{equation}
\begin{equation}
    \left < \frac{ \partial \uifluc }{ \partial x_k} \frac{ \partial \ujfluc }{ \partial x_l } \right > = \int_{\Rthree} \kappa_k \kappa_l \est_{ij}(\kvec,t) \, d\kvec
\end{equation}
\begin{equation}
    \label{eq:strain_mag_spectral}
    \langle s^2 \rangle = \langle 2 s_{ij} s_{ij} \rangle = 2 \int_{0}^\infty \kappa^2 \est(\kappa,t) \, d\kappa
\end{equation}
\end{itemize}



%-------------------------------------------------------------------------------
\section{The fluctuating velocity}
%-------------------------------------------------------------------------------
We rewrite below \cref{eq:u_lin_tran} and \cref{eq:p_lin_tran} but with the $\mathring{}$ superscript neglected for convenience. 
\begin{equation}
\frac{\partial \uifluc }{\partial t } = -\ujfluc G_{ij} - \frac{\partial \uifluc \ujfluc}{\partial x_k} A_{kj} - \frac{1}{\rho} \frac{\partial \pfluc }{\partial x_k} A_{ki} + \nu \frac{ \partial^2 \uifluc}{\partial x_p \partial x_k} A_{pj} A_{kj}.
\end{equation}
\begin{equation}
\frac{1}{\rho} \frac{\partial^2 \pfluc }{\partial x_m \partial x_l} A_{mk} A_{lk} = -2 G_{ij} \frac{\partial \ujfluc }{\partial x_t} A_{ti} - \frac{\partial^2 \uifluc \ujfluc}{\partial x_p \partial x_k} A_{pi}A_{kj}.
\end{equation}

The Fourier transform of the above two leads to
\begin{equation}
\frac{\partial \hat{u}_i }{\partial t } = -\hat{u}_j G_{ij} - i A_{kj}\kappa_k  \hat{N}_{ij} - i\frac{1}{\rho} A_{ki} \kappa_k  \hat{ p } - \nu  A_{pj}\kappa_p A_{kj}\kappa_k \hat{u}_i.
\end{equation}
\begin{equation}
-\frac{1}{\rho} A_{mk}\kappa_m A_{lk}\kappa_l \hat{ p } = -2 G_{ij} i A_{ti}\kappa_t \hat{u}_j + A_{pi}\kappa_p A_{kj}\kappa_k \hat{N}_{ij}.
\end{equation}
where $\hat{u}_i = \hat{u}_i(t,\kvec)$, $\hat{p} = \hat{p}(t,\kvec)$, and $\hat{N}_{ij} = \hat{N}_{ij}(t,\kvec)$ are the Fourier transforms of $\uifluc$, $p$, and $\uifluc \ujfluc$, respectively. 

Define the function $\hat{\kvec} = \hat{\kvec}(t,\kvec)$, which is determined by the following ODE
\begin{equation}
\frac{\partial \hat{\kappa}_i}{\partial t} = -G_{ki} \hat{\kappa}_k,
\end{equation}
with initial condition $\hat{\kvec} = \kvec$ at $t = 0$. Since the ODE is linear we can assume solutions of the form $\hat{\kappa}_i = B_{ji} \kappa_j$, where $B_{ij} = B_{ij}(t)$. Plugging this form into the ODE and the initial condition above, we require that $B_{ij}$ satisfy
\begin{equation}
\frac{d B_{ji}}{dt} + G_{ki} B_{jk} = 0,
\end{equation}
along with $B_{ij} = \delta_{ij}$ at $t=0$. Since these requirements are identical to those satisfied by $A_{ij}$, we conclude that 
\begin{equation}
\hat{\kappa}_i = A_{ji} \kappa_j,
\end{equation}
which in vector notation becomes $\hat{\kvec} = \mathbf{A}^T\kvec$. Using this definition for $\hat{\kvec}$ the evolution equation for $\hat{u}_i$ and $\hat{p}$ become
\begin{equation}
\frac{\partial \hat{u}_i }{\partial t } = -\hat{u}_j G_{ij} - i \hat{\kappa}_j  \hat{N}_{ij} - i\frac{1}{\rho} \hat{\kappa}_i  \hat{p} - \nu  \hat{\kappa}^2 \hat{u}_i,
\end{equation}
\begin{equation}
-\frac{1}{\rho} \hat{\kappa}^2 \hat{p} = -2 G_{ij} i \hat{\kappa}_i \hat{u}_j + \hat{\kappa}_i \hat{\kappa}_j \hat{N}_{ij}.
\end{equation}
Combining the above two we have
\begin{equation}
\label{vhat_evol_homo}
\frac{\partial \hat{u}_i }{\partial t } = -M_{ij}\hat{u}_j + s_i  - \nu  \hat{\kappa}^2 \hat{u}_i,
\end{equation}
where $M_{ij} = M_{ij}(t,\kvec)$ is given by
\begin{equation}
M_{ij} = G_{ij} - 2 \frac{\hat{\kappa}_i \hat{\kappa}_k}{\hat{\kappa}^2} G_{kj},
\end{equation}
$s_{i} = s_{i}(t,\kvec)$ by
\begin{equation}
s_{i} = -i \hat{\kappa}_j L_{ik}\hat{N}_{kj},
\end{equation}
and $L_{ij} = L_{ij}(t,\kvec)$ by
\begin{equation}
L_{ij} = \delta_{ij} - \frac{ \hat{\kappa}_i \hat{\kappa}_j}{\hat{\kappa}^2}.
\end{equation}
This last tensor is termed the projection operator since $L_{ij} y_j$ is the component of some generic vector $\yvec$ orthogonal to $\hat{\kvec}$. The term $-M_{ij}\hat{v}_j$, which is linear, arises from the combination of the first terms on the RHS of the velocity and pressure equations, whereas the term $s_{i}$, which is non-linear, arises from combination of the second terms on the RHS of the velocity and pressure equations.

The Fourier coefficient $\hat{N}_{ij}$ can be expressed as follows
\begin{align}
\hat{N}_{ij}(t,\kvec) &= \frac{1}{(2\pi)^3} \int_\Rthree u_i(t,\xvec) u_j(t,\xvec) e^{-i \kvec \cdot \xvec} d\xvec \nonumber \\
&= \frac{1}{(2\pi)^3} \int_\Rthree \left ( \int_\Rthree \hat{u}_i(t,\pvec) e^{i \pvec \cdot \xvec} d\pvec \right ) \left ( \int_\Rthree \hat{u}_j(t,\qvec) e^{i\qvec \cdot \xvec} d \qvec \right ) e^{-i \kvec \cdot \xvec} d\xvec \nonumber \\
&= \int_\Rthree \int_\Rthree \hat{u}_i(t,\pvec) \hat{u}_j(t,\qvec) \frac{1}{(2\pi)^3} \int_\Rthree e^{i (\pvec + \qvec) \cdot \xvec} e^{-i \kvec \cdot \xvec} d\xvec d\pvec d\qvec \nonumber \\
&=  \int_\Rthree \int_\Rthree \hat{u}_i(t,\pvec) \hat{u}_j(t,\qvec) \boldsymbol{\delta}(\pvec + \qvec - \kvec)  d\pvec d\qvec \nonumber \\
&= \int_\Rthree \hat{u}_i(t,\pvec) \hat{u}_j(t,\kvec - \pvec) d\pvec.
\end{align}

%-------------------------------------------------------------------------------
\section{The velocity-spectrum tensor}
%-------------------------------------------------------------------------------

We proceed in deriving the evolution equation for the velocity-spectrum tensor as follows,
\begin{align}
\label{E_intermediate}
\frac{\partial}{ \partial t} [ \hat{u}_i^*(t,\kvec) \hat{u}_j(t,\kvec') ] =& \frac{\partial \hat{u}_i^*(t,\kvec)}{\partial t} \hat{u}_j(t,\kvec') + \hat{u}_i^*(t,\kvec) \frac{\partial \hat{u}_j(t,\kvec')}{\partial t} \nonumber \\
=& \left [ -M_{ik}(t,\kvec) \hat{u}_k^*(t,\kvec) + s_i^*(t,\kvec) - \nu \hat{\kappa}^2(t,\kvec) \hat{u}_i^*(t,\kvec) \right ] \hat{u}_j(t,\kvec') \nonumber \\
&+ \hat{u}_i^*(t,\kvec)  \left [ -M_{jk}(t,\kvec') \hat{u}_k(t,\kvec') + s_j(t,\kvec') - \nu \hat{\kappa}^2(t,\kvec') \hat{u}_j(t,\kvec') \right ] \nonumber \\
=& -M_{ik}(t,\kvec) \hat{u}_k^*(t,\kvec) \hat{u}_j(t,\kvec') - M_{jk}(t,\kvec') \hat{u}_i^*(t,\kvec) \hat{u}_k(t,\kvec') \nonumber \\
& + \hat{u}_i^*(t,\kvec) s_j(t,\kvec') + s^*_i(t,\kvec) \hat{u}_j(t,\kvec') \nonumber \\
& - \nu \hat{\kappa}^2(t,\kvec) \hat{u}^*_i(t,\kvec) \hat{u}_j(t,\kvec') - \nu \hat{\kappa}^2(t,\kvec') \hat{u}^*_i(t,\kvec) \hat{u}_j(t,\kvec')
\end{align}
We now define the tensor $T_{ij} = T_{ij}(t,\kvec)$ using the following equation
\begin{equation}
\label{fromvs_toT}
\langle \hat{u}_i^*(t,\kvec) s_j(t,\kvec') + s^*_i(t,\kvec) \hat{u}_j(t,\kvec') \rangle = T_{ij}(t,\kvec') \boldsymbol{\delta}(\kvec' - \kvec).
\end{equation}
Taking the expectation of both sides of equation (\ref{E_intermediate}), and using equations (\ref{fromu_toE_cont}) and (\ref{fromvs_toT}) one obtains
\begin{equation}
\begin{split}
\frac{\partial E_{ij}(t,\kvec')}{\partial t} \boldsymbol{\delta}(\kvec' - \kvec) = \left [ -M_{ik}(t,\kvec) E_{kj}(t,\kvec') - M_{jk}(t,\kvec') E_{ik}(t,\kvec') + T_{ij}(t,\kvec') \right . \\
\left . - \nu \hat{\kappa}^2(t,\kvec) E_{ij}(t,\kvec') - \nu \hat{\kappa}^2(t,\kvec') E_{ij}(t,\kvec') \right ] \boldsymbol{\delta}(\kvec' - \kvec).
\end{split}
\end{equation}
Integrating over $\kvec'$ the evolution equation for the velocity-spectrum tensor $E_{ij} = E_{ij}(t,\kvec)$ becomes
\begin{equation}
\label{E_evol_homo}
\frac{\partial E_{ij}}{\partial t} = P_{ij} + T_{ij} - 2\nu \hat{\kappa}^2 E_{ij},
\end{equation}
where $P_{ij} = -M_{ik} E_{kj} - M_{jk} E_{ik}$.

\paragraph{A clarification note}
At this point it is important to note that we've been working with Fourier transforms of variables that were defined in the Rogallo reference frame, that is, those that have the superscript $\mathring{}$ (this superscript was neglected for convenience). The relationship between statistics of $\uifluc(t,\xvec)$ and $\urog_i(t,\xrogvec)$ is now discussed. For the two-point correlations, both quantities are related as
\begin{equation}
\langle \uifluc(t,\xvec) \ujfluc(t,\xvec') \rangle = \left [ \langle \urog_i(t,\xrogvec) \urog_j(t,\xrogvec') \rangle \right ]_{\xrogvec = \bold{A}\xvec, \; \xrogvec' = \bold{A}\xvec'}.
\end{equation}
Using equation (\ref{eq:tpvc_continuous}) we can express $\langle \urog_i(t,\xrogvec) \urog_j(t,\xrogvec') \rangle$ as the following integral
\begin{equation}
\langle \urog_i(t,\xrogvec) \urog_j(t,\xrogvec') \rangle = \int_\Rthree E_{ij}(t,\kvec) e^{i\kvec \cdot (\xrogvec' - \xrogvec)} d\kvec.
\end{equation}
Since
\begin{equation}
\kvec \cdot ( \mathbf{A} \xvec' - \mathbf{A} \xvec) = \kappa_iA_{ij} (x'_j - x_j) = \hat{\kappa}_j (x'_j - x_j)
\end{equation}
the two-point velocity correlation for $\uifluc$ becomes
\begin{equation}
\langle u_i(t,\xvec) u_j(t,\xvec') \rangle = \int_\Rthree E_{ij}(t,\kvec) e^{i \hat{\kvec}(t,\kvec) \cdot ( \xvec' - \xvec)} d\kvec .
\end{equation}
The Reynolds stresses are therefore obtained from the following
\begin{equation}
\rs_{ij} = \int_\Rthree \est_{ij} \, d\kvec.
\end{equation}

We now show a summary of the correspondence between the different terms in the evolution equations for the fluctuating velocity (\ref{eq:vel_fluc}), the energy spectrum (\ref{E_evol_homo}), and the Reynolds stresses (\ref{rs_transport}).

\begin{tabular} {| Sc | Sc | Sc | Sc | Sc | Sc | Sc | Sc | Sc | Sc |}
\hline
$\uifluc$: & $\dfrac{\partial \uifluc}{\partial t}$ & $\ujavg \dfrac{\partial \uifluc}{\partial x_j}$ & $-\ujfluc \dfrac{ \partial \uiavg}{\partial x_j}$ & $-\dfrac{1}{\rho}\dfrac{\partial p^{(r)}}{\partial x_i}$ & $-\dfrac{1}{\rho}\dfrac{\partial p^{(s)}}{\partial x_i}$ & $-\dfrac{\partial \uifluc \ujfluc}{\partial x_j}$ & \multicolumn{2}{c |}{$\nu \dfrac{\partial^2 \uifluc}{\partial x_j \partial x_j}$} &$\dfrac{\partial \rs_{ij}}{\partial x_j}$ \\
\hline
$\est_{ij}$: & $\dfrac{\partial \est_{ij}}{\partial t}$ & makes $\hat{\kvec}$ &  \multicolumn{2}{c |}{$P_{ij}$}  & \multicolumn{2}{c |}{$T_{ij}$} & \multicolumn{2}{c |}{$-2 \nu \hat{\kappa}^2 \est_{ij}$} & Null\\
\hline
$\rs_{ij}$: & $\dfrac{\partial \rs_{ij}}{\partial t}$ & $\ukavg\dfrac{\partial \rs_{ij}}{\partial x_k}$ & $\mathcal{P}_{ij}$ &  \multicolumn{2}{c |}{$\Pi_{ij}$}  & $\mathcal{T}_{kij}^{(u)}$ & $\epsilon_{ij}$ & $\mathcal{T}_{kij}^{(\nu)}$ & Null \\
\hline
\end{tabular}

%###############################################################################
%
\chapter{Rapid-distortion theory}
%
%###############################################################################
\begin{itemize}
\item Homogeneous RDT used much more (origin. Taylor, Prandt'l, 1930's)
\item Viscous effects can be included but often neglected
\item Inhomogeneous RDT developed by Hunt (1973, JFM) and collaborators. See Hunt \& Carruthers (1990, JFM), Goldsten, Atassi, Peake.
\end{itemize}

%-------------------------------------------------------------------------------
\section{When are mean distortions rapid enough?}
%-------------------------------------------------------------------------------
Start with the homogeneous equations for the turbulent fluctuations,
\begin{equation}
\frac{\partial \uifluc}{\partial t} + G_{jk} x_k \frac{\partial \uifluc}{\partial x_j} = -\ujfluc G_{ij} - \frac{\partial \uifluc \ujfluc}{\partial x_j} - \frac{1}{\rho} \frac{\partial \pfluc}{\partial x_i} + \nu \frac{\partial^2 \uifluc}{\partial x_j \partial x_j} 
\end{equation}
\begin{equation}
\frac{1}{\rho} \frac{\partial^2 \pfluc}{\partial x_j \partial x_j} = -2 G_{ij} \frac{\partial \ujfluc}{\partial x_i} - \frac{\partial^2 \uifluc \ujfluc}{\partial x_i \partial x_j}.
\end{equation}
We will now assume orders of magnitude for the variables above. This ordering is as follow
\begin{equation}
    u_i \sim u_0 \qquad p \sim p_0 \qquad G_{ij} \sim S \qquad t \sim S^{-1} \qquad x_i \sim l_0.  
\end{equation}
In the above, $u_0$ is the characteristic velocity of large-scale eddies, $l_0$ the length scale of large-scale eddies, $p_0$ the characteristic scale for pressure, and $S^{-1}$ the time scale of the deformation. Using the above, the terms in the pressure equation have the following order of magnitude
\begin{equation}
\underbrace{\frac{1}{\rho} \frac{\partial^2 \pfluc}{\partial x_j \partial x_j}}_{\dfrac{p_0}{\rho l_0^2}} = -\underbrace{2 G_{ij} \frac{\partial \ujfluc}{\partial x_i}}_{\dfrac{S u_0}{l_0}} - \underbrace{\frac{\partial^2 \uifluc \ujfluc}{\partial x_i \partial x_j}}_{\dfrac{u_0^2}{l_0^2}}.
\end{equation}
The linear term is of much larger magnitude when
\begin{equation}
\frac{\frac{Su_0}{l_0}}{\frac{u_0^2}{l_0^2}} \gg 1 \to \frac{S l_0}{u_0} \gg 1.
\end{equation}
If this is satisfied, then 
\begin{equation}
\frac{p_0}{\rho l_0^2} \sim \frac{S u_0}{l_0} \to p_0 \sim \rho S u_0 l_0.
\end{equation}
Using this ordering for pressure, we obtain the following orders of magnitude for the terms in the fluctuating velocity equation
\begin{equation}
\underbrace{\frac{\partial \uifluc}{\partial t}}_{\displaystyle Su_0} + \underbrace{G_{jk} x_k \frac{\partial \uifluc}{\partial x_j}}_{ \displaystyle Su_0} = -\underbrace{\ujfluc G_{ij}}_{\displaystyle S u_0} - \underbrace{\frac{\partial \uifluc \ujfluc}{\partial x_j}}_{\dfrac{u_0^2}{l_0}} - \underbrace{\frac{1}{\rho} \frac{\partial \pfluc}{\partial x_i}}_{\displaystyle Su_0} + \underbrace{\nu \frac{\partial^2 \uifluc}{\partial x_j \partial x_j}}_{\dfrac{\nu u_0}{l_0^2}}. 
\end{equation}
Again, the linear terms are of much larger magnitudes when
\begin{equation}
\frac{Su_0}{\frac{u_0^2}{l_0}} \gg 1 \to \frac{Sl_0}{u_0} \gg 1,
\end{equation}
The viscous terms can be neglected when
\begin{equation}
\frac{Su_0}{\frac{u_0\nu}{l_0^2}} \gg 1 \to \frac{Sl_0^2}{\nu} \gg 1 \to \frac{Sl_0}{u_0}\frac{u_0l_0}{\nu} \gg 1 \to \frac{Sl_0}{u_0} Re_t \gg 1.
\end{equation}
Thus, one could conclude that the equations for the fluctuating velocities and pressure can be approximated by linear equations when $Sl_0/u_0 \gg 1$, that is, when the time scale of the deformation is much faster than the time scale of the energetic eddies. The linearized equations are shown below
\begin{equation}
\label{ulin}
\frac{\partial \uifluc}{\partial t} + G_{jk}x_k \frac{\partial \uifluc}{\partial x_j} = - \ujfluc G_{ij} -\frac{1}{\rho}\frac{\partial \pfluc}{\partial x_i}
\end{equation}
\begin{equation}
\frac{1}{\rho} \frac{\partial^2 \pfluc}{\partial x_j \partial x_j} = -2G_{ij}\frac{\partial \ujfluc}{\partial x_i}.
\label{plin}
\end{equation}

However, it is important to note the following points:
\begin{itemize}
\item For rapid distortions $S^{-1}$ is a very small number. Thus, the time is of the order of $S^{-1}$ only for the very first initial period. For longer times, scaling $t$ by $S^{-1}$ and still assuming that $\tau$ is of $O(1)$ does not hold.
\item Having $\uifluc$ and $x_i$ to be of the order of $u_o$ and $l_o$ respectively implies that all velocities and lengths are of the order of $u_o$ and $l_o$. However, turbulence is a multi scale problem, and thus some velocities and lengths will scale as $u_\eta$ and $\eta$, respectively. Hence, for RDT to apply to all scales we also need to have $S\eta/u_\eta \gg 1$.
\end{itemize}

%-------------------------------------------------------------------------------
\section{Evolution of a Fourier mode}
%-------------------------------------------------------------------------------

The evolution of a velocity Fourier mode for generic homogeneous deformations is given by equation (\ref{vhat_evol_homo}). Since the governing equations for rapid distortions do not include the non-linear and viscous terms, the evolution equation for the Fourier modes simplifies to
\begin{equation}
\frac{\partial \hat{u}_i}{\partial t} = -M_{ij} \hat{u}_j 
\end{equation}
Since the evolution equation for $\hat{u}_i$ is linear, we can assume solutions of the form
\begin{equation}
\hat{u}_i = H_{ik} \hat{u}_k^{(0)}
\end{equation}
where $\hat{u}_i^{(0)} = \hat{u}_i(0,\kvec)$, and $H_{ij} = H_{ij}(t,\kvec)$ is a real non-random function that satisfies
\begin{equation}
\frac{\partial H_{ik}}{\partial t} = - M_{ij} H_{jk}
\end{equation}
with initial condition $H_{ik} = \delta_{ik}$ for $t=0$.

%-------------------------------------------------------------------------------
\section{Evolution of the spectrum}
%-------------------------------------------------------------------------------

The evolution equation for the velocity-spectrum tensor for generic homogeneous deformations is given by equation (\ref{E_evol_homo}). Again, neglecting the non-linear and viscous terms we obtain
\begin{equation}
\frac{\partial E_{ij}}{\partial t} = -M_{ik} E_{kj} - M_{jk} E_{ik},
\end{equation}
which, after plugging in for $M_{ij}$, can be re-expressed as follows
\begin{equation}
\label{evol_phi}
\frac{\partial \est_{ij}}{\partial t} = -G_{ik} \est_{kj} - G_{jk}\est_{ik} + 2 G_{pk} \left ( \frac{\hat{\kappa}_i \hat{\kappa}_p}{\hat{\kappa}^2} \est_{kj} + \frac{\hat{\kappa}_j \hat{\kappa}_p}{\hat{\kappa}^2} \est_{ik} \right ).
\end{equation}

We could, however, compute the velocity-spectrum tensor using the transfer functions $H_{ij}$, due to the linearity of the governing equations. Using equation (\ref{fromu_toE_cont}) we obtain
\begin{align}
E_{ij}(t,\kvec') \boldsymbol{\delta}(\kvec' - \kvec) & = \langle \hat{v}^*_i(t,\kvec) \hat{v}_j(t,\kvec') \rangle \nonumber \\
& = \langle H_{ik}(t,\kvec) \hat{v}_k^*(0,\kvec) H_{jp}(t,\kvec') \hat{v}_p(0,\kvec') \rangle \nonumber \\
& = H_{ik}(t,\kvec) H_{jp}(t,\kvec') \langle \hat{v}_k^*(0,\kvec) \hat{v}_p(0,\kvec') \rangle \nonumber \\
& = H_{ik}(t,\kvec) H_{jp}(t,\kvec') E_{kp}(0,\kvec') \boldsymbol{\delta}(\kvec' - \kvec).
\end{align}
Integrating both sides over $\kvec'$ we obtain
\begin{equation}
E_{ij} = H_{ik} H_{jp} E_{kp}^{(0)},
\end{equation}
where $E_{ij}^{(0)} = E_{ij}(0,\kvec)$.

The Reynolds stresses can be obtained from an alternate velocity-spectrum tensor. Define this tensor as $\tilde{\est}_{ij} = \tilde{\est}_{ij}(t,\chivec)$, such that
\begin{equation}
\est_{ij} = \left [ \tilde{\est}_{ij} \right ]_{\chivec = \hat{\kvec}}.
\end{equation}
We thus obtain the following
\begin{align}
\frac{\partial{\est_{ij}}}{\partial t} &= \left [ \frac{ \partial \tilde{\est}_{ij} }{\partial t} \right ]_{\chivec = \hat{\kvec}} + \left [ \frac{\partial \tilde{\est}_{ij} }{\partial \chi_k} \right ]_{\chivec = \hat{\kvec} } \frac{\partial \hat{\kappa}_k }{\partial t} \nonumber \\
& = \left [ \frac{\partial \tilde{\est}_{ij} }{\partial t} \right ]_{\chivec = \hat{\kvec} } - \left [ \frac{\partial \tilde{\est}_{ij} }{\partial \chi_k} \right ]_{\chivec = \hat{\kvec} } G_{rk} \hat{\kappa}_r \nonumber \\
& = \left [ \frac{\partial \tilde{\est}_{ij}}{\partial t} - \frac{\partial \tilde{\est}_{ij}}{\partial \chi_k} G_{rk}\chi_r \right ]_{\chivec = \hat{\kvec} }.
\end{align}
Using the above in equation (\ref{evol_phi}) we obtain the following evolution equation for $\tilde{\est}_{ij}$
\begin{equation}
\frac{\partial \tilde{\est}_{ij}}{\partial t} - \frac{\partial \tilde{\est}_{ij}}{\partial \chi_k} G_{rk}\chi_r =
-G_{ik} \tilde{\est}_{kj} - G_{jk}\tilde{\est}_{ik} + 2 G_{pk} \left ( \frac{\chi_i \chi_p}{\chi^2} \tilde{\est}_{kj} + \frac{\chi_j \chi_p}{\chi^2} \tilde{\est}_{ik} \right ).
\end{equation}
Jacobi's formula allows us to compute the evolution of the determinant of $A_{ij}$. Thus
\begin{align}
\frac{d}{dt} \det(A_{ij}) & = \det(A_{ij}) \text{tr}\left( A^{-1}_{ik} \frac{d A_{kj}}{dt} \right ) \nonumber \\
& = \det(A_{ij}) \text{tr} ( -A^{-1}_{ik} G_{qj} A_{kq} ) \nonumber \\
& = -\det(A_{ij}) \text{tr} (G_{qj} \delta_{iq}) \nonumber \\
& = -\det(A_{ij}) G_{kk}.
\end{align}
Since $\det(A_{ij}) = 1$ for $t = 0$, the determinant of $A_{ij}$ remains equal to one for all times $t$ when the flow is incompressible. Thus the Reynolds stresses can be computed from $\tilde{\est}_{ij}$ as well, since using integration by substitution one can show that
\begin{align}
\rs_{ij} &= \int \est_{ij} d\kvec \nonumber \\
&= \int \left [ \tilde{\est}_{ij} \right ]_{\chivec = \hat{\kvec} } d\kvec \nonumber \\
&= \int \left [ \tilde{\est}_{ij} \right ]_{\chivec = \hat{\kvec}} | \det(A_{pq}) | d\kvec \nonumber \\
&= \int \left [ \tilde{\est}_{ij} \right ]_{\chivec = \hat{\kvec}} \left | \det \left (\frac{\partial \hat{\kappa}_q}{\partial \kappa_p} \right ) \right | d\kvec \nonumber \\
&= \int \tilde{\est}_{ij} d\chivec.
\end{align}

%Consider the initial velocity field $u_{0i}(\xvec) = \uifluc(\xvec,0)$, which is decomposed using the inverse Fourier transform as follows,
%\begin{equation}
%u_{0i}(\xvec) = \int \hat{u}_{0i}(\kvec) e^{i\kvec \cdot \xvec} d\kvec.
%\label{uinit}
%\end{equation}
%The changes in $\uifluc(\xvec,t)$ with time can be determined by a non-random transfer function $Q_{ij}(\kvec,\xvec,t)$ as shown below,
%\begin{equation}
%\uifluc(\xvec,t) = \int \hat{u}_{0j}(\kvec) Q_{ij}(\kvec,\xvec,t) d\kvec
%\end{equation}
%where at $t=0$, $Q_{ij}(\kvec,\xvec,0)=\delta_{ij}e^{i\kvec \cdot \xvec}$, but for $t>0$, $Q_{ij}$ is determined by the dynamical equations.
%
%With, 
%\begin{equation}\uifluc(\xvec,t) = \int  \hat{u}_{0p}^*(\kvec) Q_{ip}^*(\kvec,\xvec,t) d\kvec \end{equation}
%and,
%\begin{equation}\ujfluc(\xvec',t) = \int  \hat{u}_{0q}(\kvec') Q_{jq}(\kvec',\xvec',t) d\kvec'\end{equation}
%the two-point velocity correlation can be expressed as follows,
%\begin{align}
%R_{ij}(\xvec,\xvec',t) & = \langle \uifluc(\xvec,t)\ujfluc(\xvec',t)\rangle\\
%& = \int \int Q_{ip}^*(\kvec,\xvec,t)Q_{jq}(\kvec',\xvec',t) \langle \hat{u}_{0p}^*(\kvec) \hat{u}_{0q}(\kvec')\rangle d\kvec d\kvec'.
%\end{align}
%Using the relationship $\langle \hat{u}_i^*(\kvec,t)\hat{u}_j(\kvec',t) \rangle = \delta(\kvec' - \kvec)\Phi_{ij}(\kvec',t)$, we rewrite the above as follows,
%\begin{equation}
%R_{ij}(\xvec,\xvec',t) = \int Q_{ip}^*(\kvec,\xvec,t)Q_{jq}(\kvec,\xvec',t) \Phi_{0pq}(\kvec) d\kvec
%\label{rij}
%\end{equation}
%
%For homogeneous turbulence in a ``suitable'' reference frame, the transfer functions can be expressed as follows,
%\begin{equation}
%Q_{ij}(\kvec,\xvec,t) = A_{ij}(\chivec,t)e^{i\chivec \cdot \xvec} 
%\label{qij}
%\end{equation}
%where $\chivec = \chivec(\kvec,t)$ is a function associated with a locally deformed wavenumber. It is defined by the following equation,
%\begin{equation}
%\label{evoChi}
%\frac{\partial \chi_i}{\partial t} + \chi_l \frac{\partial \ulavg}{\partial x_i} = 0
%\end{equation}
%with initial condition $\chivec(\kvec,0) = \kvec$.
%
%Using the expression for $Q_{ij}(\kvec,\xvec,t)$ in the definition of $u_{i}(\xvec,t)$ leads to,
%\begin{equation}
%\uifluc(\xvec,t) = \int \hat{u}_{0j}(\kvec) A_{ij}(\chivec,t)e^{i\chivec \cdot \xvec} d\kvec.
%\label{udec}
%\end{equation}
%Thus this decomposition can be thought of as an inverse Fourier transform where not only the Fourier coefficients $\hat{u}_{0j}A_{ij}$ evolve in time as usual, but one in which the wavevectors $\chivec$ evolve in time as well (see Pope, Ch. 11).
%
%Using (\ref{qij}) in the definition of $R_{ij}$ shown in (\ref{rij}), we obtain
%\begin{equation}
%R_{ij}(\xvec,\xvec',t) = \int A_{ip}^*(\chivec,t)e^{-i\chivec \cdot \xvec} A_{jq}(\chivec,t)e^{i\chivec \cdot \xvec'}  \Phi_{0pq}(\kvec) d\kvec.
%\end{equation}
%With $\xvec'=\xvec+\rvec$,
%\begin{equation}
%R_{ij}(\xvec,\xvec',t) = \int A_{ip}^*(\chivec,t) A_{jq}(\chivec,t) \Phi_{0pq}(\kvec) e^{i\chivec \cdot \rvec} d\kvec.
%\label{rdt_Rij}
%\end{equation}
%We can thus define $\tilde\Phi_{ij}(\kvec,t,\chivec)=A_{ip}^*(\chivec,t) A_{jq}(\chivec,t) \Phi_{0pq}(\kvec)$ as a ``localized'' energy spectrum, so that, 
%\begin{equation}
%R_{ij}(\xvec,\xvec',t) = \int \tilde\Phi_{ij}(\kvec,t,\chivec) e^{i\chivec \cdot \rvec} d\kvec.
%\label{locphi1}
%\end{equation}
%and, 
%\begin{equation}
%\rs = \int  \tilde\Phi_{ij}(\kvec,t,\chivec) d\kvec. 
%\label{locphi2}
%\end{equation}
%
%An analogous analysis can be carried out when the velocity is decomposed in terms of a Fourier series instead of the Fourier transform. For such case the equivalent of equation (\ref{rdt_Rij}) follows,
%\begin{equation}
%R_{ij}(\xvec,\xvec',t) = \sum_{\kvec_d} A_{ip}^*(\chivec_d,t) A_{jq}(\chivec_d,t) \hat{R}_{0pq}(\kvec_d) e^{i\chivec_d \cdot \rvec}.
%\end{equation}
%where $\kvec_d$ is now a discrete wavenumber, and $\chivec_d = \chivec_d(\kvec_d,t)$ is also discrete.
%
%Again, we can define $\tilde{R}(\kvec_d,t,\chivec_d) = A_{ip}^*(\chivec_d,t) A_{jq}(\chivec_d,t) \hat{R}_{0pq}(\kvec_d)$ as the ``localized'' Fourier coefficient of $R_{ij}$, so that, in analogy to equations (\ref{locphi1}) and (\ref{locphi2}), we have,
%\begin{equation}
%R_{ij}(\xvec,\xvec',t) = \sum_{\kvec_d} \tilde{R}(\kvec_d,t,\chivec_d) e^{i\chivec_d \cdot \rvec},
%\end{equation}
%and,
%\begin{equation}
%\rs = \sum_{\kvec_d} \tilde{R}(\kvec_d,t,\chivec_d).
%\end{equation}
%The relationship that connects the velocity spectrum tensor and the Fourier coefficients of $R_{ij}$ for the traditional Fourier decompositions also applies in this case, that is,
%\begin{equation}
%\tilde\Phi_{ij}(\kvec,t,\chi) = \sum_{\kvec_d} \delta(\kvec - \kvec_d)\tilde{R}(\kvec_d,t,\chivec_d),
%\label{connect}
%\end{equation}
%which can be confirmed by plugging in (\ref{connect}) into (\ref{locphi1}). (This is equation 11.97 in Pope.)
%
%\subsection{The pressure field}
%%--------------------------------------------------------------------------
%Similarly as for the initial velocity field, the initial pressure field $p_{0}(\xvec) = \pfluc(\xvec,0)$ can be decomposed using the inverse Fourier transform as follows,
%\begin{equation}
%p_0(\xvec) = \int \hat{p}_0(\kvec) e^{i\kvec \cdot \xvec} d\kvec.
%\label{pinit}
%\end{equation}
%Plugging-in this decomposition and that shown in (\ref{uinit}) in the RDT equation for pressure (\ref{plin}) allows the initial Fourier coefficient of pressure to be expressed as,
% \begin{equation}
% \hat{p}_0(\kvec) = \frac{1}{\kappa^2}2i\kappa_l\frac{\partial \ulavg}{\partial x_n} \hat{u}_{0n}(\kvec),
% \end{equation} and thus the initial pressure field becomes,
%\begin{equation}
%p_0(\xvec) = \int \hat{u}_{0n}(\kvec) \frac{1}{\kappa^2}2i\kappa_l\frac{\partial \ulavg}{\partial x_n} e^{i\kvec \cdot \xvec} d\kvec.
%\end{equation}
%
%Again, similarly to the velocity field, changes in $p(\xvec,t)$ with time can be determined by a non-random transfer function, 
%\begin{equation}
%\Pi_n(\kvec,\xvec,t)=\hat{p}_n(\chivec,t)e^{i\chivec\cdot\xvec}
%\end{equation}
%which leads to, 
%\begin{equation}
%\pfluc(\xvec,t) = \int \hat{u}_{0n}(\kvec) \Pi_{n}(\kvec,\xvec,t) d\kvec = \int \hat{u}_{0n}(\kvec) \hat{p}_n(\chivec,t)e^{i\chivec\cdot\xvec} d\kvec
%\label{pdec}
%\end{equation}
%At $t=0$, 
%\begin{equation}
%\Pi_{n}(\kvec,\xvec,0) = \frac{1}{\kappa^2}2i\kappa_l\frac{\partial \ulavg}{\partial x_n} e^{i\kvec \cdot \xvec},
%\end{equation}
%or
%\begin{equation}
%\hat{p}_{n}(\kvec,0) = \frac{1}{\kappa^2}2i\kappa_l\frac{\partial \ulavg}{\partial x_n}.
%\end{equation}
%
%\subsection{Evolution of the transfer functions}
%%--------------------------------------------------------------------------
%Using the spectral decompositions for velocity and pressure (Eqs. (\ref{udec}) and (\ref{pdec})) in the linearized equation (\ref{ulin}) leads to the following,
%\begin{equation}
%\frac{\partial}{\partial t} [ \hat{u}_{0n}(\kvec)A_{in}(\chivec,t) ] = -\hat{u}_{0n}(\kvec)A_{jn}(\chivec,t)\frac{\partial \uiavg}{\partial x_j} - i\chi_i\hat{u}_{0n}(\kvec)\hat{p}_n(\chivec,t).
%\end{equation}
%Similarly, using those decompositions in the linearized equation (\ref{plin}) leads to the following,
%\begin{equation}
%\hat{u}_{0n}(\kvec)\hat{p}_n(\chivec,t) = \frac{1}{\chi^2}2i\chi_l\frac{\partial \ulavg}{\partial x_j} \hat{u}_{0n}(\kvec)A_{jn}(\chivec,t)
%\end{equation}
%This shows that $\hat{p}_n(\chivec,t)$ satisfies the initial condition. Combining the above two relations leads to the evolution equation for the velocity transfer function, 
%\begin{equation}
%\frac{\partial}{\partial t} [ \hat{u}_{0n}(\kvec)A_{in}(\chivec,t)] = -\hat{u}_{0n}(\kvec)A_{jn}(\chivec,t)\frac{\partial \ulavg}{\partial x_j}\left (\delta_{il} - 2\frac{\chi_i\chi_l}{\chi^2} \right )
%\label{evoA}
%\end{equation}
%One can thus solve for the time-evolving Fourier coefficients $\hat{u}_{0n}A_{in}$, or for the transfer function $A_{in}$ after having factored out the Fourier coefficient $\hat{u}_{0n}$. It is important to note that the Fourier coefficients depend on $t$ both directly and through $\chivec$. Thus, the time derivative can be expanded as
%\begin{equation}
%\frac{\partial}{\partial t} [ \hat{u}_{0n}(\kvec) A_{in}(\chivec,t) ] =
%\left \{ \frac{\partial}{\partial t} [ \hat{u}_{0n}(\kvec)A_{in}(\xivec,t) ] \right \}_{\xivec=\chivec(\kvec,t)} + \frac{\partial \chi_j(\kvec,t)}{\partial t} \left \{ \frac{\partial}{\partial \xi_j} [ \hat{u}_{0n}(\kvec)A_{in}(\xivec,t) ] \right \}_{\xivec=\chivec(\kvec,t)}
%\end{equation}
%where $\xivec$, unlike $\chivec$, is an independent variable. Using equation (\ref{evoChi}) this is rewritten as
%\begin{equation}
%\frac{\partial}{\partial t} [ \hat{u}_{0n}(\kvec) A_{in}(\chivec,t) ] =
%\left \{ \frac{\partial}{\partial t} [\hat{u}_{0n}(\kvec)A_{in}(\xivec,t)] - \xi_l \frac{\partial \ulavg}{\partial x_j} \frac{\partial}{\partial \xi_j} [ \hat{u}_{0n}(\kvec)A_{in}(\xivec,t)] \right \}_{\xivec=\chivec(\kvec,t)}.
%\end{equation}
%Thus the analogous to equation (\ref{evoA}) but in this new reference frame follows
%\begin{equation}
% \frac{\partial}{\partial t} [ \hat{u}_{0n}(\kvec)A_{in}(\xivec,t) ] - \xi_l \frac{\partial \ulavg}{\partial x_j} \frac{\partial}{\partial \xi_j} [ \hat{u}_{0n}(\kvec)A_{in}(\xivec,t) ] = -\hat{u}_{0n}(\kvec)A_{jn}(\xivec,t) \frac{\partial \ulavg}{\partial x_j}\left (\delta_{il} - 2\frac{\xi_i\xi_l}{\xi^2} \right ).
%\end{equation}
%
%Taking the derivative of the ``localized'' energy spectrum $\tilde\Phi_{ij} = \tilde\Phi_{ij}(\kvec,t,\chivec)$, and using the evolution equation for $A_{ij}(\chivec,t)$ (\ref{evoA}), one obtains the evolution equation
%\begin{equation}
%\frac{\partial \tilde\Phi_{ij}}{\partial t} = -\frac{\partial \uiavg}{\partial x_k}\tilde\Phi_{kj} - \frac{\partial \ujavg}{\partial x_k}\tilde\Phi_{ik} + 2\frac{\partial \ulavg}{\partial x_k} \left ( \frac{\chi_i\chi_l}{\chi^2} \tilde\Phi_{kj} + \frac{\chi_j\chi_l}{\chi^2} \tilde\Phi_{ik} \right ).
%\end{equation}
%The evolution equation for $\tilde\Phi_{ij} = \tilde\Phi_{ij}(\kvec,t,\xivec)$ then takes the following form,
%\begin{equation}
%\frac{\partial \tilde\Phi_{ij}}{\partial t} - \xi_l \frac{\partial \ulavg}{\partial x_m} \frac{\partial \tilde\Phi_{ij}}{\partial \xi_m} = -\frac{\partial \uiavg}{\partial x_k} \tilde\Phi_{kj} - \frac{\partial \ujavg}{\partial x_k} \tilde\Phi_{ik} + 2\frac{\partial \ulavg}{\partial x_k} \left ( \frac{\xi_i\xi_l}{\xi^2} \tilde\Phi_{kj} + \frac{\xi_j\xi_l}{\xi^2} \tilde\Phi_{ik} \right ).
%\end{equation}

%%%%%%%%%%%%%%%%%%%%%%%%%%%%%%%%%%%%%%%%%%%%%%%%%%%%%%%%%%%%%%%%%%%%%%%%%
%                                                                       %
%                                                                       %
%                                                                       %
%                                                                       %
\part{Inhomogeneous turbulence}
%                                                                       %
%                                                                       %
%                                                                       %
%                                                                       %
%%%%%%%%%%%%%%%%%%%%%%%%%%%%%%%%%%%%%%%%%%%%%%%%%%%%%%%%%%%%%%%%%%%%%%%%%

%###############################################################################
%
\chapter{Free shear flows}
%
%###############################################################################

%###############################################################################
%
\chapter{Wall-bounded flows}
%
%###############################################################################

%-------------------------------------------------------------------------------
\section{Fully-developed flows}
%-------------------------------------------------------------------------------

%--------------------------------------------
\subsection{Channels}
%--------------------------------------------

\subsubsection{Balance of forces}
%--------------------------------------------

For fully developed channel flow, with channel half height $\delta$, the streamwise and wall normal momentum equations simplify to the following
\begin{equation}
0 = -\frac{1}{\rho}\frac{\partial \pavg}{\partial x} + \nu \frac{d^2 \langle U \rangle}{d y^2} - \frac{d \langle uv \rangle}{d y}
\end{equation}
\begin{equation}
0 = -\frac{1}{\rho}\frac{\partial \pavg}{\partial y} - \frac{d \langle vv \rangle}{d y}
\end{equation}
Differentiating the wall normal equation with respect to $x$ leads to
\begin{equation}
\frac{\partial}{\partial y} \frac{\partial \pavg}{\partial x} = 0,
\end{equation}
that is, the pressure drop is independent of $y$. Integrating the streamwise equation with respect to y, from zero to some location y, leads to
\begin{align}
\rho \nu \frac{d \langle U \rangle}{d y} - \rho\langle uv \rangle &= \left . \rho \nu \frac{d \langle U \rangle}{d y} \right |_{wall} + \frac{\partial \pavg}{\partial x} y \nonumber \\
& = \tau_{wall} + \frac{\partial \pavg}{\partial x} y.
\end{align}
Evaluating at $y = \delta$ shows that
\begin{equation}
\frac{\partial \pavg}{\partial x} = -\frac{\tau_{wall}}{\delta}.
\end{equation}
Thus, the integrated momentum equation becomes
\begin{equation}
\rho \nu \frac{d \langle U \rangle}{d y} - \rho\langle uv \rangle = \tau_{wall} \left ( 1 - \frac{y}{\delta} \right ).
\end{equation}
Using the following wall scales
\begin{equation}
u_\tau = \sqrt{\frac{\tau_{wall}}{\rho}} \qquad \delta_\nu = \frac{\nu}{u_\tau} 
\end{equation}
and the friction Reynolds number $Re_\tau = u_\tau \delta/\nu$ the integrated momentum equation can be re-expressed in wall units as follows
\begin{equation}
\frac{du^+}{dy^+} - \langle uv \rangle^+ = 1 - \frac{y^+}{Re_\tau}.
\end{equation}

\subsubsection{Channel laws}
%--------------------------------------------

We choose the set of physical variables associated with the channel to consist of the following: $\rho$, $\nu$, $\delta$, $u_{\tau}$ (corresponds to the pressure gradient), $y$, and $d \langle U \rangle/dy$. Buckingham $\Pi$ theorem tells us that there are three non-dimensional groups formed from the above variables, and they are related by a single equation. This is shown below
\begin{equation}
\label{dudy_channel}
\frac{d\langle U \rangle}{dy} = \frac{u_\tau}{y} \Phi \left ( \frac{y}{\delta_\nu}, \frac{y}{\delta} \right ).
\end{equation}
It is postulated that very close to the wall one encounters the inner layer ($y/\delta < 0.1$) throughout which the velocity is independent of channel scales such as $\delta$. Thus, equation (\ref{dudy_channel}) becomes
\begin{equation}
\label{inner_layer}
\frac{d\langle U \rangle}{dy} = \frac{u_\tau}{y} \Phi_I \left ( \frac{y}{\delta_\nu} \right ).
\end{equation}
It is also postulated that far from the wall one encounters the outer layer ($y/\delta_\nu > 50$) throughout which the velocity is independent of the wall scales such as $\delta_\nu$. Thus equation (\ref{dudy_channel}) becomes 
\begin{equation}
\label{outer_layer}
\frac{d\langle U \rangle}{dy} = \frac{u_\tau}{y} \Phi_O \left (\frac{y}{\delta} \right ).
\end{equation}

Rewriting the inner layer in wall units one obtains
\begin{equation}
\frac{d u^+}{dy^+} = \frac{1}{y^+} \Phi_I ( y^+ ).
\end{equation}
and thus
\begin{equation}
\label{law_wall}
u^+ = \int_0^{y^+} \frac{1}{y'} \Phi_I (y') dy'.
\end{equation}
The integral above is referred to as the \textbf{law of the wall}, and experimental data shows this integral to be universal. At the wall $u^+=0$ and $du^+/dy^+ = 1$ (since $\tau_{wall} = \rho \nu d\langle U \rangle/dy$). Thus a Taylor series expansion about $y^+=0$ shows that
\begin{equation}
u^+ = y^+ + H.O.T.
\end{equation}
Indeed, simulations show $u^+=y^+$ up to around $y^+ < 5$, and this region is referred to as the viscous sublayer.

The overlap between the inner layer and the outer layer is described by the \textbf{log law}, though the limits of the log law may expand beyond those of the inner and outer layer (i.e $y/\delta_\nu > 30$, $y/\delta < 0.3$). Within this layer the effect of both wall and channel scales is negligible and hence $\Phi_I$ becomes a constant equal to $\kappa^{-1}$. Thus
\begin{equation}
\frac{du^+}{dy^+} = \frac{1}{\kappa y^+} .
\end{equation}
Evaluating the law of the wall we obtain
\begin{equation}
u^+ = \int_0^{a} \frac{1}{y'} \Phi_I (y') dy' + \int_a^{y^+} \frac{1}{\kappa y'} dy' = C + \frac{1}{\kappa} [ \ln(y^+) - \ln(a)] = \frac{1}{\kappa} \ln(y^+) + B,
\end{equation}
where $a < y^+$ is a location within the log law and $C$ and $B$ are constants.

Integrating equation (\ref{outer_layer}) between $y$ and $\delta$ leads to the \textbf{velocity-defect law}. That is
\begin{equation}
\frac{U_0 - \langle U \rangle}{u_\tau} = \int_y^\delta \frac{1}{y'} \Phi_O(\frac{y'}{\delta}) dy' = \int_{y/\delta}^1 \frac{1}{y''} \Phi_O(y'') dy''.
\end{equation}
The integral above, as opposed to the law of the wall, is not expected to be universal. Within the log law $\Phi_O$ becomes a constant equal to $\kappa^{-1}$. Hence evaluating the velocity-defect law leads to
\begin{equation}
\frac{U_0 - \langle U \rangle}{u_\tau} = \int_{y/\delta}^b \frac{1}{\kappa y''} dy'' + \int_b^1 \frac{1}{y''}\Phi_O(y'')dy'' = \frac{1}{\kappa} [\ln(b) - \ln(y/\delta)] + C_1 = -\frac{1}{\kappa} \ln(y/\delta) + B_1
\end{equation}
where $b > y/\delta$ is a location within the log law and $C_1$ and $B_1$ are constants.

\begin{table}
\centering
\begin{tabular}{ | c | c | c |}
  \hline
  Layer & Limits & Law \\
  \hline
  Inner & $y/\delta < 0.1$ & Law of the wall \\
  \hline
  Overlap & $y/\delta < 0.1$, $y^+ > 50$ & Log law (expands to $y/\delta < 0.3$, $y^+>30$) \\
  \hline
  Outer & $y^+ > 50$ & Velocity-defect law\\
  \hline
\end{tabular}
\caption{Summary of wall layers and corresponding laws.}
\end{table}

%--------------------------------------------
\subsection{Pipes}
%--------------------------------------------

For fully developed pipe flow the streamwise and radial momentum equations simplify to the following
\begin{equation}
0 =  -\frac{1}{\rho} \frac{\partial \pavg}{\partial z} + \nu \left ( \frac{\partial^2 \langle u_z \rangle}{\partial r^2} + \frac{1}{r} \frac{\partial \langle u_z \rangle}{\partial r} \right ) - \frac{\partial \langle u_z u_r \rangle}{\partial r} - \frac{\langle u_z u_r \rangle}{r}.
\end{equation}
\begin{equation}
0 = -\frac{1}{\rho} \frac{\partial \pavg}{\partial r} - \frac{\partial \langle u_ru_r \rangle}{\partial r} - \frac{\langle u_ru_r \rangle}{r} + \frac{\langle u_\theta u_\theta \rangle}{r}
\end{equation} 
Differentiating the radial equation with respect to $z$ leads to
\begin{equation}
\frac{\partial}{\partial r} \frac{\partial \pavg}{\partial z} = 0,
\end{equation}
that is, the pressure drop is independent of $r$. 
We now rewrite the streamwise equation as follows
\begin{equation}
0 =  -\frac{1}{\rho} \frac{\partial \pavg}{\partial z} + \nu \frac{1}{r} \frac{\partial}{\partial r} \left (r \frac{\partial \langle u_z \rangle}{\partial r} \right ) - \frac{1}{r} \frac{\partial r \langle u_z u_r \rangle}{\partial r}.
\end{equation}
Multiplying the above by r, and then integrating with respect to $r$, from some location $r$ to the radius of the pipe $R$ leads to
\begin{equation}
\rho \nu R \frac{\partial \langle u_z \rangle}{\partial r} - \left ( \rho \nu r \frac{\partial \langle u_z \rangle}{\partial r} - r \rho \langle u_z u_r \rangle \right )= \frac{\partial \pavg}{\partial z} \frac{R^2}{2} - \frac{\partial \pavg}{\partial z} \frac{r^2}{2}
\end{equation}
which is rewritten as
\begin{equation}
-R \tau_{wall} - \left ( \rho \nu r \frac{\partial \langle u_z \rangle}{\partial r} - r \rho \langle u_z u_r \rangle \right )= \frac{\partial \pavg}{\partial z} \frac{R^2}{2} - \frac{\partial \pavg}{\partial z} \frac{r^2}{2}.
\end{equation}
Evaluation at $r = 0$ shows that
\begin{equation}
\frac{\partial \pavg}{\partial z} = -\frac{2\tau_{wall}}{R}
\end{equation}
Thus, the integrated momentum equation becomes
\begin{equation}
\rho \nu \frac{\partial \langle u_z \rangle}{\partial r} - \rho \langle u_z u_r \rangle = -\tau_{wall} \frac{r}{R}
\end{equation}
which is of equivalent form to that of the channel if one uses the substitution $r = R - r'$ since then each of the first and second terms of the left-hand side would change signs. 

%--------------------------------------------
\subsection{Ducts}
%--------------------------------------------

%%%%%%%%%%%%%%%%%%%%%%%%%%%%%%%%%%%%%%%%%%%%%%%%%%%%%%%%%%%%%%%%%%%%%%%%%
%                                                                       %
%                                                                       %
%                                                                       %
%                                                                       %
\part{Modeling and Simulation}														             %
%                                                                       %
%                                                                       %
%                                                                       %
%                                                                       %
%%%%%%%%%%%%%%%%%%%%%%%%%%%%%%%%%%%%%%%%%%%%%%%%%%%%%%%%%%%%%%%%%%%%%%%%%

%###############################################################################
%
\chapter{RANS Modeling}
%
%###############################################################################
This chapter focuses on models for the Reynolds stresses themselves, without use of the Boussinesq approximation such as in Eddy-Viscosity Models (EVMs). Nonetheless a few comments are first mentioned regarding EVMs. The $k-\epsilon$ model, which was quite popular some long time ago, has the following characteristics:
\begin{itemize}
\item One of the first two-equation models that showed moderate success across in a variety of engineering problems.
\item Lack of sensitivity to adverse pressure gradients, where the model over predicts the shear stress and thus delays or completely prevents separation (Menter 1993).
\item Numerical stiffness of the low-Reynolds number variants (Menter 1993).
\item There are no natural or direct boundary conditions for the $\epsilon$ equation, and the boundary conditions that have actually been proposed introduce additional non-linearities that can upset the numerical procedure (Menter 1993). 
\item The exact equation for the dissipation, evaluated at the wall, includes unclosed higher order correlations such as the dissipation of dissipation (Speziale 1992).  
\end{itemize}
The $k-\omega$ model, another popular two-equation closure, has the following characteristics:
\begin{itemize}
\item Significantly better performance under adverse pressure gradient conditions than the $k-\epsilon$ model.
\item Increased computational robustness than $k-\epsilon$ for integration up to a solid boundary (Speziale 1992).
\item The leading order terms of the exact $\omega$ equations at the wall are of closed form (Speziale 1992). 
\item  Solutions of TKE are asymptotically inconsistent near a solid wall.
\item There is a strong sensitivity to free stream boundary values for $w$ specified outside the shear layer (Menter 1993).
\end{itemize}

%-------------------------------------------------------------------------------
\section{The Theory of Matrix Polynomials Applied to Reynolds-Stress Modleing}
%-------------------------------------------------------------------------------

\label{sec:polynomial_expansions}
Assume $\Pi_{ij}$ is a model for the physical quantity $\Sigma_{ij}$. It is quite common for differential and algebraic Reynolds-stress models to define $\Pi_{ij}$ as an algebraic functional relationship whose inputs are tensors $\bold{C}^{(1)},...,\bold{C}^{(n)}$ and scalars $s^{(1)},...,s^{(m)}$. Thus,
\begin{equation}
\Pi_{ij} = F_{ij}(s^{(1)}, ..., s^{(m)},\bold{C}^{(1)}, ..., \bold{C}^{(n)}).
\end{equation}
An assumption that can be used to narrow down the functional form of $F_{ij}$ is to have both the exact physical tensor and its model satisfy the same transformation properties. For example, if $\Sigma_{ij}$ is objective we could enforce $\Pi_{ij}$ to be objective as well. An objective tensor is one that satisfies $\tilde{\Sigma}_{ij} = O_{ip}O_{jq}\Sigma_{pq}$ under Euclidean transformations, where $\tilde{\Sigma}_{ij}$ is such tensor in the transformed reference frame and $O_{ij}$ is a time-dependent rotation matrix.

Enforcing such assumption would require $F_{ij}$ to consist of a sum of terms as follows
\begin{equation}
\label{poly_expan}
F_{ij} = \sum_{k} f^{(k)}\mathcal{F}_{ij}^{(k)}.
\end{equation}
The functions $f^{(k)}$ depend on the scalars $s^{(1)}, ...,s^{(m)}$ only, and each $\mathcal{F}_{ij}^{(k)}$ is of the form
\begin{equation}
\label{tensor_product}
\mathcal{F}_{ij}^{(k)} = c^{(a_k)}_{ip}c^{(b_k)}_{pq}c^{(c_k)}_{qr}...
\end{equation}
where $a_k,b_k,c_k,...$ can each take any value between one and $n$. Additionally, the scalars $s^{(1)},...,s^{(m)}$ and the tensors $\bold{C}^{(1)},...,\bold{C}^{(n)}$ would need to be objective, where an objective scalar is one that satisfies $\tilde{s} = s$. If all of these requirements are satisfied, then we can show that
\begin{align}
\tilde{\Pi}_{ij} & = F_{ij}(\tilde{s}^{(1)}, ..., \tilde{s}^{(m)},\tilde{\bold{C}}^{(1)}, ..., \tilde{\bold{C}}^{(n)}) \nonumber \\
& = F_{ij}(s^{(1)}, ..., s^{(m)},\bold{Q} \bold{C}^{(1)}\bold{Q}^T, ..., \bold{Q} \bold{C}^{(n)}\bold{Q}^T) \nonumber \\
& = Q_{ip}Q_{jq} F_{pq}(s^{(1)}, ..., s^{(m)}, \bold{C}^{(1)}, ..., \bold{C}^{(n)}) \nonumber \\
&= Q_{ip}Q_{jq} \Pi_{pq}, 
\end{align}
which shows $\Pi_{ij}$ is indeed objective. It is relevant to note that $\Pi_{ij} = F_{ji}$ is also a valid form.

%--------------------------------------------
\subsection{Models as a function of one tensor}
%--------------------------------------------

%--------------------------------------------
\subsection{Models as a function of two tensors}
%--------------------------------------------
We now focus on a model for the non-dimensional anisotropic Reynolds stresses $\nars_{ij}$. We further assume that $\nars_{ij}$ depends only on the non-dimensionalized anisotropic rate-of-strain tensor $\Stau_{ij}$, and the non-dimensionalized rate-of-rotation tensor $\Wtau_{ij}$. 

As shown in Spencer \& Rivlin (The theory of matrix polynomials and its applications to the mechanics of isotropic continua, 1959) one can use the Cayley-Hamilton theorem to show that any product of the form shown in (\ref{tensor_product}) that consists of  the two tensors $\mathbf{\Stau}$ and $\mathbf{\Wtau}$ can be expanded as a linear combination of a finite number of tensors. For this case such tensors are
\begin{gather}
\mathbf{I}, \nonumber \\
\mathbf{\Stau}, \qquad \mathbf{\Wtau}, \nonumber \\
\mathbf{\Stau}^2, \qquad \mathbf{\Wtau}^2, \qquad \mathbf{\Stau} \mathbf{\Wtau}, \qquad \mathbf{\Wtau} \mathbf{\Stau}, \nonumber \\
\mathbf{\Wtau} \mathbf{\Stau}^2, \qquad \mathbf{\Stau}^2 \mathbf{\Wtau}, \qquad \mathbf{\Wtau}^2 \mathbf{\Stau}, \qquad \mathbf{\Stau} \mathbf{\Wtau}^2, \nonumber \\
\mathbf{\Wtau}^2 \mathbf{\Stau}^2, \qquad \mathbf{\Stau}^2 \mathbf{\Wtau}^2, \qquad \mathbf{\Wtau} \mathbf{\Stau} \mathbf{\Wtau}^2, \qquad \mathbf{\Stau} \mathbf{\Wtau} \mathbf{\Stau}^2, \nonumber \\
\mathbf{\Stau} \mathbf{\Wtau}^2 \mathbf{\Stau}^2, \qquad \mathbf{\Wtau} \mathbf{\Stau}^2 \mathbf{\Wtau}^2.
\end{gather}
Thus, the infinite sum for $F_{ij}$ shown in (\ref{poly_expan}) becomes a finite linear combination as follows
\begin{equation}
\label{rs_algebraic_model}
F_{ij} = \sum_{k = 1}^{17} g^{(k)}\mathcal{G}_{ij}^{(k)},
\end{equation}
where the tensors $\mathcal{G}_{ij}^{(k)}$ are shown above.

Since the Reynolds stresses are symmetric, its model must be symmetric as well. The symmetric form is given by $F_{ij} + F_{ji}$. Combining terms in the sums for $F_{ij}$ and $F_{ji}$ that have the same coefficient $g^{(k)}$ we obtain
\begin{equation}
F_{ij} + F_{ji} = \sum_{k=1}^{12} h^{(k)} \mathcal{H}_{ij}^{(k)}
\end{equation}
where each of the $\mathcal{H}_{ij}^{(k)}$'s is shown below.
\begin{gather}
\mathbf{I}, \nonumber \\
\mathbf{\Stau}, \nonumber \\
\mathbf{\Stau}^2, \qquad \mathbf{\Wtau}^2, \qquad \mathbf{\Stau} \mathbf{\Wtau} - \mathbf{\Wtau} \mathbf{\Stau}, \nonumber \\
\mathbf{\Wtau} \mathbf{\Stau}^2 - \mathbf{\Stau}^2 \mathbf{\Wtau}, \qquad \mathbf{\Wtau}^2 \mathbf{\Stau} + \mathbf{\Stau} \mathbf{\Wtau}^2 \nonumber \\
\mathbf{\Wtau}^2 \mathbf{\Stau}^2 + \mathbf{\Stau}^2 \mathbf{\Wtau}^2, \qquad \mathbf{\Wtau} \mathbf{\Stau} \mathbf{\Wtau}^2 - \mathbf{\Wtau}^2 \mathbf{\Stau} \mathbf{\Wtau}, \qquad \mathbf{\Stau} \mathbf{\Wtau} \mathbf{\Stau}^2 - \mathbf{\Stau}^2 \mathbf{\Wtau} \mathbf{\Stau} ,\nonumber \\
\mathbf{\Stau} \mathbf{\Wtau}^2 \mathbf{\Stau}^2 + \mathbf{\Stau}^2 \mathbf{\Wtau}^2 \mathbf{\Stau}, \qquad \mathbf{\Wtau} \mathbf{\Stau}^2 \mathbf{\Wtau}^2 - \mathbf{\Wtau}^2 \mathbf{\Stau}^2 \mathbf{\Wtau}.
\end{gather}
As shown in Spencer \& Rivlin (1959) the sum $\mathbf{\Stau} \mathbf{\Wtau}^2 \mathbf{\Stau}^2 + \mathbf{\Stau}^2 \mathbf{\Wtau}^2 \mathbf{\Stau}$ can be expresses as a linear combination of the other terms, and hence can be dropped from the set above. Since $\nars_{ij}$ is anisotropic, a model for this tensor needs to be anisotropic as well. This is given by 
\begin{equation}
F_{ij} + F_{ji} - \frac{2}{3} F_{kk} \delta_{ij} = \sum_{k=1}^{10} p^{(k)} \hat{\mathcal{P}}_{ij}^{(k)}
\end{equation}
where the deviatoric tensors $\hat{\mathcal{P}}_{ij}^{(k)}$ are shown in table blah. The above sum is thus used as the model for $\nars_{ij}$.

\begin{center}
\begin{tabular} {| Sl | Sl | }
\hline
$\hat{\boldsymbol{\mathcal{P}}}^{(1)} = \mathbf{\Stau}$ & $\hat{\boldsymbol{\mathcal{P}}}^{(6)} = \mathbf{\Wtau}^2 \mathbf{\Stau} + \mathbf{\Stau} \mathbf{\Wtau}^2 - \dfrac{2}{3} \{ \mathbf{\Wtau}^2 \mathbf{\Stau} \} \mathbf{I}$ \\
\hline
$\hat{\boldsymbol{\mathcal{P}}}^{(2)} = \mathbf{\Stau} \mathbf{\Wtau} - \mathbf{\Wtau} \mathbf{\Stau}$ & $\hat{\boldsymbol{\mathcal{P}}}^{(7)} = \mathbf{\Wtau} \mathbf{\Stau} \mathbf{\Wtau}^2 - \mathbf{\Wtau}^2 \mathbf{\Stau} \mathbf{\Wtau}$ \\
\hline
$\hat{\boldsymbol{\mathcal{P}}}^{(3)} = \mathbf{\Stau}^2 - \frac{1}{3} \{\mathbf{\Stau}^2 \} \mathbf{I}  $ & $ \hat{\boldsymbol{\mathcal{P}}}^{(8)} = \mathbf{\Stau} \mathbf{\Wtau} \mathbf{\Stau}^2 - \mathbf{\Stau}^2 \mathbf{\Wtau} \mathbf{\Stau} $ \\
\hline
$ \hat{\boldsymbol{\mathcal{P}}}^{(4)} = \mathbf{\Wtau}^2 - \frac{1}{3} \{\mathbf{\Wtau}^2 \} \mathbf{I} $ & $ \hat{\boldsymbol{\mathcal{P}}}^{(9)} = \mathbf{\Wtau}^2 \mathbf{\Stau}^2 + \mathbf{\Stau}^2 \mathbf{\Wtau}^2 - \frac{2}{3} \{ \mathbf{\Wtau}^2 \mathbf{\Stau}^2 \} \mathbf{I} $ \\
\hline
$ \hat{\boldsymbol{\mathcal{P}}}^{(5)} = \mathbf{\Wtau} \mathbf{\Stau}^2 - \mathbf{\Stau}^2 \mathbf{\Wtau} \qquad$ & $ \hat{\boldsymbol{\mathcal{P}}}^{(10)} = \mathbf{\Wtau} \mathbf{\Stau}^2 \mathbf{\Wtau}^2 - \mathbf{\Wtau}^2 \mathbf{\Stau}^2 \mathbf{\Wtau}$ \\
\hline
\end{tabular}
\end{center}

The ten tensors above constitute the so-called integrity basis for Reynolds stresses modeled as a function of $\Stau_{ij}$ and $\Wtau_{ij}$.

%-------------------------------------------------------------------------------
\section{Differential Reynolds-stress models}
%-------------------------------------------------------------------------------

There are three main components that need to be modeled to obtain closure for the Reynolds-stress differential transport equation. These are the velocity-pressure-gradient tensor $\Pi_{ij}$, the dissipation tensor $\epsilon_{ij}$, and the turbulent transport $\mathcal{T}^{(u)}_{kij}$.

%--------------------------------------------
\subsection{Velocity-pressure-gradient tensor}
%--------------------------------------------

A model for the pressure-rate-of-strain tensor needs to be:
\begin{enumerate}
\item Symmetric,
\item Traceless,
\item Objective,
\item Linear in $G_{ij}$ (the rapid term is linear in $G_{ij}$ and the slow term depends only on fluctuating quantities)
\item Capable of guaranteeing realizability for $\rs$,
\item Linear in $\rs_{ij}$ for the RDT limit.
\end{enumerate}
As described in section (\ref{sec:polynomial_expansions}), if we require a model to satisfy a minimum of tensor properties, then there is a finite number of linearly independent tensors, termed the integrity basis, that can be used to define the model. Thus, a model for the non-dimensional pressure-rate-of-strain tensor $\redi_{ij}/\epsilon$ that satisfies requirements 1through 4 from above, and depends on the non-dimensional variables $\Stau_{ij} = \tau \Sdev_{ij}$, $\Wtau_{ij} = \tau W_{ij}$, and $b_{ij}$, would be a polynomial expansion of the following form (Speziale 1991)
\begin{equation}
\frac{\redi_{ij}}{\epsilon} = \sum_{k=1}^7 h^{(k)} \mathcal{H}^{(k)}_{ij},
\end{equation}
where the functions $h^{(k)}$ depend on model coefficients and invariants of the input tensors, and $\mathcal{H}^{(1)}_{ij} ... \mathcal{H}^{(7)}_{ij}$ are the eight members of the integrity basis. These members are shown below
\begin{center}
\begin{tabular} {| Sl | Sl | }
\hline
$\boldsymbol{\mathcal{H}}^{(1)} = \mathbf{b}$ & $\boldsymbol{\mathcal{H}}^{(5)} = \mathbf{\Wtau} \mathbf{b} - \mathbf{b} \mathbf{\Wtau}$ \\
\hline
$\boldsymbol{\mathcal{H}}^{(2)} = \mathbf{b}^2 - \frac{1}{3} \{ \mathbf{b}^2 \} \mathbf{I}$ & $\boldsymbol{\mathcal{H}}^{(6)} = \mathbf{\Stau} \mathbf{b}^2 + \mathbf{b}^2 \mathbf{\Stau} - \frac{2}{3} \{ \mathbf{\Stau} \mathbf{b}^2 \} \mathbf{I}$ \\
\hline
$\boldsymbol{\mathcal{H}}^{(3)} = \mathbf{\Stau} $ & $ \boldsymbol{\mathcal{H}}^{(7)} = \mathbf{\Wtau} \mathbf{b}^2 - \mathbf{b}^2 \mathbf{\Wtau} $ \\
\hline
$ \boldsymbol{\mathcal{H}}^{(4)} = \mathbf{\Stau} \mathbf{b} + \mathbf{b} \mathbf{\Stau} - \frac{2}{3} \{ \mathbf{\Stau} \mathbf{b} \} \mathbf{I}  \qquad$ & \\
\hline
\end{tabular}
\end{center}

The most generic linear-in-$\mathbf{b}$ model would have $h^{(2)} = h^{(6)} = h^{(7)} = 0$. The LRR model by Lumley (1975) has the following $h^{(k)}$ coefficients
\begin{equation}
h^{(1)} = -2C_R \qquad h^{(3)} = \frac{4}{5} \qquad h^{(4)} = \frac{18 C_2 + 12}{11} \qquad h^{(5)} = \frac{20 - 14 C_2}{11}.
\end{equation}
The SSG model by Speziale (1991) has the following coefficients
\begin{equation}
h^{(1)} =  -C_1 - C_1^*\frac{\mathcal{P}}{\epsilon} \qquad h^{(2)} = C_2 \qquad h^{(3)} = C_3 - C_3^* \{\mathbf{b}^2 \} \qquad h^{(4)} = C_4 \qquad h^{(5)} = C_5,
\end{equation}
and is thus a non-linear model that does not satisfy the fifth requirement shown above.

%--------------------------------------------
\subsection{Dissipation}
%--------------------------------------------

The simplest model for the dissipation is to assume an isotropic form due to high Reynolds numbers. That is
\begin{equation}
\epsilon_{ij} = \frac{2}{3} \epsilon \delta_{ij}.
\end{equation}
An alternate model is that of Rotta (1951), which is 
\begin{equation}
\epsilon_{ij} = \frac{\rs_{ij}}{k} \epsilon.
\end{equation}
A third approach is to express the dissipation as follows
\begin{equation}
\epsilon_{ij} = \frac{2}{3} \epsilon \delta_{ij} + \hat{\epsilon}_{ij},
\end{equation}
where $\hat{\epsilon}_{ij}$ is the anisotropic dissipation. The anisotropic dissipation can thus be combined with the pressure-rate of strain so that a model is formulated for the sum $\redi_{ij} + \hat{\epsilon}_{ij}$.

%--------------------------------------------
\subsection{Turbulent transport}
%--------------------------------------------
Common models for the turbulent transport $\mathcal{T}^{(u)}_{ijk}$ are those of Mellor and Herring (1973) and Hanjalic and Launder (1972). The term $\mathcal{T}^{(p)}_i$ that appears in the pressure transport tensor is modeled as $c\mathcal{T}^{(u)}_{ijj}$, where $c$ is some constant. However, if both $\mathcal{T}^{(u)}_{ijk}$ and $\mathcal{T}^{(p')}_{ijk}$ are to be treated together, then common models for such sum are those of Shir (1973) and Daly and Harlow (1970),

%-------------------------------------------------------------------------------
\section{Algebraic Reynolds-stress models}
%-------------------------------------------------------------------------------

A common method for obtaining an algebraic Reynolds stresses model is to use the quasi-equilibrium assumption
\begin{equation}
\frac{D b_{ij}}{Dt} = 0,
\end{equation}
and to model the Reynolds stress transport terms using the transport terms of the TKE, as shown below
\begin{equation}
\frac{\partial}{\partial x_k} \left ( \mathcal{T}^{(u)}_{kij} + \mathcal{T}^{(p)}_{kij} + \mathcal{T}^{(\nu)}_{kij} \right ) = \frac{\rs_{ij}}{k} \frac{\partial}{\partial x_k} \left ( \mathcal{T}^{(u)}_{k} + \mathcal{T}^{(p)}_{k} + \mathcal{T}^{(\nu)}_{k} \right ).
\end{equation} 
The above states that the inhomogeneous contribution due to the transport terms of the Reynolds stresses can be modeled in terms of the inhomogeneous contribution due to the transport terms of the turbulent kinetic energy. Thus equation (\ref{anisotropy_eq}) simplifies to the following
\begin{equation}
\label{algebraic_rs}
0 = \mathcal{P}_{ij} + \redi_{ij} - \epsilon_{ij} - \frac{\rs_{ij}}{k} \left ( \mathcal{P} - \epsilon \right ).
\end{equation}

For incompressible flows, the production over dissipation can be expressed in terms of $\Stau_{ij}$ and $\Wtau_{ij}$ as follows
\begin{align}
\frac{\mathcal{P}_{ij}}{\epsilon} &= \frac{1}{\epsilon} \left (-\frac{2}{3} k \frac{\partial \ujavg}{\partial x_i} - 2kb_{ik}(S_{jk} + W_{jk}) - \frac{2}{3} k \frac{\partial \uiavg}{\partial x_j} - 2kb_{jk}(S_{ik} + W_{ik}) \right ) \nonumber \\
& = -\frac{4}{3} \mathbf{\Stau} - 2( \mathbf{\Stau} \mathbf{b} + \mathbf{b} \mathbf{\Stau}) - 2 (\mathbf{\Wtau} \mathbf{b} - \mathbf{b} \mathbf{\Wtau} ).
\end{align}
Assuming the most generic linear model for the pressure-rate-rate-of-strain, equation (\ref{algebraic_rs}) becomes
\begin{equation}
a_1 \mathbf{b} + a_2 \mathbf{\Stau} + a_3 \left (\mathbf{\Stau} \mathbf{b} + \mathbf{b} \mathbf{\Stau} - \frac{2}{3} \{ \mathbf{\Stau} \mathbf{b} \} \mathbf{I} \right) + a_4 (\mathbf{\Wtau} \mathbf{b} - \mathbf{b} \mathbf{\Wtau}) = 0,
\end{equation}
where
\begin{equation}
a_1 = \left ( \frac{h^{(1)}}{2} +1 - \frac{\mathcal{P}}{\epsilon} \right ) \qquad a_2 = \frac{1}{2} \left (-\frac{4}{3} + h^{(3)} \right ) \qquad a_3 = \frac{1}{2} \left (-2 + h^{(4)} \right ) \qquad a_4 = \frac{1}{2} \left (-2 + h^{(5)} \right ).
\end{equation}
Linearity in $\mathbf{b}$ is required so that when the polynomial expansion (\ref{poly_expansion}) is used to replace $\mathbf{b}$ a system of linear equations is obtained for the coefficients of such expansion.

The EASM of Rumsey \& Gatski (cite) uses the linearize SSG model to compute the pressure-rate-of-strain tensor. Thus, we have
\begin{equation}
a_1 = - \left ( \frac{C_1^*}{2} + 1\right ) \frac{\mathcal{P}}{\epsilon} + 1 - \frac{C_1}{2} \qquad a_2 = \frac{1}{2} \left (-\frac{4}{3} + C_3 \right ) \qquad a_3 = \frac{1}{2} \left (-2 + c_4 \right ) \qquad a_4 = \frac{1}{2} \left (-2 + C_5 \right ).
\end{equation}
On the other hand, the EARSM of Wallin \& Johansson (cite) uses Rotta's model to compute the slow-pressure-rate-of-strain tensor and the LRR model for the rapid part.  Thus, the $a_i$ coefficients become
\begin{equation}
a_1 = \left ( -C_R + 1 - \frac{\mathcal{P}}{\epsilon} \right ) \qquad a_2 = -\frac{4}{15} \qquad a_3 = -\frac{5 - 9 C_2}{11} \qquad a_4 = -\frac{7C_2 + 1}{11}
\end{equation}
For such model, a value for $C_2$ of $5/9$ can be chosen, which simplifies the implicit algebraic equation to
\begin{equation}
\label{implicit_rs}
N \mathbf{b} = -\frac{3}{5} \mathbf{\Stau} - (\mathbf{\Wtau} \mathbf{b} - \mathbf{b} \mathbf{\Wtau}),
\end{equation}
where 
\begin{equation}
\label{n_def}
N = \frac{9}{4} \left ( C_R - 1 + \frac{\mathcal{P}}{\epsilon} \right ).
\end{equation}
Equation (\ref{implicit_rs}) is an implicit algebraic equation for the Reynolds stresses since the factor N depends on $\mathcal{P}/\epsilon$, which itself depends on $b_{ij}$. 

The method for obtaining and explicit expression for $b_{ij}$ from equation (\ref{implicit_rs}) is the following. First, the polynomial expansion (\ref{poly_expansion}) is used to replace $b_{ij}$ in the implicit equation, except for the stresses that define $N$ since this variable is left as a yet undetermined function. All higher order tensors in this new equation are replaced by combinations of elements of the integrity basis using the Cayley-Hamilton theorem. By matching coefficients corresponding to the same element of the integrity basis one obtains expressions for the weights $p^{(k)}$ as a function of $N$. Finally, the polynomial expansion for $b_{ij}$ with the new  expressions for $p^{(k)}$ is used in the definition of $N$ (\ref{n_def}). This leads to a non-linear algebraic equation for N and its solution is used in the expressions for $p^{(k)}$.

%###############################################################################
%
\chapter{Probability Density Function Methods}
%
%###############################################################################

%-------------------------------------------------------------------------------
\section{The Eulerian Modeling Approach}
%-------------------------------------------------------------------------------

The transport equation for the one-point, one-time, Eulerian PDF $f=f(\Vvec ;\xvec,t)$ is derived in Appendix H of Pope (2000) and is shown below
\begin{equation}
\frac{\partial f}{\partial t} + V_i\frac{\partial f}{\partial x_i} = -\frac{\partial}{\partial V_i} \left ( f \left < \left . \frac{DU_i}{Dt} \right | \Vvec \right > \right ).
\end{equation}
This is a mathematical identity that contains no physics. The physics comes in by replacing the material derivative of $U_i$ using the Navier-Stokes equations, which leads to
\begin{equation}
\label{pdf_ns}
\frac{\partial f}{\partial t} + V_i\frac{\partial f}{\partial x_i} = \left ( \frac{1}{\rho} \frac{\partial \pavg}{\partial x_i} -\nu \frac{\partial^2 \uiavg}{\partial x_k \partial x_k} \right) \frac{\partial f}{\partial V_i} - \frac{\partial}{\partial V_i} \left ( f \left < \left . - \frac{1}{\rho} \frac{\partial \pfluc}{\partial x_i} +  \nu \frac{\partial ^2 \uifluc}{\partial x_k \partial x_k} \right | \Vvec \right > \right ).
\end{equation}
The equation above can be multiplied by $V_j$ and then averaged over the velocity sample space to obtain the RANS momentum equation. For the first term on the left hand side one would obtain
\begin{equation}
\int V_j \frac{\partial f}{\partial t} d\Vvec = \frac{\partial}{\partial t} \int V_j f d\Vvec = \frac{\partial \ujavg}{\partial t},
\end{equation} 
whereas for the second term one obtains
\begin{equation}
\int V_j V_i \frac{\partial f}{\partial x_i} d\Vvec = \frac{\partial}{\partial x_i} \int V_j V_i f d\Vvec = \frac{\partial \langle U_j U_i \rangle}{\partial x_i}.
\end{equation} 
For the terms on the right hand side we make use of the identity
\begin{equation}
\int B_k \frac{\partial}{\partial V_i} ( f A_j ) d\Vvec = - \int \frac{\partial B_k}{\partial V_i} A_j f d\Vvec,
\end{equation}
where $B_k = B_k(\Vvec)$ and $A_j = A_j(\Vvec)$.
Thus, the first term on the right hand side becomes
\begin{align}
\int \left ( \frac{1}{\rho} \frac{\partial \pavg}{\partial x_i} -\nu \frac{\partial^2 \uiavg}{\partial x_k x_k} \right) V_j \frac{\partial f}{\partial V_i} d\Vvec & = \left (- \frac{1}{\rho} \frac{\partial \pavg}{\partial x_i} + \nu \frac{\partial^2 \uiavg}{\partial x_k \partial x_k} \right) \delta_{ij} \int f d\Vvec \nonumber \\
& = -\frac{1}{\rho} \frac{\partial \pavg}{\partial x_j} + \nu \frac{\partial ^2 \ujavg}{\partial x_k x_k}.
\end{align}
Given $f_{\Uvec \phi} = f_{\Uvec \phi}(\Vvec, \psi; \xvec, t)$ the joint PDF of $\Uvec$ and $\phi$, and $f_{\phi | \Uvec} = f_{\phi | \Uvec}(\psi | \Vvec; \xvec, t)$ the conditional PDF for $\phi$ conditioned on $\Vvec$, we write the following 
\begin{equation}
\langle \phi \rangle = \int \int \psi f_{\Uvec \phi} d\psi d\Vvec = \int \int \psi f_{\phi | \Uvec} f d\psi d\Vvec = \int \langle \phi | \Vvec \rangle f d\Vvec,
\end{equation}
The derivation for the second term on the right hand side thus proceeds as follows
\begin{align}
-\int V_j \frac{\partial}{\partial V_i} \left ( f \left < \left . - \frac{1}{\rho} \frac{\partial \pfluc}{\partial x_i} + \nu \frac{\partial^2 \uifluc}{\partial x_k \partial x_k} \right | \Vvec \right > \right ) d\Vvec &= \delta_{ij} \int \left < \left . - \frac{1}{\rho} \frac{\partial \pfluc}{\partial x_i} + \nu \frac{\partial^2 \uifluc}{\partial x_k \partial x_k} \right | \Vvec \right > f d\Vvec \nonumber \\
& = \left < - \frac{1}{\rho} \frac{\partial \pfluc}{\partial x_j} + \nu \frac{\partial^2 \ujfluc}{\partial x_k \partial x_k} \right > \nonumber \\
& = 0.
\end{align}

Given the PDF one can compute $\uiavg$ and $\rs_{ij}$. The averaged pressure $\pavg$ can then be obtained from a poisson equation. Thus, the last term in the PDF equation (\ref{pdf_ns}) is the only unclosed term, and it is typically modeled using the Generalized Langevin Model (GLM), as shown below
\begin{equation}
\label{GLM}
- \frac{\partial}{\partial V_i} \left ( f \left < \left . - \frac{1}{\rho} \frac{\partial \pfluc}{\partial x_i} +  \nu \frac{\partial^2 \uifluc}{\partial x_k \partial x_k} \right | \Vvec \right > \right ) = -\frac{\partial}{\partial V_i} [f G_{ij} (V_j - \ujavg)] + \frac{1}{2} C_0 \epsilon \frac{\partial^2 f}{\partial V_j \partial V_j}.
\end{equation}
The coefficients $G_{ij} = G_{ij}(\xvec,t)$ and $C_0 = C_0(\xvec,t)$ are specific to each of the different versions of the GLM. As with the exact unclosed term, the modeled terms vanish when the RANS momentum equations are formed. 

Multiplying equation (\ref{pdf_ns}) by $v_j v_k$ and integrating over the velocity space one obtains the corresponding transport equation for the Reynolds stresses
\begin{equation}
\frac{\partial \rs_{ij}}{\partial t} + \ukavg \frac{\partial \rs_{ij}}{\partial x_k} = \mathcal{P}_{ij} - \frac{\partial \mathcal{T}^{(u)}_{kij} }{\partial x_k} + G_{ik} \rs_{kj} + G_{jk} \rs_{ki} + C_0 \epsilon \delta_{ij}.
\end{equation} 
In other words, the velocity-pressure-gradient tensor, the dissipation tensor, and viscous diffusion are modeled by the GLM as follows
\begin{equation}
\Pi_{ij} - \epsilon_{ij} - \frac{\partial \mathcal{T}^{(\nu)}_{kij} }{\partial x_k} = G_{ik} \rs_{kj} + G_{jk} \rs_{ki} + C_0 \epsilon \delta_{ij}
\end{equation}
This shows that to each GLM there is a corresponding DRSM. The accuracy of the above framework ultimately depends on how accurately the modeled PDF represents the exact PDF.

The Eulerian modeling approach thus proceeds as follows. From the exact equations for the instantaneous fluid properties, such as the Navier-Stokes equations for velocity, unclosed statistical equations are obtained, such as the RANS equation for $\uiavg$ or $\rs_{ij}$, or the transport equation for the Eulerian PDF $f$. The unclosed terms in these equations are then modeled in terms of the known variables.

%-------------------------------------------------------------------------------
\section{The Stochastic Lagrangian Approach}
%-------------------------------------------------------------------------------

An alternate modeling approach, termed the stochastic Lagrangian approach, can be followed to derive a closed modeled conservation equation for the Eulerian PDF. As for the Eulerian modeling approach, mean quantities such as $\uiavg$ or $\rs_{ij}$ are known variables since they can be obtained from the Eulerian PDF. The mean pressure can be obtained by solving its corresponding Poisson equation, and it is thus also referred to as a known variable. 

The starting point is the Lagrangian evolution equations for the position $X_i^+ = X_i^+(t,\yvec)$ and velocity $U_i^+ = U_i^+(t,\yvec)$ of a fluid particle that at time $t_0$ was located at $\yvec$. These evolution equations are shown below 
\begin{equation}
\frac{\partial X_i^+}{\partial t} = U_i^+ 
\end{equation}
\begin{equation}
\frac{\partial U_i^+}{\partial t} = \left (-\frac{1}{\rho} \frac{\partial \pavg}{\partial x_i} + \nu \frac{\partial^2 \uiavg}{\partial x_k \partial x_k} \right)_{\xvec = \Xvec^+} + \left ( - \frac{1}{\rho} \frac{\partial \pfluc}{\partial x_i} +  \nu \frac{\partial^2 \uifluc}{\partial x_k \partial x_k} \right)_{\xvec = \Xvec^+}.
\end{equation}
The first term in parenthesis of the right-hand side of the velocity evolution equation consists of known variables, whereas the second term consists of unknown terms and thus requires modeling. The above equations can be used to derive a transport equation for a Lagrangian PDF, from which an Eulerian PDF follows. These derivations will be shown in this section. 

%--------------------------------------------
\subsection{Lagrangian probablity density functions}
%--------------------------------------------

The inverse of $\Xvec^+$ is $\Yvec^+ = \Yvec^+(t,\zvec)$, which gives the initial location of a fluid particle that at time $t$ is located at position $\zvec$. Thus,
\begin{equation}
\Xvec^+(t,\Yvec^+(t,\zvec)) = \zvec,
\end{equation}
which represents the location of a particle at time $t$, whose original location corresponds to such particle being located at $\zvec$ for time $t$.

The Lagrangian PDF is given by $f_L = f_L(\Vvec,\xvec;t|\bold{y})$, which is conditioned on the fluid particle being located at $\yvec$ at some initial time $t_0$. Here $\Vvec$ and $\xvec$ are the sample space variables for $\Uvec^+$ and $\Xvec^+$. $f_L$ can be obtained as follows 
\begin{equation}
f_L = \langle f'_L \rangle,
\end{equation}
where $f'_L = f'_L(\Vvec,\xvec;t|\bold{y})$ is the fine-grained Lagrangian PDF, which is given by
\begin{equation}
f'_L = \delta(\bold{U}^+ - \Vvec) \delta(\bold{X}^+ - \xvec).
\end{equation}

We will now derive the relationship between the Lagrangian and Eulerian PDFs. We start with
\begin{align}
\int f'_L(\Vvec,\xvec;t|\yvec) d\yvec &= \int \delta(\bold{U}^+(t,\bold{y}) - \Vvec) \delta(\bold{X}^+(t,\bold{y}) - \xvec) d\yvec \nonumber \\
&= \int \delta(\bold{U}(t,\bold{X}^+(t,\yvec)) - \Vvec) \delta(\bold{X}^+(t,\yvec) - \xvec) d\yvec \nonumber \\
&= \int \delta(\bold{U}(t,\bold{X}^+(t,\yvec)) - \Vvec) \delta(\bold{X}^+(t,\yvec) - \xvec) | \det D\bold{X^+} | d\yvec,
\end{align}
where we have introduced $| \det D\bold{X^+} |$, which is the absolute value of the determinant of the Jacobean $\partial \bold{X}^+/\partial \yvec$, and is equal to one for incompressible flows. Using integration by substitution we obtain
\begin{equation}
\int f'_L(\Vvec,\xvec;t|\yvec) d\yvec = \int \delta(\bold{U}(t,\zvec) - \Vvec) \delta(\zvec - \xvec) d\zvec = \delta(\Uvec(t,\xvec) - \Vvec)
\end{equation}
Since $\delta(\Uvec(t,\xvec) - \Vvec)$ is the fine-grained Eulerian PDF, taking the expectation of the above we finally obtain
\begin{equation}
\label{eul_lag}
\int f_L(\Vvec,\xvec;t|\yvec) d\yvec = f(\Vvec;\xvec,t).
\end{equation}

%--------------------------------------------
\subsection{Diffusion processes}
%--------------------------------------------

A Markov process is a stochastic process whose joint PDF satisfies the following
\begin{equation}
f_{N-1}(V_N;t_N| V_{N-1}, t_{N-1}, V_{N-2}, t_{N-2}, ... , V_1, t_1) = f_1(V_N; t_N | V_{N-1}, t_{N-1}). 
\end{equation}
That is, the probability distribution for the random variable at time $t$ depends only on one single previous realization at time $t_{N-1} < t$, and is independent of all the realizations at previous times less than $t_{N-1}$.

Consider the operator $\Delta_h$ which operates on $U = U(t)$, such that $\Delta_hU(t) = U(t+h) - U(t)$. Using such operator define the set of infinitesimal parameters $B_n = B_n(V,t)$
\begin{equation}
B_n = \lim_{h\to0} \frac{1}{h} \langle ( \Delta_h U )^n | U = V \rangle.
\end{equation}
A diffusion process is a Markov process with drift coefficient $a = a(V,t)$ and diffusion coefficient $b = b(V,t)$ given by
\begin{equation}
a = B_1 \qquad
b^2 = B_2,
\end{equation}
and $B_i = 0$ for all $i \ge 3$. Diffusion processes  are governed by the following stochastic differential equation
\begin{equation}
dU(t) = a[U(t),t] dt + b[U(t),t] dW(t)
\end{equation}
where $dW(t)$ is the infinitesimal increment for a Wiener process. The corresponding conditional PDF $f_1 = f_1(V;t | V_1,t_1)$ satisfies the Fokker-Plank equation
\begin{equation}
\frac{\partial f_1}{\partial t} = - \frac{\partial a f_1}{\partial V} + \frac{1}{2} \frac{\partial ^2 b^2 f_1}{\partial V^2}.
\end{equation}
Multiplying the above by $f(V_1;t_1)$ and integrating over $V_1$ shows that the marginal PDF $f = f(V;t)$ satisfies the same Fokker-Plank equation.

For a vector-valued diffusion process the sample paths are given by $\mathbf{U} =  \mathbf{U}(t)$. The drift coefficient is given by $a_i = a_i(\Vvec,t)$, where
\begin{equation}
a_i = \lim_{h\to0} \frac{1}{h} \langle ( \Delta_h U_i )^n | \Uvec = \Vvec \rangle.
\end{equation}
The coefficient $b_{ij} = b_{ij}(\Vvec, t)$ is defined such that $b_{ik}b_{jk} = B_{ij}$, where $B_{ij} = B_{ij}(\Vvec,t)$ is the diffusion coefficient and is given by
\begin{equation}
B_{ij} = \lim_{h\to0} \frac{1}{h} \langle \Delta_h U_i \Delta_h U_j  | \Uvec = \Vvec \rangle.
\end{equation}
The associated stochastic differential equation is
\begin{equation}
dU_i(t) = a_i[\bold{U}(t),t] dt + b_{ij} [\bold{U}(t),t] dW_j(t),
\end{equation}
and the Fokker-Plank equation for the conditional PDF $f_1 = f_1(\Vvec; t| \Vvec^{(1)}, t^{(1)})$ is
\begin{equation}
\label{fokkerplank_3d}
\frac{\partial f_1}{\partial t} = - \frac{\partial a_i f_1}{\partial V_i} + \frac{1}{2} \frac{\partial^2 B_{ij} f_1}{\partial V_i V_j}.
\end{equation}
As with the scalar case, the marginal PDF is obtained by multiplying the above by $f(\bold{V}^{(1)};t^{(1)})$ and integrating over $\bold{V}^{(1)}$. Such PDF solves the exact same evolution equation as the conditional PDF.

%--------------------------------------------
\subsection{Evolution equations for the Lagrangian PDF}
%--------------------------------------------

Consider the six-dimensional diffusion process $\Zvec^* = \Zvec^*(t)$  that describes the position $\Xvec^* = \Xvec^*(t)$ and velocity $\Uvec^* = \Uvec^*(t)$ of a fluid particle. Thus
\begin{equation}
\bold{Z}^* = \begin{pmatrix}
  Z_1^* \\
  Z_2^* \\
  Z_3^* \\
  Z_4^* \\
  Z_5^* \\
  Z_6^*
 \end{pmatrix}
=\begin{pmatrix}
  X_1^* \\
  X_2^* \\
  X_3^* \\
  U_1^* \\
  U_2^* \\
  U_3^*
 \end{pmatrix}
\end{equation}
The corresponding sample space is given by the vector $\{x_1, x_2, x_3, V_1, V_2, V_3\}$. The vectors $\Xvec^*$ and $\Uvec^*$ evolve according to
\begin{gather}
\frac{d X^*_i}{dt} = U^*_i, \\
dU_i^* = -\left ( \frac{1}{\rho} \frac{\partial \pavg}{\partial x_i} + \nu \frac{\partial^2 \uiavg}{\partial x_k \partial x_k} \right )_{\xvec = \Xvec^*} dt + G_{ij} (U_j^* - \ujavg) dt + (C_0 \epsilon)^{1/2} dW_i,
\end{gather} 
where the coefficient $G_{ij}$ and the expectation $\ujavg$ are evaluated at $\Xvec^*$. The drift term involving the pressure gradient and the laplacian of $\uiavg$ comes from the Navier-Stokes equations, whereas the remaining terms model the fluctuating pressure gradient and the Laplacian of the fluctuating velocity.

The stochastic process $\bold{Z}^*$ evolves according the general diffusion equation
\begin{equation}
dZ^*_i(t) = a_i[\bold{Z}^*(t),t] dt + b_{ij}[\bold{Z}^*(t),t] dW_j(t),
\end{equation}
where 
\begin{equation}
\bold{a}[\bold{Z}^*(t),t)] =  \begin{pmatrix}
  Z^*_4(t) \\
  Z^*_5(t)  \\
  Z^*_6(t) \\
-\left ( \frac{1}{\rho} \frac{\partial \pavg}{\partial x_1} + \nu \frac{\partial^2 \langle U_1 \rangle}{\partial x_k \partial x_k} \right ) + G_{1j} (Z_{(j+3)}^*(t) - \ujavg) \\
-\left ( \frac{1}{\rho} \frac{\partial \pavg}{\partial x_2} + \nu \frac{\partial^2 \langle U_2 \rangle}{\partial x_k \partial x_k} \right ) + G_{2j} (Z_{(j+3)}^*(t) - \ujavg) \\
-\left ( \frac{1}{\rho} \frac{\partial \pavg}{\partial x_3} + \nu \frac{\partial^2 \langle U_3 \rangle}{\partial x_k \partial x_k} \right ) + G_{3j} (Z_{(j+3)}^*(t) - \ujavg)
 \end{pmatrix}
\end{equation}
and
\begin{equation}
\mathbf{b}[\bold{Z}^*(t),t] = \begin{pmatrix}
  0 & 0 & 0 & 0 & 0 & 0 \\
  0 & 0 & 0 & 0 & 0 & 0 \\
  0 & 0 & 0 & 0 & 0 & 0 \\
  0 & 0 & 0 & (C_0 \epsilon)^{1/2}  & 0 & 0 \\
  0 & 0 & 0 & 0 & (C_0 \epsilon)^{1/2}  & 0 \\
  0 & 0 & 0 & 0 & 0 & (C_0 \epsilon)^{1/2} 
 \end{pmatrix}.
\end{equation}
The Eulerian fields for the pressure gradient, viscous diffusion, and $G_{ij}$ that appear in $\avec[\Zvec^*(t),t)]$ are evaluated at $\xvec = \Xvec^*$. From the expression for $\bvec$ we obtain
\begin{equation}
B_{ij} = \begin{pmatrix}
  0 & 0 & 0 & 0 & 0 & 0 \\
  0 & 0 & 0 & 0 & 0 & 0 \\
  0 & 0 & 0 & 0 & 0 & 0 \\
  0 & 0 & 0 & C_0 \epsilon  & 0 & 0 \\
  0 & 0 & 0 & 0 & C_0 \epsilon  & 0 \\
  0 & 0 & 0 & 0 & 0 & C_0 \epsilon
 \end{pmatrix}.
\end{equation}
Using equation (\ref{fokkerplank_3d}) it is straightforward to derive the Fokker-Plank equation for the conditional PDF $f_1(\xvec,\Vvec;t | \xvec^{(1)}, \Vvec^{(1)}, t^{(1)})$. Assuming a fixed value of $t^{(1)} = t_0$, and labeling $\xvec^{(1)}$ by $\yvec$ and $\Vvec^{(1)}$ by $\Wvec$, we re-label this PDF as $f_L(\Vvec, \xvec; t | \Wvec, \yvec)$. Its Fokker-Plank equation follows
\begin{align}
\label{fokkerplank_xVx1V1}
\frac{\partial}{\partial t} f_L(\Vvec, \xvec; t | \Wvec, \yvec) = &-V_i\frac{\partial} {\partial x_i}f_L(\Vvec, \xvec; t | \Wvec, \yvec) + \left (\frac{1}{\rho} \frac{\partial \pavg}{\partial x_i}  -\nu \frac{\partial^2 \uiavg}{\partial x_k \partial x_k} \right ) \frac{\partial}{\partial V_i}f_L(\Vvec, \xvec; t | \Wvec, \yvec) \nonumber \\
&- G_{ij}\frac{\partial}{\partial V_i} [(V_j - \ujavg) f_L(\Vvec, \xvec; t | \Wvec, \yvec) ] + \frac{1}{2} C_0 \epsilon \frac{\partial ^2}{\partial V_i V_i} f_L(\Vvec, \xvec; t | \Wvec, \yvec ).
\end{align}
The pressure gradient, viscous diffusion, and $G_{ij}$ are now functions of time and the sample space variable $\xvec$. The Lagrangian PDF $f_L$ can be obtained from $f_L(\Vvec, \xvec; t| \Wvec, \yvec)$ as shown below
\begin{align}
f_L(\Vvec, \xvec; t | \yvec) &= \frac{ f_L(\Vvec, \xvec; t, \yvec)}{f_L(\yvec)} \nonumber \\
& = \frac{ \int f_L(\Vvec, \xvec; t, \Wvec, \yvec) d\Wvec }{f_L(\yvec)} \nonumber \\
& =  \int f_L(\Vvec, \xvec; t | \Wvec, \yvec) \frac{ f_L(\Wvec, \yvec) }{  f_L(\yvec)  } d\Wvec 
\end{align}
Multiplying equation (\ref{fokkerplank_xVx1V1}) by $f_L(\Wvec, \yvec) / f_L(\yvec)$ and integrating over $\Wvec$ shows that $f_L$ still satisfies the given Fokker-Plank, that is
\begin{equation}
\frac{\partial f_L}{\partial t} = -V_i\frac{\partial f_L} {\partial x_i} + \left ( \frac{1}{\rho} \frac{\partial \pavg}{\partial x_i} - \nu \frac{\partial^2 \uiavg}{\partial x_k \partial x_k} \right ) \frac{\partial f_L}{\partial V_i} - G_{ij}\frac{\partial}{\partial V_i} [(V_j - \ujavg) f_L] + \frac{1}{2} C_0 \epsilon \frac{\partial ^2 f_L}{\partial V_i V_i}.
\end{equation}
Using equation (\ref{eul_lag}) we finally obtain the associated Eulerian PDF
\begin{equation}
\label{GLM2}
\frac{\partial f}{\partial t}  + V_i\frac{\partial f} {\partial x_i} =  \left ( \frac{1}{\rho} \frac{\partial \pavg}{\partial x_i} - \nu \frac{\partial^2 \uiavg}{\partial x_k \partial x_k} \right ) \frac{\partial f}{\partial V_i} - G_{ij}\frac{\partial}{\partial V_i} [(V_j - \ujavg) f] + \frac{1}{2} C_0 \epsilon \frac{\partial ^2 f}{\partial V_i V_i}.
\end{equation}
This is equation is identical to the evolution equation (\ref{GLM}) for the Eulerian PDF using the GLM.

%--------------------------------------------
\subsection{Particle methods}
%--------------------------------------------
We consider a large number of statistically identical particles whose PDFs are denoted by  $f_L$, as previously described. They are originally distributed uniformly over a domain $\mathcal{V}$, and thus the PDF of initial position becomes
\begin{equation}
f_L(\yvec) = \frac{1}{\mathcal{V}}
\end{equation}
so as to satisfy the normalization condition.

Ultimately, we are interested in Eulerian quantities such as the mean velocity, which is obtained as follows
\begin{equation}
\uiavg = \int V_i f d\Vvec.
\end{equation}
Instead of computing the corresponding Eulerian PDF for the system of particles we make use of the following
\begin{align}
f(\Vvec; \xvec,t) &= \int f_L(\Vvec, \xvec; t|\yvec) d\yvec \nonumber \\
& = \int f_L(\Vvec, \xvec; t|\yvec) d\yvec \frac{f_L(\yvec)}{f_L(\yvec)} \nonumber \\
& = \int f_L(\Vvec, \xvec; t| \yvec) f_L(\yvec) d\yvec \frac{1}{f_L(\yvec)} \nonumber \\
& = \int f_L(\Vvec, \xvec; t, \yvec) d\yvec \frac{1}{f_L(\yvec)} \nonumber \\
& = f_L(\Vvec, \xvec; t) \frac{1}{f_L(\yvec)} \nonumber \\
& = f_{L_x}(\Vvec| \xvec; t) \frac{f_L(\xvec; t)}{f_L(\yvec)}.
\end{align}
We note that $f_L(\Vvec,\xvec;t)$ is the probability of seeing a particle with velocity $\Vvec$ and position $\xvec$ at time $t$, irrespective of where that particle originated. The PDF $f_{L_x} = f_{L_x}(\Vvec|\xvec;t)$ refers to the probability of seeing a particle with velocity $\Vvec$ given that the particle, irrespective of where it originated, is located at position $\xvec$. We now require that $f_L(\xvec;t) = f_L(\yvec)$ (the distribution of particles over space remains uniform) and thus we can compute statistical quantities, such as the mean velocity, as follows
\begin{equation}
\uiavg = \int V_i f_{L_x} d\Vvec.
\end{equation}
That is, we have stablished an equality between $f$ and $f_{L_x}$. If we were to use the GLM to model the particle evolution equations, the PDF $f_{L_x}$ would also satisfy eq. (\ref{GLM2}).

%-------------------------------------------------------------------------------
\section{PDFs for the fluctuating velocity}
%-------------------------------------------------------------------------------

The Eulerian PDF  for $\uvec$ is given by $g = g(\vvec; \xvec,t)$, where $\vvec$ is the sample space for $\uvec$. The PDF $g$ contains no information on $\langle \Uvec \rangle$, but the information provided by both $g$ and $\langle \Uvec \rangle$ is identical to that obtained from $f$. The relationship between $f$ and $g$ is as follows
\begin{equation}
\label{eul_g_f}
g(\vvec; \xvec,t) = f(\langle \Uvec \rangle + \vvec; \xvec,t).
\end{equation}
It is important to note that similar relationships follow for multi-variable PDFs. For example, consider the joint PDFs $g(\qvec,\vvec; \xvec, t)$ and $f(\qvec, \Vvec; \xvec, t)$, where $\qvec$ is the sample space variable for $- \frac{1}{\rho} \nabla \pvec +  \nu \Delta \uvec$. Thus, these two PDFs are related by 
\begin{equation}
\label{eul_g_f_joint}
g(\qvec, \vvec; \xvec,t) = f(\qvec, \langle \Uvec \rangle + \vvec; \xvec,t).
\end{equation}

We begin the derivation of the transport equation for the fluctuating Eulerian PDF as follows 
\begin{align}
\frac{\partial g}{\partial t} + ( \uiavg + v_i ) \frac{\partial g}{\partial x_i} = & \left ( \frac{\partial f}{\partial V_j} \right )_{\Vvec = \langle \Uvec \rangle + \vvec} \frac{\partial \ujavg}{\partial t} + \left (\frac{\partial f}{\partial t} \right )_{\Vvec = \langle \Uvec \rangle + \vvec} \nonumber \\
&+ ( \uiavg + v_i ) \left [ \left ( \frac{\partial f}{\partial V_j} \right )_{\Vvec = \langle \Uvec \rangle + \vvec} \frac{\partial \ujavg}{\partial x_i} + \left (\frac{\partial f}{\partial x_i} \right )_{\Vvec = \langle \Uvec \rangle + \vvec} \right ].
\end{align}
Regrouping the above we obtain
\begin{equation}
\frac{\partial g}{\partial t} + ( \uiavg + v_i ) \frac{\partial g}{\partial x_i} = \left ( \frac{\partial f}{\partial t} + V_i \frac{\partial f}{\partial x_i} \right )_{\Vvec = \langle \Uvec \rangle + \vvec} + \left ( \frac{\partial f}{\partial V_j} \right )_{\Vvec = \langle \Uvec \rangle + \vvec} \left [ \frac{\partial \ujavg}{\partial t} + (\uiavg + v_i) \frac{\partial \ujavg}{\partial x_i} \right ]
\end{equation}
To proceed further we need the following identity. For any function $H(\Vvec; \xvec, t)$ we have
\begin{equation}
\label{deriv_v}
\frac{\partial H(\langle \Uvec \rangle + \vvec; \xvec, t)}{\partial v_i} = \left ( \frac{\partial H(\Vvec; \xvec, t)}{\partial V_j} \right)_{\Vvec = \langle \Uvec \rangle + \vvec} \frac{\partial ( \ujavg + v_j) }{\partial v_i} = \left ( \frac{\partial H(\Vvec; \xvec,t )}{\partial V_i} \right)_{\Vvec = \langle \Uvec \rangle + \vvec}.
\end{equation}
The transport equation for $g$ then becomes
\begin{equation}
\frac{\partial g}{\partial t} + ( \uiavg + v_i ) \frac{\partial g}{\partial x_i} = \left ( \frac{\partial f}{\partial t} + V_i \frac{\partial f}{\partial x_i} \right )_{\Vvec = \langle \Uvec \rangle + \vvec} + \frac{\partial g}{\partial v_j} \left [ \frac{\partial \ujavg}{\partial t} + (\uiavg + v_i) \frac{\partial \ujavg}{\partial x_i} \right ]
\end{equation}
Using equation (\ref{pdf_ns}) the first term on the right-hand side becomes
\begin{multline}
\left ( \frac{\partial f}{\partial t} + V_i \frac{\partial f}{\partial x_i} \right )_{\Vvec = \langle \Uvec \rangle + \vvec} = \left ( \frac{1}{\rho} \frac{\partial \pavg}{\partial x_j} -\nu \frac{\partial^2 \ujavg}{\partial x_k \partial x_k} \right) \left (\frac{\partial f}{\partial V_j} \right)_{\Vvec = \langle \Uvec \rangle + \vvec} \\
- \left [ \frac{\partial}{\partial V_j} \left ( f \left < \left . - \frac{1}{\rho} \frac{\partial \pfluc}{\partial x_j} +  \nu \frac{\partial ^2 \ujfluc}{\partial x_k \partial x_k} \right | \Vvec \right > \right ) \right ]_{\Vvec = \langle \Uvec \rangle + \vvec} 
\end{multline}
Using the identity shown in (\ref{deriv_v}) the above becomes
\begin{multline}
\left ( \frac{\partial f}{\partial t} + V_i \frac{\partial f}{\partial x_i} \right )_{\Vvec = \langle \Uvec \rangle + \vvec} = \left ( \frac{1}{\rho} \frac{\partial \pavg}{\partial x_j} -\nu \frac{\partial^2 \ujavg}{\partial x_k \partial x_k} \right) \frac{\partial g}{\partial v_j} \\
- \frac{\partial}{\partial v_j} \left ( g \left < \left . - \frac{1}{\rho} \frac{\partial \pfluc}{\partial x_j} +  \nu \frac{\partial ^2 \ujfluc}{\partial x_k \partial x_k} \right | \langle \Uvec \rangle + \vvec \right > \right ). \end{multline}
Using equation (\ref{eul_g_f_joint}) the conditional expectation in the last term can be re-expressed as
\begin{equation}
\int \qvec\, f(\qvec | \langle \Uvec \rangle + \vvec; \xvec, t) d\qvec = \int \qvec\, g(\qvec | \vvec; \xvec, t) d\qvec = \left < \left . - \frac{1}{\rho} \nabla \pvec +  \nu \Delta \uvec \right | \vvec \right >.
\end{equation}
Using the RANS momentum equation one finally obtains the transport equation for $g$
\begin{equation}
\frac{\partial g}{\partial t} + ( \uiavg + v_i ) \frac{\partial g}{\partial x_i} = v_i \frac{\partial \ujavg}{\partial x_i} \frac{\partial g}{\partial v_j} - \frac{\partial \rs_{jk}}{\partial x_k} \frac{\partial g}{\partial v_j} - \frac{\partial}{\partial v_j} \left ( g \left < \left . - \frac{1}{\rho} \frac{\partial \pfluc}{\partial x_j} +  \nu \frac{\partial ^2 \ujfluc}{\partial x_k \partial x_k} \right | \vvec \right > \right ).
\end{equation}

As before, the term that needs modeling is the last one on the right-hand side. If, during the derivation of the transport equation for $g$, the modeled equation for $f$ based on the GLM is used, then the modeled equation for $g$ corresponding to the GLM becomes
\begin{equation}
\frac{\partial g}{\partial t} + ( \uiavg + v_i ) \frac{\partial g}{\partial x_i} = v_i \frac{\partial \ujavg}{\partial x_i} \frac{\partial g}{\partial v_j} - \frac{\partial \rs_{jk}}{\partial x_k} \frac{\partial g}{\partial v_j} - \frac{\partial g G_{jk} v_k}{\partial v_j}  + \frac{1}{2} C_0 \epsilon \frac{\partial^2 g}{\partial v_k \partial v_k}.
\end{equation}

We can also set up a relationship between the fluctuating Eulerian and fluctuating Lagrangian PDFs. The fluctuating Lagrangian PDF $g_L = g_L(\vvec, \xvec; t|\yvec)$ is the joint PDF of $\bold{X}^+(t,\yvec)$ and the fluctuating velocity $\uvec^+ = \uvec^+(t,\yvec)$, where $\uvec^+ = \Uvec^+ - \langle  \Uvec^+ \rangle$. The evolution equations for these variables are
\begin{equation}
\frac{\partial X_i^+}{\partial t} = \uiavg_{\xvec = \Xvec^+} + \uifluc^+ ,
\end{equation}
\begin{equation}
\frac{\partial \uifluc^+}{\partial t} = -\ujfluc^+ \left ( \frac{\partial \uiavg}{\partial x_j} \right)_{\xvec = \Xvec^+} + \left ( \frac{\partial \rs_{ij}}{\partial x_j} \right )_{\xvec = \Xvec^+} + \left ( - \frac{1}{\rho} \frac{\partial \pfluc}{\partial x_i} + \nu \frac{\partial^2 \uifluc}{\partial x_j \partial x_j} \right)_{\xvec = \Xvec^+}.
\end{equation}
As with the Eulerian PDFs, we can relate the Lagrangian PDF and the fluctuating Lagrangian PDF as follows
\begin{equation}
\label{lag_g_f}
g_L(\vvec,\xvec; t|\yvec) = f_L(\langle \Uvec(\xvec,t) \rangle + \vvec, \xvec; t|\yvec).
\end{equation}
Thus, plugging in equations (\ref{eul_g_f}) and (\ref{lag_g_f}) into equation (\ref{eul_lag}), we finally obtain
\begin{equation}
\label{fluc_eul_lag}
\int g_L(\vvec,\xvec; t|\yvec) d\yvec = g(\vvec; \xvec,t).
\end{equation}

For homogeneous turbulence the fluctuating Eulerian PDF becomes independent of $\xvec$, that is, $g = g^{(h)}$, where $g^{(h)} = g^{(h)}(\vvec;t)$. As with the two-point velocity correlation $\tpvc_{ij}(\xvec,\yvec,t)$, which for homogeneous turbulence does not depend on $\xvec$ and $\yvec$ separately but on $\xvec - \yvec$, the fluctuating Lagrangian PDF $g_L$  also depends on $\xvec - \yvec$ only. Thus, we write
\begin{equation}
g_L(\vvec ,\xvec ; t |\yvec) = h_L(\vvec, \xvec - \yvec; t),
\end{equation}
where $h_L(\vvec, \xvec - \yvec; t)$ is the joint PDF for the fluctuating velocity $\uvec^+(t,\yvec)$ and the separation distance $\Xvec^+(t,\yvec) - \yvec$. Using equation (\ref{fluc_eul_lag}) we can show that
\begin{equation}
g^{(h)}(\vvec;t) =  \int g_L(\vvec,\xvec;t|\yvec)dy = \int h_L(\vvec, \xvec - \yvec; t) d\yvec = \int h_L(\vvec, \rvec;t) d\rvec,
\end{equation}
where we used integration by substitution for the variable $\yvec(\rvec) = \xvec - \rvec$. The last integral above constitutes the fluctuating Lagrangian PDF for $\uvec^+(t,\yvec)$, which we label as $g_L^{(h)} = g_L^{(h)}(\vvec; t)$. Thus, for homogeneous turbulence, the PDFs for $\uvec(t,\xvec)$ and $\uvec^+(t,\yvec)$ are equal, that is $g^{(h)} = g_L^{(h)}$.

%###############################################################################
%
\chapter{Large-eddy simulation}
%
%###############################################################################

%-------------------------------------------------------------------------------
\section{Filtering}
%-------------------------------------------------------------------------------
The filtering operation for a homogeneous filter is as follows
\begin{equation}
\overline{U}_i(\xvec) = \int_{\Rthree} G(\rvec) U_i(\xvec - \rvec ) d\rvec
\end{equation}
Taking the fourier transform of the above leads to
\begin{equation}
\mathcal{F}\{ \overline{U}_i(\xvec) \} = (2 \pi)^3 \mathcal{F} \{ G(\xvec) \} \mathcal{F} \{ U_i(\xvec) \}
\end{equation}
due to the convolution theorem. If we define $\hat{\overline{U}}_i (\kvec) = \mathcal{F}\{ \overline{U}_i(\xvec) \}$, $\hat{G}(\kvec) = (2 \pi )^3 \mathcal{F} \{ G(\xvec) \}$, and $\hat{U}_i(\kvec) = \mathcal{F} \{ U_i(\xvec) \}$, then the above becomes
\begin{equation}
\hat{\overline{U}}_i (\kvec) = \hat{G}(\kvec) \hat{U}_i(\kvec).
\end{equation}
That is, filtering in physical space is equivalent to multiplication by $\hat{G}(\kvec)$ in spectral space.

The filtered two-point velocity correlation is given by $\overline{\tpvc}_{ij}(\xvec, \rvec) = \langle \overline{u}_i(\xvec) \overline{u}_j(\xvec + \rvec) \rangle$. For homogeneous turbulence this can be expresses as
\begin{equation}
\label{eq:filt_tpvc}
\overline{\tpvc}_{ij}(\rvec) = \int_\Rthree \int_\Rthree G(\avec) G(\bvec) \tpvc_{ij}(\rvec + \avec - \bvec) \, d\avec d\bvec.
\end{equation}
The filtered spectrum is then defined as
\begin{equation}
\label{eq:filt_vst}
\overline{\est}_{ij}(\kvec) = \frac{1}{ (2\pi)^3 } \int_\Rthree \overline{\tpvc}_{ij} (\rvec) e^{-i \kvec \cdot \rvec} \, d\rvec.
\end{equation}
Thus, by plugging in \cref{eq:filt_tpvc} into \cref{eq:filt_vst} one can show that the filtered spectrum is related to the velocity-spectrum tensor as follows
\begin{equation}
\overline{\est}_{ij} (\kvec) = \hat{G}(\kvec)^2 \est_{ij}(\kvec).
\end{equation}
If we assume the filter is isotropic in real space, then it'll also be isotropic in spectral space. Thus the above is rewritten as $\overline{\est}_{ij} (\kvec) = \hat{G}( \kappa )^2 \est_{ij}(\kvec)$, which allows us to obtain
\begin{equation}
\overline{\est}(\kappa) = \hat{G}(\kappa)^2 \est(\kappa).
\end{equation}

The most common filters are the box filter 
\begin{equation}
G( r ) = \frac{6 }{\pi  \Delta^3} H ( \frac{\Delta}{2} - | r | ),
\end{equation}
the Gaussian filter
\begin{equation}
G( r ) = \frac{ 1 }{ ( 2 \pi \sigma^2 )^{3/2} } \exp \left( -\frac{ r^2 }{ 2 \sigma^2  } \right ),
\end{equation}
with $\sigma^2 = \Delta^2/12$, and the Sharp spectral filter
\begin{equation}
\hat{G}( \kappa ) = H( \frac{ \pi }{ \Delta } - | \kappa | ).
\end{equation}
A few important aspects need to be noted. All of the filters above are parameterized by the filter width $\Delta$. Also, the box filter represents an average over a sphere instead of a box. Finally, the spectral filter annihilitates all Fourier modes with wave-vector magnitude greater than $\kappa_c = \pi / \Delta$, where $\kappa_c$ is labeled the cut-off wavenumber.


%-------------------------------------------------------------------------------
\section{Dynamics of filtered variables}
%-------------------------------------------------------------------------------


%--------------------------------------------
\subsection{Rate-of-strain tensor}
%--------------------------------------------
The filtered rate-of-strain tensor is given by
\begin{equation}
\overline{S}_{ij} = \frac{1}{2} \left ( \frac{\partial \ures_i}{\partial x_j} + \frac{\partial \ures_j}{\partial x_i} \right ),
\end{equation}
and the characteristic filtered rate of strain is given by $\overline{S} = \left ( 2 \overline{S}_{ij} \overline{S}_{ij} \right ) ^{1/2}$. 

Using \cref{eq:strain_mag_spectral} for a filtered isotropic turbulent field, we obtain
\begin{equation}
\langle \overline{S}^2 \rangle = 2 \int_0^\infty \kappa^2 \overline{E}(\kappa) \,d\kappa = 2 \int_0^\infty \kappa^2 \hat{G}(\kappa)^2 E(\kappa) \,d\kappa.
\end{equation}
The factor $\kappa^2$ attenuates the integrand for small wave numbers, whereas the term $\hat{G}^2(\kappa)$ suppresses the integrand for wave numbers higher than the cuttoff wave number $\kappa_c$. Thus, most of the contributions towards $\langle \overline{S}^2 \rangle$ come from wave numbers around $\kappa_c$. Assuming $\kappa_c$ is in the inertial subrange, once can use the Kolmogorov spectrum $E = C \epsilon^{2/3} \kappa^{-5/3}$ to obtain
\begin{equation}
\label{eq:S2_estimate}
\langle \overline{S}^2 \rangle = \left ( 2 \int_0^\infty \kappa^{1/3} \hat{G}(\kappa)^2 \,d\kappa \right ) C \epsilon^{2/3} = a_f C \epsilon^{2/3} \Delta^{-4/3},
\end{equation}
where 
\begin{equation}
a_f = 2 \int_0^\infty ( \kappa \Delta )^{1/3} \hat{G}(\kappa)^2 \Delta \, d\kappa 
\end{equation}
A transformation of variables reveals that
\begin{equation}
a_f = 2 \int_0^\infty s^{1/3} \hat{G} (s/\Delta )^2\, ds.
\end{equation}
Since $\kappa$ is always multiplied by $\Delta$ in the definitions of $\hat{G}$, $a_f$ becomes independent of $\Delta$, and dependent on the type of filter only.

%-------------------------------------------------------------------------------
\subsection{Filtered Navier-Stokes equations}
%-------------------------------------------------------------------------------

Assuming a homogeneous filter, spatial differentiation and filtering commute. Thus, the continuity equation follows
\begin{equation}
\frac{\partial \uiavg}{\partial x_i} = 0,
\end{equation}
and the momentum equation in non-conservative form is
\begin{equation}
\frac{\partial \overline{U}_i}{\partial t} + \overline{U}_j \frac{\partial \overline{U}_i}{\partial x_j} = -\frac{1}{\rho} \frac{\partial \overline{P}}{\partial x_i} + \nu \frac{\partial^2 \overline{U}_i}{\partial x_j \partial x_j} - \frac{\partial \rs_{ij}}{\partial x_j}.
\end{equation}
In the above, $\rs_{ij} = \overline{ U_i U_j} - \overline{U}_i \overline{U}_j$ is the residual-stress or subgrid-scale (SGS) tensor.

%--------------------------------------------
\subsection{Residual stresses}
%--------------------------------------------
As states in section blah blah, the residual or subgrid-scale stress is given by $\rs_{ij} = \overline{ U_i U_j} - \ures_i \ures_j$. The residual kinetic energy is $k_r = \frac{1}{2} \rs_{ii}$ and the anisotropic residual-stress tensor is defined using $\rs_{ij} = \frac{2}{3} k_r \delta_{ij} + \ars_{ij}$.

Two common decompositions of the residual stresses are those of Leonard (1974)
\begin{equation}
\rs_{ij} = L_{ij} + C_{ij} + R_{ij},
\end{equation}
and Germano (1986)
\begin{equation}
\rs_{ij} = \mathcal{L}^\circ_{ij} + \mathcal{C}^\circ_{ij} + \mathcal{R}^\circ_{ij}.
\end{equation}
The tensors 
\begin{equation}
L_{ij} = \qquad \text{and} \qquad \mathcal{L}^\circ_{ij} = 
\end{equation}
are the Leonard stresses, 
\begin{equation}
C_{ij} = \qquad \text{and} \qquad \mathcal{C}^\circ_{ij} = 
\end{equation}
the cross tresses, and 
\begin{equation}
R_{ij} = \qquad \text{and} \qquad \mathcal{R}^\circ_{ij} = 
\end{equation}
the SGS Reynolds stresses. In general $L_{ij}$ and $C_{ij}$ are not Galilean invariant, and thus the Germano decomposition is preferred.

%--------------------------------------------
\subsection{Kinetic energy}
%--------------------------------------------
A decomposition of the filtered kinetic energy is obtained in analogy to the RANS case. That is
\begin{equation}
\text{RANS:}\quad \langle E \rangle = \bar{E} + k \qquad \text{LES:} \quad \overline{E} = E_f + k_r,
\end{equation}
where $\overline{E}$ is the filtered kinetic energy $\frac{1}{2} \overline{ U_k U_k}$, $E_f$ is the kinetic energy of the filtered velocity field $\frac{1}{2} \ures_k \ures_k$ and $k_r$ is the residual kinetic energy.

Below we compare the transport equations for $\bar{E}$ and $E_f$
\begin{equation}
\text{RANS:} \quad \frac{\bar{D} \bar{E}}{\bar{D} t} = - \mathcal{P} - \bar{\epsilon} - \frac{\partial}{\partial x_i} \left ( \ujavg \rs_{ij} + \uiavg \frac{\pavg}{\rho} - 2\nu \ujavg \bar{S}_{ij} \right ),
\end{equation}
\begin{equation}
\text{LES:} \quad \frac{\overline{D} E_f}{ \overline{D} t} = - \mathcal{P}_r - \epsilon_f - \frac{\partial}{\partial x_i} \left ( \ures_j \rs_{ij} + \ures_i \frac{\pres}{\rho} - 2 \nu \ures_j \overline{S}_{ij} \right ) ,
\end{equation}
where 
\begin{equation}
\mathcal{P} = -\rs_{ij} \bar{S}_{ij} \qquad \bar{\epsilon} = 2 \nu \bar{S}_{ij} \bar{S}_{ij}
\end{equation}
and
\begin{equation}
\mathcal{P}_r = -\rs_{ij} \overline{S}_{ij} \qquad \epsilon_f = 2 \nu \overline{S}_{ij} \overline{S}_{ij}.
\end{equation}

%-------------------------------------------------------------------------------
\section{The Smagorinsky model}
%-------------------------------------------------------------------------------
In an analogy to the Boussinesq approximation of RANS models, the Smagorinsky model follows
\begin{equation}
a_{ij} = -2 \nu_r \overline{S}_{ij}
\end{equation}
where $\nu_r$ is the eddy viscosity of the residual motions. In analogy now to the mixing length hypothesis, the eddy viscosity is expressed as follows
\begin{equation}
\nu_r = l_S^2 \overline{S}
\end{equation}
where $l_S = C_S \Delta$ is the length scale and $C_s$ is the Smagorinsky coefficient. Thus, the expected rate of production of residual kinetic energy is given by $\langle P_r \rangle = l_S^2 \langle \overline{S}^3 \rangle$.

The scaling $l_S \sim \Delta$ that results from the definition can be proved by assuming that $\langle P_r \rangle = \epsilon$. In this case
\begin{align}
l_S = \left ( \frac{\epsilon}{ \langle \overline{S}^3 \rangle} \right ) ^{1/2} = \left ( \frac{ \left (\frac{\langle \overline{S}^2 \rangle \Delta^{4/3} }{a_f C} \right )^{3/2}}{ \langle \overline{S}^3 \rangle} \right ) ^{1/2} = \frac{\Delta}{ (a_f C)^{3/4}} \left ( \frac{ \langle \overline{S}^2 \rangle^{3/2}}{\langle \overline{S}^3 \rangle} \right )^{1/2},
\end{align}
where we have used \cref{eq:S2_estimate} to replace $\epsilon$. Assuming $\langle \overline{S} \rangle^{3/2} = \langle \overline{S}^3 \rangle$ and using the value of $a_f$ corresponding to the sharp spectral filter one obtains
\begin{equation}
l_S \approx 0.17 \Delta,
\end{equation}
which allows us to identify the value of $0.17$ for $C_S$.

\section{Numerical aspects LES}
For explicit LES, the physical modeling and numerics are separate issues, whereas for implicit LES these two aspects are entwined. Regarding the numerical aspects of explicit LES, an appropriate grid resolution and numerical scheme is required to accurately capture the time evolution of $\Wvec$. One of the first questions to ask is then how refined should the mesh be? The answer depends, among other factors, on the filter width $\Delta$. For example, the larger the value of $\Delta$, the smoother the filtered velocity is, and therefore the coarser the mesh can be. Thus, before one can answer the question of how refined the mesh should be, one first needs to address the question of how to determine $\Delta$.

A user could employ different methods to specify the parameter $\Delta$ of the simulation of interest. One method would be to pick $\Delta$ so that it falls in the inertial range. This is in agreement with the basis of LES, that is, to resolve the large scales in the energy-containing range, while modeling the remaining smaller scales in the universal equilibrium range. However, it is difficult to determine a priori the bounds of the inertial range, and thus a $\Delta$ within those bounds. An alternate method is the one briefly described in Pope (Ten questions concerning the large-eddy simulation of turbulent flows, New Journal of Physics, 2004). In this case, $\Delta$ is adaptively computed so that a specified percentage of turbulent kinetic energy is resolved at any given point. For example, if we want to resolve 80\% of the turbulent kinetic energy, then $\Delta$ would be adaptively changed until the ratio
\begin{equation}
M(\xvec,t) = \frac{k_r(\xvec,t)}{K(\xvec,t) + k_r(\xvec,t)}
\end{equation}
reaches a value just below 0.2, where $k_r$ is the turbulent kinetic energy of the residual motions, and $K$ the turbulent kinetic energy of the filtered velocity ($K = \frac{1}{2} w_i w_i$). The challenge with this approach is that a good method to compute $K$ and estimate $k_r$ are needed.

Once $\Delta$ is specified, and we thus have a way to gauge how much the resolved scales fluctuate, one can then tackle the issue of how refined the mesh should be so that the numerical error introduced by the numerical scheme is minimized. This of course depends on the numerical method being used. When using low-order numerical schemes, the grid would need to be finer to provide a similar amount of numerical error than when using a high-order numerical scheme. We will first consider spectral schemes, which, for the appropriate problems, provide the lowest numerical error. 

Assume we have used a sharp spectral filter with filter width $\Delta$ on a one-dimensional flow field over a domain of length $L$. Thus, modes with wavevector up to $\kappa_c = \frac{\pi}{\Delta}$ are present in the filtered velocity field, and need to be captured by the numerical scheme. If we use a grid with spacing $h$, then the largest mode that the spectral scheme can capture is $\kappa_{N/2} = \frac{\pi}{h}$. Thus, we want to pick a value of $h$ so that $\kappa_{N/2}=\kappa_c$, which means $h$ needs to be equal to $\Delta$.

If we instead use a non-spectral scheme, a finer mesh spacing would be needed ...


%###############################################################################
%
\chapter{Direct Numerical Simulation}
%
%###############################################################################

%-------------------------------------------------------------------------------
\section{Resolution requirements}
%-------------------------------------------------------------------------------

\subsection{All flows}
\begin{itemize}
\item Compute two point correlations, which give an indication of the spatial extent of the structures. Make sure the distance from the peak of the correlation to where it levels off to zero is at most half of the domain, since the size of the structures is roughly twice this distance.

\end{itemize}

\subsection{Homogeneous flows}
\begin{itemize}
\item For isotropic turbulence $\mathcal{L} > 8 L_{11}$, where $\mathcal{L}$ is the size of the box and $L_{11}$ is the integral length scale. This leads to
\begin{equation}
\kappa_0 L_{11} = \frac{2 \pi}{\mathcal{L}} L_{11} < \frac{\pi}{4} .
\end{equation}

\item To account for the smallest scales $\kappa_{\text{max}} \eta > 1.5$ This leads to
\begin{equation}
\Delta x = \frac{\mathcal{L}}{N} = \frac{ \mathcal{L}2 }{2\pi N} \pi = \frac{\pi}{\kappa_{\text{max}}} < \frac{\pi}{1.5} \eta .
\end{equation}
\end{itemize}

\subsection{Inhomogeneous flows}
\begin{itemize}
\item For wall bounded flows check $\Delta x^+$, $\Delta y^+$, and $\Delta z^+$ to make sure they are within a ``recommended'' range.
\end{itemize}

For the interior of the flow check that $h/\eta < 12$ where $h = (h_x h_y h_z)^{1/3}$ so that you have at least two points to resolve scales of size $24 \eta$. Those scales correspond to a wave number $0.26/\eta$, which, given isotropic turbulence and a carefully devised model spectrum, Pope (2001) shows is the value at which the maximum dissipation takes place.


%%%%%%%%%%%%%%%%%%%%%%%%%%%%%%%%%%%%%%%%%%%%%%%%%%
%																			                    %
%																			                    %
\appendix																		             %
%																			                    %
%																			                    %
%%%%%%%%%%%%%%%%%%%%%%%%%%%%%%%%%%%%%%%%%%%%%%%%%%

%###############################################################################
%
\chapter{Vectors in rotating reference frames}
%
%###############################################################################

%-------------------------------------------------------------------------------
\section{Basic properties}
%-------------------------------------------------------------------------------
%--------------------------------------------
\subsection{Position}
%--------------------------------------------
Consider the Eucledian transformation
\begin{equation}
\label{eq:x_rot}
\tilde{x}^+_i(\tilde{t}) = Q_{ij}(\tilde{t}-c) x^+_j(\tilde{t}-c) + b_i(\tilde{t}-c) 
\end{equation}
where $x^+(t)$ is the Lagrangian position of a fluid particle, and $\tilde{x}^+(\tilde{t})$ is the Lagrangian position of the same particle but in a rotating reference frame. The transformation above amounts to a rotation of the reference frame determined by the orthogonal matrix $\bold{Q}(t)$, a translation of the reference frame determined by the vector $\bold{b}(t)$, and a translation in time given by $\tilde{t} = t + c$. 

%--------------------------------------------
\subsection{Velocity}
%--------------------------------------------
Since the velocity of a fluid particle is given by 
\begin{equation}
u^+_i(t) = \frac{d x^+_i(t)}{dt},
\end{equation}
the components of velocity in the non-inertial reference frame are defined as
\begin{equation}
\tilde{u}^+_i(\tilde{t}) = \frac{d \tilde{x}^+_i(\tilde{t})}{d \tilde{t}} .
\end{equation}
Thus, the relationship between the velocity components in the different reference frames is given by
\begin{align}
\label{eq:rot_vel_general}
\tilde{u}^+_i(\tilde{t}) &= \frac{d}{d\tilde{t}} [Q_{ij}(\tilde{t} - c) x^+_j(\tilde{t} - c) + b_i(\tilde{t} - c)] \nonumber \\
& = Q_{ij}(\tilde{t} - c) U^+_j(\tilde{t} - c) + \dot{Q}_{ij}(\tilde{t} - c) x^+_j(\tilde{t} - c) + \dot{b}_i(\tilde{t} - c).
\end{align}
If we assume there is no translation in space or time, the above reduces to
\begin{equation}
\label{eq:rot_vel_inter}
\tilde{u}^+_i = Q_{ij} u^+_j + \dot{Q}_{ij} x^+_j, 
\end{equation}
where each quantity above depends on time $t$. Multiplying by $Q^{-1}_{ki}$ we obtain
\begin{equation}
Q_{ik} \tilde{u}^+_i = u^+_k + Q_{ik}\dot{Q}_{ij}x^+_j.
\end{equation}
Noting that the tensor $Q_{ik} \dot{Q}_{ij}$ is antisymmetric, we obtain
\begin{equation}
\label{eq:rot_vel}
u^+_k = Q_{ik} \tilde{u}^+_i + Q_{ij}\dot{Q}_{ik}x^+_j.
\end{equation}

We now define the angular velocity tensor $\Omega_{ij}$ and the angular velocity vector $\Omega_i$ using the following relationship 
\begin{equation}
\label{eq:angular_definition}
\Omega_{ij} = \epsilon_{jik}\Omega_{k} = Q_{kj}\dot{Q}_{ki}.
\end{equation}
Thus (\ref{eq:rot_vel}) can be expressed in terms of the angular velocity tensor as
\begin{equation}
\label{eq:rot_vel_angular_tensor}
u^+_k = Q_{ik} \tilde{u}^+_i + \Omega_{kj}x^+_j,
\end{equation}
or in terms of the angular velocity vector as
\begin{equation}
\label{rot_vel_angular_vector}
u^+_k = Q_{ik} \tilde{u}^+_i + \epsilon_{jki}\Omega_ix^+_j.
\end{equation}

%--------------------------------------------
\subsection{Acceleration}
%--------------------------------------------
We assume again no translation in space or time. Taking the derivative of both sides of \cref{eq:rot_vel}
\begin{equation}
    \frac{du^+_k}{dt} = \dot{Q}_{ik} \tilde{u}^+_i + Q_{ik} \frac{d\tilde{u}^+_i}{dt} + \dot{Q}_{ik} \frac{dQ_{ij} x^+_j}{dt} + Q_{ij}\ddot{Q}_{ik} x^+_j.
\end{equation}
Using \cref{eq:x_rot} the above is re-written as
\begin{equation}
    \frac{du^+_k}{dt} = Q_{ik} \frac{d\tilde{u}^+_i}{dt} + 2 \dot{Q}_{ik} \tilde{u}^+_i + Q_{ij}\ddot{Q}_{ik} x^+_j.
\end{equation}
Re-writing the last term on the right-hand-side above leads to
\begin{equation}
    \frac{du^+_k}{dt} = Q_{ik} \frac{d\tilde{u}^+_i}{dt} + 2 \dot{Q}_{ik} \tilde{u}^+_i + \frac{d Q_{ij}\dot{Q}_{ik}}{dt} x^+_j - \dot{Q}_{ij} \dot{Q}_{ik} x^+_j.
\end{equation}
We now show that
\begin{align}
    \dot{Q}_{ij} \dot{Q}_{ik} & = \delta_{pi}\dot{Q}_{pj} \dot{Q}_{ik}  \nonumber \\
    & = Q_{pq}Q^{-1}_{qi} \dot{Q}_{pj} \dot{Q}_{ik}  \nonumber \\
    & = Q_{pq} \dot{Q}_{pj} Q_{iq} \dot{Q}_{ik} .
\end{align}
Thus, in terms of the angular velocity tensor, we have
\begin{equation}
    \frac{du^+_k}{dt} = Q_{ik} \frac{d\tilde{u}^+_i}{dt} + 2 \dot{Q}_{ik} \tilde{u}^+_i + \frac{d \Omega_{kj}}{dt} x^+_j - \Omega_{kq} \Omega_{jq} x^+_j.
\end{equation}
In terms of the angular velocity vector, the above becomes
\begin{equation}
    \frac{du^+_k}{dt} = Q_{ik} \frac{d\tilde{u}^+_i}{dt} + 2 \dot{Q}_{ik} \tilde{u}^+_i + \epsilon_{jki}\frac{d \Omega_{i}}{dt} x^+_j - \epsilon_{qki}\Omega_{i} \epsilon_{qjp}\Omega_{p} x^+_j,
\end{equation}
which, upon a simple re-arrangement of indices, gives
\begin{equation}
\label{eq:rot_acl_angular_vector}
    \frac{du^+_k}{dt} = Q_{ik} \frac{d\tilde{u}^+_i}{dt} + 2 \dot{Q}_{ik} \tilde{u}^+_i + \epsilon_{kij}\frac{d \Omega_{i}}{dt} x^+_j + \epsilon_{kiq}\Omega_{i} \epsilon_{qpj}\Omega_{p} x^+_j.
\end{equation}

%-------------------------------------------------------------------------------
\section{Unit vectors}
%-------------------------------------------------------------------------------
Let the orthogonal basis for the inertial reference frame be denoted by $\hat{a}^{(1)}$, $\hat{a}^{(2)}$, and $\hat{a}^{(3)}$, and the orthogonal basis for the rotating reference frame by $\hat{b}^{(1)}$, $\hat{b}^{(2)}$, and $\hat{b}^{(3)}$. We know that the unit vector $\hat{b}^{(i)}$ has components $\tilde{b}_j^{(i)}$ in the rotating reference frame, and components $b_j^{(i)}$ in the inertial reference frame. The vector is the same whether expressed in the rotating or inertial reference frames, that is
\begin{equation}
\tilde{b}_k^{(i)} \hat{b}^{(k)} = b_k^{(i)} \hat{a}^{(k)}
\end{equation}
Since $\tilde{b}_k^{(i)} = \delta_{ik}$, we rewrite the above as
\begin{equation}
\label{eq:unit_trans}
\hat{b}^{(i)} = b_k^{(i)} \hat{a}^{(k)} = Q_{jk} \tilde{b}_j^{(i)} \hat{a}^{(k)} = Q_{ik} \hat{a}^{(k)}.
\end{equation}
Doting both sides by $\hat{a}^{(j)}$ shows that 
\begin{equation}
Q_{ij} = \hat{a}^{(j)} \cdot \hat{b}^{(i)}.
\end{equation}

One can also multiply both sides of \cref{rot_vel_angular_vector} by $\hat{a}^{(k)}$ and use \cref{eq:unit_trans} to obtain
\begin{equation}
    \label{eq:uine_urot_temp}
u^+_k \hat{a}^{(k)} = \tilde{u}^+_i \hat{b}^{(i)} + \epsilon_{jki}\Omega_ix^+_j \hat{a}^{(k)}.
\end{equation}
If we introduce the following notation
\begin{align}
    \vec{u}^+_{ine} &= u^+_k \hat{a}^{(k)}, \\
    \vec{u}^+_{rot} &= \tilde{u}^+_i \hat{b}^{(i)}, \\
    \vec{\Omega} &= \Omega_i \hat{a}^{(i)}, \\
    \vec{x}^+ &= x_i^+ \hat{a}^{(i)}.
\end{align}
then the \cref{eq:uine_urot_temp} is writen as 
\begin{equation}
\label{uine_urot}
\vec{u}^+_{ine}  = \vec{u}^+_{rot}  + \vec{\Omega} \times \vec{x}^+.
\end{equation}
Similarly, multiplying both sides of \cref{eq:rot_acl_angular_vector} gives
\begin{equation}
\label{eq:aine_arot_inter}
    \frac{du^+_k}{dt} \hat{a}^{(k)} = \frac{d\tilde{u}^+_i}{dt}\hat{b}^{(i)} + 2 \dot{Q}_{ik} \tilde{u}^+_i \hat{a}^{(k)} + \epsilon_{kij}\frac{d \Omega_{i}}{dt} x^+_j \hat{a}^{(k)} + \epsilon_{kiq}\Omega_{i} \epsilon_{qpj}\Omega_{p} x^+_j \hat{a}^{(k)}.
\end{equation}
Using \cref{eq:angular_definition} one can show that $\dot{Q}_{ik} = \epsilon_{pjk} \Omega_p Q_{ij}$. Thus, the second term on the left-hand-side above can be written as
\begin{align}
    2 \dot{Q}_{ik} \tilde{u}^+_i \hat{a}^{(k)} &= 2 \epsilon_{pjk} \Omega_p Q_{ij} \tilde{u}^+_i \hat{a}^{(k)} \nonumber \\
    &= 2 \Omega_p Q_{ij} \tilde{u}^+_i \hat{a}^{(p)} \times \hat{a}^{(j)} \nonumber \\
    &= 2 \Omega_p \hat{a}^{(p)} \times \tilde{u}^+_i \hat{b}^{(i)}.
\end{align}
Therefore, in vector notation \cref{eq:aine_arot_inter} becomes
\begin{equation}
    \label{eq:aine_arot}
    \vec{a}^+_{ine} = \vec{a}^+_{rot} + 2 \vec{\Omega} \times \vec{u}^+_{rot} + \dot{\vec{\Omega}} \times \vec{x}^+ + \vec{\Omega} \times (\vec{\Omega} \times \vec{x}^+),
\end{equation}
where $\vec{a}^+_{ine} = \frac{du^+_k}{dt} \hat{a}^{(k)}$, $\vec{a}^+_{rot} = \frac{d\tilde{u}^+_i}{dt}\hat{b}^{(i)}$, and $\dot{\vec{\Omega}} = \frac{d\Omega_i}{dt} \hat{a}^{(i)}$.

%-------------------------------------------------------------------------------
\section{Eulerian variables}
%-------------------------------------------------------------------------------
We now introduce the Eulerian counterpart to the Lagrangian variables. In the inertial reference frame these are $\rho(t,\xvec)$, $u_i(t,\xvec)$, and $p(t,\xvec)$. They are defined by the following expressions
\begin{align}
    \rho^+ &= \rho(t,\xvec^+) \\
    u_i^+ &= u_i(t,\xvec^+) \\
    p^+ &= p(t,\xvec^+).
\end{align}
Similarly for the rotating reference frame, we have $\tilde{\rho}(t,\tilde{\xvec})$, $\tilde{u}_i(t,\tilde{\xvec})$, and $\tilde{p}(t,\tilde{\xvec})$. These are defined by
\begin{align}
    \tilde{\rho}^+ &= \tilde{\rho}(t,\tilde{\xvec}^+) \\
    \tilde{u}_i^+ &= \tilde{u}_i(t,\tilde{\xvec}^+) \\
    \tilde{p}^+ &= \tilde{p}(t,\tilde{\xvec}^+).
\end{align}
We now use the transformation rules for the Lagrangian variables to derive the transformation rules for the Eulerian variables. The transformation rules for the Lagrangian variables are
\begin{align}
    \rho^+ &= \tilde{\rho}^+ \\
    u_k^+ &= Q_{ik} \tilde{u}_i^+ + Q_{ij} \dot{Q}_{ik} x_j^+ \\
    p^+ &= \tilde{p}^+.
\end{align}
The second equation above is the previously derived \cref{eq:rot_vel}. Using the definition of the Eulerian variables, the above is re-written as
\begin{align}
    \rho(t,\xvec^+) &= \tilde{\rho}(t,\tilde{\xvec}^+) \label{eq:eul_rho_temp} \\
    u_i(t,\xvec^+) &= Q_{ik} \tilde{u}_i(t,\tilde{\xvec}^+)  + Q_{ij} \dot{Q}_{ik} x_j^+ \label{eq:eul_vel_temp}\\
    p(t,\xvec^+) &= \tilde{p}(t,\tilde{\xvec}^+). \label{eq:pres_temp}
\end{align}
Using \cref{eq:x_rot}, we get
\begin{align}
    \rho(t,\xvec^+) &= \tilde{\rho}(t,\Qvec \xvec^+) \\
    u_i(t,\xvec^+) &= Q_{ik} \tilde{u}_i(t,\Qvec \xvec^+)  + Q_{ij} \dot{Q}_{ik} x_j^+ \\
    p(t,\xvec^+) &= \tilde{p}(t,\Qvec \xvec^+)+.
\end{align}
Since the above holds for any $\xvec^+$, we finally re-write it as
\begin{align}
    \rho(t,\xvec) &= \tilde{\rho}(t,\Qvec \xvec) \label{eq:eul_rho_trans}\\
    u_i(t,\xvec) &= Q_{ik} \tilde{u}_i(t,\Qvec \xvec)  + Q_{ij} \dot{Q}_{ik} x_j \label{eq:eul_vel_trans}\\
    p(t,\xvec) &= \tilde{p}(t,\Qvec \xvec). \label{eq:eul_pres_trans}
\end{align}

%-------------------------------------------------------------------------------
\section{Rotating Navier-Stokes equations}
%-------------------------------------------------------------------------------
We first obtain a few set of relationships that will later be used in deriving the rotating Navier-Stokes equations. Taking the derivative of \cref{eq:eul_pres_trans} gives
\begin{align}
    \frac{\partial p(t,\xvec)}{\partial x_i} &= \frac{ \partial \tilde{p}(t,\Qvec \xvec)}{\partial x_i} \nonumber \\
    &= \frac{\partial}{\partial x_i} \tilde{p}(t,Q_{1j}x_j, Q_{2j}x_j, Q_{3j} x_3) \nonumber \\
    &= \frac{\partial Q_{1j}x_j}{\partial x_i} \left [ \frac{\partial \tilde{p}(t,\tilde{\xvec})}{\partial \tilde{x}_1} \right]_{ \tilde{\xvec} = \Qvec \xvec} + \frac{\partial Q_{2j}x_j}{\partial x_i} \left [ \frac{\partial \tilde{p}(t,\tilde{\xvec})}{\partial \tilde{x}_2} \right ]_{\tilde{\xvec} = \Qvec \xvec} + \frac{\partial Q_{3j}x_j}{\partial x_i} \left [ \frac{\partial \tilde{p}(t,\tilde{\xvec})}{\partial \tilde{x}_3} \right ]_{\tilde{\xvec} = \Qvec \xvec} \nonumber \\
    &= Q_{ji} \left [ \frac{\partial \tilde{p}(t,\tilde{\xvec})}{\partial \tilde{x}_j} \right ]_{\tilde{\xvec} = \Qvec \xvec}
\end{align}
If we evaluate the above at $\xvec = \xvec^+$, we get
\begin{equation}
    \left [ \frac{\partial p(t,\xvec)}{\partial x_i} \right]_{\xvec = \xvec^+} = Q_{ji} \left [ \frac{\partial \tilde{p}(t,\tilde{\xvec})}{\partial \tilde{x}_j} \right]_{\tilde{\xvec} = \tilde{\xvec}^+}.
\end{equation}
Multiplying both sides by $\hat{a}^{(i)}$ gives
\begin{equation}
\label{eq:ns_pressure_rot_transform}
    \left [ \frac{\partial p(t,\xvec)}{\partial x_i} \right]_{\xvec = \xvec^+} \hat{a}^{(i)} = \left [ \frac{\partial \tilde{p}(t,\tilde{\xvec})}{\partial \tilde{x}_j} \right]_{\tilde{\xvec} = \tilde{\xvec}^+} \hat{b}^{(j)}.
\end{equation}
For the shear-stress tensor, we have
\begin{equation}
    \label{eq:ns_shear_stress_rot_transform}
    \left [ \frac{\partial \tau_{ij}(t,\xvec)}{\partial x_j} \right]_{\xvec = \xvec^+} \hat{a}^{(i)} = \left [ \frac{\partial \tilde{\tau}_{ij}(t,\tilde{\xvec})}{\partial \tilde{x}_j} \right]_{\tilde{\xvec} = \tilde{\xvec}^+} \hat{b}^{(i)}.
\end{equation}
We also use \cref{eq:aine_arot} to write
\begin{align}
    \frac{du_i^+}{dt} \hat{a}^{(i)} &= \frac{d\tilde{u}_i^+}{dt} \hat{b}^{(i)} + 2 \vec{\Omega} \times \left ( \tilde{u}_i^+ \hat{b}^{(i)} \right ) + \dot{\vec{\Omega}} \times \vec{x}^+ + \vec{\Omega} \times \left ( \vec{\Omega} \times \vec{x}^+ \right ) \nonumber \\
    &= \frac{d\tilde{u}_i^+}{dt} \hat{b}^{(i)} + 2 \vec{\Omega} \times \left ( \tilde{u}_i^+ \hat{b}^{(i)} \right ) + \dot{\vec{\Omega}} \times \left ( \tilde{x}_i^+ \hat{b}^{(i)}\right ) + \vec{\Omega} \times \left [ \vec{\Omega} \times \left ( \tilde{x}_i^+ \hat{b}^{(i)}\right ) \right ]
\end{align}
Expressing the above in terms of Eulerian quantities gives
\begin{multline}
    \label{eq:aine_arot_eulerian}
    \left [ \frac{\partial u_i(t,\xvec)}{\partial t} + u_j(t,\xvec) \frac{\partial u_i(t,\xvec)}{\partial x_j} \right ]_{\xvec = \xvec^+} \hat{a}^{(i)} = \left [ \frac{\partial \tilde{u}_i(t,\tilde{\xvec})}{\partial t} + \tilde{u}_j(t,\tilde{\xvec}) \frac{\partial \tilde{u}_i(t,\tilde{\xvec})}{\partial \tilde{x}_j} \right ]_{\tilde{\xvec} = \tilde{\xvec}^+} \hat{b}^{(i)} \\
    + 2 \vec{\Omega} \times \left \{ \left [\tilde{u}_i(t,\tilde{\xvec}) \right ]_{\tilde{\xvec} = \tilde{\xvec}^+} \hat{b}^{(i)} \right \} + \dot{\vec{\Omega}} \times \left [ \left ( \tilde{x}_i \right)_{\tilde{\xvec} = \tilde{\xvec}^+} \hat{b}^{(i)}\right ] + \vec{\Omega} \times \left \{ \vec{\Omega} \times \left [ \left ( \tilde{x}_i \right )_{\tilde{\xvec} = \tilde{\xvec}^+} \hat{b}^{(i)}\right ] \right \}
\end{multline}

The conservation of momentum equation is given by
\begin{equation}
    \rho(t,\xvec) \left [\frac{\partial u_i(t,\xvec)}{\partial t} + u_j(t,\xvec) \frac{\partial u_i(t,\xvec)}{\partial x_j} \right ] = -\frac{\partial p(t,\xvec)}{\partial x_i} + \frac{\partial \tau_{ij}(t,\xvec)}{\partial x_j}.
\end{equation}
Evaluating the above at $\xvec = \xvec^+$ can be written as
\begin{equation}
    \rho(t,\xvec^+) \left [\frac{\partial u_i(t,\xvec)}{\partial t} + u_j(t,\xvec) \frac{\partial u_i(t,\xvec)}{\partial x_j} \right ]_{\xvec = \xvec^+} = -\left [ \frac{\partial p(t,\xvec)}{\partial x_i} \right]_{\xvec = \xvec^+} + \left [ \frac{\partial \tau_{ij}(t,\xvec)}{\partial x_j} \right ]_{\xvec = \xvec^+}.
\end{equation}
Multiplying both sides by $\hat{a}^{(i)}$ gives
\begin{multline}
    \rho(t,\xvec^+) \left [\frac{\partial u_i(t,\xvec)}{\partial t} + u_j(t,\xvec) \frac{\partial u_i(t,\xvec)}{\partial x_j} \right ]_{\xvec = \xvec^+} \hat{a}^{(i)} \\
    = -\left [ \frac{\partial p(t,\xvec)}{\partial x_i} \right]_{\xvec = \xvec^+} \hat{a}^{(i)} + \left [ \frac{\partial \tau_{ij}(t,\xvec)}{\partial x_j} \right ]_{\xvec = \xvec^+} \hat{a}^{(i)}.
\end{multline}
Using \cref{eq:eul_rho_temp,eq:ns_pressure_rot_transform,eq:ns_shear_stress_rot_transform} gives
\begin{multline}
    \left [ \tilde{\rho}(t,\tilde{\xvec}) \right]_{\tilde{\xvec} = \tilde{\xvec}^+} \left [\frac{\partial u_i(t,\xvec)}{\partial t} + u_j(t,\xvec) \frac{\partial u_i(t,\xvec)}{\partial x_j} \right ]_{\xvec = \xvec^+} \hat{a}^{(i)} \\
    = -\left [ \frac{\partial \tilde{p}(t,\tilde{\xvec})}{\partial \tilde{x}_i} \right]_{\tilde{\xvec} = \tilde{\xvec}^+} \hat{b}^{(i)} + \left [ \frac{\partial \tilde{\tau}_{ij}(t,\tilde{\xvec})}{\partial \tilde{x}_j} \right ]_{\tilde{\xvec} = \tilde{\xvec}^+} \hat{b}^{(i)}.
\end{multline}
Using \cref{eq:aine_arot_eulerian} we obtain
\begin{multline}
    \left [ \tilde{\rho}(t,\tilde{\xvec}) \right]_{\tilde{\xvec} = \tilde{\xvec}^+} \left ( \left [ \frac{\partial \tilde{u}_i(t,\tilde{\xvec})}{\partial t} + \tilde{u}_j(t,\tilde{\xvec}) \frac{\partial \tilde{u}_i(t,\tilde{\xvec})}{\partial \tilde{x}_j} \right ]_{\tilde{\xvec} = \tilde{\xvec}^+} \hat{b}^{(i)} + 2 \vec{\Omega} \times \left \{ \left [\tilde{u}_i(t,\tilde{\xvec}) \right ]_{\tilde{\xvec} = \tilde{\xvec}^+} \hat{b}^{(i)} \right \} \right . \\
    \left . + \dot{\vec{\Omega}} \times \left [ \left ( \tilde{x}_i \right)_{\tilde{\xvec} = \tilde{\xvec}^+} \hat{b}^{(i)}\right ] + \vec{\Omega} \times \left \{ \vec{\Omega} \times \left [ \left ( \tilde{x}_i \right )_{\tilde{\xvec} = \tilde{\xvec}^+} \hat{b}^{(i)}\right ] \right \} \right ) = -\left [ \frac{\partial \tilde{p}(t,\tilde{\xvec})}{\partial \tilde{x}_i} \right]_{\tilde{\xvec} = \tilde{\xvec}^+} \hat{b}^{(i)} + \left [ \frac{\partial \tilde{\tau}_{ij}(t,\tilde{\xvec})}{\partial \tilde{x}_j} \right ]_{\tilde{\xvec} = \tilde{\xvec}^+} \hat{b}^{(i)}.
\end{multline}
Finally, since this holds for any $\tilde{\xvec}^+$, we get
\begin{multline}
    \tilde{\rho}(t,\tilde{\xvec}) \left \{ \left [ \frac{\partial \tilde{u}_i(t,\tilde{\xvec})}{\partial t} + \tilde{u}_j(t,\tilde{\xvec}) \frac{\partial \tilde{u}_i(t,\tilde{\xvec})}{\partial \tilde{x}_j} \right ] \hat{b}^{(i)} + 2 \vec{\Omega} \times \left [ \tilde{u}_i(t,\tilde{\xvec}) \hat{b}^{(i)} \right ] \right . \\
    \left . + \dot{\vec{\Omega}} \times \left ( \tilde{x}_i \hat{b}^{(i)}\right ) + \vec{\Omega} \times \left [ \vec{\Omega} \times \left ( \tilde{x}_i \hat{b}^{(i)} \right ) \right ] \right \} = -\frac{\partial \tilde{p}(t,\tilde{\xvec})}{\partial \tilde{x}_i} \hat{b}^{(i)} + \frac{\partial \tilde{\tau}_{ij}(t,\tilde{\xvec})}{\partial \tilde{x}_j} \hat{b}^{(i)}.
\end{multline}

%-------------------------------------------------------------------------------
\section{Eulerian variables (OLD!!!!!)}
%-------------------------------------------------------------------------------
We now show the transformation rules for Eulerian rather than Lagrangian variables. The Eulerian velocity in the inertial reference frame $u_i(t,\xvec)$ and the Eulerian velocity in the rotating reference frame $\tilde{u}_i(\tilde{t}, \tilde{\xvec})$ are defined by the following expressions
\begin{equation}
    u_i^+(t) = u_i(t, \xvec^+(t)),
\end{equation}
\begin{equation}
    \tilde{u}^+_i(\tilde{t}) = \tilde{u}_i(\tilde{t},\tilde{\xvec}^+(\tilde{t})).
\end{equation}

Given the definition of the Eulerian velocities, \cref{eq:rot_vel_general} for the velocity transformation becomes
\begin{equation}
     \tilde{u}_i(\tilde{t}, \tilde{\xvec}^+(\tilde{t})) = Q_{ij}(\tilde{t} - c) u_j(\tilde{t} - c,\xvec^+(\tilde{t}-c)) + \dot{Q}_{ij}(\tilde{t} - c) x^+_j(\tilde{t} - c) + \dot{b}_i(\tilde{t} - c).
\end{equation}
We now use \cref{eq:x_rot}, but in the form $x_j^+(\tilde{t} - c) = Q_{kj}(\tilde{t} - c) \tilde{x}_k^+(\tilde{t}) - Q_{kj}(\tilde{t} - c) b_k(\tilde{t}-c)$. Plugging this into the above gives
\begin{align}
\tilde{u}_i(\tilde{t},\tilde{\xvec}^+(\tilde{t})) &= Q_{ij}(\tilde{t} - c) u_j(\tilde{t} - c,\bold{Q}^T(\tilde{t} - c)\tilde{\xvec}^+(\tilde{t}) - \bold{Q}^T(\tilde{t} - c) \bvec(\tilde{t}-c)) \nonumber \\
& + \dot{Q}_{ij}(\tilde{t} - c) Q_{kj}(\tilde{t} -c) \tilde{x}_k^+(\tilde{t}) - \dot{Q}_{ij}(\tilde{t} - c) Q_{kj}(\tilde{t} - c) b_k(\tilde{t} - c) + \dot{b}_i(\tilde{t} - c).
\end{align}
Since the above holds for any $\tilde{\xvec}^+(\tilde{t})$ we obtain
\begin{align}
\tilde{u}_i(\tilde{t},\tilde{\xvec}) &= Q_{ij}(\tilde{t} - c) u_j(\tilde{t} - c,\bold{Q}^T(\tilde{t} - c)\tilde{\xvec} - \bold{Q}^T(\tilde{t} - c) \bvec(\tilde{t}-c)) \nonumber \\
& + \dot{Q}_{ij}(\tilde{t} - c) Q_{kj}(\tilde{t} -c) \tilde{x}_k - \dot{Q}_{ij}(\tilde{t} - c) Q_{kj}(\tilde{t} - c) b_k(\tilde{t} - c) + \dot{b}_i(\tilde{t} - c).
\end{align}
This is re-written in the simpler notation
\begin{equation}
\label{eq:rot_vel_eulerian}
\tilde{u}_i = Q_{ij}u_j + \dot{Q}_{ij} Q_{kj} \tilde{x}_k - \dot{Q}_{ij}Q_{kj}b_k + \dot{b}_i,
\end{equation}
with the assumption that variables in the inertial reference frame are evaluated as follows
\begin{align}
    u_i &= u_i(\tilde{t} - c,\bold{Q}^T\tilde{\xvec} - \bold{Q}^T \bvec) \nonumber \\ 
    Q_{ij} & = Q_{ij}(\tilde{t} - c) \nonumber \\
    \dot{Q}_{ij} &= \left [ \frac{dQ_{ij}(t)}{dt} \right ]_{(\tilde{t} - c)} \nonumber \\
    b_i &= b_i(\tilde{t} - c) \nonumber \\
    \dot{b}_i &= \left [ \frac{db_i(t)}{dt} \right ]_{(\tilde{t} - c)}.
\end{align}

%-------------------------------------------------------------------------------
\section{Rotating turbulence (OLD!!!!!!)}
%-------------------------------------------------------------------------------
Given \cref{eq:rot_vel_eulerian}, the mean velocity in the transforming reference frame is expressed in terms of the mean velocity in the inertial reference as follows
\begin{equation}
\label{rot_mean_vel}
\langle \tilde{U}_i \rangle = Q_{ij} \langle U_j \rangle + \dot{Q}_{ij} Q_{kj} \tilde{x}_k - \dot{Q}_{ij} Q_{kj} b_k + \dot{b}_i.
\end{equation}
The mean velocity in the inertial reference frame in terms of the mean velocity in the transforming reference frame is given by
\begin{equation}
\langle U_q \rangle = Q_{iq} \langle \tilde{U}_i \rangle  - Q_{iq}\dot{Q}_{ij} Q_{kj} \tilde{x}_k + Q_{iq}\dot{Q}_{ij} Q_{kj} b_k - Q_{iq} \dot{b}_i.
\end{equation}
Since $Q_{iq}\dot{Q}_{ij}$ is an anti-symmetric tensor, we have $Q_{iq}\dot{Q}_{ij} = -Q_{ij}\dot{Q}_{iq}$. Using the fact that $Q_{ij}Q_{kj} = \delta_{ki}$ the above becomes
\begin{equation}
\langle U_q \rangle = Q_{iq} \langle \tilde{U}_i \rangle + \dot{Q}_{iq} \tilde{x}_i -  \dot{Q}_{iq}b_i - Q_{iq} \dot{b}_i.
\end{equation}

The fluctuating velocity, which in the inertial reference frame is given by $\uifluc = U_i - \uiavg$, and in the transformed reference frame by $\tilde{u}_i = \tilde{U}_i - \langle \tilde{U}_i \rangle$, then transforms as
\begin{equation}
\tilde{u}_i = Q_{ij} u_j .
\end{equation}
Thus the reynolds stresses in the new reference frame $\tilde{\rs}_{ij} = \langle \tilde{u}_i \tilde{u}_j \rangle$ are given by 
\begin{equation}
\tilde{\rs}_{ij} = Q_{ip}Q_{jq} \rs_{pq}.
\end{equation}

The material derivative of the Reynolds stresses in the rotating reference frame thus becomes
\begin{align}
\frac{D \tilde{\rs}_{ij}}{D \tilde{t}} &= \frac{\partial \tilde{\rs}_{ij} }{\partial \tilde{t}} + \langle \tilde{U}_k \rangle \frac{\partial \tilde{\rs}_{ij} }{\partial \tilde{x}_k} \nonumber \\
& = \dot{Q}_{ip}Q_{jq} \rs_{pq} + Q_{ip} \dot{Q}_{jq} \rs_{pq} + Q_{ip}Q_{jq} \left ( \frac{\partial \rs_{pq} }{\partial \tilde{t}} + \langle \tilde{U}_k \rangle \frac{\partial \rs_{pq} }{\partial \tilde{x}_k} \right ) 
\end{align}
Since the Reynolds stresses $\rs_{pq}$ depend on $(\tilde{t} - c,\bold{Q}^T\tilde{\xvec} - \bold{Q}^T \bvec)$, we have
\begin{align}
\frac{\partial \rs_{pq} }{\partial \tilde{t}} &=  \frac{\partial \rs_{pq} }{\partial t} + (\dot{Q}_{ki} \tilde{x}_k - \dot{Q}_{ki} b_k - Q_{ki}\dot{b}_k) \frac{\partial \rs_{pq} }{\partial x_i}, \nonumber \\
\langle \tilde{U}_k \rangle \frac{\partial \rs_{pq} }{\partial \tilde{x}_k} &= Q_{ki} \langle \tilde{U}_k \rangle \frac{\partial \rs_{pq} }{\partial x_i} .
\end{align}
Thus, we obtain
\begin{align}
\frac{D \tilde{\rs}_{ij} }{D \tilde{t}} &= \dot{Q}_{ip}Q_{jq} \rs_{pq} + Q_{ip} \dot{Q}_{jq} \rs_{pq} + Q_{ip}Q_{jq} \left ( \frac{\partial \rs_{pq} }{\partial t} + \langle U_k \rangle \frac{\partial \rs_{pq} }{\partial x_k} \right ) \nonumber \\
& = \dot{Q}_{ip}Q_{jq} \rs_{pq} + Q_{ip} \dot{Q}_{jq} \rs_{pq} + Q_{ip}Q_{jq} \frac{D \rs_{pq} }{D t} 
\end{align}
The first two terms on the right hand side can be expressed in terms of the Reynolds stresses in the rotating reference frame as follows
\begin{align}
\label{trans_eq_additional_terms}
\dot{Q}_{ip}Q_{jq} \rs_{pq} + Q_{ip} \dot{Q}_{jq} \rs_{pq} &= \dot{Q}_{ip}Q_{jq} Q_{kp} \tilde{\rs}_{ks} Q_{sq} + Q_{ip} \dot{Q}_{jq} Q_{sp} \tilde{\rs}_{sk} Q_{kq} \nonumber \\
& = \dot{Q}_{ip}Q_{kp} \tilde{\rs}_{kj} + \dot{Q}_{jq}Q_{kq} \tilde{\rs}_{ik} \nonumber \\
& = -\tilde{\Lambda}_{ik} \tilde{\rs}_{kj} + \tilde{\rs}_{ik} \tilde{\Lambda}_{kj},
\end{align}
where $\tilde{\Lambda}_{ij} = Q_{ik}\dot{Q}_{jk}$. Note the similarity between $\tilde{\Lambda}_{ij} = Q_{ik}\dot{Q}_{jk}$ and $\Omega_{ij} = Q_{kj}\dot{Q}_{ki}$. In fact, these two are related as follows $\tilde{\Lambda}_{ij} = Q_{ik}\Omega_{kl}Q_{jl}$. The transport equation for the Reynolds stresses in the rotating reference frame thus becomes
\begin{equation}
\label{reynolds_stress_transport_eq_noninertial}
\frac{D \tilde{\rs}_{ij} }{D \tilde{t}} + \tilde{\Lambda}_{ik} \tilde{\rs}_{kj} - \tilde{\rs}_{ik} \tilde{\Lambda}_{kj} = Q_{ip}Q_{jq} \frac{D \rs_{pq}}{D t} . 
\end{equation}

An objective scalars $s$, vector $b_i$, or tensor $c_{ij}$ is one which transforms as follows
\begin{equation}
\tilde{s} = s \qquad \tilde{b}_i = Q_{ij}b_j \qquad \tilde{c}_{ij} = Q_{ip}c_{pq}Q_{jq}. 
\end{equation}
Thus, the Reynolds stresses are objective, but the velocity vector is not. Using equation (\ref{rot_mean_vel}) it is easy to show that the mean-flow gradients transform as follow
\begin{equation}
\label{grad_transformation}
\frac{\partial \langle \tilde{U}_i \rangle}{\partial \tilde{x}_k} = Q_{ip} \frac{\partial \upavg}{\partial \tilde{x}_k} + \dot{Q}_{ip}Q_{kp} = Q_{ip}Q_{kt} \frac{\partial \upavg}{\partial x_t} + \dot{Q}_{ip}Q_{kp}.
\end{equation}
Thus, one can easily show that the rate-of-strain tensor is objective, whereas the rate-of-rotation tensor transforms as follows
\begin{equation}
\tilde{W}_{ij} = Q_{ip}W_{pq}Q_{jq} + \dot{Q}_{ik}Q_{jk},
\end{equation}
which is equivalent to
\begin{equation}
\tilde{W}_{ij} + \tilde{\Lambda}_{ij}= Q_{ip}W_{pq}Q_{jq}.
\end{equation}
We now define $\Lambda_{ij} = 0$, which implies $\Lambda_{ij}$ is not objective. This is a sensible decision given that $\Lambda_{ij}$ is not a typical tensor such as $\rs_{ij}$. Thus we can rewrite the above as follows
\begin{equation}
\tilde{W}_{ij} + \tilde{\Lambda}_{ij}= Q_{ip} (W_{pq} + \Lambda_{pq} )Q_{jq},
\end{equation}
which shows $W^*_{ij} = W_{ij} + \Lambda_{ij}$ is an objective tensor. Similarly, equation (\ref{reynolds_stress_transport_eq_noninertial}) shows that 
\begin{equation}
\frac{D \rs_{ij}}{Dt} + \Lambda_{ik} \rs_{kj} - \rs_{ik} \Lambda_{kj}
\end{equation}
is an objective tensor.

We will now show how the Reynolds stress production, redistribution, dissipation and transport terms transform under a coordinate transformation. We start with production, which involves gradients of the mean flow. Using equation (\ref{grad_transformation}) we obtain
\begin{align}
-\tilde{\rs}_{jk} \frac{\partial \langle \tilde{U}_i \rangle}{\partial \tilde{x}_k}  &= -Q_{jq}Q_{ks} \rs_{qs} \left ( Q_{ip}Q_{kt} \frac{\partial \upavg}{\partial x_t} + \dot{Q}_{ip}Q_{kp} \right ) \nonumber \\
& = -Q_{ip}Q_{jq} \rs_{qt} \frac{\partial \upavg}{\partial x_t} - \dot{Q}_{ip}Q_{jq} \rs_{pq}
\end{align}
We thus have
\begin{align}
\label{production_transformation}
\tilde{\mathcal{P}}_{ij} = -\tilde{\rs}_{ik} \frac{\partial \langle \tilde{U}_j \rangle}{\partial \tilde{x}_k} - \tilde{\rs}_{jk} \frac{\partial \langle \tilde{U}_i \rangle}{\partial \tilde{x}_k}
& = Q_{ip}Q_{jq} \mathcal{P}_{pq} - Q_{ip}\dot{Q}_{jq} \rs_{pq}  - \dot{Q}_{ip}Q_{jq} \rs_{pq} \nonumber \\
& = Q_{ip}Q_{jq} \mathcal{P}_{pq} - \tilde{\rs}_{ik} \tilde{\Lambda}_{kj}  + \tilde{\Lambda}_{ik} \tilde{\rs}_{kj}.
\end{align}
The last step was obtain using (\ref{trans_eq_additional_terms}). The above is rewritten as follows
\begin{equation}
-\tilde{\rs}_{ik} (\tilde{S}_{kj} - \tilde{W}^*_{kj}) - (\tilde{S}_{ik} + \tilde{W}^*_{ik} ) \tilde{\rs}_{kj} = Q_{ip}Q_{jq} \mathcal{P}_{pq}
\end{equation}
which shows $ -\rs_{ik} (S_{kj} - W^*_{kj}) - (S_{ik} + W^*_{ik} ) \rs_{kj} $ is objective.

The next term to consider is the velocity-pressure-gradient correlation, which involves gradients of pressure. We note that pressure, as a scalar, transforms as
\begin{equation}
\tilde{P} = P,
\end{equation}
from which it follows that $\langle \tilde{P} \rangle = \pavg$ and $\tilde{p} = p$. The same holds for the density $\rho$ and kinematic viscosity $\nu$. The expression below transforms as follow
\begin{equation}
\tilde{u}_i \frac{\partial \tilde{p}}{\partial \tilde{x}_j} = Q_{ip}Q_{jq}u_p \frac{\partial p}{\partial x_q},
\end{equation}
which leads to the following transformation for the velocity-pressure-gradient correlation
\begin{equation}
\tilde{\Pi}_{ij} = -\frac{1}{\tilde{\rho}} \left < \tilde{u}_i \frac{\partial \tilde{p}}{\partial \tilde{x}_j}  + \tilde{u}_j \frac{\partial \tilde{p}}{\partial \tilde{x}_i} \right > = -Q_{ip}Q_{jq} \frac{1}{\rho} \left < \upfluc \frac{\partial p}{\partial x_q} + \uqfluc \frac{\partial p}{\partial x_p}\right > = Q_{ip}Q_{jq} \Pi_{pq}.
\end{equation}
Similarly, the dissipation transforms as follows
\begin{equation}
\tilde{\epsilon}_{ij} = 2\tilde{\nu} \left < \frac{\partial \tilde u_i}{\partial \tilde{x}_k} \frac{\partial \tilde u_j}{\partial \tilde{x}_k} \right > = Q_{ip}Q_{jq} Q_{kt} Q_{ks} 2\nu \left < \frac{\partial \upfluc}{\partial x_t} \frac{\partial \uqfluc}{\partial x_s} \right > = Q_{ip}Q_{jq} \epsilon_{pq}.
\end{equation}
Finally, the transport terms transform as follows
\begin{align}
\frac{\partial}{\partial \tilde{x}_k} \left ( \langle \tilde{u}_i \tilde{u}_j \tilde{u}_k \rangle - \nu \frac{\partial \tilde{\rs}_{ij} }{\partial \tilde{x}_k} \right) &= Q_{ip}Q_{jq}Q_{kr} Q_{ks} \frac{\partial}{\partial x_s} \left ( \langle \upfluc \uqfluc \urfluc \rangle - \nu \frac{\partial \rs_{pq} }{\partial x_r} \right) \nonumber \\
& = Q_{ip}Q_{jq} \frac{\partial}{\partial x_k} \left ( \langle \upfluc \uqfluc \ukfluc \rangle - \nu \frac{\partial \rs_{pq} }{\partial x_k} \right).
\end{align}
We can thus conclude that 
\begin{align}
Q_{iq}Q_{jq} \frac{D \rs_{pq}}{Dt} &= Q_{iq}Q_{jq}\mathcal{P}_{pq} + Q_{iq}Q_{jq}\Pi_{pq} - Q_{iq}Q_{jq}\epsilon_{pq} - Q_{iq}Q_{jq}\frac{\partial}{\partial x_k} \left ( \mathcal{T}^{(u)}_{kpq} + \mathcal{T}^{(\nu)}_{kpq} \right ) \nonumber \\
& =  -\tilde{\rs}_{ik} (\tilde{S}_{kj} - \tilde{W}^*_{kj}) - (\tilde{S}_{ik} + \tilde{W}^*_{ik} ) \tilde{\rs}_{kj} + \tilde{\Pi}_{ij} - \tilde{\epsilon}_{ij} - \frac{\partial}{\partial \tilde{x}_k} \left ( \tilde{\mathcal{T}}^{(u)}_{kij} + \tilde{\mathcal{T}}^{(\nu)}_{kij} \right ).
\end{align}
Using the above in equation (\ref{reynolds_stress_transport_eq_noninertial}) leads to the transport equation for the Reynolds stresses in a rotating reference frame where every single term in the equation contains quantities evaluated in the rotating reference frame
\begin{align}
\frac{D \tilde{\rs}_{ij} }{D \tilde{t}} &+ \tilde{\Lambda}_{ik} \tilde{\rs}_{kj} - \tilde{\rs}_{ik} \tilde{\Lambda}_{kj} \nonumber \\
&=  -\tilde{\rs}_{ik} (\tilde{S}_{kj} - \tilde{W}^*_{kj}) - (\tilde{S}_{ik} + \tilde{W}^*_{ik} ) \tilde{\rs}_{kj} + \tilde{\Pi}_{ij} - \tilde{\epsilon}_{ij} - \frac{\partial}{\partial \tilde{x}_k} \left ( \tilde{\mathcal{T}}^{(u)}_{kij} + \tilde{\mathcal{T}}^{(\nu)}_{kij} \right ).
\end{align}


%###############################################################################
%
 \chapter{RANS equations in cylindrical coordinates}
%
%###############################################################################
The continuity equation follows
\begin{equation}
\frac{1}{r} \frac{\partial \left ( r \langle u_r \rangle \right )}{\partial r} + \frac{1}{r} \frac{\partial \langle u_\theta \rangle}{\partial \theta} + \frac{\partial \langle u_z \rangle}{\partial z} = 0.
\end{equation} 
The momentum equations follow
\begin{align}
&\frac{\partial \langle u_r \rangle}{\partial t} + \langle u_r \rangle \frac{\partial \langle u_r \rangle}{\partial r} + \frac{\langle u_\theta \rangle}{r} \frac{\partial \langle u_r \rangle}{\partial \theta} + \langle u_z \rangle \frac{\partial \langle u_r \rangle}{\partial z} - \frac{\langle u_\theta \rangle^2}{r} = \nonumber \\ 
&-\frac{1}{\rho} \frac{\partial \pavg}{\partial r} + \nu \left ( \frac{\partial^2 \langle u_r \rangle}{\partial r^2} + \frac{1}{r^2} \frac{\partial^2 \langle u_r \rangle}{\partial \theta^2} + \frac{\partial^2 \langle u_r \rangle}{\partial z^2} + \frac{1}{r} \frac{\partial \langle u_r \rangle}{\partial r} - \frac{2}{r^2} \frac{\partial \langle u_\theta \rangle }{\partial \theta} - \frac{\langle u_r \rangle}{r^2} \right ) \nonumber \\
&- \frac{\partial \langle u_ru_r \rangle}{\partial r} - \frac{1}{r} \frac{\partial \langle u_r u_\theta \rangle}{\partial \theta} - \frac{\partial \langle u_r u_z \rangle}{\partial z} - \frac{\langle u_ru_r \rangle}{r} + \frac{\langle u_\theta u_\theta \rangle}{r}
\end{align} 
\begin{align}
&\frac{\partial \langle u_\theta \rangle}{\partial t} + \langle u_r \rangle \frac{\partial \langle u_\theta \rangle}{\partial r} + \frac{\langle u_\theta \rangle}{r} \frac{\partial \langle u_\theta \rangle}{\partial \theta} + \langle u_z \rangle \frac{\partial \langle u_\theta \rangle}{\partial z} + \frac{\langle u_r \rangle \langle u_\theta\rangle}{r} = \nonumber \\ 
&-\frac{1}{\rho r} \frac{\partial \pavg}{\partial \theta} + \nu \left ( \frac{\partial^2 \langle u_\theta \rangle}{\partial r^2} + \frac{1}{r^2} \frac{\partial^2 \langle u_\theta \rangle}{\partial \theta^2} + \frac{\partial^2 \langle u_\theta \rangle}{\partial z^2} + \frac{1}{r} \frac{\partial \langle u_\theta \rangle}{\partial r} + \frac{2}{r^2} \frac{\partial \langle u_r \rangle }{\partial \theta} - \frac{\langle u_\theta \rangle}{r^2} \right ) \nonumber \\
&- \frac{\partial \langle u_\theta u_r \rangle}{\partial r} - \frac{1}{r} \frac{\partial \langle u_\theta u_\theta \rangle}{\partial \theta} - \frac{\partial \langle u_\theta u_z \rangle}{\partial z} - 2\frac{\langle u_ru_\theta \rangle}{r}
\end{align} 
\begin{align}
&\frac{\partial \langle u_z \rangle}{\partial t} + \langle u_r \rangle \frac{\partial \langle u_z \rangle}{\partial r} + \frac{\langle u_\theta \rangle}{r} \frac{\partial \langle u_z \rangle}{\partial \theta} + \langle u_z \rangle \frac{\partial \langle u_z \rangle}{\partial z} =  \nonumber \\
&-\frac{1}{\rho} \frac{\partial \pavg}{\partial z} + \nu \left ( \frac{\partial^2 \langle u_z \rangle}{\partial r^2} + \frac{1}{r^2} \frac{\partial^2 \langle u_z \rangle}{\partial \theta^2} + \frac{\partial^2 \langle u_z \rangle}{\partial z^2} + \frac{1}{r} \frac{\partial \langle u_z \rangle}{\partial r} \right ) \nonumber \\
&- \frac{\partial \langle u_z u_r \rangle}{\partial r} - \frac{1}{r} \frac{\partial \langle u_z u_\theta \rangle}{\partial \theta} - \frac{\partial \langle u_z u_z \rangle}{\partial z} - \frac{\langle u_z u_r \rangle}{r}.
\end{align}

%###############################################################################
%
\chapter{Rogallo's initial condition}
%
%###############################################################################
The Rogallo method \cite{rogallo1981} can be used to initialize an isotropic turbulent field---it specifies the fourier coefficients (both the magnitude and direction) of the velocity field.

%-------------------------------------------------------------------------------
\section{The magnitude}
%-------------------------------------------------------------------------------
For isotropic turbulence, the energy spectrum tensor takes the form
\begin{equation}
    \est_{ij}(\kvec,t) = \frac{\est(\kappa,t)}{4 \pi \kappa^2} \left ( \delta_{ij} - \frac{\kappa_i \kappa_j}{\kappa^2} \right ).
\end{equation}
Thus, given \cref{eq:continuous_uiuj_spectral}, the kinetic energy can be obtained using
\begin{equation}
    k = \frac{1}{2} \int_\Rthree \est_{ii}(\kvec,t) \, d\kvec = \int_\Rthree \frac{\est(\kappa,t)}{4 \pi \kappa^2} \, d\kvec.
\end{equation}
We now want to approximate this integral discretely. If one discretizes the continuous wave-vector space by the discrete wave vector space (i.e. $\kvec_C \to \kvec_D$), and uses a cubic domain of length $2 \pi$, then 
\begin{equation}
    \Delta^3_{\kvec} = \left ( \kappa_x^{(i+1)} - \kappa_x^{i} \right ) \left ( \kappa_y^{(j+1)} - \kappa_y^{j} \right ) \left ( \kappa_z^{(k+1)} - \kappa_z^{k} \right ) = 1,
\end{equation}
for all $i,j,k$. Thus, the integral of the energy spectrum tensor above can be approximated as
\begin{equation}
    k = \sum_{\kvec} \frac{\est(\kappa,t)}{4 \pi \kappa^2}.
\end{equation}
On the other hand, from the discrete point of view, the kinetic energy is given by \cref{eq:discrete_k_spectral}, that is
\begin{equation}
    k = \sum_{\kvec} \frac{1}{2} \hat{\tpvc}_{ii}(\kvec,t).
\end{equation}
Thus, comparing the last two equations, we can make the following approximation
\begin{equation}
    \overline{ \hat{u}_i^*(\kvec,t) \hat{u}_i(\kvec,t)} = \frac{\est(\kappa,t)}{2 \pi \kappa^2}.
\end{equation}
Given a target energy spectrum $E(\kappa,t)$, the above can be used to initialize the magnitude of the Fourier coefficients. The energy spectrum for this flow---computed using \cref{eq:num_comp_spectra}---would match the target spectrum. 

Note that for 2D turbulence, the energy spectrum tensor for isotropic turbulence is of the form
\begin{equation}
    \est_{ij}(\kvec,t) = \frac{\est(\kappa,t)}{\pi \kappa} \left ( \delta_{ij} - \frac{\kappa_i \kappa_j}{\kappa^2} \right ),
\end{equation}
and thus, the kinetic energy can be obtained using
\begin{equation}
    k = \frac{1}{2} \int_\Rtwo \est_{ii}(\kvec,t) \, d\kvec = \int_\Rtwo \frac{\est(\kappa,t)}{2 \pi \kappa} \, d\kvec.
\end{equation}
Following the same arguments as for the 3D case, we obtain
\begin{equation}
    \overline{ \hat{u}_i^*(\kvec,t) \hat{u}_i(\kvec,t)} = \frac{\est(\kappa,t)}{\pi \kappa}.
\end{equation}

%-------------------------------------------------------------------------------
\section{The direction}
%-------------------------------------------------------------------------------

\bibliography{library}
\bibliographystyle{plain}

\end{document}



