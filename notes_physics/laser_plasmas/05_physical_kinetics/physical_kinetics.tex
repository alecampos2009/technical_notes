\documentclass[a4paper,11pt]{report}
\usepackage{fullpage}

\usepackage{"../../../info/packages"}
\usepackage{"../../../info/nomenclature"}

\usepackage{scalerel}
\usepackage{subfiles}

\newcommand{\xvecdot}{\dot{\xvec}}
\newcommand{\xdot}{\dot{x}}
\newcommand{\qvecdot}{\dot{\qvec}}
\newcommand{\qdot}{\dot{q}}

\setlength{\cellspacetoplimit}{3pt}
\setlength{\cellspacebottomlimit}{3pt}

\title{Physical Kinetics}
\author{Alejandro Campos}

\begin{document}
\maketitle
\tableofcontents

%%%%%%%%%%%%%%%%%%%%%%%%%%%%%%%%%%%%%%%%%%%%%%%%%%%%%%%%%%%%%%%%%%%%%%%%%
\appendix
%%%%%%%%%%%%%%%%%%%%%%%%%%%%%%%%%%%%%%%%%%%%%%%%%%%%%%%%%%%%%%%%%%%%%%%%%

%########################################################################
\chapter{Lagrangian and Eulerian PDFs}
%########################################################################

%------------------------------------------------------------------------
\section{Eulerian PDF}
%------------------------------------------------------------------------
Consider an Eulerian velocity field $\uvec = \uvec(\xvec,t)$. The Eulerian PDF $f = f(\Vvec; \xvec,t)$ gives the probability that the velocity field will have a value of $\Vvec$ at location $\xvec$ and at time $t$. We'll also introduce the fine-grained Eulerian PDF $f' = f'(\Vvec;\xvec,t)$, which is defined as 
\begin{equation}
    f'(\Vvec; \xvec, t) = \delta(\uvec(\xvec,t) - \Vvec).
\end{equation}
Note: a delta function of a 3D argument means the following $\delta(\avec) = \delta(a_1) \delta(a_2) \delta(a_3) $. The Eulerian PDF can be obtained from the fine-grained Eulerian PDF using 
\begin{equation}
    \label{eq:fine_eul_pdf}
    f(\Vvec; \xvec, t)=\langle f'(\Vvec; \xvec, t) \rangle.
\end{equation}
The proof is as follows,
\begin{align}
    \langle f'(\Vvec; \xvec, t) \rangle &= \langle \delta( \uvec(\xvec,t) - \Vvec) \rangle \nonumber \\
    &= \int \delta( \Vvec' - \Vvec) f(\Vvec';\xvec,t) d\Vvec' \nonumber \\
    &= f(\Vvec; \xvec, t).
\end{align}

%------------------------------------------------------------------------
\section{Lagrangian PDF}
%------------------------------------------------------------------------
Consider a Lagrangian particle with velocity $\uvec^+ = \uvec^+(t,\yvec)$ and position $\xvec^+(t,\yvec)$. The Lagrangian PDF $f_L = f_L(\Vvec, \xvec; t | \yvec)$ gives the probability that the particle that started at location $\yvec$ at the reference time $t_0$ will have a velocity $\Vvec$ and position $\xvec$ at time $t$. We'll also introduce the fine-grained Eulerian PDF $f'_L = f'_L(\Vvec, \xvec; t | \yvec)$, which is defined as 
\begin{equation}
    f'_L(\Vvec, \xvec; t | \yvec) = \delta(\uvec^+(t,\yvec) - \Vvec) \delta(\xvec^+(t,\yvec) - \xvec).
\end{equation}
Note: a delta function of a 3D argument means the following $\delta(\avec) = \delta(a_1) \delta(a_2) \delta(a_3) $. The Lagrangian PDF can be obtained from the fine-grained Lagrangian PDF using
\begin{equation}
    \label{eq:fine_lag_pdf}
    f_L(\Vvec, \xvec; t | \yvec) = \langle f'_L(\Vvec, \xvec; t | \yvec) \rangle.
\end{equation} 
The proof is as follows,
\begin{align}
    \langle f'_L(\Vvec, \xvec; t | \yvec) \rangle &= \langle \delta( \uvec^+(t,\yvec) - \Vvec) \delta(\xvec^+(t,\yvec) - \xvec) \rangle \nonumber \\
    &= \int \delta( \Vvec' - \Vvec) \delta ( \xvec' - \xvec) f(\Vvec', \xvec';t | \yvec) d\Vvec' d\xvec' \nonumber \\
    &= f_L(\Vvec, \xvec; t | \yvec).
\end{align}

%------------------------------------------------------------------------
\section{Relation between Lagrangian and Eulerian PDFs}
%------------------------------------------------------------------------
As a quick side note, we mention that the inverse of $\xvec^+$ is $\yvec^+ = \yvec^+(t,\zvec)$, which gives the initial location of a fluid particle that at time $t$ is located at position $\zvec$. Thus, $\xvec^+(t,\yvec^+(t,\zvec)) = \zvec$.

We begin as follows
\begin{align}
\int f'_L(\Vvec,\xvec;t|\yvec) \, d\yvec &= \int \delta(\uvec^+(t,\yvec) - \Vvec) \delta(\xvec^+(t,\yvec) - \xvec) \, d\yvec \nonumber \\
&= \int \delta(\uvec(\xvec^+(t,\yvec),t) - \Vvec) \delta(\xvec^+(t,\yvec) - \xvec) \, d\yvec \nonumber \\
&= \int \delta(\uvec(\xvec^+(t,\yvec),t) - \Vvec) \delta(\xvec^+(t,\yvec) - \xvec) | \det D \xvec^+ | \, d\yvec,
\end{align}
where we have introduced $| \det D \xvec^+ |$, which is the absolute value of the determinant of the Jacobean $\partial \xvec^+/\partial \yvec$, and is equal to one for incompressible flows. Using integration by substitution we obtain
\begin{equation}
\int f'_L(\Vvec,\xvec;t|\yvec) \, d\yvec = \int \delta(\uvec(\zvec,t) - \Vvec) \delta(\zvec - \xvec) \, d\zvec = \delta(\uvec(\xvec,t) - \Vvec)
\end{equation}
Given the definition of $f'(\Vvec; \xvec, t)$, we have
\begin{equation}
    \label{eq:fine_eul_lag_pdf}
    \int f'_L(\Vvec,\xvec;t|\yvec) \, d\yvec = f'(\Vvec; \xvec, t).
\end{equation}
Taking the expectation of the above we obtain
\begin{equation}
    \label{eq:eul_lag_pdf}
    \int f_L(\Vvec,\xvec;t|\yvec) \, d\yvec = f(\Vvec;\xvec,t).
\end{equation}

A summary of all of the relations derived thus far is given by the following graph
\setlength{\unitlength}{1cm}
\begin{center}
    \begin{picture}(12,2.5)(0,0)
        \put(0.5,0){Eulerian PDF}
        \put(8,0){Lagrangian PDF}
            \put(8.0,0.1){\vector(-1,0){5.0}}
            \put(4.5,0.2){\cref{eq:eul_lag_pdf}}
        \put(-0.5,2){Eulerian fine-grained PDF}
            \put(1.5,1.9){\vector(0,-1){1.5}}
            \put(1.6,1.2){\cref{eq:fine_eul_pdf}}
        \put(7,2){Lagrangian fine-grained PDF}
            \put(9.0,1.9){\vector(0,-1){1.5}}
            \put(9.1,1.2){\cref{eq:fine_lag_pdf}}
            \put(7.0,2.1){\vector(-1,0){3.0}}
            \put(4.7,2.2){\cref{eq:fine_eul_lag_pdf}}
    \end{picture}
\end{center}

%------------------------------------------------------------------------
\section{Evolution equation for fine-grained Eulerian PDF}
%------------------------------------------------------------------------

%------------------------------------------------------------------------
\section{Evolution equation for fine-grained Lagrangian PDF}
%------------------------------------------------------------------------

\end{document}
