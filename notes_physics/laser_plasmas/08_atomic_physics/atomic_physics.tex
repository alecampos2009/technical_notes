\documentclass[a4paper,11pt]{article}
\usepackage{fullpage}

\usepackage{"../../../info/packages"}
\usepackage{"../../../info/nomenclature"}

\title{Atomic Physics}
\author{Alejandro Campos}

\begin{document}

\maketitle
\tableofcontents

%------------------------------------------------------------------------
\section{NLTE Equation of state}
%------------------------------------------------------------------------

%------------------------------------------------------------------------
\section{NLTE opacities}
%------------------------------------------------------------------------
Various photon phenomena are listed below
\begin{itemize}
    \item Absorption or destruction 
    \begin{itemize}
        \item Bound-bound excitation (\textbf{stimulated absorption})
        \item Bound-free ionization (\textbf{photoionization})
        \item Free-free photo-absorption (\textbf{inverse breemstrahlung})
        \item \textbf{Pair production}: Beta-Heitler ($\gamma \to e^- + e^+$), linear Breit-Wheeler ($\gamma + \gamma' \to e^- + e^+$), non-linear Breit-Wheeler ($\gamma + n\gamma' \to e^- + e^+$).
    \end{itemize}
    \item Emission or creation
    \begin{itemize}
        \item Bound-bound de-excitation (\textbf{stimulated emission}, \textbf{spontaneous emission})
        \item Free-bound recombination
        \item Free-free photo-emission (\textbf{breemstrahlung from Coulomb collisions})
        \item \text{Cyclotron, Synchrotron, and Betatron radiation}
        \item \textbf{Pair annihilation}
          
    \end{itemize}
    \item Scattering
    \begin{itemize}
        \item \textbf{Rayleigh scattering}: elastic scattering of a photon from an atom or molecule whose size is less than that of the wavelength of the photon. 
        \item \textbf{Mie scattering}: same as Rayleigh scattering but for cases where the sizes of the atoms or molecules are comparable to the wavelength of the incoming photon.
        \item \textbf{Raman scattering}: inelastic scattering of a photon from a molecule. The interaction changes the molecule's vibrational, rotational, or electron energy.
        \item \textbf{Brillouin scattering}: inelastic scattering of a photon caused by its interaction with material waves in a medium (i.e. mass oscillation modes, charge displacement modes, magnetic spin oscillation modes). 
        \item \textbf{Compton scattering}: inelastic scattering of a photon from a charged particle. 
        \item \textbf{Thomson scattering}: low-energy limit of Compton scattering. The photon energy and the particle's kinetic energy do not change as a result of the scattering. Can be explained with classical electrodynamics.
    \end{itemize}
\end{itemize}

\end{document}
