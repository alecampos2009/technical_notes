\documentclass[a4paper,11pt]{report}
\usepackage{fullpage}
%\usepackage{hyperref}

\usepackage{../../../info/packages}
\usepackage{../../../info/nomenclature}

\usepackage{scalerel}

\title{Hydrodynamics}
\author{Alejandro Campos}

\begin{document}
\maketitle
\tableofcontents

%########################################################################
\chapter{Mixtures}
%########################################################################

%------------------------------------------------------------------------
\section{Multi-material vs. multi-fluid vs. multi-species}
%------------------------------------------------------------------------
Multi-material, multi-fluid, and multi-species refer to different aspects of a fluid model. These are not mutually exclusive, for example, one can have a multi-material multi-fluid model, or a multi-material single-fluid model. Below are some definitions:
\begin{enumerate}
    \item Multi-species: the fluid is composed of multiple species.
    \item Multi-fluid: a velocity equation is solved for each species. 
    \item Multi-material: the species are grouped into bunches that share some general properties, such as EOS and thermal conductivity. Either a multi-fluid or single-fluid formulation can be multi-material. 
\end{enumerate}

Multi-material or multi-fluid problems are always multi-species. Single-material or single-fluid problems can also be multi-species, but are not necessarily always so. Thus, we use the multi-species label only when dealing with the single-material single-fluid model, to clarify whether we are using that model for cases where there are multiple species or not.

%------------------------------------------------------------------------
\section{Definition of multi-material variables}
%------------------------------------------------------------------------
We define a material $k$ as a collection of species $s$. Rather than labeling species using the single subscript $s$, we'll use the subscript $k,i,s$ and $k,e$, where the former denotes the ion species $s$ that belongs to material $k$, and the latter denotes the electron species that belongs to material $k$. The number of ion species within a material is denoted as $\mathcal{N}_k$, and the total number of materials as $\mathcal{M}$.

For multi-material problems there is a large number of new variables that are used. These are summarized below.
\subsubsection{Mass densities}

\begin{equation}
    \rho = \frac{\mathsf{M}}{\mathsf{V}}.
\end{equation}
\begin{equation}
    \rho_k= \frac{\mathsf{M}_k}{\mathsf{V}_k}.
\end{equation}
\begin{equation}
    \label{eq:mm_mass_den_kis}
    \rho_{k,i,s}= \frac{\mathsf{M}_{k,i,s}}{\mathsf{V}_{k,i,s}}.
\end{equation}
\begin{equation}
    \label{eq:mm_mass_den_ke}
    \rho_{k,e}= \frac{\mathsf{M}_{k,e}}{\mathsf{V}_k}.
\end{equation}

\subsubsection{Number densities}

\begin{equation}
    n = \frac{\mathsf{N}}{\mathsf{V}}.
\end{equation}
\begin{equation}
    n_k = \frac{\mathsf{N}_k}{\mathsf{V}_k}.
\end{equation}
\begin{equation}
    \label{eq:mm_num_den_kis}
    n_{k,i,s} = \frac{\mathsf{N}_{k,i,s}}{\mathsf{V}_{k,i,s}}.
\end{equation}
\begin{equation}
    \label{eq:mm_num_den_ke}
    n_{k,e} = \frac{\mathsf{N}_{k,e}}{\mathsf{V}_k}.
\end{equation}

\subsubsection{Relation between mass and number densities}

\begin{equation}
    \label{eq:mm_mass_number_densities_i}
    m_{k,i,s} n_{k,i,s} = \rho_{k,i,s}.
\end{equation}
\begin{equation}
    \label{eq:mm_mass_number_densities_e}
    m_e n_{k,e} = \rho_{k,e}.
\end{equation}

\subsubsection{Material volume and mass fractions}

\begin{equation}
    \eta_k = \frac{\mathsf{V}_k}{\mathsf{V}},
\end{equation}
\begin{equation}
    Y_k = \frac{\mathsf{M}_k}{\mathsf{M}}.
\end{equation}
\begin{equation}
    \rho_k = \rho \frac{Y_k}{\eta_k}.
\end{equation}
 
\subsubsection{Species volume and mass fractions}

\begin{equation}
    \eta_{k,i,s} = \frac{\mathsf{V}_{k,i,s}}{\mathsf{V}_k}.
\end{equation}
\begin{equation}
    Y_{k,i,s} = \frac{\mathsf{M}_{k,i,s}}{\mathsf{M}_k}.
\end{equation}
\begin{equation}
    \label{eq:mm_tri_relation_kis}
    \rho_{k,i,s} = \rho_k \frac{Y_{k,i,s}}{\eta_{k,i,s}}.
\end{equation}

\subsubsection{Energies}

\begin{equation}
    e = \frac{\mathsf{IE}}{\mathsf{M}}.
\end{equation}
\begin{equation}
    e_k = \frac{\mathsf{IE}_k}{\mathsf{M}_k}.
\end{equation}
\begin{equation}
    e_{k,i,s} = \frac{\mathsf{IE}_{k,i,s}}{\mathsf{M}_{k,i,s}}.
\end{equation}
\begin{equation}
    e_{k,e} = \frac{\mathsf{IE}_{k,e}}{\mathsf{M}_k}.
\end{equation}

\subsubsection{Total ion variables}

\begin{equation}
    \label{eq:sf_vars_eta_ki}
    \eta_{k,i} = \sum_s^{\mathcal{N}_k} \eta_{k,i,s} \approx 1, 
\end{equation}
\begin{equation}
    \label{eq:sf_vars_Y_ki}
    Y_{k,i} = \sum_s^{\mathcal{N}_k} Y_{k,i,s} \approx 1, 
\end{equation}
\begin{equation}
    \label{eq:sf_vars_n_ki}
    n_{k,i} = \sum_s^{\mathcal{N}_k} \eta_{k,i,s} n_{k,i,s}. 
\end{equation}
\begin{equation}
    e_{k,i} = \sum_s^{\mathcal{N}_k} Y_{k,i,s} e_{k,i,s}.
\end{equation}
\begin{equation}
    h_{k,i} = \sum_s^{\mathcal{N}_k} Y_{k,i,s} h_{k,i,s}.
\end{equation}
\begin{equation}
    \label{eq:sf_vars_p_ki}
    p_{k,i} = \sum_s^{\mathcal{N}_k} \eta_{k,i,s} p_{k,i,s}.
\end{equation}
\begin{equation}
    \label{eq:sf_vars_rho_ki}
    \rho_{k,i} = \sum_s^{\mathcal{N}_k} \eta_{k,i,s} \rho_{k,i,s}.
\end{equation}
\begin{equation}
    \label{eq:sf_vars_t_ki}
    \tvec_{k,i} = \sum_s^{\mathcal{N}_k} \eta_{k,i,s} \tvec_{k,i,s}.
\end{equation}
\begin{equation}
    \label{eq:sf_vars_sig_ki}
    \sigmavec_{k,i} = \sum_s^{\mathcal{N}_k} \eta_{k,i,s} \sigmavec_{k,i,s}.
\end{equation}
\begin{equation}
    \label{eq:sf_vars_R_ki}
    \Rvec_{k,i} = \sum_s^{\mathcal{N}_k} \Rvec_{k,i,s}.
\end{equation}
\begin{equation}
    \label{eq:sf_vars_q_ki}
    \qvec_{k,i} = \sum_s^{\mathcal{N}_k} \qvec_{k,i,s}.
\end{equation}
\begin{equation}
    \label{eq:sf_vars_Q_ki}
    Q_{k,i} = \sum_s^{\mathcal{N}_k} Q_{k,i,s}.
\end{equation}

%------------------------------------------------------------------------
\section{Interpretation of multi-material variables}
%------------------------------------------------------------------------
Consider the volume $\mathsf{V}_{\Omega,k}$ occupied by material $k$ in some subdomain $\Omega$. This can be computed by using the volume fraction $\eta_k = \eta_k(t,\xvec)$ as shown below
\begin{equation}
    \mathsf{V}_{\Omega,k} = \int_\Omega \eta_k \, d\xvec.
\end{equation}
Now consider the mass $\mathsf{M}_{\Omega,k}$ of material $k$ in some subdomain $\Omega$. This can be computed by summing up the masses $m_i$ of the atoms or molecules of material $k$ in this subdomain, or by using the mass fraction $Y_k = Y_k(t,\xvec)$, as shown below 
\begin{equation}
    \mathsf{M}_{\Omega,k} = \sum_{i \in \Omega, i \in k} m_i = \int_\Omega \rho Y_k \, d\xvec.
\end{equation}
In the above $i \in \Omega, i \in k$ denotes the atoms or molecules of material $k$ located in the subdomain $\Omega$. Finally, consider the internal energy $\mathsf{IE}_{\Omega,k}$ of material $k$ in some subdomain $\Omega$. This internal energy can be computed by summing up the internal energies $\epsilon_i$ of the atoms or molecules of material $k$ in this subdomain, or by using the specific internal energy $e_k = e_k(t,\xvec)$, as shown below
\begin{equation}
    \mathsf{IE}_{\Omega,k} = \sum_{i \in \Omega, i \in k} \epsilon_i = \int_\Omega \rho Y_k e_k \, d\xvec.
\end{equation}

We can now write 
\begin{equation}
    \label{eq:atomistic_mix_eta}
    \eta_k = \lim_{\Omega \to \epsilon} \frac{1}{V_\Omega} \int_\Omega \eta_k \, d\xvec = \lim_{\Omega \to \epsilon} \frac{ \mathsf{V}_{\Omega,k} }{ \mathsf{V}_\Omega }.
\end{equation}
That is, the volume fraction is the volume occupied by material $k$ within a small domain divided by the volume of that small domain as the domain becomes sufficiently small.
For mass fraction, we have
\begin{equation}
    \label{eq:atomistic_mix_Y}
    Y_k = \frac{\rho Y_k}{\rho} = \lim_{\Omega \to \epsilon} \frac{ \frac{1}{\mathsf{V}_\Omega} \int_\Omega \rho Y_k \, d\xvec }{ \frac{1}{\mathsf{V}_\Omega} \int_\Omega \rho \, d\xvec } = \lim_{\Omega \to \epsilon} \frac{ \sum_{i \in \Omega, i \in k} m_i }{ \sum_{i \in \Omega} m_i } = \lim_{\Omega \to \epsilon} \frac{ \mathsf{M}_{\Omega,k} }{ \mathsf{M}_\Omega }.
\end{equation}
That is, the mass fraction is the mass of material $k$ contained in a small domain divided by the mass contained in that small domain as the domain becomes sufficiently small. Similarly for specific internal energy of an material, we have
\begin{equation}
    \label{eq:atomistic_mix_e}
    e_k = \frac{\rho Y_k e_k}{\rho Y_k } = \lim_{\Omega \to \epsilon} \frac{ \frac{1}{\mathsf{V}_\Omega} \int_\Omega \rho Y_k e_k \, d\xvec }{ \frac{1}{\mathsf{V}_\Omega} \int_\Omega \rho Y_k \, d\xvec } = \lim_{\Omega \to \epsilon} \frac{ \sum_{i \in \Omega, i \in k} e_i }{ \sum_{i \in \Omega, i \in k} m_i } = \lim_{\Omega \to \epsilon} \frac{ \mathsf{IE}_{\Omega,k} }{ \mathsf{M}_{\Omega,k} }.
\end{equation}
That is, the specific internal energy of an material is the internal energy of material $k$ contained in a small domain divided by the mass of material $k$ in that small domain as the domain becomes sufficiently small.

%------------------------------------------------------------------------
\section{Multi-fluid formulation}
%------------------------------------------------------------------------
%--------------------------------------------
\subsection{Fluid equations derived from kinetic equation}
%--------------------------------------------
\label{sec:fluid_from_kinetic}
The starting point are fluid equations derived from the kinetic equation (see general notes) complemented with the definitions of current and charge density as well as an ideal equation of state. These are summarized below.
\begin{itemize}
    \item Particle density
    \begin{equation}
        \label{eq:ktof_cons_mass}
        \frac{\partial n_s}{\partial t} + \nabla \cdot \left ( n_s \uvec_s \right ) = \hat{S}_s,
    \end{equation}

    \item Momentum
    \begin{equation}
        \label{eq:ktof_cons_mom}
        \frac{\partial m_s n_s \uvec_s}{\partial t} + \nabla \cdot \left( m_s n_s \uvec_s \uvec_s \right) - Z_s e n_s ( \Evec + \uvec_s \times \Bvec ) = \nabla \cdot \sigmavec_s + \Rvec_s + \hat{\Mvec}_s,
    \end{equation}   
    
    \item Internal Energy
    \begin{equation}
        \label{eq:ktof_cons_ie}
        \frac{\partial}{\partial t} \left ( \frac{3}{2} p_s \right ) + \nabla \cdot \left ( \frac{3}{2} p_s \uvec_s \right ) = \sigmavec_s : \nabla \uvec_s - \nabla \cdot \qvec_s + Q_{s} + \hat{Q}_s - \uvec_s \cdot \hat{\Mvec}_s + m_s K_s \hat{S}_s .  
    \end{equation}

    \item Current density
    \begin{equation}
        \label{eq:ktof_current}
        \Jvec = \sum^{\mathcal{N}}_s Z_s e n_s \uvec_s.
    \end{equation}
    
    \item Charge density
    \begin{equation}
        \label{eq:ktof_charge}
        \rho_q = \sum^{\mathcal{N}}_s Z_s e n_s.
    \end{equation}

    \item Equation of state
    \begin{equation}
        \label{eq:ktof_eos}
        p_s = n_s k_B T_s.
    \end{equation}

\end{itemize}

For the treatment in this section, we'll use the following assumptions:
\begin{enumerate}
    \item No sources ($\hat{S}_s = \hat{\Mvec}_s = \hat{Q}_s = 0$ for all $s$).
\end{enumerate}

%--------------------------------------------
\subsection{Multi-material multi-fluid model}
%--------------------------------------------
We note that the number densities in \cref{sec:fluid_from_kinetic} use as their denominator the total volume. We would instead like to express densities using variables such as that defined in \cref{eq:mm_num_den_kis}, where the denominator is only the volume of the species under consideration. The correspondence between these two densities is $n_s \to \eta_k \eta_{k,i,s} n_{k,i,s}$. Using an ideal equation of state, we have
\begin{equation}
    p_i = n_i k_B T_i \to \eta_k \eta_{k,i,s} n_{k,i,s} k_B T_{k,i,s} = \eta_k \eta_{k,i,s} p_{k,i,s},
\end{equation}
where $p_{k,i,s} = n_{k,i,s} k_B T_{k,i,s}$. The same argument above applies to the stress tensor since it is defined in terms of the pressure. Finally, the analogous is used for electron quantities as well. Thus, the governing equations in \cref{sec:fluid_from_kinetic} are now written as
\begin{equation}
    \label{eq:mf_mm_ni}
    \frac{\partial \eta_k \eta_{k,i,s} n_{k,i,s}}{\partial t} + \nabla \cdot \left ( \eta_k \eta_{k,i,s} n_{k,i,s} \uvec_{k,i,s} \right ) = 0,
\end{equation}
\begin{equation}
    \label{eq:mf_mm_ne}
    \frac{\partial \eta_k n_{k,e}}{\partial t} + \nabla \cdot \left ( \eta_k n_{k,e} \uvec_{k,e} \right ) = 0,
\end{equation}
\begin{multline}
    \label{eq:mf_mm_ui}
    \frac{\partial m_{k,i,s} \eta_k \eta_{k,i,s} n_{k,i,s} \uvec_{k,i,s}}{\partial t} + \nabla \cdot \left( m_{k,i,s} \eta_k \eta_{k,i,s} n_{k.i,s} \uvec_{k,i,s} \uvec_{k,i,s} \right) \\
    - Z_{k,i,s} e \eta_k \eta_{k,i,s} n_{k,i,s} ( \Evec + \uvec_{k,i,s} \times \Bvec ) = \nabla \cdot \left( \eta_k \eta_{k,i,s} \sigmavec_{k,i,s} \right) + \Rvec_{k,i,s},
\end{multline}
\begin{multline}
    \label{eq:mf_mm_ue}
    \frac{\partial m_{k,e} \eta_k n_{k,e} \uvec_{k,e}}{\partial t} + \nabla \cdot \left( m_{k,e} \eta_k n_{k,e} \uvec_{k,e} \uvec_{k,e} \right) \\
    + e \eta_k n_{k,e} ( \Evec + \uvec_{k,e} \times \Bvec ) = \nabla \cdot \left( \eta_k \sigmavec_{k,e} \right) + \Rvec_{k,e},
\end{multline}
\begin{equation}
    \label{eq:mf_mm_iei}
    \frac{\partial}{\partial t} \left ( \frac{3}{2} \eta_k \eta_{k,i,s} p_{k,i,s} \right ) + \nabla \cdot \left ( \frac{3}{2} \eta_k \eta_{k,i,s} p_{k,i,s} \uvec_{k,i,s} \right ) = \eta_k \eta_{k,i,s} \sigmavec_{k,i,s} : \nabla \uvec_{k,i,s} - \nabla \cdot \qvec_{k,i,s} + Q_{k,i,s},
\end{equation}
\begin{equation}
    \label{eq:mf_mm_iee}
    \frac{\partial}{\partial t} \left ( \frac{3}{2} \eta_k p_{k,e} \right ) + \nabla \cdot \left ( \frac{3}{2} \eta_k p_{k,e} \uvec_{k,e} \right ) = \eta_k \sigmavec_{k,e} : \nabla \uvec_{k,e} - \nabla \cdot \qvec_{k,e} + Q_{k,e},
\end{equation}
\begin{equation}
    \label{eq:mf_mm_curr_density}
    \Jvec_k = e \eta_k \left( \sum_s^{\mathcal{N}_k} Z_{k,i,s} \eta_{k,i,s} n_{k,i,s} \uvec_{k,i,s} - n_{k,e} \uvec_{k,e} \right),
\end{equation}
\begin{equation}
    \Jvec = \sum_k^\mathcal{M} \Jvec_k
\end{equation}
\begin{equation}
    \label{eq:mf_mm_mass_density}
    \rho_{q,k} = e \eta_k \left( \sum_s^{\mathcal{N}_k} Z_{k,i,s} \eta_{k,i,s} n_{k,i,s} - n_{k,e} \right),
\end{equation}
\begin{equation}
    \rho_q = \sum_k^\mathcal{M} \rho_{q,k},
\end{equation}
\begin{equation}
    \label{eq:mf_mm_eos_ion}
    p_{k,i,s} = n_{k,i,s} k_B T_{k,i,s},
\end{equation}
\begin{equation}
    \label{eq:mf_mm_eos_elec}
    p_{k,e} = n_{k,e} k_B T_{k,e}.
\end{equation}

%--------------------------------------------
\subsection{Single-material multi-fluid model}
%--------------------------------------------
\label{sec:mf_np1fluid_equations}
The equations are the same as the multi-material formulation, but there is only one material now and hence the $k$ subscript can be dropped and all $\eta_k = 1$. Thus we have 
\begin{equation}
    \label{eq:mf_np1fluid_ni}
    \frac{\partial \eta_{i,s} n_{i,s}}{\partial t} + \nabla \cdot \left ( \eta_{i,s} n_{i,s} \uvec_{i,s} \right ) = 0,
\end{equation}
\begin{equation}
    \label{eq:mf_np1fluid_ne}
    \frac{\partial n_e}{\partial t} + \nabla \cdot \left ( n_e \uvec_e \right ) = 0,
\end{equation}
\begin{equation}
    \label{eq:mf_np1fluid_ui}
    \frac{\partial m_{i,s} \eta_{i,s} n_{i,s} \uvec_{i,s}}{\partial t} + \nabla \cdot \left( m_{i,s} \eta_{i,s} n_{i,s} \uvec_{i,s} \uvec_{i,s} \right) - Z_{i,s} e \eta_{i,s} n_{i,s} ( \Evec + \uvec_{i,s} \times \Bvec ) = \nabla \cdot \left( \eta_{i,s} \sigmavec_{i,s} \right) + \Rvec_{i,s},
\end{equation}
\begin{equation}
    \label{eq:mf_np1fluid_ue}
    \frac{\partial m_e n_e \uvec_e}{\partial t} + \nabla \cdot \left( m_e n_e \uvec_e \uvec_e \right) + e n_e ( \Evec + \uvec_e \times \Bvec ) = \nabla \cdot \sigmavec_e + \Rvec_e,
\end{equation}
\begin{equation}
    \label{eq:mf_np1fluid_iei}
    \frac{\partial}{\partial t} \left ( \frac{3}{2} \eta_{i,s} p_{i,s} \right ) + \nabla \cdot \left ( \frac{3}{2} \eta_{i,s} p_{i,s} \uvec_{i,s} \right ) = \eta_{i,s} \sigmavec_{i,s} : \nabla \uvec_{i,s} - \nabla \cdot \qvec_{i,s} + Q_{i,s},
\end{equation}
\begin{equation}
    \label{eq:mf_np1fluid_iee}
    \frac{\partial}{\partial t} \left ( \frac{3}{2} p_e \right ) + \nabla \cdot \left ( \frac{3}{2} p_e \uvec_e \right ) = \sigmavec_e : \nabla \uvec_e - \nabla \cdot \qvec_e + Q_e,
\end{equation}
\begin{equation}
    \label{eq:mf_np1fluid_curr_density}
    \Jvec = e \left(\sum_s^\mathcal{N} Z_{i,s} \eta_{i,s} n_{i,s} \uvec_{i,s} - n_e \uvec_e \right),
\end{equation}
\begin{equation}
    \label{eq:mf_np1fluid_mass_density}
    \rho_q = e \left( \sum_s^\mathcal{N} Z_{i,s} \eta_{i,s} n_{i,s} - n_e \right),
\end{equation}
\begin{equation}
    \label{eq:mf_np1fluid_eos_ion}
    p_{i,s} = n_{i,s} k_B T_{i,s},
\end{equation}
\begin{equation}
    \label{eq:mf_np1fluid_eos_elec}
    p_e = n_e k_B T_e.
\end{equation}

%--------------------------------------------
\subsection{Single-material two-fluid model}
%--------------------------------------------
If there is only one ion species, then we have the two-fluid model. For this case $\eta_{i,s} n_{i,s}$, $\eta_{i,s} p_{i,s}$, and $\eta_{i,s} \sigmavec_{i,s}$ simplify to $\eta_i n_i$, $\eta_i p_i$, and $\eta_i \sigmavec_i$, respectively. Since $\eta_i \approx 1$, we have 
\begin{equation}
    \label{eq:mf_2fluid_ni}
    \frac{\partial n_i}{\partial t} + \nabla \cdot \left ( n_i \uvec_i \right ) = 0,
\end{equation}
\begin{equation}
    \label{eq:mf_2fluid_ne}
    \frac{\partial n_e}{\partial t} + \nabla \cdot \left ( n_e \uvec_e \right ) = 0,
\end{equation}
\begin{equation}
    \label{eq:mf_2fluid_ui}
    \frac{\partial m_i n_i \uvec_i}{\partial t} + \nabla \cdot \left ( m_i n_i \uvec_i \uvec_i \right) - Z e n_i ( \Evec + \uvec_i \times \Bvec ) = \nabla \cdot \sigmavec_i + \Rvec_i,
\end{equation}
\begin{equation}
    \label{eq:mf_2fluid_ue}
    \frac{\partial m_e n_e \uvec_e}{\partial t} + \nabla \cdot \left ( m_e n_e \uvec_e \uvec_e \right) + e n_e ( \Evec + \uvec_e \times \Bvec ) = \nabla \cdot \sigmavec_e + \Rvec_e,
\end{equation}
\begin{equation}
    \label{eq:mf_2fluid_iei}
    \frac{\partial}{\partial t} \left ( \frac{3}{2} p_i \right ) + \nabla \cdot \left ( \frac{3}{2} p_i \uvec_i \right ) = \sigmavec_i : \nabla \uvec_i - \nabla \cdot \qvec_i + Q_i,
\end{equation}
\begin{equation}
    \label{eq:mf_2fluid_iee}
    \frac{\partial}{\partial t} \left ( \frac{3}{2} p_e \right ) + \nabla \cdot \left ( \frac{3}{2} p_e \uvec_e \right ) = \sigmavec_e : \nabla \uvec_e - \nabla \cdot \qvec_e + Q_e,
\end{equation}
\begin{equation}
    \label{eq:mf_2fluid_curr_density}
    \Jvec = e (Z n_i \uvec_i - n_e \uvec_e),
\end{equation}
\begin{equation}
    \label{eq:mf_2fluid_mass_density}
    \rho_q = e (Z n_i - n_e).
\end{equation}
\begin{equation}
    \label{eq:mf_2fluid_eos_ion}
    p_i = n_i k_B T_i,
\end{equation}
\begin{equation}
    \label{eq:mf_2fluid_eos_elec}
    p_e = n_e k_B T_e.
\end{equation}

%------------------------------------------------------------------------
\section{Single-fluid formulation}
%------------------------------------------------------------------------
A single-fluid formulation is defined as that for which there is a single velocity vector. For the single-fluid formulation, we'll use the following assumptions
\begin{enumerate}
    \item $m_{k,e} \approx 0$ since $m_{k,e} \ll m_{k,i,s}$. \label{eq:sf_no_e_mass} 
    \item Quasi-neutrality, i.e.\@ $\sum_s^{\mathcal{N}_k} Z_{k,i,s} \eta_{k,i,s} n_{k,i,s} = n_{k,e}$. \label{eq:sf_quasi_neutrality}
\end{enumerate}

%--------------------------------------------
\subsection{Single-fluid variables}
%--------------------------------------------
\label{sec:sf_vars}
We now introduce the following single-fluid variables
\begin{equation}
    \label{eq:sf_vars_u_ki}
    \uvec_{k,i} = \sum_s^{\mathcal{N}_k} Y_{k,i,s} \uvec_{k,i,s} 
\end{equation}
\begin{equation}
    \label{eq:sf_vars_u}
    \uvec = \sum_k^\mathcal{M} Y_k \uvec_{k,i}.
\end{equation}
\begin{equation}
    \label{eq:sf_vars_e_k}
    e_k = e_{k,i} + e_{k,e}.
\end{equation}
\begin{equation}
    \label{eq:sf_vars_e}
    e = \sum_k^\mathcal{M} Y_k e_k.
\end{equation}
\begin{equation}
    h_k = h_{k,i} + h_{k,e}.
\end{equation}
\begin{equation}
    h = \sum_k^\mathcal{M} Y_k h_k.
\end{equation}

%--------------------------------------------
\subsection{Total sum of collisional terms}
%--------------------------------------------
In the general notes, we derived the following two expressions
\begin{equation}
    \sum_s^\mathcal{N} \Rvec_s = 0,
\end{equation}
\begin{equation}
    \sum_s^\mathcal{N} Q_s = - \sum_s^\mathcal{N} \uvec_s \cdot \Rvec_s.
\end{equation}
Using the current notation, the above two are re-written as
\begin{equation}
    \label{eq:sf_sum_mom_collisions_mat}
    \sum_k^\mathcal{M} \left( \sum_s^{\mathcal{N}_k} \Rvec_{k,i,s} + \Rvec_{k,e} \right) = 0.
\end{equation}
\begin{equation}
    \label{eq:sf_sum_energy_collisions_mat}
    \sum_k^\mathcal{M} \left( \sum_s^{\mathcal{N}_k} Q_{k,i,s} + Q_{k,e} \right) = - \sum_k^\mathcal{M} \left( \sum_s^{\mathcal{N}_k} \uvec_{k,i,s} \cdot \Rvec_{k,i,s} + \uvec_{k,e} \cdot \Rvec_{k,e} \right).
\end{equation}

%--------------------------------------------
\subsection{The general model}
%--------------------------------------------
\label{sec:sf_sm_ms_general}

\subsubsection{Mass}

Multiplying \cref{eq:mf_mm_ni} by $m_{k,i,s}$ and using \cref{eq:mm_mass_number_densities_i} gives
\begin{equation*}
    \frac{\partial \eta_k \eta_{k,i,s} \rho_{k,i,s}}{\partial t} + \nabla \cdot \left( \eta_k \eta_{k,i,s} \rho_{k,i,s} \uvec_{k,i,s} \right) = 0.
\end{equation*}
Using \cref{eq:mm_tri_relation_kis}, the above becomes
\begin{equation*}
    \frac{\partial \eta_k \rho_k Y_{k,i,s}}{\partial t} + \nabla \cdot \left( \eta_k \rho_k Y_{k,i,s} \uvec_{k,i,s} \right) = 0.
\end{equation*}
We introduce the species diffusion vector
\begin{equation}
    \label{eq:sf_species_diffusion_vector}
    \jvec_{k,i,s} = \eta_k \rho_k Y_{k,i,s} \wvec_{k,i,s},
\end{equation}
where 
\begin{equation}
    \label{eq:sf_species_velocity_shift}
    \wvec_{k,i,s} = \uvec_{k,i,s} - \uvec.
\end{equation}
The mass-conservation equation is then written as
\begin{equation*}
    \frac{\partial \eta_k \rho_k Y_{k,i,s}}{\partial t} + \nabla \cdot \left( \eta_k \rho_k Y_{k,i,s} \uvec \right) = -\nabla \cdot \jvec_{k,i,s}.
\end{equation*}
Note that adding over all $s$ of a material gives
\begin{equation*}
    \frac{\partial \eta_k \rho_k }{\partial t} + \nabla \cdot \left( \eta_k \rho_k \uvec \right) = -\nabla \cdot \jvec_{k,i},
\end{equation*}
where 
\begin{equation}
    \label{eq:sf_species_diffusion_vector_sum}
    \jvec_{k,i} = \eta_k \rho_k \wvec_{k,i},
\end{equation}
and
\begin{equation}
    \label{eq:sf_species_velocity_shift_sum}
    \wvec_{k,i} = \uvec_{k,i} - \uvec.
\end{equation}

Multiplying \cref{eq:mf_mm_ni} by $Z_{k,i,s}$ and summing over all $s$ of a material gives
\begin{equation*}
    \frac{\partial}{\partial t} \left( \eta_k \sum_s^{\mathcal{N}_k} Z_{k,i,s} \eta_{k,i,s} n_{k,i,s} \right) + \nabla \cdot \left( \eta_k \sum_s^{\mathcal{N}_k} Z_{k,i,s} \eta_{k,i,s} n_{k,i,s} \uvec_{k,i,s} \right) = 0.
\end{equation*}
Subtracting \cref{eq:mf_mm_ne} from the above and using the assumption of quasi-neutrality in \cref{eq:sf_quasi_neutrality} gives
\begin{equation*}
    \nabla \cdot \left( \eta_k \sum_s^{\mathcal{N}_k} Z_{k,i,s} \eta_{k,i,s} n_{k,i,s} \uvec_{k,i,s} - n_{k,e} \uvec_{k,e} \right) = 0,
\end{equation*}
which is equivalent to
\begin{equation*}
    \nabla \cdot \Jvec_k = 0.
\end{equation*}
Summing over all $k$ gives
\begin{equation*}
    \nabla \cdot \Jvec = 0.
\end{equation*}

\subsubsection{Momentum}

We now add \cref{eq:mf_mm_ui,eq:mf_mm_ue} and use \cref{eq:sf_no_e_mass} to obtain
\begin{multline*}
    \frac{\partial m_{k,i,s} \eta_k \eta_{k,i,s} n_{k,i,s} \uvec_{k,i,s}}{\partial t} + \nabla \cdot \left( m_{k,i,s} \eta_k \eta_{k,i,s} n_{k,i,s} \uvec_{k,i,s} \uvec_{k,i,s} \right) -e \eta_k \left( Z_{k,i,s} \eta_{k,i,s}  n_{k,i,s} - n_{k,e} \right) \Evec \\
    - e \eta_k \left( Z_{k,i,s} \eta_{k,i,s}  n_{k,i,s} \uvec_{k,i,s} - n_{k,e} \uvec_{k,e} \right) \times \Bvec = \nabla \cdot \left[ \eta_k \left( \eta_{k,i,s}  \sigmavec_{k,i,s} + \sigmavec_{k,e} \right) \right] + \Rvec_{k,i,s} + \Rvec_{k,e}.
\end{multline*}
Using \cref{eq:mm_mass_number_densities_i} the above becomes
\begin{multline*}
    \frac{\partial \eta_k \eta_{k,i,s} \rho_{k,i,s} \uvec_{k,i,s}}{\partial t} + \nabla \cdot \left( \eta_k \eta_{k,i,s} \rho_{k,i,s} \uvec_{k,i,s} \uvec_{k,i,s} \right) -e \eta_k \left( Z_{k,i,s} \eta_{k,i,s} n_{k,i,s} - n_{k,e} \right) \Evec \\
    - e \eta_k \left( Z_{k,i,s} \eta_{k,i,s} n_{k,i,s} \uvec_{k,i,s} - n_{k,e} \uvec_{k,e} \right) \times \Bvec = \nabla \cdot \left[ \eta_k \left( \eta_{k,i,s} \sigmavec_{k,i,s} + \sigmavec_{k,e} \right) \right] + \Rvec_{k,i,s} + \Rvec_{k,e}.
\end{multline*}
Using \cref{eq:mm_tri_relation_kis} we get
\begin{multline*}
    \frac{\partial \eta_k \rho_k Y_{k,i,s} \uvec_{k,i,s}}{\partial t} + \nabla \cdot \left( \eta_k \rho_k Y_{k,i,s} \uvec_{k,i,s} \uvec_{k,i,s} \right) -e \eta_k \left( Z_{k,i,s} \eta_{k,i,s} n_{k,i,s} - n_{k,e} \right) \Evec \\
    - e \eta_k \left( Z_{k,i,s} \eta_{k,i,s} n_{k,i,s} \uvec_{k,i,s} - n_{k,e} \uvec_{k,e} \right) \times \Bvec = \nabla \cdot \left[ \eta_k \left( \eta_{k,i,s} \sigmavec_{k,i,s} + \sigmavec_{k,e} \right) \right] + \Rvec_{k,i,s} + \Rvec_{k,e}.
\end{multline*}
Summing over all $s$ of a material, using the assumption of \cref{eq:sf_quasi_neutrality} and the definition of the current in \cref{eq:mf_mm_curr_density} the above simplifies to
\begin{multline*}
    \frac{\partial}{\partial t} \sum_s^{\mathcal{N}_k} \eta_k \rho_k Y_{k,i,s} \uvec_{k,i,s} + \nabla \cdot \sum_s^{\mathcal{N}_k} \eta_k \rho_k Y_{k,i,s} \uvec_{k,i,s} \uvec_{k,i,s} - \Jvec_k \times \Bvec = \\
    \nabla \cdot \left[ \eta_k \left( \sum_s^{\mathcal{N}_k} \eta_{k,i,s} \sigmavec_{k,i,s} + \sigmavec_{k,e} \right) \right] + \sum_s^{\mathcal{N}_k} \Rvec_{k,i,s} + \Rvec_{k,e}.
\end{multline*}
We now write the momentum equation as
\begin{multline*}
    \frac{\partial}{\partial t} \sum_s^{\mathcal{N}_k} \eta_k \rho_k Y_{k,i,s} \uvec_{k,i,s} + \nabla \cdot \sum_s^{\mathcal{N}_k} \left[ \eta_k \rho_k Y_{k,i,s} \uvec_{k,i,s} \uvec + \eta_k \rho_k Y_{k,i,s} \uvec_{k,i,s} \left( \uvec_{k,i,s} - \uvec \right) \right] - \Jvec_k \times \Bvec = \\
    \nabla \cdot \left[ \eta_k \left( \sum_s^{\mathcal{N}_k} \eta_{k,i,s} \sigmavec_{k,i,s} + \sigmavec_{k,e} \right) \right] + \sum_s^{\mathcal{N}_k} \Rvec_{k,i,s} + \Rvec_{k,e}.
\end{multline*}
Using the definition of $\jvec_{k,i,s}$ the above becomes
\begin{multline*}
    \frac{\partial}{\partial t} \sum_s^{\mathcal{N}_k} \eta_k \rho_k Y_{k,i,s} \uvec_{k,i,s} + \nabla \cdot \sum_s^{\mathcal{N}_k} \left( \eta_k \rho_k Y_{k,i,s} \uvec_{k,i,s} \uvec  \right) - \Jvec_k \times \Bvec = \\
    \nabla \cdot \left[ \eta_k \left( \sum_s^{\mathcal{N}_k} \eta_{k,i,s} \sigmavec_{k,i,s} + \sigmavec_{k,e} \right) \right] - \nabla \cdot \sum_s^{\mathcal{N}_k} \jvec_{k,i,s} \uvec_{k,i,s} + \sum_s^{\mathcal{N}_k} \Rvec_{k,i,s} + \Rvec_{k,e}.
\end{multline*}
We now sum over all materials to obtain
\begin{multline*}
    \frac{\partial}{\partial t} \sum_k^\mathcal{M} \sum_s^{\mathcal{N}_k} \eta_k \rho_k Y_{k,i,s} \uvec_{k,i,s} + \nabla \cdot \sum_k^\mathcal{M} \sum_s^{\mathcal{N}_k} \left( \eta_k \rho_k Y_{k,i,s} \uvec_{k,i,s} \uvec  \right) - \Jvec \times \Bvec = \\
    \nabla \cdot \sum_k^\mathcal{M} \left( \sum_s^{\mathcal{N}_k} \eta_{k,i,s} \sigmavec_{k,i,s} + \sigmavec_{k,e} \right) - \nabla \cdot \sum_k^\mathcal{M} \sum_s^{\mathcal{N}_k} \jvec_{k,i,s} \uvec_{k,i,s} + \sum_k^\mathcal{M} \left( \sum_s^{\mathcal{N}_k} \Rvec_{k,i,s} + \Rvec_{k,e} \right).
\end{multline*}
We note that
\begin{equation}
    \sum_k^\mathcal{M} \sum_s^{\mathcal{N}_k} \eta_k \rho_k Y_{k,i,s} \uvec_{k,i,s} =  \sum_k^\mathcal{M} \sum_s^{\mathcal{N}_k} \rho Y_k Y_{k,i,s} \uvec_{k,i,s} = \rho \uvec.
\end{equation}
Using the above and \cref{eq:sf_sum_mom_collisions_mat} we get
\begin{equation*}
    \frac{\partial \rho \uvec}{\partial t} + \nabla \cdot \left( \rho \uvec \uvec \right) - \Jvec \times \Bvec = \nabla \cdot \sum_k^\mathcal{M} \left[ \eta_k \left( \sum_s^{\mathcal{N}_k} \eta_{k,i,s} \sigmavec_{k,i,s} + \sigmavec_{k,e} \right) \right] - \nabla \cdot \sum_k^\mathcal{M} \sum_s^{\mathcal{N}_k} \jvec_{k,i,s} \uvec_{k,i,s}.
\end{equation*}
Finally, we write the above as
\begin{equation*}
    \frac{\partial \rho \uvec}{\partial t} + \nabla \cdot \left( \rho \uvec \uvec \right) - \Jvec \times \Bvec = \nabla \cdot \sum_k^\mathcal{M} \left[ \eta_k \left( \sigmavec_{k,i} + \sigmavec_{k,e} \right) \right] - \nabla \cdot \sum_k^\mathcal{M} \sum_s^{\mathcal{N}_k} \jvec_{k,i,s} \uvec_{k,i,s},
\end{equation*}
or 
\begin{equation*}
    \frac{\partial \rho \uvec}{\partial t} + \nabla \cdot \left( \rho \uvec \uvec \right) - \Jvec \times \Bvec = \nabla \cdot \sum_k^\mathcal{M} \eta_k \sigmavec_k - \nabla \cdot \sum_k^\mathcal{M} \sum_s^{\mathcal{N}_k} \jvec_{k,i,s} \uvec_{k,i,s},
\end{equation*}
where $\sigmavec_k = \sigmavec_{k,i} + \sigmavec_{k,e}$.

Given the assumption in \cref{eq:sf_no_e_mass}, \cref{eq:mf_mm_ue} for electron momentum becomes
\begin{equation*}
    e \eta_k n_{k,e} ( \Evec + \uvec_{k,e} \times \Bvec ) = \nabla \cdot \left( \eta_k \sigmavec_{k,e} \right) + \Rvec_{k,e},
\end{equation*}
Using the definition of the current in \cref{eq:mf_mm_curr_density}, the left-hand side can be re-written as
\begin{equation*}
    e \eta_k n_{k,e} ( \Evec + \uvec_{k,e} \times \Bvec ) = e \eta_k n_{k,e} \Evec + \left( e \eta_k \sum_s^{\mathcal{N}_k} Z_{k,i,s} \eta_{k,i,s} n_{k,i,s} \uvec_{k,i,s} - \Jvec_k \right) \times \Bvec,
\end{equation*}
and thus the electron momentum equation becomes
\begin{equation*}
    e \eta_k n_{k,e} \Evec + e \eta_k \sum_s^{\mathcal{N}_k} Z_{k,i,s} \eta_{k,i,s} n_{k,i,s} \uvec_{k,i,s} \times \Bvec = \Jvec_k \times \Bvec + \nabla \cdot \left( \eta_k \sigmavec_{k,e} \right) + \Rvec_{k,e},
\end{equation*}
Using the definition of $\jvec_{k,i,s}$, we can write
\begin{align*}
    \sum_s^{\mathcal{N}_k} Z_{k,i,s} \eta_{k,i,s} n_{k,i,s} \uvec_{k,i,s} \times \Bvec &= \sum_s^{\mathcal{N}_k} \left[ Z_{k,i,s} \eta_{k,i,} n_{k,i,s} \uvec + Z_{k,i,s} \eta_{k,i,s} n_{k,i,s} \left( \uvec_{k,i,s} - \uvec \right) \right] \times \Bvec \\
    &= \sum_s^{\mathcal{N}_k} \left( Z_{k,i,s} \eta_{k,i,s} n_{k,i,s} \uvec + Z_{k,i,s} \eta_{k,i,s} n_{k,i,s} \frac{\jvec_{k,i,s}}{\eta_k \rho_k Y_{k,i,s}} \right) \times \Bvec.
\end{align*}
Using \cref{eq:mm_mass_number_densities_i} we get
\begin{align*}
    \sum_s^{\mathcal{N}_k} Z_{k,i,s} \eta_{k,i,s} n_{k,i,s} \uvec_{k,i,s} \times \Bvec &= \sum_s^{\mathcal{N}_k} \left( Z_{k,i,s} \eta_{k,i,s} n_{k,i,s} \uvec +  Z_{k,i,s} \eta_{k,i,s} \frac{\rho_{k,i,s}}{m_{k,i,s}} \frac{\jvec_{k,i,s}}{\eta_k \rho_k Y_{k,i,s}} \right) \times \Bvec \\
    &= \sum_s^{\mathcal{N}_k} \left( Z_{k,i,s} \eta_{k,i,s} n_{k,i,s} \uvec + \frac{Z_{k,i,s} \jvec_{k,i,s}}{m_{k,i,s} \eta_k} \right) \times \Bvec \\
    &= \sum_s^{\mathcal{N}_k} Z_{k,i,s} \eta_{k,i,s} n_{k,i,s} \uvec \times \Bvec + \sum_s^{\mathcal{N}_k} \frac{Z_{k,i,s} \jvec_{k,i,s}}{m_{k,i,s} \eta_k} \times \Bvec \\
    &= n_{k,e} \uvec \times \Bvec + \sum_s^{\mathcal{N}_k} \frac{Z_{k,i,s} \jvec_{k,i,s}}{m_{k,i,s} \eta_k} \times \Bvec.
\end{align*}
Thus, the electron momentum equation becomes
\begin{equation*}
    e \eta_k n_{k,e} \left( \Evec + \uvec \times \Bvec \right) = \Jvec_k \times \Bvec + \nabla \cdot \left( \eta_k \sigmavec_{k,e} \right) + \Rvec_{k,e} - e \sum_s^{\mathcal{N}_k} \frac{Z_{k,i,s} \jvec_{k,i,s}}{m_{k,i,s}} \times \Bvec.
\end{equation*}
We define the total electron density as 
\begin{equation*}
    n_e = \sum_k^\mathcal{M} \eta_k n_{k,e}.
\end{equation*}
Thus, summing over all $k$ gives
\begin{equation*}
    e n_e \left( \Evec + \uvec \times \Bvec \right) = \Jvec \times \Bvec + \nabla \cdot \sum_k^\mathcal{M} \eta_k \sigmavec_{k,e} + \sum_k^\mathcal{M} \Rvec_{k,e} - e \sum_k^\mathcal{M} \sum_s^{\mathcal{N}_k} \frac{Z_{k,i,s} \jvec_{k,i,s}}{m_{k,i,s}} \times \Bvec.
\end{equation*}
or
\begin{equation}
    \Evec + \uvec \times \Bvec = \frac{1}{e n_e} \left( \Jvec \times \Bvec + \nabla \cdot \sum_k^\mathcal{M} \eta_k \sigmavec_{k,e} + \sum_k^\mathcal{M} \Rvec_{k,e} \right) - \sum_k^\mathcal{M} \sum_s^{\mathcal{N}_k} \frac{Z_{k,i,s} \jvec_{k,i,s}}{n_e m_{k,i,s}} \times \Bvec.
\end{equation}

\subsubsection{Internal energy}

Use the definition of $\wvec_{k,i,s}$ to re-write \cref{eq:mf_mm_iei} and \cref{eq:mf_mm_iee} as follows
\begin{multline*}
    \frac{\partial}{\partial t} \left( \frac{3}{2} \eta_k \eta_{k,i,s} p_{k,i,s} \right) + \nabla \cdot \left( \frac{3}{2} \eta_k \eta_{k,i,s} p_{k,i,s} \uvec \right) = \\
    \eta_k \eta_{k,i,s} \sigmavec_{k,i,s} : \nabla \uvec - \nabla \cdot \qvec_{k,i,s} + Q_{k,i,s} - \nabla \cdot \left( \frac{3}{2} \eta_k \eta_{k,i,s} p_{k,i,s} \wvec_{k,i,s} \right) + \eta_k \eta_{k,i,s} \sigmavec_{k,i,s} : \nabla \wvec_{k,i,s} .
\end{multline*}
\begin{multline*}
    \frac{\partial}{\partial t} \left( \frac{3}{2} \eta_k p_{k,e} \right) + \nabla \cdot \left( \frac{3}{2} \eta_k p_{k,e} \uvec \right) = \\
    \eta_k \sigmavec_{k,e} : \nabla \uvec - \nabla \cdot \qvec_{k,e} + Q_{k,e} - \nabla \cdot \left( \frac{3}{2} \eta_k p_{k,e} \wvec_{k,e} \right) + \eta_k \sigmavec_{k,e} : \nabla \wvec_{k,e}.
\end{multline*}

%--------------------------------------------
\subsection{Multi-material single-fluid model}
%--------------------------------------------
\label{sec:sf_mm_ms}

We add two extra assumptions
\begin{enumerate}
    \item All ions of a material $k$ move with the same ion velocity $\uvec_{k,i}$. \label{eq:single_ion_vel}
    \item All ions of a material $k$ have the same temperature $T_{k,i}$ \label{eq:single_ion_T}
\end{enumerate}

If all ions of a material move with a single ion velocity, then the definition of $\uvec_k$ in \cref{eq:sf_vars_u_ki} gives 
\begin{equation}
    \uvec_{k,i,s} = \uvec_{k,i},
\end{equation} 
which gives for the species diffusion vector
\begin{equation}
    \label{sec:sf_mm_ms_diffusion}
    \jvec_{k,i,s} = \rho Y_k Y_{k,i,s} \wvec_{k,i,s} = \rho Y_k Y_{k,i,s} \wvec_{k,i}.
\end{equation} 

Adding over all ion species of a material and using expressions for total ion quantities, the ion pressure equation becomes
\begin{equation*}
    \frac{\partial}{\partial t} \left( \frac{3}{2} \eta_k p_{k,i} \right) + \nabla \cdot \left( \frac{3}{2} \eta_k p_{k,i} \uvec \right) = \\ 
    \eta_k \sigmavec_{k,i} : \nabla \uvec - \nabla \cdot \qvec_{k,i} + Q_{k,i} - \nabla \cdot \left( \frac{3}{2} \eta_k p_{k,i} \wvec_{k,i} \right) + \eta_k \sigmavec_{k,i} : \nabla \wvec_{k,i}.
\end{equation*}
Similarly, the ion EOS is multiplied by $\eta_{k,i,s}$ on both sides and then added over all species of a material to obtain
\begin{equation}
    p_{k,i} = n_{k,i} k_B T_{k,i}.
\end{equation}

To summarize, we have
\begin{equation}
    \label{eq:sf_mm_ms_ni}
    \frac{\partial \eta_k \rho_k Y_{k,i,s}}{\partial t} + \nabla \cdot \left( \eta_k \rho_k Y_{k,i,s} \uvec \right) = -\nabla \cdot \jvec_{k,i,s},
\end{equation}
\begin{equation}
    \label{eq:sf_mm_ms_ne}
    \nabla \cdot \Jvec = 0,
\end{equation}
\begin{equation}
    \label{eq:sf_mm_ms_ui}
    \frac{\partial \rho \uvec}{\partial t} + \nabla \cdot \left( \rho \uvec \uvec \right) - \Jvec \times \Bvec = \nabla \cdot \sum_k^\mathcal{M} \eta_k \sigmavec_k - \nabla \cdot \sum_k^\mathcal{M} \jvec_{k,i} \uvec_{k,i},
\end{equation}
\begin{equation}
    \label{eq:sf_mm_ms_ue}
    \Evec + \uvec \times \Bvec = \frac{1}{e n_e} \left( \Jvec \times \Bvec + \nabla \cdot \sum_k^\mathcal{M} \eta_k \sigmavec_{k,e} + \sum_k^\mathcal{M} \Rvec_{k,e} \right) - \sum_k^\mathcal{M} \sum_s^{\mathcal{N}_k} \frac{Z_{k,i,s} \jvec_{k,i,s}}{n_e m_{k,i,s}} \times \Bvec,
\end{equation}
\begin{multline}
    \label{eq:sf_mm_ms_iei}
    \frac{\partial}{\partial t} \left( \frac{3}{2} \eta_k p_{k,i} \right) + \nabla \cdot \left( \frac{3}{2} \eta_k p_{k,i} \uvec \right) = \\
    \eta_k \sigmavec_{k,i} : \nabla \uvec - \nabla \cdot \qvec_{k,i} + Q_{k,i} - \nabla \cdot \left( \frac{3}{2} \eta_k p_{k,i} \wvec_{k,i} \right) + \eta_k \sigmavec_{k,i} : \nabla \wvec_{k,i},
\end{multline}
\begin{multline}
    \label{eq:sf_mm_ms_iee}
    \frac{\partial}{\partial t} \left( \frac{3}{2} \eta_k p_{k,e} \right) + \nabla \cdot \left( \frac{3}{2} \eta_k p_{k,e} \uvec \right) = \\
    \eta_k \sigmavec_{k,e} : \nabla \uvec - \nabla \cdot \qvec_{k,e} + Q_{k,e} - \nabla \cdot \left( \frac{3}{2} \eta_k p_{k,e} \wvec_{k,e} \right) + \eta_k \sigmavec_{k,e} : \nabla \wvec_{k,e},
\end{multline}
\begin{equation}
    \label{eq:sf_mm_ms_curr_density}
    \Jvec_k = e \eta_k n_{k,e} \left( \uvec_{k,i} - \uvec_{k,e} \right),
\end{equation}
\begin{equation}
    \Jvec = \sum_k^\mathcal{M} \Jvec_k,
\end{equation}
\begin{equation}
    \label{eq:sf_mm_ms_mass_density}
    \rho_{q,k} = 0,
\end{equation}
\begin{equation}
    \rho_q = \sum_k^\mathcal{M} \rho_{q,k},
\end{equation}
\begin{equation}
    \label{eq:sf_mm_ms_eos_ion}
    p_{k,i} = n_{k,i} k_B T_{k,i},
\end{equation}
\begin{equation}
    \label{eq:sf_mm_ms_eos_elec}
    p_{k,e} = n_{k,e} k_B T_{k,e}.
\end{equation}

Note that \cref{eq:sf_mm_ms_ni} gives $\eta_k \rho_k Y_{k,i,s}$. Summing over all $s$ gives $\eta_k \rho_k$. Summing over all $k$ gives $\rho$. Dividing $\eta_k \rho_k Y_{k,i,s}$ by $\eta_k \rho_k$ gives $Y_{k,i,s}$. Dividing $\eta_k \rho_k$ by $\rho$ gives $Y_k$.

%--------------------------------------------
\subsection{Single-material single-fluid multi-species model}
%--------------------------------------------
\label{sec:sf_sm_ms}

Assuming there is only one material, then $\wvec_{k,i} = \uvec_{k,i} - \uvec = 0$ and thus \cref{sec:sf_mm_ms_diffusion} gives $\jvec_{k,i,s} = 0$.  Summing over all $s$ would then give $\jvec_{k,i} = 0$. Additionally, $\eta_k=1$ and $\rho_k = \rho$. Thus, the model in \cref{sec:sf_mm_ms} becomes
\begin{equation}
    \label{eq:sf_sm_ms_ni}
    \frac{\partial \rho Y_{i,s}}{\partial t} + \nabla \cdot \left( \rho Y_{i,s} \uvec \right) = 0,
\end{equation}
\begin{equation}
    \label{eq:sf_sm_ms_ne}
    \nabla \cdot \Jvec = 0,
\end{equation}
\begin{equation}
    \label{eq:sf_sm_ms_ui}
    \frac{\partial \rho \uvec}{\partial t} + \nabla \cdot \left( \rho \uvec \uvec \right) - \Jvec \times \Bvec = \nabla \cdot \sigmavec,
\end{equation}
\begin{equation}
    \label{eq:sf_sm_ms_ue}
    \Evec + \uvec \times \Bvec = \frac{1}{e n_e} \left( \Jvec \times \Bvec + \nabla \cdot \sigmavec_e + \Rvec_e \right),
\end{equation}
\begin{equation}
    \label{eq:sf_sm_ms_iei}
    \frac{\partial}{\partial t} \left( \frac{3}{2} p_i \right) + \nabla \cdot \left( \frac{3}{2} p_i \uvec \right) = \sigmavec_i : \nabla \uvec - \nabla \cdot \qvec_i + Q_i,
\end{equation}
\begin{equation}
    \label{eq:sf_sm_ms_iee}
    \frac{\partial}{\partial t} \left( \frac{3}{2} p_e \right) + \nabla \cdot \left( \frac{3}{2} p_e \uvec \right) = \sigmavec_e : \nabla \uvec - \nabla \cdot \qvec_e + Q_e - \nabla \cdot \left( \frac{3}{2} p_e \wvec_e \right) + \sigmavec_e : \nabla \wvec_e,
\end{equation}
\begin{equation}
    \label{eq:sf_sm_ms_curr_density}
    \Jvec = e n_e \left( \uvec - \uvec_e \right),
\end{equation}
\begin{equation}
    \label{eq:sf_sm_ms_mass_density}
    \rho_q = 0,
\end{equation}
\begin{equation}
    \label{eq:sf_sm_ms_eos_ion}
    p_i = n_i k_B T_i,
\end{equation}
\begin{equation}
    \label{eq:sf_sm_ms_eos_elec}
    p_e = n_e k_B T_e.
\end{equation}

%--------------------------------------------
\subsection{Single-material single-fluid two-species model}
%--------------------------------------------
If there is only one ion species, along with the electron species, then the model in \cref{sec:sf_sm_ms} simplifies to
\begin{equation}
    \label{eq:sf_1ion_ni}
    \frac{\partial \rho}{\partial t} + \nabla \cdot \left( \rho \uvec \right) = 0.
\end{equation}\begin{equation}
    \label{eq:sf_1ion_ne}
    \nabla \cdot \Jvec = 0.
\end{equation}
\begin{equation}
    \label{eq:sf_1ion_ui}
    \frac{\partial \rho \uvec}{\partial t} + \nabla \cdot \left( \rho \uvec \uvec \right) - \Jvec \times \Bvec = \nabla \cdot \sigmavec.
\end{equation}
\begin{equation}
    \label{eq:sf_1ion_ue}
    \Evec + \uvec \times \Bvec = \frac{1}{e n_e} \left( \Jvec \times \Bvec + \nabla \cdot \sigmavec_e + \Rvec_e \right).
\end{equation}
\begin{equation}
    \label{eq:sf_1ion_iei}
    \frac{\partial}{\partial t} \left( \frac{3}{2} p_i \right) + \nabla \cdot \left( \frac{3}{2} p_i \uvec \right) = \sigmavec_i : \nabla \uvec - \nabla \cdot \qvec_i + Q_i.
\end{equation}
\begin{equation}
    \label{eq:sf_1ion_iee}
    \frac{\partial}{\partial t} \left( \frac{3}{2} p_e \right) + \nabla \cdot \left( \frac{3}{2} p_e \uvec \right) = \sigmavec_e : \nabla \uvec - \nabla \cdot \qvec_e + Q_e - \nabla \cdot \left( \frac{3}{2} p_e \wvec_e \right) + \sigmavec_e : \nabla \wvec_e.
\end{equation}\begin{equation}
    \label{eq:sf_1ion_curr_density}
    \Jvec = e n_e \left( \uvec - \uvec_e \right).
\end{equation}
\begin{equation}
    \label{eq:sf_1ion_mass_density}
    \rho_q = 0,
\end{equation}
\begin{equation}
    \label{eq:sf_1ion_eos_ion}
    p_i = n_i k_B T_i,
\end{equation}
\begin{equation}
    \label{eq:sf_1ion_eos_elec}
    p_e = n_e k_B T_e.
\end{equation}

%--------------------------------------------
\subsection{Alternate forms of the energy equations}
%--------------------------------------------

Thus far we've been working with plasmas that behave as ideal gasses with three degrees of freedom (the three translational degrees of freedom). For ideal gasses we also have 
\begin{equation}
    \label{eq:sf_alternate_p_kis}
    p_{k,i,s} = (\gamma - 1) \rho_{k,i,s} e_{k,i,s},
\end{equation}
\begin{equation}
    \label{eq:sf_alternate_p_ke}
    p_{k,e} = (\gamma - 1) \rho_k e_{k,e}.
\end{equation}
Multiplying both sides of \cref{eq:sf_alternate_p_kis} by $\eta_{k,i,s}$ and summing over all $s$ gives
\begin{equation}
    \label{eq:sf_alternate_p_ki}
    p_{k,i} = (\gamma - 1) \rho_k e_{k,i}.
\end{equation}
Since $\gamma = 5/3$ for the plasmas under consideration, the internal energy equation for ions given by \cref{eq:sf_mm_ms_iei} can now be written as
\begin{multline*}
    \frac{\partial \eta_k \rho_k e_{k,i}}{\partial t} + \nabla \cdot \left( \eta_k \rho_k e_{k,i} \uvec \right) = \eta_k \sigmavec_{k,i} : \nabla \uvec - \nabla \cdot \qvec_{k,i} + Q_{k,i} \\
    - \nabla \cdot \left( \eta_k \rho_k e_{k,i} \wvec_{k,i} \right) + \eta_k \sigmavec_{k,i} : \nabla \wvec_{k,i}.
\end{multline*}
Similarly, the internal energy equation for electrons given by \cref{eq:sf_mm_ms_iee} becomes
\begin{multline*}
    \frac{\partial \eta_k \rho_k e_{k,e}}{\partial t} + \nabla \cdot \left( \eta_k \rho_k e_{k,e} \uvec \right) = \eta_k \sigmavec_{k,e} : \nabla \uvec - \nabla \cdot \qvec_{k,}e + Q_{k,e} \\
    - \nabla \cdot \left( \eta_k \rho_k e_{k,e} \wvec_{k,e} \right) + \eta_k \sigmavec_{k,e} : \nabla \wvec_{k,e}.
\end{multline*}

We re-write the last term in the two equations above as
\begin{align*}
    \eta_k \sigmavec_{k,s} : \nabla \wvec_{k,s} &= \nabla \cdot \left( \wvec_{k,s} \cdot \eta_k \sigmavec_{k,s} \right) - \wvec_{k,s} \cdot \nabla \left( \eta_k \sigmavec_{k,s} \right) \\
    &= -\nabla \cdot (\eta_k p_{k,s} \wvec_{k,s} ) + \nabla \cdot \left( \wvec_{k,s} \cdot \eta_k \tvec_{k,s} \right) - \wvec_{k,s} \cdot \nabla \left( \eta_k \sigmavec_{k,s} \right),
\end{align*}
where $s$ is equal to $i$ or $e$. Thus, the ion and electron internal energy equations become
\begin{multline*}
    \frac{\partial \eta_k \rho_k e_{k,i}}{\partial t} + \nabla \cdot \left( \eta_k \rho_k e_{k,i} \uvec \right) = \eta_k \sigmavec_{k,i} : \nabla \uvec - \nabla \cdot \qvec_{k,i} + Q_{k,i} \\
    - \nabla \cdot \left[ \left( e_{k,i} + \frac{p_{k,i}}{\rho_k} \right) \eta_k \rho_k \wvec_{k,i} \right] + \nabla \cdot \left( \wvec_{k,i} \cdot \eta_k \tvec_{k,i} \right) - \wvec_{k,i} \cdot \nabla \left( \eta_k \sigmavec_{k,i} \right).
\end{multline*}
\begin{multline*}
    \frac{\partial \eta_k \rho_k e_{e,k}}{\partial t} + \nabla \cdot \left( \eta_k \rho_k e_{k,e} \uvec \right) = \eta_k \sigmavec_{k,e} : \nabla \uvec - \nabla \cdot \qvec_{k,e} + Q_{k,e} \\
    - \nabla \cdot \left[ \left( e_{k,e} + \frac{p_{k,e}}{\rho_k} \right) \eta_k \rho_k \wvec_{k,e} \right] + \nabla \cdot \left( \wvec_{k,e} \cdot \eta_k \tvec_{k,e} \right) - \wvec_{k,e} \cdot \nabla \left( \eta_k \sigmavec_{k,e} \right).
\end{multline*}
The enthalpies are given by
\begin{equation}
    h_{k,i,s} = e_{k,i,s} + \frac{p_{k,i,s}}{\rho_{k,i,s}},
\end{equation}
 \begin{equation}
    \label{eq:sf_alternate_hke_rho}
    h_{k,e} = e_{k,e} + \frac{p_{k,e}}{\rho_k}.
\end{equation}
Using expressions from \cref{sec:sf_vars}, we can show that
\begin{equation}
    \label{eq:sf_alternate_hki_rhok}
    h_{k,i} = \sum_s^{\mathcal{N}_k} Y_{k,i,s} h_{k,i,s} = \sum_s^{\mathcal{N}_k} Y_{k,i,s} e_{k,i,s} + \sum_{s}^{\mathcal{N}_k} \frac{ \eta_{k,i,s} p_{k,i,s}}{\rho_k} = e_{k,i} + \frac{p_{k,i}}{\rho_k}.
\end{equation}
Using \cref{eq:sf_alternate_hki_rhok,eq:sf_alternate_hke_rho} for $h_{k,i}$ and $h_{k,e}$ the energy equations are written as
\begin{multline*}
    \frac{\partial \eta_k \rho_k e_{k,i}}{\partial t} + \nabla \cdot \left( \eta_k \rho_k e_{k,i} \uvec \right) = \eta_k \sigmavec_{k,i} : \nabla \uvec - \nabla \cdot \qvec_{k,i} + Q_{k,i} \\
    - \nabla \cdot \left( h_{k,i} \eta_k \rho_k \wvec_{k,i} \right) + \nabla \cdot \left( \wvec_{k,i} \cdot \eta_k \tvec_{k,i} \right) - \wvec_{k,i} \cdot \nabla \left( \eta_k \sigmavec_{k,i} \right).
\end{multline*}
\begin{multline*}
    \frac{\partial \eta_k \rho_k e_{e,k}}{\partial t} + \nabla \cdot \left( \eta_k \rho_k e_{k,e} \uvec \right) = \eta_k \sigmavec_{k,e} : \nabla \uvec - \nabla \cdot \qvec_{k,e} + Q_{k,e} \\
    - \nabla \cdot \left( h_{k,e} \eta_k \rho_k \wvec_{k,e} \right) + \nabla \cdot \left( \wvec_{k,e} \cdot \eta_k \tvec_{k,e} \right) - \wvec_{k,e} \cdot \nabla \left( \eta_k \sigmavec_{k,e} \right).
\end{multline*}
Using \cref{eq:sf_species_diffusion_vector_sum} for $\jvec_{k,i}$ and $\jvec_{k,e} = \eta_k \rho_k \wvec_{k,e}$ the internal energy equations become
\begin{multline*}
    \frac{\partial \eta_k \rho_k e_{k,i}}{\partial t} + \nabla \cdot \left( \eta_k \rho_k e_{k,i} \uvec \right) = \eta_k \sigmavec_{k,i} : \nabla \uvec - \nabla \cdot \qvec_{k,i} + Q_{k,i} \\
    - \nabla \cdot \left( h_{k,i} \jvec_{k,i} \right) + \nabla \cdot \left( \wvec_{k,i} \cdot \eta_k \tvec_{k,i} \right) - \wvec_{k,i} \cdot \nabla \left( \eta_k \sigmavec_{k,i} \right).
\end{multline*}
\begin{multline*}
    \frac{\partial \eta_k \rho_k e_{e,k}}{\partial t} + \nabla \cdot \left( \eta_k \rho_k e_{k,e} \uvec \right) = \eta_k \sigmavec_{k,e} : \nabla \uvec - \nabla \cdot \qvec_{k,e} + Q_{k,e} \\
    - \nabla \cdot \left( h_{k,e} \jvec_{k,e} \right) + \nabla \cdot \left( \wvec_{k,e} \cdot \eta_k \tvec_{k,e} \right) - \wvec_{k,e} \cdot \nabla \left( \eta_k \sigmavec_{k,e} \right).
\end{multline*}
Finally, adding up the ion and electron internal energy equations and using \cref{eq:sf_vars_e_k} gives
\begin{multline}
    \label{eq:sf_alternate_e_k}
    \frac{\partial \eta_k \rho_k e_k}{\partial t} + \nabla \cdot \left( \eta_k \rho_k e_k \uvec \right) = \eta_k \sigmavec_k: \nabla \uvec - \nabla \cdot \left( \qvec_{k,i} + \qvec_{k,e} \right) + Q_{k,i} + Q_{k,e} \\
    - \nabla \cdot \left( h_{k,i} \jvec_{k,i} + h_{k,e} \jvec_{k,e} \right) + \nabla \cdot \left( \wvec_{k,i} \cdot \eta_k \tvec_{k,i} + \wvec_{k,e} \cdot \eta_k \tvec_{k,e} \right) - \wvec_{k,i} \cdot \nabla \left( \eta_k \sigmavec_{k,i} \right) - \wvec_{k,e} \cdot \nabla \left( \eta_k \sigmavec_{k,e} \right) .
\end{multline}

Summing \cref{eq:sf_alternate_e_k} over all materials and using \cref{eq:sf_vars_e} gives
\begin{multline*}
    \frac{\partial \rho e}{\partial t} + \nabla \cdot \left( \rho e \uvec \right) = \sum_k^\mathcal{M} \eta_k \sigma_k : \nabla \uvec - \nabla \cdot \sum_k^\mathcal{M} \left( \qvec_{k,i} + \qvec_{k,e} \right) + \sum_k^\mathcal{M} \left( Q_{k,i} + Q_{k,e} \right) \\
    - \nabla \cdot \sum_k^\mathcal{M} \left( h_{k,i} \jvec_{k,i} + h_{k,e} \jvec_{k,e} \right) + \nabla \cdot \sum_k^\mathcal{M} \left( \wvec_{k,i} \cdot \eta_k \tvec_{k,i} + \wvec_{k,e} \cdot \eta_k \tvec_{k,e} \right) \\
    - \sum_k^\mathcal{M} \left[ \wvec_{k,i} \cdot \nabla \left( \eta_k \sigmavec_{k,i} \right) - \wvec_{k,e} \cdot \nabla \left( \eta_k \sigmavec_{k,e} \right) \right].
\end{multline*}
We can now re-write \cref{eq:sf_sum_energy_collisions_mat} as 
\begin{equation}
    \sum_k^\mathcal{M} \left( Q_{k,i} + Q_{k,e} \right) = -\sum_k^\mathcal{M} \left( \uvec_{k,i} \cdot \Rvec_{k,i} + \uvec_{k,e} \cdot \Rvec_{k,e} \right).
\end{equation}
Thus, the total internal energy equation becomes
\begin{multline*}
    \frac{\partial \rho e}{\partial t} + \nabla \cdot \left( \rho e \uvec \right) = \sum_k^\mathcal{M} \eta_k \sigma_k : \nabla \uvec - \nabla \cdot \sum_k^\mathcal{M} \left( \qvec_{k,i} + \qvec_{k,e} \right) -\sum_k^\mathcal{M} \left( \uvec_{k,i} \cdot \Rvec_{k,i} + \uvec_{k,e} \cdot \Rvec_{k,e} \right) \\
    - \nabla \cdot \sum_k^\mathcal{M} \left( h_{k,i} \jvec_{k,i} + Y_{k,e} h_{k,e} \bar{\jvec}_{k,e} \right) + \nabla \cdot \sum_k^\mathcal{M} \left( \wvec_{k,i} \cdot \eta_k \tvec_{k,i} + \wvec_{k,e} \cdot \eta_k \tvec_{k,e} \right) \\
    - \sum_k^\mathcal{M} \left[ \wvec_{k,i} \cdot \nabla \left( \eta_k \sigmavec_{k,i} \right) - \wvec_{k,e} \cdot \nabla \left( \eta_k \sigmavec_{k,e} \right) \right].
\end{multline*}

To summarize we have
\begin{equation}
    \label{eq:sf_mm_ms_ni_final}
    \frac{\partial \eta_k \rho_k Y_{k,i,s}}{\partial t} + \nabla \cdot \left( \eta_k \rho_k Y_{k,i,s} \uvec \right) = -\nabla \cdot \jvec_{k,i,s},
\end{equation}
\begin{equation}
    \label{eq:sf_mm_ms_ne_final}
    \nabla \cdot \Jvec = 0,
\end{equation}
\begin{equation}
    \label{eq:sf_mm_ms_ui_final}
    \frac{\partial \rho \uvec}{\partial t} + \nabla \cdot \left( \rho \uvec \uvec \right) - \Jvec \times \Bvec = \nabla \cdot \sum_k^\mathcal{M} \eta_k \sigmavec_k - \nabla \cdot \sum_k^\mathcal{M} \jvec_{k,i} \uvec_{k,i},
\end{equation}
\begin{equation}
    \label{eq:sf_mm_ms_ue_final}
    \Evec + \uvec \times \Bvec = \frac{1}{e n_e} \left( \Jvec \times \Bvec + \nabla \cdot \sum_k^\mathcal{M} \eta_k \sigmavec_{k,e} + \sum_k^\mathcal{M} \Rvec_{k,e} \right) - \sum_k^\mathcal{M} \sum_s^{\mathcal{N}_k} \frac{Z_{k,i,s} \jvec_{k,i,s}}{n_e m_{k,i,s}} \times \Bvec,
\end{equation}
\begin{multline}
    \label{eq:sf_mm_ms_iei_final}
    \frac{\partial \eta_k \rho_k e_{k,i}}{\partial t} + \nabla \cdot \left( \eta_k \rho_k e_{k,i} \uvec \right) = \eta_k \sigmavec_{k,i} : \nabla \uvec - \nabla \cdot \qvec_{k,i} + Q_{k,i} \\
    - \nabla \cdot \left( h_{k,i} \jvec_{k,i} \right) + \nabla \cdot \left( \wvec_{k,i} \cdot \eta_k \tvec_{k,i} \right) - \wvec_{k,i} \cdot \nabla \left( \eta_k \sigmavec_{k,i} \right).
\end{multline}
\begin{multline}
    \label{eq:sf_mm_ms_iee_final}
    \frac{\partial \eta_k \rho_k e_{e,k}}{\partial t} + \nabla \cdot \left( \eta_k \rho_k e_{k,e} \uvec \right) = \eta_k \sigmavec_{k,e} : \nabla \uvec - \nabla \cdot \qvec_{k,e} + Q_{k,e} \\
    - \nabla \cdot \left( h_{k,e} \jvec_{k,e} \right) + \nabla \cdot \left( \wvec_{k,e} \cdot \eta_k \tvec_{k,e} \right) - \wvec_{k,e} \cdot \nabla \left( \eta_k \sigmavec_{k,e} \right).
\end{multline}
\begin{equation}
    \label{eq:sf_mm_ms_curr_density_final}
    \Jvec_k = e \eta_k n_{k,e} \left( \uvec_{k,i} - \uvec_{k,e} \right),
\end{equation}
\begin{equation}
    \Jvec = \sum_k^\mathcal{M} \Jvec_k,
\end{equation}
\begin{equation}
    \label{eq:sf_mm_ms_mass_density_final}
    \rho_{q,k} = 0,
\end{equation}
\begin{equation}
    \rho_q = \sum_k^\mathcal{M} \rho_{q,k},
\end{equation}
\begin{equation}
    \label{eq:sf_mm_ms_therm_eos_ion_final}
    p_{k,i} = n_{k,i} k_B T_{k,i},
\end{equation}
\begin{equation}
    \label{eq:sf_mm_ms_therm_eos_elec_final}
    p_{k,e} = n_{k,e} k_B T_{k,e},
\end{equation}
\begin{equation}
    \label{eq:sf_mm_ms_calo_eos_ion_final}
    p_{k,i} = (\gamma - 1) \rho_k e_{k,i},
\end{equation}
\begin{equation}
    \label{eq:sf_mm_ms_calo_eos_elec_final}
    p_{k,e} = (\gamma - 1) \rho_k e_{k,e}.
\end{equation}

Some extra notes on this model:
\begin{enumerate}
    \item As before the mass transport equation gives $\eta_k \rho_k Y_{k,i,s}$. Summing over all $s$ gives $\eta_k \rho_k$. Dividing $\eta_k \rho_k Y_{k,i,s}$ by $\eta_k \rho_k$ gives $Y_{k,i,s}$. Summing $\eta_k \rho_k$ over all $k$ gives $\rho$. Dividing $\eta_k \rho_k$ by $\rho$ gives $Y_k$.

    \item The ion internal energy equation gives $e_{k,i}$. 

    \item The electron internal energy equation gives $e_{k,e}$.

    \item If $\eta_k$ is known, we can compute $\rho_k$ from $\rho_k = \rho Y_k / \eta_k$.

    \item Using $\rho_k$ and $e_{k,i}$ in \cref{eq:sf_mm_ms_calo_eos_ion_final} we get $p_{k,i}$.

    \item Using $\rho_k$ and $e_{k,e}$ in \cref{eq:sf_mm_ms_calo_eos_elec_final} we get $p_{k,e}$.

    \item We can compute $h_{k,i}$ using \cref{eq:sf_alternate_hki_rhok}, which we repeat below
    \begin{equation*}
        h_{k,i} = e_{k,i} + \frac{p_{k,i}}{\rho_k}.
    \end{equation*}

    \item We can compute $h_{k,e}$ using \cref{eq:sf_alternate_hke_rho}, which we repeat below
    \begin{equation*}
        h_{k,e} = e_{k,e} + \frac{p_{k,e}}{\rho_k}.
    \end{equation*}

    \item We can compute $n_{k,i,s}$ using \cref{eq:mm_mass_number_densities_i}, which we repeat below
    \begin{equation*}
        m_{k,i,s} n_{k,i,s} = \rho_k Y_{k,i,s}.
    \end{equation*}

    \item We can compute $n_{k,i}$ using \cref{eq:sf_vars_n_ki,eq:mm_mass_number_densities_i}, as shown below
    \begin{equation}
        n_{k,i} = \sum_s^{\mathcal{N}_k} \eta_{k,i,s} n_{k,i,s} = \sum_s^{\mathcal{N}_k} \eta_{k,i,s} \frac{\rho_{k,i,s}}{m_{k,i,s}} = \sum_s^{\mathcal{N}_k} \frac{\rho_k Y_{k,i,s}}{m_{k,i,s}}.
    \end{equation}
    
    \item We can compute $n_{k,e}$ from quasi-neutrality, which we repeat below
    \begin{equation}
        \sum_s^{\mathcal{N}_k} Z_{k,i,s} \eta_{k,i,s} n_{k,i,s} = n_{k,e},
    \end{equation}

    \item Using $p_{k,i}$ and $n_{k,i}$ in \cref{eq:sf_mm_ms_therm_eos_ion_final} we get $T_{k,i}$.

    \item Using $p_{k,e}$ and $n_{k,e}$ in \cref{eq:sf_mm_ms_therm_eos_elec_final} we get $T_{k,e}$.
\end{enumerate}

Note that this model requires knowing the value of $\eta_k$. Thus the following equation is also solved
\begin{equation}
    \frac{\partial \eta_k}{\partial t} + \uvec \cdot \nabla \left( \eta_k \right) = \alpha_k,
\end{equation}
where $\alpha_k$ is a closure model.

%--------------------------------------------
\subsection{Generic EOS}
%--------------------------------------------
Rather than assuming ideal equations of state for each species, we can instead account for generic equations of state. To show this, we first start with

%########################################################################
\chapter{Mixtures}
%########################################################################
In this chapter we focus on fluids that consist of mixtures of elements, which we sometimes refer to as species and other times as materials. There are two approaches that we'll consider, multi-species hydrodynamics, where the constituent elements in the mixture are referred to as species, and multi-material hydrodynamics, which is a higher-fidelity extension of multi-species hydrodynamics and where we refer to the constituent elements of the mixture as materials. 

We introduce an additional fraction, the mole fraction $X_\alpha = X_\alpha(t,\xvec)$, as shown below
\begin{equation}
    X_\alpha = \frac{\mathsf{O}_k}{\mathsf{O}},
\end{equation}
where $\mathsf{O}_k$ is the number of moles of material $k$ at a point in space, and $\mathsf{O}$ is the number of all moles at a point in space. Since $\mathsf{M}_k = M_k \mathsf{O}$, where $M_k$ is the molar mass of material $k$, we can introduce two relationships between $X_k$ and $Y_k$:
\begin{equation}
\label{eq:y_intermsof_x}
    Y_k = \frac{M_k \mathsf{O}_k}{\sum_l M_l \mathsf{O}_l} = \frac{M_k X_k}{\sum_l M_l X_l}
\end{equation}
\begin{equation}
\label{eq:x_intermsof_y}
    X_k = \frac{\mathsf{M}_k / M_k}{\sum_l \mathsf{M}_l / M_l} =  \frac{Y_k / M_k}{ \sum_l Y_l / M_l}
\end{equation}


%------------------------------------------------------------------------
\section{Multi-species hydrodynamics}
%------------------------------------------------------------------------
\label{sec:multi_species_hydro}
The multi-species fluid model consists of the Navier-Stokes equations augmented with multiple scalars, where each scalar is the mass fraction of a species in the system. For this case, the species mass fractions are no longer passive scalars. For ideal gases, for example, they affect the value of $R$, which in turns affects the pressure, density, and temperature of the whole mixture. They also affect the enthalpy diffusion, which then alters the heat flux and internal energy. 

%---------------------------------
\subsection{Equations of state and thermodynamic variables}
%---------------------------------
We introduce the partial pressure $p_\alpha$ such that the total pressure is obtained from
\begin{equation}
    p = \sum_\alpha p_\alpha.
\end{equation} 
We'll also introduce the secondary partial pressure $\hat{p}_\alpha$ which satisfies
\begin{equation}
    p_\alpha = \eta_\alpha \hat{p}_\alpha.
\end{equation}
The thermal equation of state is now expressed as
\begin{equation}
    \label{eq:multi_species_thermal_eos}
    \hat{p}_\alpha = \phi_\alpha (\rho_\alpha, T_\alpha),
\end{equation}
and the caloric equation of state as
\begin{equation}
    \label{eq:multi_species_caloric_eos}
    e_\alpha = \psi_\alpha (\rho_\alpha, T_\alpha).
\end{equation}

There are two relations that can be derived from \cref{eq:relation_volume_density_mass}. One is obtained by first performing the sum
\begin{equation}
    \sum_\alpha \eta_\alpha \rho_\alpha = \sum_\alpha \rho Y_\alpha,
\end{equation}
which leads to
\begin{equation}
    \rho = \sum_\alpha \eta_\alpha \rho_\alpha.
\end{equation}
The other relation is obtained by performing the sum
\begin{equation}
    \sum_\alpha \frac{\eta_\alpha}{\rho} = \sum_\alpha \frac{Y_\alpha}{\rho_\alpha},
\end{equation}
which leads to
\begin{equation}
    \label{eq:from_rho_alpha_to_rho_Y}
    \frac{1}{\rho} = \sum_\alpha \frac{Y_\alpha}{\rho_\alpha}.
\end{equation}

Using \cref{eq:atomistic_e,eq:atomistic_mix_Y,eq:atomistic_mix_e} we have
\begin{equation}
    \label{eq:from_e_alpha_to_e}
    e = \sum_\alpha Y_\alpha e_\alpha.
\end{equation}
The species enthalpy $h_\alpha$ is defined in terms of the other species variables as follows
\begin{equation}
\label{eq:species_enthalpy}
    h_\alpha = e_\alpha + \frac{\hat{p}{_\alpha}}{\rho_\alpha}.
\end{equation}
We can now show that
\begin{align}
\label{eq:from_h_alpha_to_h}
     \sum_\alpha Y_\alpha h_\alpha  &= \sum_\alpha Y_\alpha e_\alpha + \sum_\alpha \frac{Y_\alpha \hat{p}_\alpha}{\rho_\alpha} \nonumber \\
     &= \sum_\alpha Y_\alpha e_\alpha + \sum_\alpha \frac{\eta_\alpha \hat{p}_\alpha}{\rho} \nonumber \\
     &= \sum_\alpha Y_\alpha e_\alpha + \sum_\alpha \frac{p_\alpha}{\rho} \nonumber \\
     &= e + \frac{p}{\rho} \nonumber \\
     &= h
\end{align}

In the absence of chemical reactions, condensation, or other processes, $Y_i$ is independent of temperature. For these cases, differentiating \cref{eq:from_e_alpha_to_e} and \cref{eq:from_h_alpha_to_h} gives (differentiating with respect to what temperature? $T_\alpha$, some common temperature $T$? I think this requires temperature equilibration across all $\alpha$ !!!)
\begin{equation}
    C_v = \sum_\alpha Y_\alpha C_{v,\alpha},
\end{equation}
\begin{equation}
    C_p = \sum_\alpha Y_\alpha C_{p,\alpha}.
\end{equation}

%---------------------------------
\subsection{The pressure-temperature (PT) equilibration model}
%---------------------------------
The thermodynamic variables that we'll be solving for are $\rho_\alpha$, $e_\alpha$, $\hat{p}_\alpha$, and $T_\alpha$, for a total of $4N$ variables, where $N$ is the number of species. We now assume an equilibrium state in which $\hat{p}_\alpha = p$ and $T_\alpha = T$ for all $\alpha$. Thus, the total number of unknowns is reduced to $2N+2$. The equations to be solved are then
\begin{equation}
    \label{eq:ms_eos_pt_algorithm_1}
    p = \phi_\alpha (\rho_\alpha, T),
\end{equation}
\begin{equation}
    \label{eq:ms_eos_pt_algorithm_2}
    e_\alpha = \psi_\alpha (\rho_\alpha, T).
\end{equation}
\begin{equation}
    \label{eq:ms_eos_pt_algorithm_3}
    \frac{1}{\rho} = \sum_\alpha \frac{Y_\alpha}{\rho_\alpha},
\end{equation}
\begin{equation}
    \label{eq:ms_eos_pt_algorithm_4}
    e = \sum_\alpha Y_\alpha e_\alpha.
\end{equation}
For the above, $\rho$, $Y_\alpha$, and $e$ are known variables provided by the fluid transport equations, and hence they are the inputs to the equilibration algorithm. \Cref{eq:ms_eos_pt_algorithm_1,eq:ms_eos_pt_algorithm_2,eq:ms_eos_pt_algorithm_3,eq:ms_eos_pt_algorithm_4} constitute $2N+2$ equations for the $2N+2$ unknowns $\rho_\alpha$, $e_\alpha$, $p$, and $T$. Note that, after solving the system of equations above, one can compute $\eta_\alpha$ using \cref{eq:relation_volume_density_mass}.

If we plug in \cref{eq:relation_volume_density_mass} into \cref{eq:ms_eos_pt_algorithm_1,eq:ms_eos_pt_algorithm_2,eq:ms_eos_pt_algorithm_3} then the equations for the equilibration algorithm can be written as 
\begin{equation}
    p = \phi_\alpha (\rho Y_\alpha / \eta_\alpha, T),
\end{equation}
\begin{equation}
    e_\alpha = \psi_\alpha (\rho Y_\alpha / \eta_\alpha, T).
\end{equation}
\begin{equation}
    1 = \sum_\alpha \eta_\alpha
\end{equation}
\begin{equation}
    e = \sum_\alpha Y_\alpha e_\alpha.
\end{equation}
For the above $\rho$, $Y_\alpha$, and $e$ are still the inputs, and $\eta_\alpha$, $e_\alpha$, $p$ and $T$ are the $2N+2$ unknowns. Note that, after solving the system of equations above, one can compute $\rho_\alpha$ using \cref{eq:relation_volume_density_mass}.

%---------------------------------
\subsection{The PT equilibration model for perfect gasses}
%---------------------------------
For an ideal gas, \cref{eq:multi_species_thermal_eos} and \cref{eq:multi_species_caloric_eos} are given by
\begin{equation}
\label{eq:eos_thermal_multicomponent_ideal}
\hat{p}_\alpha = \rho_\alpha R_\alpha T_\alpha,
\end{equation}
and
\begin{equation} 
\label{eq:eos_caloric_multicomponent_ideal}
e_\alpha = C_{v,\alpha} T_\alpha .
\end{equation}
In the above, $R_\alpha = R_u / M_\alpha$ is the ideal gas constant of each species and it satisfies $R_\alpha = C_{p,\alpha} - C_{v,\alpha}$. For the species enthalpy, we have
\begin{equation}
    h_\alpha = e_\alpha + \frac{\hat{p}_\alpha}{\rho_\alpha} = e_\alpha + R_\alpha T_\alpha = C_{v,\alpha} T_\alpha + \left ( C_{p,\alpha} - C_{v,\alpha} \right ) T_\alpha = C_{p,\alpha} T_\alpha.
\end{equation}

The thermal equation of state for the entire mixture can be derived from the thermal equation of state for each species. We first multiply \cref{eq:eos_thermal_multicomponent_ideal} by $Y_\alpha$, divide by $\rho_\alpha$, and sum over all $\alpha$. That is
\begin{equation}
\sum_\alpha p \frac{Y_\alpha}{\rho_\alpha}= \sum_\alpha Y_\alpha R_\alpha T.
\end{equation}
The above can be expressed as
\begin{equation}
p = \rho \sum_\alpha \frac{Y_\alpha}{M_\alpha} R_u T.
\end{equation}
We now define the the average molar mass of the mixture as $M = \sum_\alpha X_\alpha M_\alpha$. Using \cref{eq:y_intermsof_x}, this can be rewritten as 
\begin{equation}
    \frac{1}{M} = \sum_\alpha \frac{Y_\alpha}{M_\alpha}
\end{equation}
Thus, the thermal equation of state for the entire mixture can be expressed as
\begin{equation}
    p = \rho R T
\end{equation}
where
\begin{equation}
R = \frac{R_u}{M}
\end{equation}
Using the above, we can easily show that
\begin{equation}
    R = \sum_\alpha Y_\alpha \frac{R_u}{M_\alpha} = \sum_\alpha Y_\alpha R_\alpha = \sum_\alpha Y_\alpha \left ( C_{p,\alpha} - C_{v,\alpha} \right ) = C_p - C_v.
\end{equation}

%---------------------------------
\subsection{Summary of governing equations}
%---------------------------------
\label{sec:multi_species_eqs}

\paragraph{Conservation equations}

\begin{equation*}
    \frac{\partial\rho Y_\alpha}{\partial t}+\frac{\partial \rho Y_\alpha u_i}{\partial x_i} = -\frac{\partial j_{\alpha,i}}{\partial x_i} + w_\alpha
\end{equation*}
\begin{equation*}
    \frac{\partial \rho u_i}{\partial t} + \frac{\partial \rho u_i u_j}{\partial x_j} = \frac{\partial \sigma_{ij}}{\partial x_j} + \rho f_i
\end{equation*}
\begin{equation*}
    \frac{ \partial \rho e}{\partial t} + \frac{\partial \rho e u_j}{\partial x_j} = \sigma_{ij} \frac{\partial u_i}{\partial x_j} - \frac{\partial q_j}{\partial x_j}.
\end{equation*}

\paragraph{Transport models}

\begin{equation*}
t_{ij} = 2\mu S_{ij}^*
\end{equation*}
\begin{equation*}
q_i = -\kappa \frac{\partial T}{\partial x_i}  + \sum_\alpha h_\alpha j_{\alpha,i}
\end{equation*}
\begin{equation*}
j_{\alpha,i} = -\rho \left ( D_\alpha \frac{\partial Y_\alpha}{\partial x_i} - Y_\alpha \sum_\beta D_\beta \frac{\partial Y_\beta}{\partial x_i} \right )
\end{equation*}

\paragraph{Transport coefficients}

\begin{equation*}
\mu = \mu_0 \left ( \frac{T}{T_0} \right )^n
\end{equation*}
\begin{equation*}
\kappa = \frac{\mu C_p}{Pr}
\end{equation*}
\begin{equation*}
D_\alpha = \frac{\mu}{\rho Sc_\alpha}
\end{equation*}

\paragraph{Equations of state}

\begin{equation*}
    p = \phi_\alpha (\rho_\alpha, T),
\end{equation*}
\begin{equation*}
    e_\alpha = \psi_\alpha (\rho_\alpha, T).
\end{equation*}
\begin{equation*}
    \frac{1}{\rho} = \sum_\alpha \frac{Y_\alpha}{\rho_\alpha},
\end{equation*}
\begin{equation*}
    e = \sum_\alpha Y_\alpha e_\alpha.
\end{equation*}

\paragraph{Additional relations}

\begin{equation*}
    \sigma_{ij} = -p \delta_{ij} + t_{ij}
\end{equation*}
\begin{equation*}
    S^*_{ij} = \frac{1}{2} \left ( \frac{\partial u_i}{\partial x_j} + \frac{\partial u_j}{\partial x_i} \right ) - \frac{1}{3} \frac{\partial u_k}{\partial x_k} \delta_{ij}
\end{equation*}
\begin{equation*}
    h_\alpha = e_\alpha + \frac{p}{\rho_\alpha}
\end{equation*}
\begin{equation*}
    1 = \sum_\alpha Y_\alpha 
\end{equation*}

%---------------------------------
\subsection{Mass-fraction equations}
%---------------------------------
\label{sec:mass_frac_methods}
There are two methods that are often used to compute the total density $\rho$ and the mass fractions $Y_\alpha$. These two methods are equivalent in paper, as will be shown below, but when implemented numerically can lead to different behavior. Since $\rho$ and $Y_\alpha$ constitute $1+N$ unknowns, $1+N$ equations are required.

\subsubsection{Method 1}
This is the method shown in \cref{sec:multi_species_eqs}, that is, one first solves for $\rho Y_\alpha$ using
\begin{equation*}
    \frac{\partial\rho Y_\alpha}{\partial t}+\frac{\partial \rho Y_\alpha u_i}{\partial x_i} = -\frac{\partial j_{\alpha,i}}{\partial x_i} + w_\alpha.\,
\end{equation*}
and assumes the following constraint
\begin{equation*}
    \sum_\alpha Y_\alpha = 1.
\end{equation*}
The densities then follow from $\rho = \sum_\alpha \rho Y_\alpha$ and the mass fractions from $Y_\alpha = \rho Y_\alpha / \rho$. We can then write
\begin{align*}
    \frac{\partial \rho}{\partial t} + \frac{\partial \rho u_i}{\partial x_i} &= \frac{\partial}{\partial t} \sum_\alpha \rho Y_\alpha + \frac{\partial}{\partial x_i} \sum_\alpha \rho Y_\alpha u_i \\
    &= \sum_\alpha \left( \frac{\partial \rho Y_\alpha}{\partial t} + \frac{\partial \rho Y_\alpha u_i}{\partial x_i} \right) \\
    &= \sum_\alpha \left( -\frac{\partial j_{\alpha,i}}{\partial x_i} + w_\alpha \right).
\end{align*}
Assuming $\sum_\alpha j_{\alpha,i} = 0$ and $\sum_\alpha w_\alpha = 0$, we have
\begin{equation*}
    \frac{\partial \rho}{\partial t} + \frac{\partial \rho u_i}{\partial x_i} = 0.
\end{equation*}

\subsubsection{Method 2}
In this method one first solves for $\rho$ and $\rho Y_\alpha$ using 
\begin{equation*}
    \frac{\partial \rho}{\partial t} + \frac{\partial \rho u_i}{\partial x_i} = 0,
\end{equation*}
and    
\begin{equation*}
    \frac{\partial\rho Y_\alpha}{\partial t}+\frac{\partial \rho Y_\alpha u_i}{\partial x_i} = -\frac{\partial j_{\alpha,i}}{\partial x_i} + w_\alpha.
\end{equation*}
The mass fractions then follow from $Y_\alpha = \rho Y_\alpha / \rho$. We can write
\begin{align*}
    \frac{\partial Y_\alpha}{\partial t} + u_i \frac{\partial Y_\alpha}{\partial x_i} &= \frac{1}{\rho^2} \left( \frac{\partial \rho Y_\alpha}{\partial t} \rho - \frac{\partial \rho}{\partial t} \rho Y_\alpha \right) + u_i \frac{1}{\rho^2} \left( \frac{\partial \rho Y_\alpha}{\partial x_i} \rho - \frac{\partial \rho}{\partial x_i} \rho Y_\alpha \right)\\
    &= \frac{1}{\rho} \left[ \left( \frac{\partial \rho Y_\alpha}{\partial t} + u_i \frac{\partial \rho Y_\alpha}{\partial x_i} \right) - Y_\alpha \left( \frac{\partial \rho}{\partial t} + u_i \frac{\partial \rho}{\partial x_i} \right) \right] \\
    &= \frac{1}{\rho} \left( \frac{\partial \rho Y_\alpha}{\partial t} + u_i \frac{\partial \rho Y_\alpha}{\partial x_i} + \rho Y_\alpha \frac{\partial u_i}{\partial x_i} \right) \\
    &= \frac{1}{\rho} \left( -\frac{\partial j_{\alpha,i}}{\partial x_i} + w_\alpha \right).
\end{align*}
If we assume $\sum_\alpha j_{\alpha,i} = 0$ and $\sum_\alpha w_\alpha = 0$, then we have
\begin{equation}
    \frac{\partial}{\partial t} \sum_\alpha Y_\alpha + u_i \frac{\partial}{\partial x_i} \sum_\alpha Y_\alpha = 0.
\end{equation} 
That is, $\sum_\alpha Y_\alpha$ is conserved along a streamline. If the initial and boundary conditions satisfy $\sum_\alpha Y_\alpha=1$, then this equality holds for all time and space.

\begin{table}
    \centering
    \caption{Variables for multi-species and multi-material hydrodynamics.}
    \label{tab:multi_species_vs_multi_material}
    \def\arraystretch{1.5}
    \begin{tabular}{ c | c | c }
        Variable  & M-S solution & M-M solution \\
        \hhline{=|=|=}
        $\rho$ & method 1 or 2 from \cref{sec:mass_frac_methods} &  method 1 or 2 from \cref{sec:mass_frac_methods} \\
        \hline
        $Y_\alpha$ & method 1 or 2 from \cref{sec:mass_frac_methods} & method 1 or 2 from \cref{sec:mass_frac_methods} \\
        \hline
        $e$ & PDE given by \cref{eq:internal_energy_sigma} & from $Y_\alpha$ and $e_\alpha$ using \cref{eq:from_e_alpha_to_e}\\
        \hline
        $p_\alpha$ & equal to $p$ & thermal eos, \cref{eq:multi_species_thermal_eos} \\
        \hline
        $T_\alpha$ & equal to $T$ & caloric eos, \cref{eq:multi_species_caloric_eos} \\
        \hline
        $p$ & PT-equilibration algorithm & n/a \\
        \hline
        $T$ & PT-equilibration algorithm & n/a \\
        \hline
        $\rho_\alpha$ & PT-equilibration algorithm & algebraic relation given by \cref{eq:relation_volume_density_mass} \\
        \hline
        $e_\alpha$ & PT-equilibration algorithm & PDE given by \cref{eq:multi_mat_hydro_species_energy} \\
        \hline
        $\eta_\alpha$ & algebraic relation given by \cref{eq:relation_volume_density_mass} & PDE given by \cref{eq:multi_mat_hydro_vol_fractions} \\
    \end{tabular}
\end{table}

\end{document}