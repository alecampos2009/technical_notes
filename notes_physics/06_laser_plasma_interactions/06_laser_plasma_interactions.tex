\documentclass[a4paper,11pt]{report}
\usepackage{fullpage}

\usepackage{"../../info/packages"}
\usepackage{"../../info/nomenclature"}

\title{Laser Plasma Interactions}
\author{Alejandro Campos}

\begin{document}
\maketitle
\tableofcontents

%----------------------------------------------------------------------------------------------------------------------
\chapter{Governing equations}
%----------------------------------------------------------------------------------------------------------------------
The starting point are the multi-fluid conservation laws and the Maxwell equations. We assume there are two species: electrons and ions. Additionally, we assume homentropic flow, and no sources, stresses, and collisions. Thus, the governing equations are
\begin{equation}
    \label{eq:pwaves_ion_density}
    \frac{\partial n_i}{\partial t} + \nabla \cdot \left (n_i \uvec_i \right ) = 0,
\end{equation}
\begin{equation}
    \label{eq:pwaves_electron_density}
    \frac{\partial n_e}{\partial t} + \nabla \cdot \left (n_e \uvec_e \right ) = 0,
\end{equation}
\begin{equation}
    \label{eq:pwaves_ion_momentum}
    \frac{\partial m_i n_i \uvec_i}{\partial t} + \nabla \cdot \left ( m_i n_i \uvec_i \uvec_i \right ) - Ze n_i \left ( \Evec + \uvec_i \times \Bvec \right ) = -\nabla p_i,
\end{equation}
\begin{equation}
    \label{eq:pwaves_electron_momentum}
    \frac{\partial m_e n_e \uvec_e}{\partial t} + \nabla \cdot \left ( m_e n_e \uvec_e \uvec_e \right ) + e n_e \left ( \Evec + \uvec_e \times \Bvec \right ) = -\nabla p_e.
\end{equation}
\begin{equation}
    p_i = C_i n_i^{\gamma_i},
\end{equation}
\begin{equation}
    p_e = C_e n_e^{\gamma_e},
\end{equation}
\begin{equation}
    \label{eq:pwaves_maxwell_3}
    \nabla \cdot \Evec = \frac{\rho_e}{\epsilon_0} 
\end{equation}
\begin{equation}
    \label{eq:p_waves_maxwell_4}
    \nabla \cdot \Bvec = 0.
\end{equation}
\begin{equation}
    \label{eq:p_waves_maxwell_1}
    \nabla \times \Evec = -\frac{ \partial \Bvec}{\partial t}
\end{equation}
\begin{equation}
    \label{eq:p_waves_maxwell_2}
    \nabla \times \Bvec = \mu_0 \Jvec + \mu_0 \epsilon_0 \frac{\partial \Evec}{\partial t}
\end{equation}
\begin{equation}
    \label{eq:p_waves_curr_density}
    \Jvec = e (Z n_i \uvec_i - n_e \uvec_e)
\end{equation}
\begin{equation}
    \label{eq:p_waves_mass_density}
    \rho_e = e (Z n_i - n_e) 
\end{equation}

%----------------------------------------------------------------------------------------------------------------------
\chapter{Electron-plasma and ion-acoustic waves}
%----------------------------------------------------------------------------------------------------------------------
%-------------------------------------------------------------------------------
\section{Linearization}
%-------------------------------------------------------------------------------
\label{sec:p_waves_linearization}
The following decompositions will be used in the derivation of electron-plasma and ion-acoustic waves:
\begin{align}
    n_i &= n_{i0} + n_{i1}, \nonumber \\
    n_e &= n_{e0} + n_{e1}, \nonumber \\
    p_i &= p_{i0} + p_{i1}, \nonumber \\
    p_e &= p_{e0} + p_{e1}, \nonumber \\
    \uvec_i &= \uvec_{i0} + \uvec_{i1}, \nonumber \\
    \uvec_e &= \uvec_{e0} + \uvec_{e1}, \nonumber \\
    \Evec &= \Evec_0 + \Evec_1, \nonumber \\
    \Bvec &= \Bvec_0 + \Bvec_1.
\end{align}
For these decompositions, we'll assume
\begin{enumerate}
    \item Terms with a subscript 1 are small and thus products of two small quantities can be neglected. \label{it:p_waves_assumption_1}
    \item $\uvec_{i0}$, $\uvec_{e0}$, $\Evec_0$, and $\Bvec_0$ are zero. \label{it:p_waves_assumption_2}
    \item $n_{i0}$, $n_{e0}$, $p_{i0}$, and $p_{e0}$ are uniform in space and time. \label{it:p_waves_assumption_3}
\end{enumerate}

Using the variable decompositions in the electron density equation, we have
\begin{equation*}
    \frac{\partial n_{e0} + n_{e1}}{\partial t} + \nabla \cdot \left [ \left ( n_{e0} + n_{e1} \right )  \left ( \uvec_{e0} + \uvec_{e1} \right ) \right ] = 0.
\end{equation*}
Using assumptions in \cref{it:p_waves_assumption_1,it:p_waves_assumption_2,it:p_waves_assumption_3}, the above simplifies to
\begin{equation}
    \label{eq:p_waves_e_den_linearized}
    \frac{\partial n_{e1}}{\partial t} + \nabla \cdot \left (n_{e0} \uvec_{e1} \right ) = 0.
\end{equation}
Using the variable decompositions in the ion density equation, we have
\begin{equation*}
    \frac{\partial n_{i0} + n_{i1}}{\partial t} + \nabla \cdot \left [ \left ( n_{i0} + n_{i1} \right )  \left ( \uvec_{i0} + \uvec_{i1} \right ) \right ] = 0.
\end{equation*}
Given the assumptions in \cref{it:p_waves_assumption_1,it:p_waves_assumption_2,it:p_waves_assumption_3}, the above simplifies to
\begin{equation}
    \label{eq:p_waves_i_den_linearized}
    \frac{\partial n_{i1}}{\partial t} + \nabla \cdot \left (n_{i0} \uvec_{i1} \right ) = 0.
\end{equation}

Using the variable decompositions in the electron momentum equation, we have 
\begin{multline*}
    \frac{\partial}{\partial t} \left [ m_e \left ( n_{e0} + n_{e1} \right ) \left ( \uvec_{e0} + \uvec_{e1} \right ) \right ] + \nabla \cdot \left [ m_e \left ( n_{e0} + n_{e1} \right ) \left ( \uvec_{e0} + \uvec_{e1} \right ) \left ( \uvec_{e0} + \uvec_{e1} \right ) \right ] \\
    + e \left ( n_{e0} + n_{e1} \right ) \left [ \left ( \Evec_0 + \Evec_1 \right ) + \left ( \uvec_{e0} + \uvec_{e1} \right ) \times \left ( \Bvec_0 + \Bvec_1 \right ) \right ] = -\nabla \left ( p_{e0} + p_{e1} \right ).
\end{multline*}
Given the assumptions in \cref{it:p_waves_assumption_1,it:p_waves_assumption_2,it:p_waves_assumption_3}, the above simplifies to
\begin{equation}
    \label{eq:p_waves_e_mom_linearized}
    \frac{\partial n_{e0} \uvec_{e1}}{\partial t} + \frac{e n_{e0}}{m_e} \Evec_1 = - \frac{1}{m_e} \nabla p_{e1} .
\end{equation}
Using the variable decompositions in the ion momentum equation, we have 
\begin{multline*}
    \frac{\partial}{\partial t} \left [ m_i \left ( n_{i0} + n_{i1} \right ) \left ( \uvec_{i0} + \uvec_{i1} \right ) \right ] + \nabla \cdot \left [ m_i \left ( n_{i0} + n_{i1} \right ) \left ( \uvec_{i0} + \uvec_{i1} \right ) \left ( \uvec_{i0} + \uvec_{i1} \right ) \right ] \\
    - Z e \left ( n_{i0} + n_{i1} \right ) \left [ \left ( \Evec_0 + \Evec_1 \right ) + \left ( \uvec_{i0} + \uvec_{i1} \right ) \times \left ( \Bvec_0 + \Bvec_1 \right ) \right ] = -\nabla \left ( p_{i0} + p_{i1} \right ).
\end{multline*}
Given the assumptions in \cref{it:p_waves_assumption_1,it:p_waves_assumption_2,it:p_waves_assumption_3}, the above simplifies to
\begin{equation}
    \label{eq:p_waves_i_mom_linearized}
    \frac{\partial n_{i0} \uvec_{i1}}{\partial t} - \frac{Z e n_{i0}}{m_i} \Evec_1 = - \frac{1}{m_i}\nabla p_{i1}.
\end{equation}

We'll often need to take the gradient of the ion and electron pressure. We begin by showing
\begin{equation*}
    \nabla p_\alpha = C_\alpha \gamma_\alpha n_\alpha^{\gamma_\alpha-1} \nabla n_\alpha = C_\alpha \gamma_\alpha \frac{n_\alpha^{\gamma_\alpha}}{n_\alpha} \nabla n_\alpha = \gamma_\alpha \frac{p_\alpha}{n_\alpha} \nabla n_\alpha,
\end{equation*}
or
\begin{equation*}
    n_\alpha \nabla p_\alpha = \gamma_\alpha p_\alpha \nabla n_\alpha,
\end{equation*}
where $\alpha = i,e$. Using the variable decompositions, we have
\begin{equation*}
    \left ( n_{\alpha 0} + n_{\alpha 1} \right ) \nabla \left ( p_{\alpha 0} + p_{\alpha 1} \right ) = \gamma_\alpha \left ( p_{\alpha 0} + p_{\alpha 1} \right ) \nabla \left ( n_{\alpha 0} + n_{\alpha 1} \right ).
\end{equation*}
Given the assumptions in \cref{it:p_waves_assumption_1,it:p_waves_assumption_3}, the above simplifies to
\begin{equation}
    \label{eq:p_waves_alpha_pressure_linearized}
    n_{\alpha 0} \nabla p_{\alpha 1} = \gamma_\alpha p_{\alpha 0} \nabla n_{\alpha 1}.
\end{equation}
Thus, for electrons we have
\begin{equation}
    \label{eq:p_waves_e_pressure_linearized}
    n_{e0} \nabla p_{e1} = \gamma_e p_{e0} \nabla n_{e1},
\end{equation}
and for ions
\begin{equation}
    \label{eq:p_waves_i_pressure_linearized}
    n_{i0} \nabla p_{i1} = \gamma_i p_{i0} \nabla n_{i1}.
\end{equation}

An alternate derivation to obtain the previous results is to use an equation of state for the large pressure and density of the following form
\begin{equation}
    \label{eq:p_waves_eos}
    p_{\alpha 0} = n_{\alpha 0} k_B T_\alpha,
\end{equation}
where $\alpha = i,e$. Note that the temperature is one whole quantity, that is, it is not split into large and small terms. Let's assume that the large quantities also satisfy isentropic flow, that is $p_{\alpha 0} = C n^{\gamma_\alpha}_{\alpha 0}$. Thus, we can show that
\begin{align*}
    p_{\alpha 1} &= p_{\alpha} - p_{\alpha 0} \\
    &= C n_\alpha^{\gamma_\alpha} - p_{\alpha 0} \\
    &= C (n_{\alpha 0} + n_{\alpha 1})^{\gamma_\alpha} - p_{\alpha 0} \\
    &= C n_{\alpha 0}^{\gamma_\alpha} \left ( 1 + \frac{n_{\alpha 1}}{n_{\alpha 0}} \right )^{\gamma_\alpha} - p_{\alpha 0} \\
    &= p_{\alpha 0} \left ( 1 + \frac{n_{\alpha 1}}{n_{\alpha 0}} \right )^{\gamma_\alpha} - p_{\alpha 0}.
\end{align*} 
Since $n_{\alpha 1} / n_{\alpha 0}$ is small, we can use the binomial series to proceed as follows
\begin{align*}
    p_{\alpha 1} &= p_{\alpha 0} \left ( 1 + \gamma_\alpha \frac{n_{\alpha 1}}{n_{\alpha 0}} \right ) - p_{\alpha 0} \\
    &= \gamma_\alpha \frac{p_{\alpha 0}}{n_{\alpha 0}} n_{\alpha 1}.
\end{align*}
With $p_{\alpha 0}$ and $n_{\alpha 0}$ constant, this then gives \cref{eq:p_waves_alpha_pressure_linearized}. Finally, using \cref{eq:p_waves_eos} in the above we get
\begin{equation}
    p_{\alpha 1} = \gamma_\alpha k_B T_\alpha n_{\alpha 1},
\end{equation}
which can be interpreted as the equation of state for the small quantities.

%-------------------------------------------------------------------------------
\section{Electron Plasma Waves}
%-------------------------------------------------------------------------------
On top of the assumptions in \cref{sec:p_waves_linearization}, we'll assume 
\begin{enumerate}
    \item Quasi-neutrality for the base flow, $Zn_{i0} = n_{e0}$.
    \item Uniform ion density, $n_{i1} = 0$.
\end{enumerate}

Combining \Cref{eq:p_waves_e_mom_linearized} with \cref{eq:p_waves_e_pressure_linearized} gives
\begin{equation}
    \label{eq:ep_waves_mom_linearized}
    \frac{\partial n_{e0} \uvec_{e1}}{\partial t} + \frac{e n_{e0}}{m_e} \Evec_1 = - \frac{\gamma_e p_{e0}}{n_{e0} m_e} \nabla n_{e1}.
\end{equation}
Taking the time derivative of \cref{eq:p_waves_e_den_linearized} and using \cref{eq:ep_waves_mom_linearized} leads to the the wave equation for electron density
\begin{equation}
    \label{eq:ep_waves_den_combined}
    \frac{\partial^2 n_{e1}}{\partial t^2} - \frac{e n_{e0}}{m_e} \nabla \cdot \Evec_1 = \frac{\gamma_e p_{e0}}{n_{e0} m_e} \nabla^2 n_{e1}.
\end{equation}
For electron plasma waves, we'll assume that $n_{i}$ varies in space and time so slowly that it can be assumed to be constant. That is, we assume $n_{i1} = 0$. Thus, Gauss's law now takes the form
\begin{equation*}
    \nabla \cdot \Evec_1 = \frac{e}{\epsilon_0} \left ( Z n_{i0} - n_{e0} - n_{e1} \right ).
\end{equation*}
Using the quasi-neutrality assumption ($Zn_{i0} = n_{e0}$)
\begin{equation}
    \label{eq:ep_waves_efield_divergence}
    \nabla \cdot \Evec_1 = -\frac{e}{\epsilon_0} n_{e1}.
\end{equation}
Plugging the above in the electron wave equation we obtain
\begin{equation*}
    \frac{\partial^2 n_{e1}}{\partial t^2} + \frac{e^2 n_{e0}}{m_e \epsilon_0} n_{e1} = \frac{\gamma_e p_{e0}}{n_{e0} m_e} \nabla^2 n_{e1},
\end{equation*}
or
\begin{equation}
    \frac{\partial^2 n_{e1}}{\partial t^2} + w^2_{pe} n_{e1} - \frac{\gamma_e p_{e0}}{n_{e0} m_e} \nabla^2 n_{e1} = 0.
\end{equation}

Assuming a mode of the form $n_{e1} = \hat{n}_{e1} \exp [ i \left ( \kvec_e \cdot \xvec - w t \right ) ]$, where $\kvec_e$ is the wave vector of the electron-plasma wave, gives the following dispersion relation
\begin{equation}
    w^2 - w^2_{pe} - \frac{\gamma_e p_{e0}}{n_{e0} m_e}  k_e^2 = 0.
\end{equation}
If we define the thermal velocity without the factor of two, that is, $v_{T\alpha} = \sqrt{k_BT_\alpha / m_\alpha}$, then we have
\begin{equation}
    w^2 - w^2_{pe} - \gamma_e k_e^2 v^2_{T_{e0}} = 0.
\end{equation}

%-------------------------------------------------------------------------------
\section{Ion Acoustic Waves}
%-------------------------------------------------------------------------------
On top of the assumptions in \cref{sec:p_waves_linearization}, we'll assume 
\begin{enumerate}
    \item Quasi-neutrality for the base flow, $Zn_{i0} = n_{e0}$.
    \item Approximate quasi-neutrality for the fluctuations, $Z n_{i1} \approx n_{e1}$.
    \item Negligible electron mass, $m_e \to 0$.
\end{enumerate}

Combining \cref{eq:p_waves_i_mom_linearized} with \cref{eq:p_waves_i_pressure_linearized} gives
\begin{equation}
    \label{eq:ia_waves_mom_linearized}
    \frac{\partial n_{i0} \uvec_{i1}}{\partial t} - \frac{Z e n_{i0}}{m_i} \Evec_1 = - \frac{\gamma_i p_{i0}}{n_{i0} m_i} \nabla n_{i1}.
\end{equation}

Taking the time derivative of \cref{eq:p_waves_i_den_linearized} and using \cref{eq:ia_waves_mom_linearized} leads to the the wave equation for ion density
\begin{equation}
    \label{eq:ia_waves_den_combined}
    \frac{\partial^2 n_{i1}}{\partial t^2} + \frac{Z e n_{i0}}{m_i} \nabla \cdot \Evec_1 = \frac{\gamma_i p_{i0}}{n_{i0} m_i} \nabla^2 n_{i1}.
\end{equation}
For this case, we assume that the mass of the electron, which is significantly smaller than that of the ions, is negligible. Thus, \cref{eq:p_waves_e_mom_linearized} simplifies to 
\begin{equation}
    \label{eq:ia_waves_E}
    e n_{e0} \Evec_1 = - \frac{\gamma_e p_{e0}}{n_{e0}} \nabla n_{e1}.
\end{equation}
Plugging in the above in the ion wave equation we obtain
\begin{equation*}
    \frac{\partial^2 n_{i1}}{\partial t^2} = \frac{Z n_{i0}}{ n_{e0}} \frac{\gamma_e p_{e0}}{n_{e0} m_i} \nabla^2 n_{e1} + \frac{\gamma_i p_{i0}}{n_{i0} m_i} \nabla^2 n_{i1}.
\end{equation*}
Due to quasi-neutrality, we have $Z n_{i0} = n_{e0}$ and $Z n_{i1} \approx n_{e1}$, which gives
\begin{equation*}
    \frac{\partial^2 n_{i1}}{\partial t^2} = \frac{1}{m_i} \left ( \frac{Z \gamma_e p_{e0}}{n_{e0}} + \frac{\gamma_i p_{i0}}{n_{i0}} \right ) \nabla^2 n_{i1}.
\end{equation*}
Since $p_{i0}/n_{i0} = k_B T_i$ and $p_{e0}/n_{e0} = k_B T_e$, we finally have
\begin{equation}
    \frac{\partial^2 n_{i1}}{\partial t^2} - \left ( \frac{Z \gamma_e k_B T_e + \gamma_i k_B T_i}{m_i} \right ) \nabla^2 n_{i1} = 0.
\end{equation}

Assuming a mode of the form $n_{i1} = \hat{n}_{i1} \exp [ i \left ( \kvec_i \cdot \xvec - w t \right ) ]$, where $\kvec_i$ is the wave vector of the ion-acoustic wave, we obtain the following dispersion relation
\begin{equation}
    w^2 - k_i^2 v_s^2 = 0,
\end{equation}
where
\begin{equation}
    v_s = \sqrt{ \frac{Z \gamma_e k_B T_e + \gamma_i k_B T_i }{m_i} }.
\end{equation}

%----------------------------------------------------------------------------------------------------------------------
\chapter{Longitudinal and transverse waves}
%----------------------------------------------------------------------------------------------------------------------
%-------------------------------------------------------------------------------
\section{Definitions}
%-------------------------------------------------------------------------------
The Helmholtz decomposition for a function $\Fvec = \Fvec(\xvec,t)$ is of the following form
\begin{equation}
    \Fvec = \Fvec_l + \Fvec_t,
\end{equation}
where $\Fvec_l = \Fvec_l(\xvec,t)$ is the longitudinal component and $\Fvec_t = \Fvec_t(\xvec,t)$ the transverse component. These are defined by
\begin{align}
    \nabla \times \Fvec_l &= 0, \label{eq:ltw_plasma_long_def}\\
    \nabla \cdot \Fvec_t &= 0. \label{eq:ltw_plasma_tran_def}
\end{align}
We'll assume that any vector function $\Gvec = \Gvec(\xvec,t)$ can be expressed as the real component of
\begin{equation}
    \label{eq:em_more_general_wave_form}
    \Gvec = \hat{\Gvec} \exp \left [ i \left ( \int_0^x k_x(x') \, dx' + \int_0^y k_y(y') \, dy' + \int_0^z k_z(z') \, dz'  - wt \right ) \right ].
\end{equation}
For the above, $w$ a frequency constant in time and space, and $k_x = k_x(x)$, $k_y = k_y(y)$, and $k_z = k_z(z)$ form the wave vector $\kvec = [k_x, k_y, k_z]$. $\hat{\Gvec} = \hat{\Gvec}(\xvec,t)$ is a complex vector where the real and complex components point in the same direction. Additionally, we enforce the constraint that if $\nabla \times \Gvec = 0$, then $\nabla \times \hat{\Gvec} = 0$, and similarly, if $\nabla \cdot \Gvec = 0$, then $\nabla \cdot \hat{\Gvec} = 0$. We'll often use a different subscript in the wave vectors and frequencies of different waves. For example, we'll use $\kvec_e$ for electron-plasma waves, $\kvec_i$ for ion-acoustic waves, $\kvec_L$ for laser waves, and $\kvec_s$ for scattered waves. Similarly for the frequencies $w_e$, $w_i$, $w_L$, $w_s$.

We note that
\begin{multline}
    \nabla \exp \left [ i \left ( \int_0^x k_x(x') \, dx' + \int_0^y k_y(y') \, dy' + \int_0^z k_z(z') \, dz'  - wt \right ) \right ] \\
    = i \kvec \exp \left [ i \left ( \int_0^x k_x(x') \, dx' + \int_0^y k_y(y') \, dy' + \int_0^z k_z(z') \, dz'  - wt \right ) \right ]
\end{multline}
Given the identity $\nabla \times (\Avec f) = (\nabla \times \Avec ) f - \Avec \times (\nabla f)$, we can show that
\begin{align}
    \label{eq:ltw_no_curl}
    \nabla \times \Gvec &= \nabla \times \left [ \hat{\Gvec} \exp (...) \right ] \nonumber \\
    &= \left ( \nabla \times \hat{\Gvec} \right ) \exp (...) - \hat{\Gvec} \times \left [ \nabla \exp (...) \right ] \nonumber \\
    &= \left ( \nabla \times \hat{\Gvec} \right ) \exp (...) - \hat{\Gvec} \times \left [ i\kvec \exp (...) \right ] \nonumber \\
    &= \left ( \nabla \times \hat{\Gvec} \right ) \exp (...) + i \kvec \times \Gvec.
\end{align}
Given the identity $\nabla \cdot (\Avec f) = ( \nabla \cdot \Avec) f + \Avec \cdot (\nabla f)$, we can show that
\begin{align}
    \label{eq:ltw_no_div}
    \nabla \cdot \Gvec &= \nabla \cdot \left [ \hat{\Gvec} \exp (...) \right ] \nonumber \\
    &= \left ( \nabla \cdot \hat{\Gvec} \right ) \exp (...) + \hat{\Gvec} \cdot \left [ \nabla \exp (...) \right ] \nonumber \\
    &= \left ( \nabla \cdot \hat{\Gvec} \right ) \exp (...) + \hat{\Gvec} \cdot \left [ i \kvec \exp (...) \right ] \nonumber \\
    &= \left ( \nabla \cdot \hat{\Gvec} \right ) \exp (...) + i \kvec \cdot \Gvec.
\end{align}
By definition, $\Fvec_l$ has no curl and $\Fvec_t$ has no divergence. As mentioned earlier, we then require $\hat{\Fvec}_l$ to have no curl and $\hat{\Fvec}_t$ to have no divergence. Using this in \cref{eq:ltw_no_curl,eq:ltw_no_div} allow us to write
\begin{align}
    \kvec \times \Fvec_l = 0 \label{eq:ltw_plasma_long_wavevector} \\
    \kvec \cdot \Fvec_t = 0 \label{eq:ltw_plasma_trans_wavevector}.
\end{align}

The first expression above says $\Fvec_l$ is parallel to $\kvec$ and the second says $\Fvec_t$ is orthogonal to $\kvec$. Thus, $\Fvec_l \cdot \Fvec_t = 0$. We will often have situations where $\nabla \times \Fvec = \nabla \cdot \Fvec = 0$, which by its own does not imply $\Fvec = 0$. However, using \cref{eq:ltw_no_curl,eq:ltw_no_div}, this translates to to $\kvec \times \Fvec = \kvec \cdot \Fvec =  0$. The latter equality states that $\kvec$ and $\Fvec$ are orthogonal, that is, the angle between them is $90^\circ$. The former equality leads to $|\Fvec| \sin(90^\circ) = 0$, which in turn means $\Fvec = 0$. To summarize,
\begin{equation}
    \label{eq:ltw_general_null_vector}
    \nabla \times \Fvec = \nabla \cdot \Fvec = 0 \to \Fvec = 0.
\end{equation}

For some cases we'll further restrict $\hat{\Gvec}$ in \cref{eq:em_more_general_wave_form} such that $\hat{\Gvec} = \hat{\Gvec} (\xvec)$, that is, the time dependence of the wave is fully captured by the $\exp(-iwt)$ term. For this case, we'll often re-write the expression for $\Gvec$ as
\begin{equation}
    \label{eq:em_semi_general_wave_form}
    \Gvec = \tilde{\Gvec} \exp (-iwt),
\end{equation}
where $\tilde{\Gvec} = \tilde{\Gvec}(\xvec)$ is given by
\begin{equation}
    \tilde{\Gvec} = \hat{\Gvec} \exp \left [ i \left ( \int_0^x k_x(x') \, dx' + \int_0^y k_y(y') \, dy' + \int_0^z k_z(z') \, dz' \right ) \right ].
\end{equation}
Finally, a further simplification occurs when $\kvec$ and $\hat{\Gvec}$ are assumed to be constant in space. For this case, the expression for $\Gvec$ becomes
\begin{equation}
    \label{eq:em_general_wave_form}
    \Gvec = \hat{\Gvec} \exp \left [i\left ( \kvec \cdot \xvec - wt \right ) \right ].
\end{equation}
These are the so-called plane waves. We end with the cautionary note that the second gradients of $\Gvec$ in \cref{eq:em_more_general_wave_form} are not necessarily the same as those of $\Gvec$ in \cref{eq:em_general_wave_form}.
%-------------------------------------------------------------------------------
\section{Electron-plasma and ion-acoustic waves}
%-------------------------------------------------------------------------------
For both electron-plasma and ion-acoustic waves we can assume the magnetic field does not change. Thus, Faraday's law gives
\begin{equation}
    \nabla \times \Evec = \nabla \times \Evec_t = 0.
\end{equation}
By definition, $\nabla \cdot \Evec_t = 0$. Thus, using \cref{eq:ltw_general_null_vector} we get $\Evec_t = 0$, that is, $\Evec = \Evec_l$. 

For electron-plasma waves, we can use \cref{eq:em_general_wave_form} to write \cref{eq:ep_waves_mom_linearized} in spectral form and thus obtain
\begin{equation}
-i w n_{e0} \hat{\uvec}_{e1} + \frac{e n_{e0}}{m_e} \hat{\Evec}_{1,l} = -\kvec_e \frac{\gamma_e p_{e0}}{n_{e0} m_e} \hat{n}_{e1}. 
\end{equation}
Since the second term on the left-hand side and the term on the right-hand side point along $\kvec_e$, $\hat{\uvec}_{e1}$ also points along $\kvec_e$, that is, $\hat{\uvec}_{e1} = \hat{\uvec}_{e1,l}$.

For ion-acoustic waves, we can use \cref{eq:em_general_wave_form} to write \cref{eq:ia_waves_mom_linearized} in spectral form and thus obtain
\begin{equation}
-i w n_{i0} \hat{\uvec}_{i1} - \frac{Z e n_{i0}}{m_i} \hat{\Evec}_{1,l} = -\kvec_i \frac{\gamma_i p_{i0}}{n_{i0} m_i} \hat{n}_{i1}. 
\end{equation}
Since the second term on the left-hand side and the term on the right-hand side point along $\kvec_i$, $\hat{\uvec}_{i1}$ also points along $\kvec_i$, that is, $\hat{\uvec}_{i1} = \hat{\uvec}_{i1,l}$. Finally, we note that the electric field being purely longitudinal is in agreement with \cref{eq:ia_waves_E}.

%----------------------------------------------------------------------------------------------------------------------
\chapter{Electromagnetic waves in plasmas}
%----------------------------------------------------------------------------------------------------------------------
\label{sec:electromagnetic_waves_plasmas}
In introductory electrodynamics, one typically studies electromagnetic waves in vacuum, that is, for cases where $\rho_e = \Jvec = 0$. In this section we relax both of these assumptions. Consider the electric and magnetic fields as well as the scalar and vector potentials, which satisfy
\begin{equation}
    \label{eq:emp_general_E_potential}
    \Evec = -\nabla \phi - \frac{\partial \Avec}{\partial t},
\end{equation}
\begin{equation}
    \label{eq:emp_general_B_potential}
    \Bvec = \nabla \times \Avec.
\end{equation}
For the above, we choose $\nabla \cdot \Avec = 0$. Using the fact that the magnetic field is solenoidal, we have
\begin{equation*}
    \nabla \cdot \Bvec = \nabla \cdot \Bvec_l + \nabla \cdot \Bvec_t = \nabla \cdot \Bvec_l = 0.
\end{equation*}
However, by definition, $\nabla \times \Bvec_l = 0$ as well. Thus, using \cref{eq:ltw_general_null_vector}, we have $\Bvec_l = 0$. The same argument applies to the vector potential, and thus $\Avec_l = 0$. For the electric field, we have
\begin{equation*}
    \nabla \cdot \Evec = \nabla \cdot \Evec_l + \nabla \cdot \Evec_t = \nabla \cdot \Evec_l .
\end{equation*}
Taking the divergence of \cref{eq:emp_general_E_potential}, we get
\begin{equation}
    \nabla \cdot \Evec = \nabla \cdot \left ( -\nabla \phi \right ).
\end{equation}
Combining the last two equations gives
\begin{equation*}
    \nabla \cdot \left ( \Evec_l + \nabla \phi \right ) = 0.
\end{equation*}
By definition, we also have
\begin{equation*}
    \nabla \times \left ( \Evec_l + \nabla \phi \right ) = 0.
\end{equation*}
Thus, using \cref{eq:ltw_general_null_vector}, we have $\Evec_l = -\nabla \phi$. A similar argument can be used to show $\Evec_t = -\partial \Avec / \partial t$. Our goal in this section will be to determine equations for $\Evec_l$, $\Evec_t$ and $\Bvec$.

We'll begin with the conservation of charge equation
\begin{equation*}
    \frac{\partial \rho_e}{\partial t} + \nabla \cdot \Jvec = 0,
\end{equation*}
which we re-write as
\begin{equation*}
    \frac{\partial \rho_e}{\partial t} + \nabla \cdot \Jvec_l = 0,
\end{equation*}
Using Poisson's equation $\nabla^2\phi = -\rho_e / \epsilon_0$ in the above, we get
\begin{equation*}
    \frac{\partial}{\partial t} \left ( -\epsilon_0 \nabla^2 \phi \right ) + \nabla \cdot \Jvec_l = 0,
\end{equation*}
or
\begin{equation*}
    \nabla \cdot \left ( \frac{\partial \nabla \phi}{\partial t} - \frac{1}{\epsilon_0} \Jvec_l \right ) = 0.
\end{equation*}
However, by definition, we also have
\begin{equation*}
    \nabla \times \left ( \frac{\partial \nabla \phi}{\partial t} - \frac{1}{\epsilon_0} \Jvec_l \right ) = 0.
\end{equation*}
Using \cref{eq:ltw_general_null_vector}, we conclude
\begin{equation}
    \label{eq:emp_general_longitudinal_J}
    \frac{\partial \nabla \phi}{\partial t} = \frac{1}{\epsilon_0} \Jvec_l.
\end{equation}
This gives the equation for $\Evec_l$, namely,
\begin{equation}
    \label{eq:emp_general_long_E}
    \frac{\partial^2 \Evec_l}{\partial t^2} + \frac{1}{\epsilon_0} \frac{\partial \Jvec_l}{\partial t} = 0.
\end{equation}

Both $\Evec_t$ and $\Bvec$ can be extracted from $\Avec$, so now we proceed to find an equation for the transverse vector potential. Ampere's law with Maxwell's correction gives
\begin{equation*}
    \nabla \times \left ( \nabla \times \Avec \right ) = \mu_0 \Jvec + \mu_0 \epsilon_0 \frac{\partial \Evec}{\partial t}.
\end{equation*}
The above is re-written as
\begin{equation*}
    \nabla \left ( \nabla \cdot \Avec \right ) - \nabla^2 \Avec = \mu_0 \Jvec + \mu_0 \epsilon_0 \left ( -\frac{\partial \nabla \phi}{\partial t} - \frac{\partial^2 \Avec}{\partial t^2} \right ),
\end{equation*}
which gives
\begin{equation*}
    \frac{\partial^2 \Avec}{\partial t} - \frac{1}{\mu_0 \epsilon_0} \nabla^2 \Avec = \frac{1}{\epsilon_0} \Jvec - \frac{\partial \nabla \phi}{\partial t},
\end{equation*}
or
\begin{equation*}
    \frac{\partial^2 \Avec}{\partial t} - c_0^2 \nabla^2 \Avec = \frac{1}{\epsilon_0} \Jvec - \frac{\partial \nabla \phi}{\partial t},
\end{equation*}
where $c_0 = 1/\sqrt{\mu_0 \epsilon_0}$. Expanding the current density as $\Jvec = \Jvec_l + \Jvec_t$, and using \cref{eq:emp_general_longitudinal_J}, we get
\begin{equation}
    \label{eq:emp_general_trans_vec_pot}
    \frac{\partial^2 \Avec}{\partial t} - c_0^2 \nabla^2 \Avec = \frac{1}{\epsilon_0} \Jvec_t.
\end{equation}
Using the functional form in \cref{eq:em_general_wave_form} for $\Avec$ and $\Jvec_t$ gives
\begin{equation}
    -w^2 \hat{\Avec} + k^2 c_0^2 \hat{\Avec}= \frac{1}{\epsilon_0} \hat{\Jvec}_t.
\end{equation}
That is, $\Avec$ and $\Jvec_t$ point in the same direction.

Taking the time derivative of \cref{eq:emp_general_trans_vec_pot} gives the equation for $\Evec_t$, that is
\begin{equation}
    \label{eq:emp_general_trans_E}
    \frac{\partial^2 \Evec_t}{\partial t^2} - c_0^2 \nabla^2 \Evec_t + \frac{1}{\epsilon_0} \frac{\partial \Jvec_t}{\partial t} = 0.
\end{equation}
Taking the curl of \cref{eq:emp_general_trans_vec_pot} gives the equation for $\Bvec$, that is
\begin{equation}
    \label{eq:emp_general_trans_B}
    \frac{\partial^2 \Bvec}{\partial t^2} - c_0^2 \nabla^2 \Bvec - \frac{1}{\epsilon_0} \nabla \times \Jvec_t = 0.
\end{equation}
Using the functional form in \cref{eq:em_general_wave_form} for $\Evec_t$, $\Bvec$ and $\Jvec_t$ gives
\begin{equation}
    -w^2 \hat{\Evec}_t + k^2 c_0^2 \hat{\Evec}_t - \frac{iw}{\epsilon_0} \hat{\Jvec}_t = 0,
\end{equation}
\begin{equation}
    -w^2 \Bvec + k^2 c_0^2 \Bvec - \frac{i}{\epsilon_0} \kvec \times \Jvec_t = 0.
\end{equation}
That is, $\Evec_t$ points in the same direction as $\Jvec_t$, which as shown before points in the same direction as $\Avec$. Additionally, $\Bvec$ points in the direction of $\kvec \times \Jvec_t$, that is, it is orthogonal to $\Evec_t$.

We briefly note that taking the curl of \cref{eq:p_waves_maxwell_1}, and using \cref{eq:p_waves_maxwell_2}, gives the wave equation for the total electric field $\Evec$, that is 
\begin{equation}
    \frac{\partial^2 \Evec}{\partial t^2} - c_0^2 \nabla^2 \Evec + c_0^2 \nabla (\nabla \cdot \Evec) + \frac{1}{\epsilon_0} \frac{\partial \Jvec}{\partial t} = 0.
\end{equation}
The above can be considered as the sum of the following three equations
\begin{align*}
    \frac{\partial^2 \Evec_l}{\partial t^2} + \frac{1}{\epsilon_0} \frac{\partial \Jvec_l}{\partial t} &= 0, \\
    \frac{\partial^2 \Evec_t}{\partial t^2} - c_0^2 \nabla^2 \Evec_t + \frac{1}{\epsilon_0} \frac{\partial \Jvec_t}{\partial t} &= 0, \\
    -c_0^2 \nabla^2 \Evec_l + c_0^2 \nabla (\nabla \cdot \Evec_l) &= 0.
\end{align*}
The first is the equation for the longitudinal electric field, that is \cref{eq:emp_general_long_E}. The second is the equation for the transverse electric field, that is \cref{eq:emp_general_trans_E}. The third equation above follows from the vector identity $\nabla \times (\nabla \times \Fvec) = -\nabla^2 \Fvec + \nabla ( \nabla \cdot \Fvec )$ and the fact that $\nabla \times \Evec_l = 0$.

It will often be the case that transverse waves will oscillate at such a fast rate that the ions, which have a large inertia, will be unable to react quickly enough. Thus, we can assume $\uvec_{i,t} = 0$. Given the definition of the current density in \cref{eq:p_waves_curr_density}, the transverse current density is expressed as $\Jvec_t = e \left (Z n_i \uvec_{i,t} - n_e \uvec_{e,t} \right )$, which now simplifies to 
\begin{equation}
    \label{eq:emp_transverse_current}
    \Jvec_t = -e n_e \uvec_{e,t}.
\end{equation}
Thus, the transverse electron velocity $\uvec_{e,t}$ points in the same direction as $\Jvec_t$, which is the same direction as $\Evec_t$ and $\Avec$. The next section focuses on deriving an expression for $\uvec_{e,t}$. 

We begin with \cref{eq:pwaves_electron_momentum}, the electron momentum equation, which, due to the electron continuity equation, can be written as
\begin{equation*}
    m_e n_e \frac{\partial\uvec_e}{\partial t} + m_e n_e \uvec_e \cdot \nabla \uvec_e + e n_e \left ( \Evec + \uvec_e \times \Bvec \right ) = -\nabla p_e,
\end{equation*}
or
\begin{equation*}
    \frac{\partial\uvec_e}{\partial t} + \uvec_e \cdot \nabla \uvec_e + \frac{e}{m_e} \left ( \Evec + \uvec_e \times \Bvec \right ) = -\frac{1}{n_e m_e}\nabla p_e,
\end{equation*}
Using the scalar and vector potentials we have
\begin{equation*}
    \frac{\partial\uvec_e}{\partial t} + \uvec_e \cdot \nabla \uvec_e +\frac{e}{m_e} \left [ -\nabla \phi - \frac{\partial \Avec}{\partial t} + \uvec_e \times \left ( \nabla \times \Avec \right ) \right ] = -\frac{1}{n_e m_e} \nabla p_e.
\end{equation*}
Using the vector identity $\nabla \left ( F^2 / 2 \right ) = \Fvec \times \left ( \nabla \times \Fvec \right ) + \Fvec \cdot \nabla \Fvec$, we write the above as
\begin{equation}
    \frac{\partial\uvec_e}{\partial t} - \uvec_e \times \left ( \nabla \times \uvec_e \right ) + \nabla \left (\frac{u_e^2}{2} \right ) + \frac{e}{m_e} \left [ -\nabla \phi - \frac{\partial \Avec}{\partial t} + \uvec_e \times \left ( \nabla \times \Avec \right ) \right ] = -\frac{1}{n_e m_e} \nabla p_e,
\end{equation}
which is equivalent to 
\begin{equation}
    \label{eq:emp_electron_momentum}
    \frac{\partial\uvec_e}{\partial t} - \uvec_e \times \left ( \nabla \times \uvec_{e,t} \right ) + \nabla \left (\frac{u_e^2}{2} \right ) + \frac{e}{m_e} \left [ -\nabla \phi - \frac{\partial \Avec}{\partial t} + \uvec_e \times \left ( \nabla \times \Avec \right ) \right ] = -\frac{1}{n_e m_e} \nabla p_e.
\end{equation}
We'll now introduce a more specific coordinate system. We'll be dealing with at most three waves at a time: a laser wave, a scattered wave, and a plasma wave (either electron-plasma or ion-acoustic wave). We'll assume all three of these waves lie on a so-called base plane. That is, $\kvec_e$ (or $\kvec_i$), $\kvec_L$, and $\kvec_s$ all point along this plane. We now choose the main transverse direction, that is, the direction of $\uvec_{e,t}$, $\Jvec_t$, $\Evec_t$, and $\Avec$ to be the direction orthogonal to this plane, so that these vectors are orthogonal to any $\kvec$. As an aside, we note that the longitudinal and transverse components of the electron velocity can belong to different waves. That is
\begin{align}
    \uvec_{e,l} &= \hat{\uvec}_{e,l} \exp \left [ i (\kvec_p \cdot \xvec - w_pt) \right ] \\
    \uvec_{e,t} &= \hat{\uvec}_{e,t} \exp \left [ i (\kvec_q \cdot \xvec - w_qt) \right ].
\end{align}
The vectors $\nabla \left ( u_e^2 / 2 \right )$, $\nabla \phi$ and $\nabla p_e$ are all by definition longitudinal. As \cref{eq:ltw_plasma_long_wavevector} states, longitudinal vectors point along their wave vectors. Since we chose all wave vectors to be confined to the base plane, $\nabla \left ( u_e^2 / 2 \right )$, $\nabla \phi$ and $\nabla p_e$ do not have a component along the main transverse direction. As a result, the component of \cref{eq:emp_electron_momentum} along the main transverse direction simplifies to
\begin{equation}
    \label{eq:emp_electron_momentum_transverse}
    \frac{\partial\uvec_{e,t}}{\partial t} - \uvec_{e,l} \times \left ( \nabla \times \uvec_{e,t} \right ) + \frac{e}{m_e} \left [ - \frac{\partial \Avec}{\partial t} + \uvec_{e,l} \times \left ( \nabla \times \Avec \right ) \right ] = 0.
\end{equation}
Using $c = w/k$, we show the following scalings 
\begin{align}
    \frac{1}{c^2} \frac{\partial \uvec_{e,t}}{\partial t} &= -\frac{iw \uvec_{e,t}}{c^2} \sim i \frac{\uvec_{e,t}}{c} k , \nonumber \\
    \frac{1}{c^2} \uvec_{e,l} \times \left (\nabla \times \uvec_{e,t} \right ) & = \frac{i \uvec_{e,l} \times \left ( \kvec \times \uvec_{e,t} \right ) }{c^2} \sim i \frac{\uvec_{e,l}}{c} \frac{\uvec_{e,t}}{c} k , \nonumber \\
    \frac{1}{c^2} \frac{\partial \Avec}{\partial t} &= -\frac{iw \Avec}{c^2} \sim i \frac{\Avec}{c} k , \nonumber \\
    \frac{1}{c^2} \uvec_{e,l} \times \left ( \nabla \times \Avec \right ) &= \frac{i \uvec_{e,l} \times \left ( \kvec \times \Avec \right )}{c^2} \sim i \frac{\uvec_{e,l}}{c} \frac{\Avec}{c} k .
\end{align}
Thus, assuming $\uvec_{e,l} \ll c$, the terms involving the double cross product are smaller than those involving the time derivative. As a result, \cref{eq:emp_electron_momentum_transverse} becomes
\begin{equation}
    \frac{\partial\uvec_{e,t}}{\partial t} - \frac{e}{m_e} \frac{\partial \Avec}{\partial t} = 0.
\end{equation}
Using \cref{eq:em_semi_general_wave_form}, the above is equivalent to 
\begin{equation}
    -i w \uvec_{e,t} +i w \frac{e \Avec}{m_e} = 0,
\end{equation}
which upon re-arranging gives
\begin{equation}
    \label{eq:emp_transverse_velocity}
    \uvec_{e,t} = \frac{e\Avec}{m_e}.
\end{equation}

Using both the transverse current given by \cref{eq:emp_transverse_current} and the transverse velocity given by \cref{eq:emp_transverse_velocity}, \cref{eq:emp_general_trans_vec_pot} can be re-written as
\begin{equation*}
    \frac{\partial^2 \Avec}{\partial t} - c_0^2 \nabla^2 \Avec = -\frac{e n_e}{\epsilon_0} \uvec_{e,t} = -\frac{e^2 n_e}{\epsilon_0 m_e} \Avec.
\end{equation*}
We now use the decomposition $n_e = n_{e0} + n_{e1}$, where $n_{e0}$ is time independent. The above becomes
\begin{equation}
    \label{eq:emp_general_trans_vec_pot_complete}
    \frac{\partial^2 \Avec}{\partial t} + w_{pe}^2 \Avec - c_0^2 \nabla^2 \Avec = -\frac{e^2 n_{e1}}{\epsilon_0 m_e} \Avec,
\end{equation}
where $w_{pe}^2 = e^2 n_{e0} / m_e \epsilon_0$.

%----------------------------------------------------------------------------------------------------------------------
\chapter{Electromagnetic waves in a stable plasma}
%----------------------------------------------------------------------------------------------------------------------
%-------------------------------------------------------------------------------
\section{The vector potential}
%-------------------------------------------------------------------------------
\label{sec:semp_vector_potential}
We start with \cref{eq:emp_general_trans_vec_pot_complete}, but focus on the stable-plasma case, that is, $n_{e1} = 0$. Thus, we have
\begin{equation}
    \label{eq:semp_general_trans_vec_pot_complete}
    \frac{\partial^2 \Avec}{\partial t} + w_{pe}^2 \Avec - c_0^2 \nabla^2 \Avec = 0.
\end{equation}
We note that $n_{e0}$ is only time independent, that is, it is still allowed to vary across space. As a result, $w_{pe}^2$ is also allowed to vary across space. Using \cref{eq:em_semi_general_wave_form} for the vector potential, \cref{eq:semp_general_trans_vec_pot_complete} becomes
\begin{equation*}
    -w^2 \Avec + w_{pe}^2 \Avec - c_0^2 \nabla^2 \Avec = 0.
\end{equation*}
We re-write the above as
\begin{equation*}
    \frac{w^2}{c_0^2} \Avec - \frac{w^2}{c_0^2} \frac{w_{pe}^2}{w^2} \Avec + \nabla^2 \Avec = 0.
\end{equation*}
Defining $\epsilon = 1 - w_{pe}^2 / w^2$, we ultimately get
\begin{equation}
    \label{eq:semp_general_trans_vec_pot_complete_spec}
    \frac{w^2}{c_0^2} \epsilon \Avec + \nabla^2 \Avec = 0.
\end{equation}

We now consider the case of a \textit{uniform} stable plasma, that is, a plasma where $n_{e0}$ is uniform across space, and thus $w_{pe}$ and $\epsilon$ are also uniform across space. Using the standard plane-wave expression $\Avec = \hat{\Avec} \exp[i (\kvec \cdot \xvec - wt)]$ in \cref{eq:semp_general_trans_vec_pot_complete_spec} gives the following dispersion relation
\begin{equation}
    \label{eq:semp_uniform_dispersion_relation}
    \frac{w^2}{c_0^2} \epsilon = k^2.
\end{equation}
We expand the above to obtain
\begin{equation*}
    w^2 - w_{pe}^2 = c_0^2 k^2.
\end{equation*}
Taking the derivative $\partial / \partial k$ on both sides we get
\begin{equation*}
    2w \frac{\partial w}{\partial k} = 2 c_0^2 k,
\end{equation*}
which in turn gives the following expressions for the group velocity $v_g$
\begin{equation}
    v_g = \frac{c_0^2 k}{w}.
\end{equation}
Using \cref{eq:semp_uniform_dispersion_relation}, we can also write the above as
\begin{equation}
    v_g = \frac{c_0^2 k}{w} = c_0^2 \frac{\sqrt{\epsilon}}{c_0} = c_0 \sqrt{\epsilon}.
\end{equation}

%-------------------------------------------------------------------------------
\section{The electric field}
%-------------------------------------------------------------------------------
Taking the time derivative of \cref{eq:semp_general_trans_vec_pot_complete} gives the equation for $\Evec_t$, that is
\begin{equation}
    \label{eq:semp_general_trans_E_complete}
    \frac{\partial^2 \Evec_t}{\partial t^2} + w_{pe}^2 \Evec_t - c_0^2 \nabla^2 \Evec_t = 0.
\end{equation}
Using \cref{eq:em_semi_general_wave_form} for the electric field, the time derivative in \cref{eq:semp_general_trans_E_complete} evaluates such that 
\begin{equation*}
    -w^2 \Evec_t + w_{pe}^2 \Evec_t - c_0^2 \nabla^2 \Evec_t = 0.
\end{equation*}
We re-write the above as 
\begin{equation*}
    \frac{w^2}{c_0^2} \Evec_t - \frac{w^2}{c_0^2} \frac{w_{pe}^2}{w^2} \Evec_t + \nabla^2 \Evec_t = 0,
\end{equation*}
which becomes
\begin{equation}
    \label{eq:semp_general_trans_E_complete_spec}
    \frac{w^2}{c_0^2} \epsilon \Evec_t + \nabla^2 \Evec_t = 0.
\end{equation}

%-------------------------------------------------------------------------------
\section{The magnetic field}
%-------------------------------------------------------------------------------
Taking the curl of \cref{eq:semp_general_trans_vec_pot_complete} gives the equation for $\Bvec$, that is
\begin{equation}
    \label{eq:semp_general_trans_B_complete}
    \frac{\partial^2 \Bvec}{\partial t^2} + w_{pe}^2 \Bvec - c_0^2 \nabla^2 \Bvec + \nabla w_{pe}^2 \times \Avec = 0.
\end{equation}
Using \cref{eq:em_semi_general_wave_form} for the magnetic field, the time derivative in \cref{eq:semp_general_trans_B_complete} evaluates such that 
\begin{equation*}
    -w^2 \Bvec + w_{pe}^2 \Bvec - c_0^2 \nabla^2 \Bvec + \nabla w_{pe}^2 \times \Avec = 0.
\end{equation*}
We re-write the above as
\begin{equation*}
    \frac{w^2}{c_0^2} \Bvec - \frac{w^2}{c_0^2} \frac{w_{pe}^2}{w^2} \Bvec + \nabla^2 \Bvec - \frac{w^2}{c_0^2} \nabla \left ( \frac{w_{pe}^2}{w^2} \right ) \times \Avec = 0,
\end{equation*}
which becomes
\begin{equation*}
    \frac{w^2}{c_0^2} \epsilon \Bvec + \nabla^2 \Bvec + \frac{w^2}{c_0^2} \nabla \epsilon \times \Avec = 0.
\end{equation*}
Using \cref{eq:semp_general_trans_vec_pot_complete_spec} we get 
\begin{equation*}
    \frac{w^2}{c_0^2} \epsilon \Bvec + \nabla^2 \Bvec - \frac{1}{\epsilon} \nabla \epsilon \times \nabla^2 \Avec = 0.
\end{equation*}
The vector identity $\nabla \times \left ( \nabla \times \Fvec \right ) = \nabla \left ( \nabla \cdot \Fvec \right ) - \nabla^2 \Fvec$ gives $\nabla \times \left ( \nabla \times \Avec \right ) = -\nabla^2 \Avec $, or
\begin{equation}
    \label{eq:semp_b_to_a_vec_identity}
    \nabla \times \Bvec = -\nabla^2 \Avec.
\end{equation}
Thus, we finally get
\begin{equation}
    \label{eq:semp_general_trans_B_complete_spec}
    \frac{w^2}{c_0^2} \epsilon \Bvec + \nabla^2 \Bvec + \frac{1}{\epsilon} \nabla \epsilon \times \left ( \nabla \times \Bvec \right ) = 0.
\end{equation}

As a side note, we can use the expressions above to write Ampere's law in a new form. We combine the vector identity \cref{eq:semp_b_to_a_vec_identity} with \cref{eq:semp_general_trans_vec_pot_complete_spec} to obtain
\begin{equation*}
    \nabla \times \Bvec = \frac{w^2}{c_0^2} \epsilon \Avec.
\end{equation*}
Using \cref{eq:em_semi_general_wave_form} for the vector potential, the expression $\Evec_t = -\partial \Avec / \partial t$ gives
\begin{equation*}
    \Evec_t = iw \Avec.
\end{equation*}
Thus, the curl of $\Bvec$ can be expressed as
\begin{equation}
    \nabla \times \Bvec = -i \frac{w}{c_0^2} \epsilon \Evec_t.
\end{equation}

As mentioned in \cref{sec:semp_vector_potential}, for a uniform stable plasma we have $\epsilon$ equal to a constant. Thus, \cref{eq:semp_general_trans_B_complete_spec} becomes identical to \cref{eq:semp_general_trans_E_complete_spec}, that is, the wave forms of $\Evec_t$ and $\Bvec$ are the same. 

%----------------------------------------------------------------------------------------------------------------------
\chapter{Stimulated Raman and Brillouin instabilities}
%----------------------------------------------------------------------------------------------------------------------
%-------------------------------------------------------------------------------
\section{Linearization}
%-------------------------------------------------------------------------------
The following decompositions will be used in the derivation of stimulated Raman and Brillouin instabilities:
\begin{align}
    n_i &= n_{i0} + n_{i1}, \nonumber \\
    n_e &= n_{e0} + n_{e1}, \nonumber \\
    p_i &= p_{i0} + p_{i1}, \nonumber \\
    p_e &= p_{e0} + p_{e1}, \nonumber \\
    \uvec_{i,l} &= \uvec_{i0,l} + \uvec_{i1,l}, \nonumber \\
    \uvec_{e,l} &= \uvec_{e0,l} + \uvec_{e1,l}, \nonumber \\
    \Evec_l &= \Evec_{0,l} + \Evec_{1,l}, \nonumber \\
    \Avec &= \Avec_L + \Avec_s.
\end{align}
For these decompositions, we'll assume
\begin{enumerate}
    \item Terms with a subscript 1 are small and thus products of two small quantities can be neglected. \label{it:p_instabilities_assumption_1}
    \item $\uvec_{i0,l}$, $\uvec_{e0,l}$, and $\Evec_{0,l}$,are zero. \label{it:p_instabilities_assumption_2}
    \item $n_{i0}$, $n_{e0}$, $p_{i0}$, and $p_{e0}$ are uniform in space and time. \label{it:p_instabilities_assumption_3}
\end{enumerate}

Thus, unlike the previous section, we do not assume the plasma is stable, that is, we assume fluctuations such as $n_{e1}$ are small but non zero. $\Avec_L$ is the vector potential associated with the laser light, and $\Avec_s$ is the potential associated with the scattered light. For linearization purposes, we'll assume $\Avec_s$ is small. 

Using the decomposition for $\Avec$, \cref{eq:emp_general_trans_vec_pot_complete} is written as 
\begin{equation}
    \frac{\partial^2 \Avec_L}{\partial t} + \frac{\partial^2 \Avec_s}{\partial t} + w_{pe}^2 \Avec_L + w_{pe}^2 \Avec_s - c_0^2 \nabla^2 \Avec_L - c_0^2 \nabla^2 \Avec_s = -\frac{e^2 n_{e1}}{\epsilon_0 m_e} \Avec_L -\frac{e^2 n_{e1}}{\epsilon_0 m_e} \Avec_s,
\end{equation}
Dropping products of small quantities we have
\begin{equation}
    \label{eq:srs_temp1}
    \frac{\partial^2 \Avec_L}{\partial t} + \frac{\partial^2 \Avec_s}{\partial t} + w_{pe}^2 \Avec_L + w_{pe}^2 \Avec_s - c_0^2 \nabla^2 \Avec_L - c_0^2 \nabla^2 \Avec_s = -\frac{e^2 n_{e1}}{\epsilon_0 m_e} \Avec_L.
\end{equation}
We'll assume the laser light is stable, that is, it satisfies \cref{eq:semp_general_trans_vec_pot_complete}, which we re-write below
\begin{equation}
    \label{eq:srs_temp2}
    \frac{\partial^2 \Avec_L}{\partial t} + w_{pe}^2 \Avec_L - c_0^2 \nabla^2 \Avec_L = 0.
\end{equation}
Thus, \cref{eq:srs_temp1} becomes
\begin{equation}
    \frac{\partial^2 \Avec_s}{\partial t} + w_{pe}^2 \Avec_s - c_0^2 \nabla^2 \Avec_s = -\frac{e^2 n_{e1}}{\epsilon_0 m_e} \Avec_L.
\end{equation}
The above shows that the fluctuating $n_{e1}$ couples with the laser light to serve as a source for the scattered light.

The electron density equation is now written as
\begin{equation*}
    \frac{\partial n_{e0} + n_{e1}}{\partial t } + \nabla \cdot \left [ \left ( n_{e0} + n_{e1} \right )\left ( \uvec_{e,t} + \uvec_{e0,l} + \uvec_{e1,l} \right ) \right ] = 0,
\end{equation*}
which, since $\uvec_{e,t}$ is transverse, can be written as
\begin{equation*}
    \frac{\partial n_{e0} + n_{e1}}{\partial t } + \uvec_{e,t} \cdot \nabla \left (n_{e0} + n_{e1} \right ) + \nabla \cdot \left [ \left ( n_{e0} + n_{e1} \right )\left ( \uvec_{e0,l} + \uvec_{e1,l} \right ) \right ] = 0.
\end{equation*}
Given the assumptions in \cref{it:p_instabilities_assumption_1,it:p_instabilities_assumption_2,it:p_instabilities_assumption_3}, the above simplifies to
\begin{equation*}
    \frac{\partial n_{e1}}{\partial t } + \uvec_{e,t} \cdot \nabla n_{e1} + \nabla \cdot \left ( n_{e0} \uvec_{e1,l} \right ) = 0.
\end{equation*}
Since $\uvec_{e,t}$ and $\nabla n_{e1}$ are orthogonal, we finally have
\begin{equation}
    \label{eq:p_instabilities_e_den_linearized}
    \frac{\partial n_{e1}}{\partial t} + \nabla \cdot \left ( n_{e0} \uvec_{e1,l} \right ) = 0.
\end{equation}

The ion density equation is now written as
\begin{equation*}
    \frac{\partial n_{i0} + n_{i1}}{\partial t } + \nabla \cdot \left [ \left ( n_{i0} + n_{i1} \right )\left ( \uvec_{i,t} + \uvec_{i0,l} + \uvec_{i1,l} \right ) \right ] = 0,
\end{equation*}
As stated in \cref{sec:electromagnetic_waves_plasmas}, it is often the case that transverse waves oscillate at such a fast rate that the ions, which have large inertia, are unable to react on comparable time scales. Thus, we can assume $\uvec_{i,t} = 0$,
\begin{equation*}
    \frac{\partial n_{i0} + n_{i1}}{\partial t } + \nabla \cdot \left [ \left ( n_{i0} + n_{i1} \right )\left ( \uvec_{i0,l} + \uvec_{i1,l} \right ) \right ] = 0.
\end{equation*}
Given the assumptions in \cref{it:p_instabilities_assumption_1,it:p_instabilities_assumption_2,it:p_instabilities_assumption_3}, the above simplifies to
\begin{equation}
    \label{eq:p_instabilities_i_den_linearized}
    \frac{\partial n_{i1}}{\partial t } + \nabla \cdot \left ( n_{i0} \uvec_{i1,l} \right ) = 0.
\end{equation}

Consider the electron momentum equation. Subtracting \cref{eq:emp_electron_momentum_transverse} from \cref{eq:emp_electron_momentum} gives
\begin{equation}
    \label{eq:emp_electron_momentum_longitudinal}
    \frac{\partial\uvec_{e,l}}{\partial t} - \uvec_{e,t} \times \left ( \nabla \times \uvec_{e,t} \right ) + \nabla \left (\frac{u_e^2}{2} \right ) + \frac{e}{m_e} \left [ -\nabla \phi +\uvec_{e,t} \times \left ( \nabla \times \Avec \right ) \right ] = -\frac{1}{n_e m_e} \nabla p_e.
\end{equation}
Since $\uvec_{e,t} = e \Avec / m_e$, the above simplifies to
\begin{equation*}
    \frac{\partial\uvec_{e,l}}{\partial t} + \nabla \left (\frac{u_e^2}{2} \right ) - \frac{e}{m_e} \nabla \phi = -\frac{1}{n_e m_e} \nabla p_e,
\end{equation*}
or 
\begin{equation*}
    \frac{\partial\uvec_{e,l}}{\partial t} + \nabla \left (\frac{u_e^2}{2} \right ) + \frac{e}{m_e} \Evec_l = -\frac{1}{n_e m_e} \nabla p_e.
\end{equation*}
Since $\uvec_{e,l}$ and $\uvec_{e,t}$ are orthogonal $u_e^2 = \uvec_e \cdot \uvec_e = u_{e,l}^2 + u_{e,t}^2$. The electron momentum equation is then 
\begin{equation*}
    \frac{\partial\uvec_{e,l}}{\partial t} + \nabla \left (\frac{u_{e,l}^2 + u_{e,t}^2}{2} \right ) + \frac{e}{m_e} \Evec_l = -\frac{1}{n_e m_e} \nabla p_e.
\end{equation*}
Given the assumptions in \cref{it:p_instabilities_assumption_1,it:p_instabilities_assumption_2,it:p_instabilities_assumption_3}, the above simplifies to
\begin{equation*}
    \frac{\partial n_{e0} \uvec_{e1,l}}{\partial t} + n_{e0} \nabla \left (\frac{u_{e,t}^2}{2} \right ) + \frac{e n_{e0}}{m_e} \Evec_{1,l} = -\frac{1}{m_e} \nabla p_{e1} .
\end{equation*}
For the transverse electron velocity we have
\begin{equation*}
    u_{e,t}^2 = \left ( \frac{e \Avec}{m_e} \right ) \cdot \left ( \frac{e \Avec}{m_e} \right ) = \frac{e^2}{m_e^2} \left ( \Avec_L \cdot \Avec_ L + 2 \Avec_L \cdot \Avec_s + \Avec_s \cdot \Avec_s \right ).
\end{equation*}
Since the product of small quantities can be neglected, the $\Avec_s \cdot \Avec_s$ term is dropped. We'll also ignore the $\Avec_L \cdot \Avec_L$ term, given that $\Avec_L$ is stable and thus its magnitude does not play a critical role in the growth of the instabilities. Thus, the electron momentum equation becomes
\begin{equation}
    \label{eq:p_instabilities_e_mom_linearized}
    \frac{\partial n_{e0} \uvec_{e1,l}}{\partial t} + \frac{e^2 n_{e0}}{m_e^2} \nabla \left (\Avec_L \cdot \Avec_s \right ) + \frac{e n_{e0}}{m_e} \Evec_{1,l} = -\frac{1}{m_e} \nabla p_{e1}.
\end{equation}

Consider now the ion momentum equation, given by \cref{eq:pwaves_ion_momentum}, which we re-write below as 
\begin{equation*}
    \frac{\partial n_i \uvec_i}{\partial t} + \nabla \cdot \left ( n_i \uvec_i \uvec_i \right ) - \frac{Ze n_i}{m_i} \left ( \Evec + \uvec_i \times \Bvec \right ) = -\frac{1}{m_i} \nabla p_i,
\end{equation*}
The longitudinal component of the above is
\begin{equation*}
    \frac{\partial n_i \uvec_{i,l}}{\partial t} + \left [ \nabla \cdot \left (n_i \uvec_i \uvec_i \right ) \right ]_l - \frac{Z e n_i}{m_i} \left ( \Evec_l + \uvec_{i,t} \times \Bvec \right ) = -\frac{1}{m_i} \nabla p_i,
\end{equation*}
where $[\cdot]_l$ denotes longitudinal component. Since $\uvec_{i,t} = 0$, we have
\begin{equation*}
    \frac{\partial n_i \uvec_{i,l}}{\partial t} + \left [ \nabla \cdot \left (n_i \uvec_i \uvec_i \right ) \right ]_l - \frac{Z e n_i}{m_i} \Evec_l = -\frac{1}{m_i} \nabla p_i.
\end{equation*}
Using the variable decompositions, we have
\begin{multline*}
    \frac{\partial}{\partial t} \left [ \left ( n_{i0} + n_{i1} \right ) \left ( \uvec_{i0,l} + \uvec_{i1,l} \right ) \right ] \\
    + \left \{ \nabla \cdot \left [ \left (n_{i0} + n_{i1} \right ) \left ( \uvec_{i,t} + \uvec_{i0,l} + \uvec_{i1,l} \right ) \left ( \uvec_{i,t} + \uvec_{i0,l} + \uvec_{i1,l} \right ) \right ] \right \}_l \\
    - \frac{Z e}{m_i} \left ( n_{i0} + n_{i1} \right ) \left ( \Evec_{0,l} + \Evec_{1,l} \right ) = - \frac{1}{m_i} \nabla \left ( p_{i0} + p_{i1} \right ).
\end{multline*}
Given the assumptions in \cref{it:p_instabilities_assumption_1,it:p_instabilities_assumption_2,it:p_instabilities_assumption_3}, the above simplifies to
\begin{equation}
    \label{eq:p_instabilities_i_mom_linearized}
    \frac{\partial n_{i0} \uvec_{i1,l}}{\partial t} - \frac{Z e n_{i0}}{m_i} \Evec_{1,l} = - \frac{1}{m_i} \nabla p_{i1}.
\end{equation}

%-------------------------------------------------------------------------------
\section{Stimulated Raman Scattering}
%-------------------------------------------------------------------------------
We employ the same assumptions as for the electron-plasma waves, that is 
\begin{enumerate}
    \item Quasi-neutrality for the base flow, $Zn_{i0} = n_{e0}$.
    \item Uniform ion density, $n_{i1} = 0$.
\end{enumerate}

Combining \cref{eq:p_instabilities_e_mom_linearized} with \cref{eq:p_waves_e_pressure_linearized} gives 
\begin{equation}
    \label{eq:srs_mom_linearized}
    \frac{\partial n_{e0} \uvec_{e1,l}}{\partial t} + \frac{e^2 n_{e0}}{m_e^2} \nabla \left (\Avec_L \cdot \Avec_s \right ) + \frac{e n_{e0}}{m_e} \Evec_{1,l} = -\frac{\gamma_e p_{e0}}{n_{e0} m_e} \nabla n_{e1}.
\end{equation}
Taking the time derivative of \cref{eq:p_instabilities_e_den_linearized} and using \cref{eq:srs_mom_linearized} leads to the wave equation for electron density
\begin{equation*}
    \frac{\partial^2 n_{e1}}{\partial t^2} - \frac{e^2 n_{e0}}{m_e^2} \nabla^2 \left (\Avec_L \cdot \Avec_s \right ) - \frac{e n_{e0}}{m_e} \nabla \cdot \Evec_{1,l} = \frac{\gamma_e p_{e0}}{n_{e0} m_e} \nabla^2 n_{e1}.
\end{equation*}
As before, using \cref{eq:ep_waves_efield_divergence} we obtain
\begin{equation*}
    \frac{\partial^2 n_{e1}}{\partial t^2} - \frac{e^2 n_{e0}}{m_e^2} \nabla^2 \left (\Avec_L \cdot \Avec_s \right ) + \frac{e^2 n_{e0}}{m_e \epsilon_0} n_{e1} = \frac{\gamma_e p_{e0}}{n_{e0} m_e} \nabla^2 n_{e1}.
\end{equation*}
or
\begin{equation}
    \frac{\partial^2 n_{e1}}{\partial t^2} + w_{pe}^2 n_{e1} - \frac{\gamma_e p_{e0}}{n_{e0} m_e} \nabla^2 n_{e1} =   \frac{e^2 n_{e0}}{m_e^2} \nabla^2 \left (\Avec_L \cdot \Avec_s \right ).
\end{equation}
Thus, the scattered laser light $\Avec_s$ couples with the laser light to serve as a source for the electron-plasma wave.

%-------------------------------------------------------------------------------
\section{Stimulated Brillouin Scattering}
%-------------------------------------------------------------------------------
We employ the same assumptions as for the ion-acoustic waves, that is
\begin{enumerate}
    \item Quasi-neutrality for the base flow, $Zn_{i0} = n_{e0}$.
    \item Approximate quasi-neutrality for the fluctuations, $Z n_{i1} \approx n_{e1}$.
    \item Negligible electron mass, $m_e \to 0$.
\end{enumerate}
Combining \cref{eq:p_instabilities_i_mom_linearized} with \cref{eq:p_waves_i_pressure_linearized} gives
\begin{equation}
    \label{eq:sbs_mom_linearized}
    \frac{\partial n_{i0} \uvec_{i1,l}}{\partial t} - \frac{Z e n_{i0}}{m_i} \Evec_{1,l} = - \frac{\gamma_i p_{i0}}{n_{i0} m_i} \nabla n_{i1}.
\end{equation}
Taking the time derivative of \cref{eq:p_instabilities_i_den_linearized} and using \cref{eq:sbs_mom_linearized} leads to the wave equation for ion density
\begin{equation}
    \frac{\partial^2 n_{i1}}{\partial t^2} + \frac{Z e n_{i0}}{m_i} \nabla \cdot \Evec_{1,l} = \frac{\gamma_i p_{i0}}{n_{i0} m_i} \nabla^2 n_{i1}.
\end{equation}
For this case, we assume that the mass of the electron, which is significantly smaller than that of the ions, is negligible. Thus, \cref{eq:p_instabilities_e_mom_linearized} simplifies to 
\begin{equation}
    \frac{e^2 n_{e0}}{m_e} \nabla \left (\Avec_L \cdot \Avec_s \right ) + e n_{e0} \Evec_{1,l} = -\frac{\gamma_e p_{e0}}{n_{e0}} \nabla n_{e1}.
\end{equation}
Plugging in the above in the ion wave equation we obtain
\begin{equation}
    \frac{\partial^2 n_{i1}}{\partial t^2} = \frac{Z n_{i0}}{n_{e0}} \frac{\gamma_e p_{e0}}{n_{e0} m_i} \nabla^2 n_{e1} + \frac{\gamma_i p_{i0}}{n_{i0} m_i} \nabla^2 n_{i1} + \frac{Z e^2 n_{i0}}{m_i m_e} \nabla^2 \left (\Avec_L \cdot \Avec_s \right ).
\end{equation}
Due to quasi-neutrality, we have $Zn_{i0} = n_{e0}$ and $Zn_{i1} \approx n_{e1}$, which gives
\begin{equation}
    \frac{\partial^2 n_{i1}}{\partial t^2} = \frac{1}{m_i} \left ( \frac{Z \gamma_e p_{e0}}{n_{e0}} + \frac{\gamma_i p_{i0}}{n_{i0}} \right ) \nabla^2 n_{i1} + \frac{Z e^2 n_{i0}}{m_i m_e} \nabla^2 \left (\Avec_L \cdot \Avec_s \right ).
\end{equation}
Since $p_{i0}/n_{i0} = k_B T_{i0}$ and $p_{e0} / n_{e0} = k_B T_{e0}$, we finally have
\begin{equation}
    \frac{\partial^2 n_{i1}}{\partial t^2} - \left ( \frac{Z \gamma_e k_B T_{e0} + \gamma_i k_B T_{i0}}{m_i} \right ) \nabla^2 n_{i1} = \frac{Z e^2 n_{i0}}{m_i m_e} \nabla^2 \left (\Avec_L \cdot \Avec_s \right ).
\end{equation}
Thus, the scattered laser light $\Avec_s$ couples with the laser light to serve as a source for the ion-acoustic wave.


\bibliographystyle{plainnat}
\bibliography{library}
\end{document}
