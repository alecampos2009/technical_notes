\documentclass[a4paper,12pt]{article}
\usepackage{graphicx,subfigure,amssymb,amstext,amsmath}
\begin{document}

%===========================================================================
\section{Initial Value Problems (IVP's)}
Denoted by conditions that are specified at a point in space.
\subsection{Linear}
%-----------------------------------------------------------------------------------------------------------------------------------
	\subsubsection{First Order}
%-----------------------------------------------------------------------------------------------------------------------------------
	\subsubsection{Second Order}
		
		$y''+p(t)y'+q(t)y=g(t)$ and $y(t_o)=y_o$, $y'(t_o)=y'_o$ has a unique solution in the interval where $p(t)$, $q(t)$, and $g(t)$ are continuous.
		
		Lets form the matrix $\bold{X}= \left( \begin{array}{cc}
y_1 & y_2 \\
y'_1 & y'_2 \\
 \end{array} \right)$, where $y_1$ and $y_2$ are solutions to the ODE. 
 
 The Wronskian is defined as $W(\bold{X})(t)=det(\bold{X}(t))$
		
		\begin{itemize}
		\item Homogenous ODE 
		
		Def: one for which zero is a solution. Look like $y''+p(t)y'+q(t)y=0$
		
		If $W(\bold{X})(t_o)\neq 0$, then $c_1$ and $c_2$ are well defined for any $y_o$ and $y'_o$ specified at $t_o$, and therefore $y=c_1y_1+c_2y_2$ is the solution to any $y_o$ and $y'_o$, which makes it the general solution.
		
		If $W(\bold{X})(t_o)\neq 0$ then $y_1$ and $y_2$ are independent, and are the fundamental solutions, and $\bold{X}$ becomes the fundamental matrix.
		
		
			\begin{itemize}
			\item Constant Coefficient
			
			Assume $e^{rt}$, gives characteristic equation.
				\begin{itemize}
				\item Real distinct roots $r=r_1,r_2$: $y=c_1e^{r_1t}+c_2e^{r_2t}$
				\item Real repeated roots $r_1=r_2$: $y=c_1e^{r_1t}+c_2te^{r_1t}$
				\item Complex roots $r=\lambda \pm i\mu$: $y=c_1e^{\lambda t}cos(\mu t) + c_2 e^{\lambda t} sin(\mu t)$
				\end{itemize}
			\item Non-Constant Coefficient
			\end{itemize}
		\item Nonhomogeneous ODE  
		
		Def: not a homogeneous ODE. Look like $y''+p(t)y'+q(t)y=g(t)$
		
		The general solution is $y=c_1y_1+c_2y_2+y_p$
			\begin{itemize}
			\item Undetermined Coefficients (works for constant coefficients only)
			
			Assume a particular solution analogous to the forcing term up to undetermined coefficients, plug in the ODE to get the coefficient, form the general solution, and find coefficients through the initial conditions.
			\item Variation of Parameters
			
			$y_p(t)=u_1(t)y_1(t)+u_2(t)y_2(t)$. Compute $y^{p'}$ (assuming $u_1'(t)y_1(t)+u_2'(t)y_2(t)=0$) and $y^{p''}$, plug them in ODE to obtain $u_1'(t)y_1'(t)+u_2'(t)y_2'(t)=g(t)$. Finally, solve for $u_1(t)$ and $u_2(t)$, which gives $\left( \begin{array}{c}
u_1  \\
u_2  \end{array} \right)=\int_0^t\bold{X}(\sigma)^{-1}\left( \begin{array}{c}
0  \\
g(\sigma)  \end{array} \right)d\sigma $.
			\end{itemize}
		\end{itemize}
%-----------------------------------------------------------------------------------------------------------------------------------
	\subsubsection{Higher Order}
%-----------------------------------------------------------------------------------------------------------------------------------
	\subsubsection{System of eqs.}
	
	$\bold{x'}=\bold{P(t)}\bold{x}+\bold{g(t)}$
		
	Lets form the matrix $\bold{X}=[\bold{x}^{(1)}(t)...\bold{x}^{(n)}(t)]$.
		
		\begin{itemize}
		\item Homogenous System $\bold{x'}=\bold{P(t)x}$
	
		As with second order equations, if $W(\bold{X})\neq 0$, then $\bold{x}=c_1\bold{x}^{(1)}+...c_n\bold{x}^{(n)}$ is the general solution, $\bold{x}^{(1)}...\bold{x}^{(n)}$ are linearly independent, they are the fundamental solutions, and $\bold{X}$ becomes the fundamental matrix.
			\begin{itemize}
			\item Constant Coefficient $\bold{x'}=\bold{Ax}$
			
			Assume $ \boldsymbol{\xi} e^{rt} $, gives $(\bold{A}-r \bold{I}) \boldsymbol{\xi} = 0$
				\begin{itemize}
				\item real distinct eigenvalues: $\bold{x}=c_1 \boldsymbol{\xi}^{(1)}e^{r_1t}+...+c_n \boldsymbol{\xi}^{(n)} e^{r_nt}$
				\item real repeated eigenvalues: $\bold{x}=c_1\boldsymbol{\xi}^{rt}+c_2\left[ \boldsymbol{\xi} te^{rt}+ \boldsymbol{\eta} e^{rt}\right]$, where $\boldsymbol{\xi}$ is the eigenvector and $\boldsymbol{\eta}$ satisfies $(\bold{A}-r\bold{I}) \boldsymbol{\eta} = \boldsymbol{\xi}$.
				\item complex conjugate eigenvalues ($r_1, r_2$) with corresponding complex conjugate eigenvectors ($\boldsymbol{\xi}^{(1)}, \boldsymbol{\xi}^{(2)}$): \\
				 $\bold{x}=c_1\text{Re}[ \boldsymbol{\xi}^{(1)}e^{r_1t}]+c_2 \text{Im}[ \boldsymbol{\xi}^{(1)}e^{r_1t}]+c_3 \boldsymbol{\xi}^{(3)}e^{r_3t}+...+c_n \boldsymbol{\xi}^{(n)}e^{r_nt}$ 
				\end{itemize}
			\item Non-Constant Coefficient
			\end{itemize}
		\item Nonhomogenous System
		
		The general solution is $\bold{x}=c_1\bold{x}^{(1)}(t)+...+c_n\bold{x}^{(n)}+\bold{v}(t)$
			\begin{itemize}
			\item{Diagonalization}
			
			If $\bold{A}$ can be diagonalize, perform diagonalization, form uncoupled system, solve and recombine to obtain general solution. 
			\item{Undetermined Coefficients}
			
			Assume a particular solution analogous to the forcing term up to undetermined vectors of coefficients, plug in the ODE to get these undetermined vectors. The result is the particular solution.
			\item{Variation of Parameters}
			
			$\bold{v}(t)=\bold{X}(t)\bold{u}(t)$. Follow same approach as in second order case, the end result is $\bold{u}(t)=\int_0^t\bold{X}(\sigma)^{-1}\bold{g}(\sigma)d\sigma$.
			\end{itemize}
		\end{itemize}
%-----------------------------------------------------------------------------------------------------------------------------------
\subsection{Nonlinear}

%===========================================================================
\section{Boundary Value Problems (BVP's)}
Denoted by conditions that are specified at more than one point in space.
\subsection{Linear}
\subsubsection{ First Order}
\subsubsection{Second Order}
	\begin{itemize}
	\item Sturm-Liouville BVP
	
	One that satisfies the S-L homogeneous or nonhomogenous ODE, has bc's $a_1y(a)+a_2y'(a)=0$ and $b_1y(b)+b_2y'(b)=0$, and has $p(x)>0$, $r(x)>0$. These are also known as regular Sturm Liouville BVP. 
		
	The goal is to find $\lambda$'s (known as eigenvalues) that give non-zero solutions (known as eigenfunctions) to the ODE. This is analogous to finding $\lambda$'s that give non-zero solutions (known as eigenvectors) to the matrix equation $\bold{A}\xi=\lambda \xi$.
	
	Common S-L solutions : Fourrier, Bessel, Chebyshev, Legendre, Hermite, Laguerre. 
		\begin{itemize}
		\item Homogenous Sturm-Liouville 
		
		$My=-\lambda^2y$ where $M=\frac{1}{r(x)}\left[\frac{d}{dx}\left(p(x)\frac{d}{dx}\right)+q(x)\right]$
			\begin{itemize}
			\item To solve, find general solution, apply boundary conditions to figure out eigenvalues that will give non zero solutions, obtain the eigenfunctions up to a multiplicative constant.
			\item Infinite series of real, non-negative, distinct eigenvalues $\lambda_1<\lambda_2<...$
			\item One to one correspondence between eigenvalue and LI eigenfunction.
			\item All eigenfunctions of S-L problem are orthogonal $\int_a^b r(x)\phi_m(x)\phi_n(x)dx=N_m\delta_{nm}$, where $N_m=\int_a^br(x)\phi_m^2(x)dx$
			\item A square integrable function $f$ can be expanded as $f=\displaystyle\sum_{n=1}^{\infty}c_n \phi_n(x)$, where $c_m=\frac{\int_a^br(x)f(x)\phi_m(x)dx}{\int_a^br(x)\phi_m^2(x)dx}$
			\end{itemize}
		\item Nonhomogenous Sturm-Liouville
		
		$My=-\mu^2y+f(x)$
			\begin{itemize}
			\item Solution is of the form: $\phi=\displaystyle\sum_{n=1}^{\infty}b_n \phi_n(x)$
			\item If $\mu \neq \lambda_n$ for all $n$, then $b_n=\frac{c_n}{\mu^2-\lambda_n^2}$, where $c_n=\int_a^br(x)f(x)\phi_n(x)dx$.
			\item If $\mu=\lambda_m$
				\begin{itemize}
				\item If $c_m\neq0$, then no solution
				\item If $c_m=0$, $b_n$ is arbitrary, and there are infinite solutions
				\end{itemize}
			\end{itemize}
			
		\item Singular Sturm-Liouville
			\begin{itemize}
			\item A SL problem can be rewritten as $y''+\frac{p'(x)}{p(x)}y'+\frac{q(x)+\lambda^2r(x)}{p(x)}y=0$.
			\item A singular point of an ODE is one where the coefficients blow up. For this case, $p(x) = 0$.
			\item A Singular SL problem is one where we allow a singularity at either or both of the boundaries (i.e. p=0), and the bc at the singular point is one that ensures the following is satisfied $\lim_{x \to a or b} p(x)(y'_ny_m-y'_my_n)\to0$ 
			\item Since we now allow $p=0$ at boundaries, this is an extension of the regular S-L BVP.
			\item A boundary condition that ensures $\lim_{x \to a or b} p(x)(y'_ny_m-y'_my_n)\to0$  is satisfied basically forces the general solution to exclude the singular fundamental solution, that is, the one that blows up.
			\end{itemize}
		\end{itemize}
	\item Other BVP's (i.e. do not satisfy Sturm-Liouville ODE or bc's, are higher order, etc.)
	\end{itemize}
\subsubsection{Higher Order}
\subsection{Nonlinear}
%===========================================================================	
\end{document}